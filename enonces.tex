\documentclass[12pt]{article}
\usepackage[french]{babel}
\usepackage[utf8]{inputenc}
\usepackage[T1]{fontenc}
\usepackage{amsmath}
\usepackage{amssymb}
\usepackage{color}
\definecolor{persianplum}{rgb}{0.44, 0.11, 0.11}
\usepackage{url}
\usepackage[breaklinks]{hyperref}
\hypersetup{
	colorlinks=true,
	linkcolor=persianplum,
	filecolor=blue,
	citecolor=black,      
	urlcolor=cyan,
}

\textwidth=18cm \textheight=23cm \oddsidemargin=-1.00cm
\evensidemargin=-1.00cm
\parindent=1cm
\topmargin=-2cm

\usepackage{amsthm}
\newtheorem{exercise}{Exercice}[section]
\theoremstyle{remark}
\newtheorem*{solution}{Solution}
\theoremstyle{remark}
\newtheorem{remark}{Remarque}

\newcommand{\K}{\mathbb{K}}
\newcommand{\R}{\mathbb{R}}
\newcommand{\C}{\mathbb{C}}
\newcommand{\Q}{\mathbb{Q}}
\newcommand{\N}{\mathbb{N}}
\newcommand{\Z}{\mathbb{Z}}
\newcommand{\U}{\mathbb{U}}
\renewcommand{\L}{\mathcal{L}}
\newcommand{\im}{\emph{Im }}
\DeclareMathOperator{\sgn}{sgn}
\newcommand{\vertiii}[1]{{\left\vert\kern-0.25ex\left\vert\kern-0.25ex\left\vert{}#1 
	\right\vert\kern-0.25ex\right\vert\kern-0.25ex\right\vert}}
\newcommand{\function}[5]{
	$$
	\begin{array}{rccl}
		#1: & #2 & \to & #3 \\
		& #4 & \mapsto & #5
	\end{array}
	$$
}

\begin{document}

\begin{titlepage}
	\centering
	\vspace*{\fill}
	\Huge \textit{\textbf{Exercices MP/MP$^*$}}
	\vspace*{\fill}
\end{titlepage}

\cleardoublepage

\tableofcontents

\cleardoublepage

\section{Algèbre Générale}

\begin{exercise}
	Soit $(G,\cdot)$ un groupe tel que $\exists p\in\N$ tel que
	$f_p,f_{p+1},f_{p+2}$ soient des morphismes où
	\function{f_p}{G}{G}{x}{x^p}
	Montrer que $G$ est un groupe abélien.
\end{exercise}

\begin{remark}
	\phantom{}
	\begin{itemize}
		\item Pour $(\Sigma_{3},\circ)$, on a $f_{0},f_{1},f_{6}$ des morphismes
		mais $\Sigma_{3}$ n'est pas commutatif.
		\item Si $f_{2}$ est un morphisme, pour tout $x,y\in G^2$, on a 
		\begin{align*}
			(xy)^{2}
			&=xy xy\\
			&=x^2y^2
		\end{align*}
		d'où $xy=yx$.
	\end{itemize}
\end{remark}

\begin{exercise}
	Soit $(G,\cdot)$ un groupe fini. Soit $A=\{x\in G,~\omega(x)\text{ est impair}\}$ où
	$\omega(x)$ désigne l'ordre de $x$. Montrer que $A$ est non vide, et que
	$x\mapsto x^2$ est une permutation de $A$.
\end{exercise}

\begin{exercise}
	Soit $\sigma\in\Sigma_n$. On note $\theta(\sigma)$ le nombre d'orbite de
	$\sigma$. Montrer que le nombre minimal de transposition dont $\sigma$ est
	le produit est $n-\theta(\sigma)$.
\end{exercise}

\begin{exercise}
	Soit $(n,m)\in(\N^*)^2$. Combien y a-t-il de morphismes de groupe de
	$\Bigl(\Z/n\Z, +\Bigr)\to\Bigl(\Z/m\Z, +\Bigr)$ ?
\end{exercise}

\begin{remark}
	Exemple pour $f:\Bigl(\Z/4\Z, +\Bigr)\to\Bigl(\Z/6\Z, +\Bigr)$. On note
	$f(\bar{1})=\tilde{x}$, d'où $\tilde{4x}=\tilde{0}$ et $3\mid x$, donc
	$x\in\{0,3\}$. Ainsi, on a ou bien $f=f_{0}:\bar{l}\mapsto \tilde{0}$, ou
	bien $f=f_{1}:\bar{l}\mapsto \tilde{3l}$.
\end{remark}

\begin{exercise}
	Soit $(G,\cdot)$ un groupe abélien fini. Soit $P=\prod_{x\in G}x$. Montrer
	que $P=e_{G}$ (élément neutre de $G$) sauf dans un cas très particulier.
\end{exercise}

\begin{exercise}
	Soit $G$ un sous-groupe additif de $\R$. On suppose qu'il existe un nombre
	fini $n$ d'ensembles de la forme $(x+G)_{x\in\R}$ avec $x+G=\{x+y,~y\in G\}$. Montrer que $G=\R$.
\end{exercise}

\begin{exercise}
	Soit $n\in\N^*$. Combien y a-t-il d'automorphismes de $\Bigl(\Z/n\Z,
	+\Bigr)$ ?
\end{exercise}

\begin{exercise}
	Soit $(G,\cdot)$ un groupe fini et $\varphi$ un morphisme de $G\to G$.
	Montrer que $\vert G\vert=\vert\im\varphi\vert\times\vert\ker\varphi\vert$.
	En déduire que $\ker\varphi=\ker\varphi^2$ si et seulement si
	$\im\varphi=\im\varphi^2$. 
\end{exercise}

\begin{exercise}
	Soit $(G,\cdot)$ un groupe fini d'ordre $n$, et $m\in\N$ tel que $n\wedge m=1$.
	Montrer que pour tout $y\in G$, il existe un unique $x\in G$ tel que $x^m=y$.
\end{exercise}

\begin{exercise}
	Soit $(G,\cdot)$ un groupe fini. Pour $g\in G$, on note 
	$$C(g)=\{hgh^{-1},~h\in G\}$$
	et 
	$$S_{g}=\{x\in G,~xg=gx\}$$
	\begin{enumerate}
		\item Montrer que $S_{g}$ est un sous-groupe de $G$.
		\item Montrer que $\vert G\vert=\vert S_{g}\vert\times\vert C(g)\vert$.
		\item On note $Z(G)=\{x\in G,~\forall y\in G,~xy=yx\}$. Montrer que
		$Z(G)$ est un sous-groupe de $G$, et que pour tout $g\in G$,
		$Z(G)\subset S_{g}$.
		\item On suppose que $\vert G\vert=p^{\alpha}$ où $p$ est premier et
		$\alpha\geqslant1$. Montrer que $\vert Z(G)\vert\neq1$. On pourra
		utiliser le fait que $x\mathcal{R}y$ si et seulement si il existe $h\in
		G$ tel que $y=hxh^{-1}$ est une relation d'équivalence.
		\item On suppose que $\vert G\vert=p^{2}$. Montrer que $G$ est abélien
		et qu'il est isomorphe à $\Z/p^2\Z$ ou à $\Bigl(\Z/p\Z\Bigr)^2$.
	\end{enumerate}
\end{exercise}

\begin{remark}
	Les groupes de cardinal $p^3$ ne sont pas nécessairement abélien, un exemple
	est donné par $D_{4}$, le groupe des isométries du carré (qui est de cardinal
	$2^3=8$).
\end{remark}

\begin{exercise}
	Trouver tous les morphismes de $(\Z,+)$ (respectivement $(\Q,+)$) dans
	$(\Q_{+}^*,\times)$. On pourra poser, pour $p$ premier et $n\in\Z$,
	$\nu_{p}(n)$ la puissance de $p$ dans la décomposition en produit de
	facteurs premiers de $n$.
\end{exercise}

\begin{exercise}
	Soit $G$ un groupe engendré par deux éléments $x,y\neq e_{G}$ tels que
	$x^5=e_{G}$ et $xy=y^2x$. Montrer que $\vert G\vert=155=5\times31$ et qu'il
	est unique à un isomorphisme près.
\end{exercise}

\begin{exercise}
	Soit $(G,\cdot)$ un groupe abélien fini. On note $N=\vee_{x\in G}\omega(x)$
	(ppcm des ordres des éléments de $G$) appelé exposant de $G$, caractérise
	par $\forall k\in\Z, (\forall x\in G,x^{k}=e)$ si et seulement si $(\forall
	x\in G,~\omega(x)\mid k)$ si et seulement si $(N\vert k)$. En particulier,
	$N\mid\vert G\vert$.

	On pose $N=p_{1}^{\alpha_{1}}\dots p_{r}^{\alpha_{r}}$ la décomposition en
	nombres premiers de $N$.
	\begin{enumerate}
		\item Soit $i\in\{1,\dots,r\}$. Justifier qu'il existe $y_{i}\in G$, tel
		que $p_{i}^{\alpha_{i}}\mid \omega(y_{i})$.
		\item Soit $i\in\{1,\dots,r\}$. Justifier qu'il existe $x_{i}\in G$, tel
		que $\omega(x_{i})=p_{i}^{\alpha_{i}}$.
		\item Montrer qu'il existe $x\in G$ tel que $\omega(x)=N$.
	\end{enumerate}
\end{exercise}

\begin{exercise}
	Soit $\K$ un corps fini commutatif, $(\K^*,\times)$ est un groupe abélien fini.
	Soit $N=\vee_{x\in \K^*}\omega(x)$ (ordre multiplicatif). On sait d'après
	l'exercice précédent qu'il existe $x_{0}\in\K^*$ tel que $\omega(x_{0})=N$.
	En étudiant le polynôme $X^{N}-1_{K}$, montrer que $(\K^*,\times)$ est
	cyclique.
	
	En exemple, soit $\Bigl(\Z/13\Z,+,\times\Bigr)$ (c'est un corps).\\
	Trouver un générateur du groupe $\Bigl(\Z/13\Z^*,\times\Bigr)$.
\end{exercise}

\begin{exercise}
	Soit $(G,\cdot)$ un groupe tel que $\forall x\in G,~x^2=e_{G}$.
	\begin{enumerate}
		\item Montrer que $G$ est abélien.
		\item Montrer que si $G$ est fini, il existe $n\in\N$ tel que $G$ soit
		isomorphe à $\Bigl(\bigl(\Z/2\Z\bigr)^n,+\Bigr)$. On pourra considérer
		une famille génératrice minimale.
	\end{enumerate}
\end{exercise}

\begin{exercise}
	Soit $(G,\cdot)$ un groupe, on appelle groupe dérivé de $G$ et on note
	$$D(G)=\{xyx^{-1}y^{-1},~(x,y)\in G^{2}\}$$.
	\begin{enumerate}
		\item Si $G$ est abélien, que vaut $D(G)$ ?
		\item Montrer que pour $n\geqslant3$, les 3-cycles engendrent
		$\mathcal{A}_{n}$ (groupe des permutations de signature égale à 1).
		\item Montrer que deux 3-cycles $(a_{1},a_{2},a_{3})$ et
		$(b_{1},b_{2},b_{3})$ sont conjugués dans $\Sigma_{n}$ (c'est-à-dire
		qu'il existe $\sigma\in\Sigma_{n}$ telle que
		$(b_{1},b_{2},b_{3})=\sigma\circ(a_{1},a_{2},a_{3})\circ\sigma^{-1})$.
		Est-ce encore vrai dans $\mathcal{A}_{n}$?
		\item En déduire $D(\Sigma_{n})$.
	\end{enumerate}
\end{exercise}

\begin{remark}
	Pour $n\geqslant 5$, on a $D(\mathcal{A}_{n})=D(\mathcal{A}_{n})$.
\end{remark}

\begin{exercise}
	Soit $(G,\cdot)$ un groupe fini de cardinal $n$.
	\begin{enumerate}
		\item Soit $g\in G$ et \function{\tau_g}{G}{G}{x}{g\cdot x}
		Montrer que \function{\tau}{G}{\Sigma(G)}{g}{\tau_g} 
		(où $\Sigma(G)$ est le groupe des permutations de $G$) est un morphisme
		injectif. En déduire que $G$ est isomorphe à un sous-groupe de
		$(\Sigma_{n},\circ)$.
		\item Montrer que $G$ est isomorphe à un sous-groupe de $(GL_{n}(\C),\times)$.
	\end{enumerate}
\end{exercise}

\begin{exercise}
	Montrer qu'il n'existe pas $(x,y,z,t,n)\in \N^{5}$ tel que
	$x^{2}+y^{2}+z^{2}=(8t+7)\times 4^{n}$.
\end{exercise}

\begin{exercise}
	Montrer que $10^{10^{n}}\equiv 4 [7]$ pour tout $n\in\N^{*}$.
\end{exercise}

\begin{exercise}
	Pour $n\in\N$, on pose $F_{n}=2^{2^{n}}+1$.
	\begin{enumerate}
		\item Montrer que pour tout $n\geqslant1$, $F_{n}=2+\prod_{k=0}^{n-1}F_{k}$.
		\item En déduire qu'il existe une infinité de nombres premiers.
	\end{enumerate}
\end{exercise}

\begin{remark}
	Si $n\neq m$, alors $F_{n}\wedge F_{m}=1$.
\end{remark}

\begin{exercise}
	Soit $U$ le groupe des inversibles de $\Z/32\Z$.
	\begin{enumerate}
		\item Quel est l'ordre de $\bar{5}$ ?
		\item Montrer que $U=gr\{\bar{-1},\bar{5}\}$ (groupe engendré) et qu'il
		est isomorphe à un groupe produit.
	\end{enumerate}
\end{exercise}

\begin{exercise}
	On note, pour $n\in\N^{*}$, $G_{n}=\{e^{\frac{2\mathrm{i}k\pi}{n}},~k\wedge n=1\}$
	l'ensemble des racines $n$-ièmes de l'unité, on définit $\mu(n)=\sum_{\xi\in
	G_{n}}\xi$.
	\begin{enumerate}
		\item Montrer que si $n\wedge m=1$, alors $\mu(nm)=\mu(m)\mu(n)$.
		\item Calculer $\mu(1)$. Que vaut $\mu(n)$ si
		$n=p_{1}^{\alpha_{1}}\dots p_{r}^{\alpha_{r}}$ (décomposition en nombres
		premiers) ?
		\item Soit $\C^{\N^{*}}$ muni de 
		\function{f\star g}{\N^*}{\C}{n}{(f\star g)(n)=\sum_{d\mid n}f(d)g(n/d)}
		Montrer que $\star$ est une loi associative et commutative, qu'elle
		admet un élément neutre noté $e$. Déterminer l'inverse de $\mu$ pour
		$\star$. On pourra calculer, pour $n\geqslant2$, $\sum_{d\mid n}\mu(d)$.
		\item Que vaut pour $n\in\N^{*}$, $\sum_{d\mid n}d\mu(d/n)$ ?
	\end{enumerate}
\end{exercise}

\begin{exercise}
	Soit $p$ premier. Montrer que
	$$\sum_{k=0}^{p}\binom{p}{k}\binom{p+k}{k}\equiv 2^{p}+1[p^{2}]$$
\end{exercise}

\begin{exercise}
	\phantom{}
	\begin{enumerate}
		\item Montrer que les sous-groupes finis de $(\U,\times)$ sont cycliques
		(où $\U$ est le cercle unité).
		\item Quels sont les sous-groupes finis de $SO_{2}(\R)$ ?
		\item Soit $G$ un sous-groupe fini de $SL_{2}(\R)$. Montrer que 
		\function{\varphi}{\R^2}{\R}{(X,Y)}{\sum_{M\in G}\langle MX,MY\rangle}
		où $\langle\cdot,\cdot\rangle$ est le produit scalaire canonique de
		$\R$. Montrer que $\varphi$ est un produit scalaire pour lequel les
		matrices de $M$ sont des isométries. En déduire que $G$ est cyclique.
	\end{enumerate}
\end{exercise}

\begin{exercise}
	Soit $E=\{x+y\sqrt{2},~x\in\N^{*},~y\in\Z,\text{ et }x^{2}-2y=1\}$.
	\begin{enumerate}
		\item Montrer que $E$ est un sous-groupe de $(\R_{+}^{*},\times)$.
		\item Montrer que $E=\{(x_{0}+y_{0}\sqrt{2})^{n},~n\in\Z\}$ où
		$x_{0}+y_{0}\sqrt{2}=\min E\cap]1,+\infty[$.
	\end{enumerate}
\end{exercise}

\begin{exercise}
	Déterminer les entiers $n\in\N^{*}$ tels que $7\mid n^{n}-3$.
\end{exercise}

\begin{exercise}
	Soit $p$ premier plus grand que 5. Soit $a\in\N$ tel que
	$1+\frac{1}{2}+\dots+\frac{1}{p-1}=\frac{a}{(p-1)!}$.
	Montrer que $p^{2}\mid a$.
\end{exercise}

\begin{exercise}
	Soit $P\in \R[X]$ tel que $\forall x\in\R$, $P(x)\geqslant0$. Montrer qu'il
	existe $(A,B)\in\R[X]^{2}$ tel que $P=A^{2}+B^{2}$.
\end{exercise}

\begin{exercise}
	\phantom{}
	\begin{enumerate}
		\item Soit $\alpha\in\R$ tel que $\frac{\alpha}{\pi}\notin\Q$. Montrer que
		$(\sin(n\alpha))_{n\in\N}$ est dense dans $[-1,1]$.
		\item Montrer qu'il y a une infinité de puissance de 2 qui commencent
		par 7 en base 10.
	\end{enumerate}
\end{exercise}

\begin{exercise}
	Soit $A$ un anneau commutatif intègre, on dit que $A$ est euclidien si et
	seulement s'il existe $v:A\setminus\{0\}\to\N$ tels que pour tout $(a,b)\in
	A\times A\setminus\{0\}$, il existe $(q,r)\in A^{2}$ tels que $a=bq+r$ et
	$v(r)<v(b)$ ou $r=0$.
	\begin{enumerate}
		\item Montrer que $\Z[\mathrm{i}]=\{a+\mathrm{i}b,~(a,b)\in\Z^{2}\}$ est euclidien.
		\item Montrer que tout anneau euclidien est principal.
	\end{enumerate}
\end{exercise}

\begin{exercise}
	\phantom{}
	\begin{enumerate}
		\item Soit $p$ premier plus grand que 3. Soit
		$\bar{x}\in\Z/p\Z\setminus\{\bar{0}\}$. Montrer que $\bar{x}$ est un carré
		dans $\Z/p\Z$ si et seulement $\bar{x}^{\frac{p-1}{2}}=\bar{1}$.
		\item En déduire qu'il existe une infinité de nombres premiers congrus à
		1 modulo 4.
	\end{enumerate}
\end{exercise}

\begin{exercise}
	Soit $P=\sum_{i=0}^{n}r_{i}X^{i}\in\Q[X]\setminus\{0\}$. On pose
	$$c(P)=\prod_{p\in\mathcal{P}}p^{\min\limits_{0\leqslant i\leqslant
	n}(\nu_{p}(r_{i}))}$$
	où $\mathcal{P}$ est l'ensemble des nombres premiers. On écrit $P=c(P)\times
	P_{1}$.
	\begin{enumerate}
		\item Montrer que $P_{1}\in\Z[X]$, que ses coefficients sont premiers
		entre eux dans leur ensemble et qu'une telle écriture est unique.
		\item Soit $(P,Q)\in
		\Bigl(\Q[X]\setminus\{0\}\Bigr)^{2}$. Montrer que $c(PQ)=c(P)c(Q)$. On
		justifiera en passant dans $\Z/p\Z[X]$ que si $p$ premier divise tous
		les coefficients de $P_{1}\times Q_{1}$, alors il divise tous les
		coefficients de $P_{1}$ ou tous ceux que $Q_{1}$ [Lemme de Gauss].
		\item En déduire que si $P\in\Z[X]$ est irréductible sur $\Z[X]$, alors
		il l'est aussi sur $\Q[X]$. La réciproque est-elle vraie ?
		\item Trouver tous les $\theta\in[0,2\pi[$ tels que
		$\frac{\theta}{\pi}\in\Q$ et $\cos(\theta)\in\Q$. Si
		$\theta\not\equiv0[\pi]$ et si $\theta=2\pi p/q$ avec $p\wedge q=1$, on
		appliquera ce qui précède à $A=X^{q}-1$ et $P=X^{2}-(2\cos(\theta))X+1$.
	\end{enumerate}
\end{exercise}

\begin{exercise}
	Soit $P\in\R[X]$ scindé sur $\R$.
	\begin{enumerate}
		\item Montrer que pour tout $\alpha\in\R$, $P+\alpha P'$ est scindé sur $\R$.
		\item Soit $R=\sum_{i=0}^{r}a_{i}X^{i}$ scindé sur $\R$. Montrer que
		$\sum_{i=0}^{r}a_{i}P^{(i)}$ l'est aussi.
	\end{enumerate}
\end{exercise}

\begin{exercise}
	Soit $P\in\R[X]$ de degré $n\geqslant1$, scindé sur $\R$. Montrer que pour
	tout $x\in\R$, $(n-1)(P'^{2})(x)\geqslant nP(x)P''(x)$.
\end{exercise}

\begin{exercise}
	\phantom{}
	\begin{enumerate}
		\item Soit $P\in\Q[X]$ irréductible sur $\Q[X]$, montrer que $R$ n'a que
		des racines simples sur $\C$. On pourra évaluer $P\wedge P'$ sur
		$\Q[X]$.
		\item Soit $A\in\Q[X]$ et $\alpha\in\C$ une racine de $A$ de
		multiplicité $m(\alpha)>d(A)/2$ où $d(A)$ est le degré de $A$. Montrer
		que $\alpha\in\Q$.
		\item Soit $A\in\Q[X]$ de degré $2m+1$. On suppose que $A$ admet une
		racine complexe de multiplicité plus grande que $m$. Montrer que $A$
		possède une racine rationnelle.
	\end{enumerate}
\end{exercise}

\begin{exercise}
	Soit $(G,\cdot)$ un groupe et $A$ une partie finie de $G$ stable pour
	$\cdot$. Montrer que $A$ est en fait un sous-groupe de $G$.
\end{exercise}

\begin{exercise}
	Soit $p$ premier plus grand que 3. Montrer que pour tout $\alpha\in\N$,
	$$(1+p)^{p^{\alpha}}\equiv 1+p^{\alpha+1}[p^{\alpha+2}]$$
\end{exercise}

\begin{exercise}
	Soit pour $n\in\N^{*}$, $\mu(n)=\sum_{\substack{k=1\\k\wedge
	n}}^{n}e^{\frac{2\mathrm{i}k\pi}{n}}=\sum_{\xi\in\Xi_{n}}\xi$ où $\Xi_{n}$
	sont les racines primitives $n$-ièmes de l'unité. On a notamment
	$\vert\Xi_{n}\vert=\varphi(n)$ (fonction d'Euler).
	\begin{enumerate}
		\item Montrer que si $m\wedge n=1$, $\mu(m\times n)=\mu(m)\times\mu(n)$.
		\item Si $n=p_{1}^{\alpha_{1}}\dots p_{r}^{\alpha_{r}}$ (décomposition
		en facteurs premiers), que vaut $\mu(n)$ ?
	\end{enumerate}
\end{exercise}

\begin{exercise}
	Montrer que pour tout $(x,y)\in\Z^{2}$, $7\neq 2x^{2}-5y^{2}$.
\end{exercise}

\begin{exercise}
	Résoudre $x^{3}=1$ dans $\Z/19\Z$.
\end{exercise}

\begin{exercise}
	Soit $n\geqslant 3$.
	\begin{enumerate}
		\item Combien y a-t-il d'inversibles dans
		$\Bigl(\Z/2^{n}\Z,+,\times\Bigr)$ ? On note
		$\Bigl(\Z/2^{n}\Z\Bigr)^{\times}$ le groupe (multiplicatif) de ses
		inversibles.
		\item Montrer que $5^{2^{n-3}}\equiv 1+2^{n-1}[2^{n}]$.
		\item Évaluer l'ordre de 5 dans $\Bigl(\Z/2^{n}\Z\Bigr)^{\times}$.
		\item Montrer que $gr\{-1\}\cap gr\{5\}=\{1\}$ où $gr$ indique le groupe
		engendré par l'ensemble. En déduire que
		$\Biggl(\Bigl(\Z/2^{n}\Z\Bigr)^{\times},\times\Biggr)$ est isomorphe à $\Bigl(\Z/2\Z\times\Z/2^{n-1}\Z,+\Bigr)$.
	\end{enumerate}
\end{exercise}

\begin{exercise}
	Soit $(G,\cdot)$ un ensemble non vide muni d'une loi interne associative. On
	suppose que
	\begin{enumerate}
		\item [(i)] $\exists e\in G,\forall x\in G,~x\cdot e=x$,
		\item [(ii)] $\forall x\in G,\exists x'\in G,~x\cdot x'=e$.
	\end{enumerate}
	Montrer que $(G,\cdot)$ est un groupe.
\end{exercise}

\begin{exercise}
	Montrer qu'il existe une infinité de multiples de 21 qui s'écrivent avec
	uniquement des 1 en base 10.
\end{exercise}

\begin{exercise}
	Soit $\K$ un corps commutatif fini. Soit $n=\vert \K^{*}\vert$.
	\begin{enumerate}
		\item Soit $d$ un diviseur de  $n$, on suppose qu'il existe $x_{0}\in
		\K^{*}$ d'ordre (multiplicatif) $d$ dans le groupe $(\K^{*},\times)$.
		Montrer qu'il existe exactement $\varphi(d)$ éléments d'ordre $d$ dans
		$(\K^{*},\times)$ ($\varphi$ indique la fonction d'Euler). On pourra
		s'intéresser au polynôme $X^{d}-1_{\K}$.
		\item En utilisant $n=\sum_{d\mid n}\varphi(d)$, montrer que
		$(\K^{*},\times)$ est cyclique.
	\end{enumerate}
\end{exercise}

\begin{exercise}
	Soit $p$ premier plus grand que 5. et $M=\Z/p\Z\setminus\{0,1\}$.
	\begin{enumerate}
		\item Montrer que \function{f}{M}{M}{x}{1-x^{-1}}
		est bien définie et calculer $f^{3}$.
		\item Montrer que -3 est un carré dans $\Z/p\Z$ si et seulement si $f$
		admet un point fixe.
		\item Montrer que -3 est un carré dans $\Z/p\Z$ si et seulement si
		$p\equiv 1[3]$ (on pourra décomposer $f$ en produit de cycles de
		supports disjoints).
	\end{enumerate}
\end{exercise}

\begin{exercise}
	Soit $x\in\R$ avec $x=\pm b_{m}b_{m-1}\dots b_{0},a_{1}a_{2}\dots a_{n}\dots$
	(écriture décimale). Montrer que $x\in\Q$ si et seulement si $\exists
	n_{0}\in\N,\exists T\in\N^{*},\forall n\geqslant n_{0},~a_{n+T}=a_{n}$ (la
	suite des décimales et périodique à partir du rang $n_{0}$).
\end{exercise}

\begin{exercise}
	On définit $H_{0}=1$ et pour tout $n\geqslant1$,
	$H_{n}=\frac{X(X-1)\dots(X-n+1)}{n!}$.
	\begin{enumerate}
		\item Montrer que $H_{n}(\Z)\subset\Z$.
		\item Soit $P\in\C[X]$. Montrer que $P(\Z)\subset\Z$ et et seulement si
		$\exists n\in\N,\exists(a_{0},\dots,a_{n})\in\Z^{n+1}$ avec $P=\sum_{k=0}^{n}a_{k}H_{k}$.
	\end{enumerate}
\end{exercise}

\begin{exercise}
	Soit $P\in\Q[X]$ irréductible sur $\Q[X]$, $\alpha\in\C$ racine de $P$.
	Montrer que $\alpha$ est racine simple de $P$. On pourra se demander, si le degré de $P$ est $n$ et
	$P=(X-\alpha)(a_{0}+a_{1}X+\dots+a_{n-1}X^{n-1})$, quels sont les
	coefficients $a_{k}$ de $\Q$ tels que $a_{k}\in\Q$.
\end{exercise}

\begin{exercise}
	Soit $P\in\Q[X]$ de degré 5 tel que $P$ admette une racine complexe $\alpha$
	d'ordre plus grand que 2. Montrer que $P$ admet au moins une racine
	rationnelle. Quels sont les entiers $n\in\N$ tels que si $P\in\Q[X]$ est de
	degré $n$ admette une racine complexe multiple, alors $P$ a une racine
	rationnelle ?
\end{exercise}

\begin{exercise}
	On définit $\Z[\mathrm{i}]=\{a+\mathrm{i}b\mid(a,b)\in\Z^{2}\}$.
	\begin{enumerate}
		\item Montrer que c'est le plus petit sous-anneau de $\C$ contenant $i$.
		\item On définit, pour $z=a+\mathrm{i}b\in\Z[\mathrm{i}]$, $\vert
		z\vert^{2}=a^{2}+b^{2}$. Montrer que $z$ est inverse dans
		$\Z[\mathrm{i}]$ si et seulement si $\vert z\vert^{2}=1$. En déduire
		l'ensemble $U$ des inversibles.
		\item 	\begin{enumerate}
					\item Montrer que pour tout $z_{0}=x_{0}+\mathrm{i}y_{0}\in\C$,
					il existe $z=a+\mathrm{i}b\in\Z[\mathrm{i}]$, $\vert
					z-z_{0}\vert^{2}\leqslant\frac{1}{2}$.
					\item Soit $(z_{1},z_{2})\in\Z[\mathrm{i}]^{2}$ avec
					$z_{2}\neq 0$. Montrer qu'il existe
					$(q,r)\in\Z[\mathrm{i}]^{2}$ tel que $z_{1}=qz_{2}+r$ et
					$\vert r\vert<\vert z_{1}\vert$. A-t-on unicité ?
					\item En déduire que $\Z[\mathrm{i}]$ est principal.
				\end{enumerate}
		\item Montrer que tout élément $z\in\Z[\mathrm{i}]\setminus\{0\}$ peut
		se décomposer en
		$z=u\times\prod_{\rho\in\mathcal{P}_{0}}\rho^{\nu_{\rho}(z)}$ où $u\in
		U$ et $\mathcal{P}_{0}$ est un ensemble d'irréductibles tel que tout
		élément de $\mathcal{P}$ (irréductibles de $\Z[\mathrm{i}]$) est associé
		à un unique élément de $\mathcal{P}_{0}$ (on pourra raisonner par
		récurrence sur $\vert z\vert^{2}\in\N)$. Montrer l'unicité de cette décomposition.
	\end{enumerate}
\end{exercise}

\begin{exercise}
	Soit $p$ premier plus grand que 3. On note $\mathbb{F}_{p}$ le corps
	$\Bigl(\Z/p\Z,+,\times\Bigr)$. On dit que $x\in\mathbb{F}_{p}^{*}$ est un
	résidu quadratique si et seulement si il existe $y\in\mathbb{F}_{p}^{*}$ tel
	que $x=y^{2}$. On note $R$ l'ensemble des résidus quadratiques.
	\begin{enumerate}
		\item Montrer que $R$ est un sous-groupe de $(\mathbb{F}_{p},\times)$ de
		cardinal $\frac{p-1}{2}$ et $a\in R$ si et seulement si
		$a^{\frac{p-1}{2}}=1$.
		\item Montrer que si $p=a^{2}+b^{2}$ avec $(a,b)\in\N^{2}$, alors
		$p\equiv 1[4]$.
		\item Montrer que, pour $k\in\{1,\dots,p-1\}$, \function{f}{\{0,\dots E(\sqrt{p})\}^{2}}{\mathbb{F}_p}{(a,b)}{a-kb}
		n'est pas injective. En déduire qu'il existe $(a_{0},b_{0})\in\{1,\dots
		E(\sqrt{p})\}^{2}$ tel que $k=a_{0}\times b_{0}^{-1}$.
		\item Soit $p$ premier tel qe $p\equiv 1[4]$. Montrer que $p$ est somme
		de deux carrés.
	\end{enumerate}
\end{exercise}

\begin{exercise}[Fermat]
	Soit $p$ premier. On sait, d'après l'exercice précédent, que $p$ est somme
	de deux carrés si et seulement si $p=2$ ou $p\equiv 1[4]$. On note
	$A=\{n\in\N^{*}\mid\exists(a,b)\in\N^{2},~n=a^{2}+b^{2}\}$.
	\begin{enumerate}
		\item Montrer que $A$ est stable par produit. On note alors $P_{1}=\{p
		\text{premier}\mid p=2\text{ ou }p\equiv 1[4]\}$ et $P_{2}=\{p\text{
		premier}\mid p\equiv3[4]\}$.
		\item Soit $n\in\N^{*}$. On suppose que pour tout $p\in P_{2}$,
		$\nu_{p}(n)$ est pair (où $\nu_{p}(n)$ la puissance de $p$ dans la décomposition en produit de
		facteurs premiers de $n$). Montrer que $n\in A$.
		\item Montrer la réciproque (pour $n\in A$, pour $p\in P_{1}\cup
		P_{2})$ tel que $\nu_{p}(n)$ est impair, on montrera que $-1$ est un
		carré dans $\mathbb{F}_{p}$.
	\end{enumerate}
\end{exercise}

\cleardoublepage
\section{Séries numériques et familles sommables}

\begin{exercise}
	Soit la suite définie par $a_{0}=1$ et pour tout $n\geqslant1$,
	$$a_{n}=2a_{\lfloor n/3\rfloor}+3a_{\lfloor n/9\rfloor}$$
	\begin{enumerate}
		\item On pose pour $p\in\N$, $b_{p}=a_{3p}$. Calculer $b_{p}$ en
		fonction de $p$.
		\item Montrer que si $3^{p}\leqslant n<3^{p+1}$, alors $a_{n}=b_{p}$.
		\item Déterminer l'ensemble des valeurs d'adhérence de
		$(\frac{a_{n}}{n})_{n\geqslant 2}$.
	\end{enumerate}
\end{exercise}

\begin{exercise}
	Soit $[a,b]\subset\in\R$ avec $a<b$ et $f:[a,b]\to[a,b]$ continue. Soit
	$x_{0}\in[a,b]$ et pour tout $n\in\N$, $x_{n+1}=f(x_{n})$.
	\begin{enumerate}
		\item Montrer que $f$ admet au moins un point fixe $l\in[a,b]$.
		\item Si $\lim\limits_{n\to+\infty}x_{n+1}-x_{n}=0$, montrer que
		l'ensemble des valeurs d'adhérence de $(x_{n})_{n\in\N}$ est un segment.
		\item En déduire que $(x_{n})_{n\in\N}$ converge si et seulement si
		$\lim\limits_{n\to+\infty}x_{n+1}-x_{n}=0$.
	\end{enumerate}
\end{exercise}

\begin{exercise}
	Soit $\theta\in[0,2\pi[$, on définit $u_{0}=e^{\mathrm{i}\theta}$ et pour
	tout $n\in\N$, $u_{n+1}=u_{n}^{2}$. Peut-on avoir $(u_{n})_{n\in\N}$
	\begin{itemize}
		\item stationnaire ?
		\item convergente ?
		\item périodique ?
		\item dense dans $\U$?
	\end{itemize}
	On pourra étudier le développement binaire de
	$\frac{\theta}{2\pi}=\sum_{k=1}^{+\infty}\frac{a_{k}}{2^{k}}$.
\end{exercise}

\begin{exercise}
	Soit $(a,b)\in\R_{+}^{2}$, étudier $u_{n}=\Bigl(\frac{\sqrt[n]{a}+\sqrt[n]{b}}{2}\Bigr)^{n^{2}}$.
\end{exercise}

\begin{exercise}
	Soit $(x_{n})_{n\in\N}\in\R_{+}^{\N}$ telle que
	$\lim\limits_{n\to+\infty}=0$ et $\sum_{n=0}^{+\infty}x_{n}=+\infty$.
	\begin{enumerate}
		\item Montrer qu'il existe $\varphi:\N\to\N$ bijective telle que
		$(x_{\varphi(n)})$ est décroissante.
		\item Montrer que pour tout $l\in\overline{\R_{+}}$, pour tout
		$\varepsilon>0$, il existe un sous-ensemble $I\subset\N$ fini tel que
		$$\Bigl\vert\sum_{k\in I}x_{k}-l\Bigr\vert\leqslant\varepsilon$$
		ou si $l=+\infty$: $\forall A>0$, il existe un sous-ensemble $I$ fini
		tel que $\sum_{k\in I}x_{k}\geqslant A$.
	\end{enumerate}
\end{exercise}

\begin{exercise}
	Soit $(u_{n})_{n\in\N}\in\R_{+}^{\N}$ telle que
	$\lim\limits_{n\to+\infty}u_{n}\times\sum_{k=0}^{n}u_{k}^{2}=1$.
	Montrer que $u_{n}\sim\frac{1}{\sqrt[3]{3n}}$. Une telle suite existe-t-elle
	?
\end{exercise}

\begin{exercise}
	Étudier $x_{n}=n-\sum_{k=1}^{n}\cosh(\frac{1}{\sqrt[]{k+n}})$.
\end{exercise}

\begin{exercise}
	Soit $(a_{n})_{n\in\N},(b_{n})_{n\in\N},(c_{n})_{n\in\N}$ des suites réelles
	telles que 
	\begin{enumerate}
		\item [(i)] $\lim\limits_{n\to+\infty}a_{n}+b_{n}+c_{n}=0$,
		\item [(ii)] $\lim\limits_{n\to+\infty}e^{a_{n}}+e^{b_{n}}+e^{c_{n}}=3$.
	\end{enumerate}
	Montrer que
	$\lim\limits_{n\to+\infty}a_{n}=\lim\limits_{n\to+\infty}b_{n}=\lim\limits_{n\to+\infty}c_{n}=0$.
	On pourra étudier $\varphi:x\mapsto e^{x}-x-1$.
\end{exercise}

\begin{exercise}
	Soit $u_{0}\in]0,1[$ et pour $n\in\N$, $u_{n+1}=u_{n}-u_{n}^{2}$. On pose
	$v_{n}=\frac{1}{u_{n}}$.
	\begin{enumerate}
		\item Montrer que $(v_{n})_{n\in\N}$ est bien définie.
		\item Montrer que $v_{n}=n+\ln(n)+O(1)$, en déduire un développement de $u_{n}$.
	\end{enumerate}
\end{exercise}

\begin{exercise}
	\phantom{}
	\begin{enumerate}
		\item Montrer que pour tout $n\geqslant 2$, il existe un unique
		$u_{n}\in\R_{+}$ tel que $u_{n}^{n}=u_{n}+n$.
		\item Montrer que $(u_{n})_{n\geqslant 2}$ converge vers $\lambda\in\R_{+}$.
		\item Donner un développement asymptotique à deux termes de $x_{n}-\lambda$.
	\end{enumerate}
\end{exercise}

\begin{exercise}
	Soit $(u_{n})_{n\in\N}$ une suite de réels positifs non tous nuls. On
	suppose que 
	$$u_n=o\Biggl(\sum_{k=0}^{n}u_{k}\Biggr)$$ 
	Soit
	$(a_{n})_{n\in\N}\in\C^{\N}$ de limite $a$. En cas d'existence, évaluer
	$$\lim\limits_{n\to+\infty}\frac{u_{n}a_{0}+u_{n-1}a_{1}+\dots+u_{0}a_{n}}{u_{0}+\dots+u_{n}}$$
\end{exercise}

\begin{exercise}
	\phantom{}
	\begin{enumerate}
		\item Soit $x\in[0,1[$, montrer qu'il existe une unique suite
		$(a_{n})_{n\geqslant 2}$ d'entiers naturels telle que 
		\begin{enumerate}
			\item [(i)] $0\leqslant a_{n}\leqslant n-1$ pour tout $n\geqslant2$,
			\item [(ii)] il existe $m\geqslant n$ tel que $a_{m}<m-1$ pour tout $n\geqslant2$,
			\item [(iii)] $x=\sum_{n=2}^{+\infty}\frac{a_{n}}{n}$.
		\end{enumerate}
		\item Donner une condition nécessaire et suffisante sur
		$(a_{n})_{n\geqslant2}$ pour que $x\in\Q$.
		\item Soit $l\in[-1,1]$, montrer qu'il existe $x\in[0,1[$ tel que $\lim\limits_{n\to+\infty}sin(n!2\pi x)=l$.
	\end{enumerate}
\end{exercise}

\begin{exercise}
	Soit $u_0>0,u_1>0$ et pour tout $n\geqslant 1$,
	$$u_{n+1}=\ln(1+u_{n})+\ln(1+u_{n-1})$$
	Étudier la suite $(u_{n})$. On pourra poser $M_{n}=\max(u_{n},u_{n-1},l)$,
	$m_{n}=\min(u_{n},u_{n-1},l)$ où $l=2\ln(1+l)$ et $l>0$.
\end{exercise}

\begin{exercise}
	Soit $(p,q)\in(\R^{*})^{2}$ avec $p/q\in\R\setminus\Q$. Soit
	$(x_{n})_{n\in\N}$ une suite réelle bornée. On suppose que
	$(e^{\mathrm{i}px_{n}})_{n\in\N}$ et $(e^{\mathrm{i}qx_{n}})_{n\in\N}$
	convergent. Montrer que $(x_{n})_{n\in\N}$ converge. Et si
	$(x_{n})_{n\in\N}$ n'est pas bornée ?
\end{exercise}

\begin{exercise}
	\phantom{}
	\begin{enumerate}
		\item Montrer que pour tout $n\geqslant1$, pour tout
		$k\in\{0,\dots,n\}$, $\binom{n}{k}\leqslant\frac{n^{k}}{k!}$.
		\item Soit $z\in\C$, montrer que 
		$$\Biggl\vert\sum_{k=0}^{n}\frac{z^{k}}{k!}-\Bigl(1+\frac{z}{n}^{n}\Bigr)\Biggr\vert\leqslant\sum_{k=0}^{n}\frac{\vert
		z\vert^{k}}{k!}-\Bigl(1+\frac{\vert z\vert}{n}\Bigr)^{n}$$
		\item En déduire $\lim\limits_{n\to+\infty}\Bigl(1+\frac{z}{n}\Bigr)^{n}$.
	\end{enumerate}
\end{exercise}

\begin{exercise}
	Soit $u_{n}=\prod_{k=2}^{n}\frac{\sqrt{k}-1}{\sqrt{k}+1}$ pour $n\geqslant
	2$. Quelle est la limite de cette suite ? Quelle est la nature de la série
	$\sum_{n\geqslant 2}u_{n}^{\alpha}$ pour $\alpha\in\R$ ?
\end{exercise}

\begin{exercise}
	Soit $(u_{n})_{n\in\N}\in\R_{+}^{\N}$ décroissante de limite nulle. Montrer
	que si $\sum u_{n}$ converge, alors $u_{n}=o\Bigl(\frac{1}{n}\Bigr)$. On
	pourra minorer $u_{n+1}+\dots+u_{2n}$. Montrer ensuite que si $\{p\in\N,
	pu_{p}\geqslant1\}$ est infini, alors $\sum u_{n}$ diverge.
\end{exercise}

\begin{exercise}
	Nature de $\sum u_{n}$ où $u_{n}=$
	\begin{enumerate}
		\item $n^{-1-\frac{1}{n}}$
		\item $\int_{0}^{\frac{\pi}{2}}t^{n}\sin(t)dt$
		\item $\sin(2\pi\frac{n!}{e})$
		\item $\frac{(-1)^{n}}{n^{\alpha}+(-1)^{n}\ln(n)}$ où $\alpha\in\R$
	\end{enumerate}
\end{exercise}

\begin{exercise}
	Montrer la convergence et calculer la somme des différentes séries
	suivantes:
	\begin{enumerate}
		\item $\sum_{n\geqslant1}\sum_{k\geqslant n}\frac{(-1)^{k}}{k}$
		\item $\sum_{n\geqslant0}\frac{1}{(3n)!}$
		\item
		$\sum_{n\geqslant1}\frac{E(n^{\frac{1}{3}})-E(n-1)^{\frac{1}{3}}}{4n-n^{\frac{1}{3}}}$
		où $E$ désigne la partie entière.
	\end{enumerate}
\end{exercise}

\begin{exercise}
	Soit $f:[1,+\infty[\to\R_{+}^{*}$ de classe $\mathcal{C}^{2}$ et telle que
	$\lim\limits_{x\to+\infty}\frac{f'(x)}{f(x)}=a<0$. Montrer la convergence de
	$\sum_{n\geqslant1}f(n)$. Donner un équivalent de
	$R_{n}=\sum_{k=n}^{+\infty} f(k)$.
\end{exercise}

\begin{exercise}
	Donner un équivalent de $S_{n}=\sum_{k=1}^{n}\frac{e^{k}}{k}$.
\end{exercise}

\begin{exercise}
	Donner la nature de $\sum u_{n}$ quand $u_{n}$ vaut
	\begin{enumerate}
		\item $\Bigl(1-\frac{1}{n}\Bigr)^{n^{\alpha}}$ où $\alpha\in\R$
		\item $\frac{1}{\sum_{k=1}^{n}\bigl(\frac{1}{k}\bigr)^{\frac{1}{k}}}$
		\item $\frac{\sin(n!\pi e)}{\ln(n)}$
	\end{enumerate}
\end{exercise}

\begin{exercise}
	Montrer la convergence et calculer la somme de $\sum u_{n}$ où $u_{n}$ vaut
	\begin{enumerate}
		\item $a\ln(n)+b\ln(n+1)+c\ln(n+2)$ pour $n\geqslant1$ (on cherchera d'abord une condition
		nécessaire et suffisante de convergence).
		\item $\frac{2^{n}}{3^{2^{n-1}}+1}$ pour $n\geqslant 1$.
		\item $\frac{k-n\lfloor\frac{k}{n}\rfloor}{k(k+1)}$.
		\item $\arctan(\frac{1}{n^{2}+n+1})$ pour $n\geqslant0$.
	\end{enumerate}
\end{exercise}

\begin{exercise}
	Soit $(u_{n})_{n\geqslant1}\in\R^{\N}$ et $v_{n}=n(u_{n}-u_{n+1})$. Montrer
	que $\sum u_{n}$ et $\sum v_{n}$ ont même nature lorsque
	\begin{enumerate}
		\item [(i)] $(nu_{n})_{n\geqslant1}$ converge vers 0 OU
		\item [(ii)] $(u_{n})_{n\geqslant 1}$ décroît et tend vers 0.
	\end{enumerate}
	Comparer alors les sommes respectives. En déduire, pour $p\geqslant1$ fixé,
	$$\sum_{n=1}^{+\infty}\frac{1}{n(n+1)\dots(n+p)}$$
\end{exercise}

\begin{exercise}
	Soit $q\in\Z$ et $v_{n}=\frac{1}{(n+q)!}\sum_{k=1}^{n}k!$. Donner la nature
	de $\sum v_{n}$. En cas de divergence, donner un équivalent des sommes partielles.
\end{exercise}

\begin{exercise}
	Soit $(a,b,c)\in (\N^{*})^{3}$, $z\in\C$, $\vert z\vert<1$. Montrer, en
	justifiant l'existence:
	$$\sum_{n=0}^{+\infty}\frac{z^{nb}}{1+z^{na+c}}=\sum_{n=0}^{+\infty}\frac{(-1)^{n}z^{nc}}{1-z^{na+b}}$$
\end{exercise}

\begin{exercise}
	Soit $\sum_{n\geqslant1} a_{n}$ une série complexe absolument convergente.
	On pose pour $q\in\N^{*}$,
	$b_q=\frac{1}{q(q+1)}(a_{1}+2a_{2}+\dots+qa_{q})$. Montrer que
	$\sum_{q\geqslant1}b_{q}$ converge et évaluer sa somme en fonction de
	$\sum_{n=1}^{+\infty}a_{n}$. On pourra poser $u_{n,q}=\frac{na_{n}}{q(q+1)}$
	si $n\leqslant q$ et 0 sinon.
\end{exercise}

\begin{exercise}
	Soit $(u_{n})_{n\geqslant1}\in\R_{+}^{\N}$ telle que $\sum u_{n}<+\infty$.
	On pose $v_{n}=\frac{1}{n(n+1)}(u_{1}+\dots+nu_{n})$ et
	$w_{n}=\sqrt[n]{u_1\times u_2\times\dots\times u_n}$. On admet que pour tout
	$n\in\N^{*}$, pour tout $(a_{1},\dots,a_{n})\in\R_{+}^{n}$, on a l'inégalité
	entre la moyenne géométrique et arithmétique:
	$$\sqrt[n]{a_{1}\dots a_{n}}\leqslant\frac{1}{n}(a_{1}+\dots+a_{n})$$
	avec égalité si et seulement si $a_{1}=\dots=a_{n}$.

	Montrer que $\sum w_{n}$ converge et que $\sum_{n=1}^{+\infty}w_{n}\leqslant
	e\sum_{n=1}^{+\infty}u_{n}$. On pourra utiliser l'exercice précédent.
	Montrer que $e$ est la "meilleure" constante possible, c'est-à-dire que si
	$\forall (u_{n})_{n\geqslant1}\in(\R_{+}^{*})^{\N^{*}}$ telle que $\sum
	u_{n}$ converge, on a $\sum w_{n}\leqslant C\sum u_{n}$ alors $C\geqslant e$.
\end{exercise}

\begin{exercise}
	\phantom{}
	\begin{enumerate}
		\item Trouver une condition nécessaire et suffisante sur $\alpha\in\R$
		pour que
		$\Bigl(\frac{1}{(p+q)^{\alpha}}\Bigr)_{(p,q)\in\N^{2}\setminus\{(0,0)\}}$
		soit sommable et exprimer alors la somme en fonction de la fonction
		$\zeta$ de Riemann.
		\item Trouver une condition nécessaire et suffisante sur $\alpha\in\R$
		pour que
		$\Bigl(\frac{1}{(p^{2}+q^{2})^{\alpha}}\Bigr)_{(p,q)\in\N^{2}\setminus\{(0,0)\}}$
		soit sommable.
	\end{enumerate}
\end{exercise}

\begin{exercise}
	Étudier la sommabilité de
	$\Bigl(\frac{1}{(m+n^{2})(m+n^{2}+1)}\Bigr)_{(m,n)\in\N^{2}}$.\\
	En déduire la
	valeur de $\sum_{n=1}^{+\infty}\frac{E(\sqrt{n})}{n(n+1)}$.
\end{exercise}

\begin{exercise}
	\phantom{}
	\begin{enumerate}
		\item Montrer que pour tout $s\in]1,+\infty[$, le produit infini
		$\prod_{k=1}^{+\infty}\frac{1}{1-\frac{1}{p_{k}^{s}}}$ converge (où les
		$p_{k}$ sont les nombres premiers). Donner sa valeur en fonction de
		$\zeta(s)$.
		\item Généraliser ce résultat à $s\in\C$ avec $\Re(s)>1$.
	\end{enumerate}
\end{exercise}

\begin{exercise}
	On note $\varphi(n)=\vert\{k\in\{1,\dots,n\},~k\wedge n=1\}\vert$. Pour quelles
	valeurs de $\alpha\in\R$ la somme $\sum \frac{\varphi(n)}{n^{\alpha}}$
	converge-t-elle ? Donner alors sa somme en fonction de $\zeta(\alpha)$.
\end{exercise}

\begin{exercise}
	Soit $(z_{n})_{n\in\N}\in(\C^{*})^{\N}$ telle que pour tout $n\neq m$,
	$\vert z_{n}-z_{m}\vert\geqslant1$. Montrer que
	$\sum_{n\in\N}\frac{1}{z_{n}^{3}}$ converge.
\end{exercise}

\begin{exercise}
	Donner la nature de $\sum_{n\geqslant1}\frac{(-1)^{E(\sqrt{n})}}{n}$.
\end{exercise}

\begin{exercise}
	Pour $(a,b)\in(\R\setminus\Z^{*})$, on définit
	$u_{n}=\frac{a(a+1)\dots(a+n)}{b(b+1)\dots(b+n)}$....
	\begin{enumerate}
		\item Donner une condition nécessaire et suffisante pour que $\sum
		u_{n}$ converge.
		\item Dans ce cas, calculer sa somme.
		\item Faire le cas où $a=-\frac{1}{2}$ et $b=1$.
	\end{enumerate}
\end{exercise}

\begin{exercise}
	Soit $u_{n}=\frac{\ln(n)}{n}$ et $v_{n}=(-1)^{n}u_{n}$ pour $n\geqslant1$.
	\begin{enumerate}
		\item Donner la nature de $\sum u_{n}$ et $\sum v_{n}$.
		\item Soit $S_{N}=\sum_{n=1}^{N}u_{n}$. Donner un équivalent de $S_{N}$
		puis développer jusqu'au o(1).
		\item Exprimer $\sum_{n=2}^{+\infty}v_{n}$ en fonction de $\gamma$
		(constante d'Euler) et $\ln(2)$.
	\end{enumerate}
\end{exercise}

\begin{exercise}
	Soit pour $n\in\N^{*}$, $q_{1}(n)$ la nombre de chiffres de l'écriture
	décimale de $n$. On définit par récurrence $q_{k+1}(n)=q_{1}(q_{k}(n))$.
	Étudier la convergence de 
	$$\sum_{n\geqslant1}\frac{1}{nq_{1}(n)q_{2}(n)\dots q_{n}(n)}$$
\end{exercise}

\begin{exercise}
	Soit $P_{n}(X)=\sum_{k=0}^{n}\frac{X^{k}}{k!}$.
	\begin{enumerate}
		\item Montrer que pour tout $n\in\N$, $P_{2n}>0$ sur $\R$ et $P_{2n+1}$
		s'annule une seule fois en $a_{2n+1}<0$.
		\item Déterminer $\lim\limits_{n\to+\infty}a_{2n+1}$.
	\end{enumerate}
\end{exercise}

\begin{exercise}
	Montrer qu'il existe un unique $x_{n}\geqslant0$ tel que
	$e^{x_{n}}=x_{n}+n$. Donne un développement asymptotique à deux termes de
	$x_{n}$ pour $n\geqslant1$.
\end{exercise}

\begin{exercise}
	Soit $(u_{n})_{n\in\N}\in(\R_{+}^{*})^{\N}$, on pose
	$S_{n}=\sum_{k=0}^{n}u_{k}$. Soit $\alpha\in\R$ et
	$v_{n}=\frac{u_{n}}{S_{n}^{\alpha}}$.
	\begin{enumerate}
		\item On suppose que $\sum u_{n}$ converge, étudier $\sum v_{n}$.
		\item On suppose que $\sum u_{n}$ diverge. Pour $\alpha=1$, montrer que
		pour tout $(n,p)\in\N^{2}$, $v_{n+1}+\dots+v_{n+p}\geqslant
		1-\frac{S_{n}}{S_{n+p}}$. En déduire que $\sum v_{n}$ diverge.
		\item On suppose que $\sum u_{n}$ diverge. Pour $\alpha>1$, on forme
		$w_{n}=\int_{S_{n-1}}^{S_{n}}\frac{dt}{t^{\alpha}}$. Montrer que $\sum
		v_{n}$ converge. Et si $\alpha<1 ?$
		\item On suppose que $\sum u_{n}$ converge. On pose
		$R_{n}=\sum_{k=n}^{+\infty}u_{k}$ et
		$w_{n}=\frac{u_{n}}{R_{n}^{\alpha}}$. Étudier la nature de $\sum w_{n}$.
	\end{enumerate}
\end{exercise}

\begin{exercise}[Principe des tiroirs de Dirichlet]
	Soit $x\in\R\setminus\Q$.
	\begin{enumerate}
		\item Soit $n\in\N^{*}$, montrer qu'il existe
		$(p,q)\in\Z\times\{1,\dots,n\}$ tel que $\bigl\vert
		x-\frac{p}{q}\bigr\vert<\frac{1}{qn}$. On pourra étudier les $n+1$ réels
		$(kx-\lfloor kx\rfloor)=(x_{k})_{0\leqslant k\leqslant n}$ et montrer
		qu'il existe $k\neq k'$ avec $\vert x_{k}-x_{k'}\vert<\frac{1}{n}$.
		\item Montrer qu'il existe
		$(p_{n},q_{n})_{n\in\N}\in\Z^{\N}\times(\N^{*})^{\N}$ telles que $\bigl\vert
		x-\frac{p_{n}}{q_{n}}\bigr\vert<\frac{1}{q_{n}^{2}}$ et
		$\lim\limits_{n\to+\infty}q_{n}=+\infty$.
		\item Étudier la convergence de la suite
		$\Bigl(\frac{1}{n\sin(n)}\Bigr)_{n\geqslant1}$ (on admet que $\pi\notin\R\setminus\Q)$.
	\end{enumerate}
\end{exercise}

\begin{exercise}
	Soit $(a_{n,p})\in\C^{(\N^{*})^{2}}$ telle que 
	\begin{enumerate}
		\item [(i)] pour tout $p\in\N^{*}$, il existe $\lim\limits_{n\to+\infty}a_{n,p}=a_{p}\in\C$,
		\item [(ii)] il existe une suite de réels positifs $(b_{p})$ donc la
		série converge telle que pour tout $(n,p)\in(\N^{*})^{2}$, $\lvert
		a_{n,p}\rvert\leqslant b_{p}$.	
	\end{enumerate}
	\begin{enumerate}
		\item Évaluer $\lim\limits_{n\to+\infty}\sum_{p=1}^{n}a_{n,p}$.
		\item Calculer $\lim\limits_{n\to+\infty}\Bigl(\bigl(\frac{1}{n}\bigr)^{n}+\bigl(\frac{2}{n}\bigr)^{n}+\dots+\bigl(\frac{n-1}{n}\bigr)^{n}\Bigr)$.
	\end{enumerate}
\end{exercise}

\begin{exercise}
	Soit $\sum_{n\geqslant1}u_{n}$ une série complexe absolument convergente.
	\begin{enumerate}
		\item Montrer que pour tout $k\geqslant1$, on peut définir $S_{k}=\sum_{n=1}^{+\infty}u_{kn}$.
		\item On suppose que pour tout $k\geqslant1$, $S_{k}=0$. Montrer que
		pour tout $n\geqslant1$, $u_{n}=0$.
	\end{enumerate}
\end{exercise}

\begin{exercise}
	Soit $f:\R\to\R$ telle que pour toute suite $(u_{n})_{n\in\N}\in\R^{\N}$, si
$\sum u_{n}$ converge, alors $\sum f(u_{n})$ converge.
\begin{enumerate}
	\item Montrer que $f(0)=0$ et que $f$ est continue en 0.
	\item Montrer qu'il existe $\alpha>0$, $\forall x\in]-\alpha,\alpha[$,
	$f(x)=-f(x)$ ($f$ est impaire au voisinage de 0).
	\item Montrer qu'il existe $\beta>0$ $\forall(x,y)\in]-\beta,\beta[^{2}$,
	$f(x+y)=f(x)+f(y)$ ($f$ est linéaire au voisinage de 0).
	\item Montrer qu'il existe $\lambda\in\R$ et $\gamma>0$ tels que $\forall
	x\in]-\gamma,\gamma[$, $f(x)=\lambda x$ ($f$ est une homothétie au voisinage
	de 0).
\end{enumerate}
\end{exercise}

\cleardoublepage
\section{Probabilités sur un univers dénombrable}
\cleardoublepage
\section{Calcul matriciel}
\cleardoublepage
\section{Réduction des endomorphismes}
\cleardoublepage
\section{Espaces vectoriels normés}
\cleardoublepage
\section{Fonction d'une variable réelle}
\cleardoublepage
\section{Suites et séries de fonctions}
\cleardoublepage
\section{Séries entières}
\cleardoublepage
\section{Intégration}
\cleardoublepage
\section{Espaces préhilbertiens}
\cleardoublepage
\section{Espaces euclidiens}
\cleardoublepage
\section{Calcul différentiel}
\cleardoublepage
\section{\'Equation différentielles linéaires}

\end{document}