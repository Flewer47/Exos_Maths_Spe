\documentclass[12pt]{article}
\usepackage[french]{babel}
\usepackage[utf8]{inputenc}
\usepackage[T1]{fontenc}
\usepackage{amsmath}
\usepackage{amssymb}
\usepackage{graphicx}
\usepackage{color}
\definecolor{persianplum}{rgb}{0.44, 0.11, 0.11}
\usepackage{url}
\usepackage[breaklinks]{hyperref}
\hypersetup{colorlinks=true, linkcolor=persianplum, filecolor=blue,
	citecolor=black,
	urlcolor=cyan,}

\textwidth=18cm \textheight=23cm \oddsidemargin=-1.00cm \evensidemargin=-1.00cm
\parindent=1cm \topmargin=-2cm

\usepackage{amsthm}
\newtheorem{exercise}{Exercice}[section]
\theoremstyle{remark}
\newtheorem*{solution}{Solution}
\theoremstyle{remark}
\newtheorem{remark}{Remarque}

\newcommand{\K}{\mathbb{K}} \newcommand{\R}{\mathbb{R}}
\newcommand{\C}{\mathbb{C}} \newcommand{\Q}{\mathbb{Q}}
\newcommand{\N}{\mathbb{N}} \newcommand{\Z}{\mathbb{Z}}
\newcommand{\U}{\mathbb{U}} \newcommand{\E}{\mathbb{E}}
\newcommand{\M}{\mathcal{M}} \renewcommand{\L}{\mathcal{L}}
\renewcommand{\P}{\mathbb{P}} \newcommand{\im}{\emph{Im}}
\DeclareMathOperator{\sgn}{sgn} \DeclareMathOperator{\diag}{diag}
\DeclareMathOperator{\rg}{rg} \DeclareMathOperator{\Tr}{Tr}
\DeclareMathOperator{\Sp}{Sp} \DeclareMathOperator{\mat}{mat}
\DeclareMathOperator{\com}{com} \DeclareMathOperator{\conv}{conv}
\newcommand{\vertiii}[1]{{\left\vert\kern-0.25ex\left\vert\kern-0.25ex\left\vert{}#1
\right\vert\kern-0.25ex\right\vert\kern-0.25ex\right\vert}}
\newcommand{\function}[5]{
	$$
	\begin{array}{rccl}
		#1: & #2 & \to & #3 \\
		& #4 & \mapsto & #5
	\end{array}
	$$
}

\begin{document}

\begin{titlepage}
	\centering
	\vspace*{\fill}
	\Huge \textit{\textbf{Exercices MP/MP$^*$}}
	\vspace*{\fill}
\end{titlepage}

\cleardoublepage

\tableofcontents

\cleardoublepage

\section{Algèbre Générale}

\begin{exercise}
	Soit $(G,\cdot)$ un groupe tel que $\exists p\in\N$ tel que
	$f_p,f_{p+1},f_{p+2}$ soient des morphismes où
	\function{f_p}{G}{G}{x}{x^p}
	Montrer que $G$ est un groupe abélien.
\end{exercise}

\begin{exercise}
	Soit $(G,\cdot)$ un groupe fini. Soit $A=\{x\in G,~\omega(x)\text{est
	impair}\}$ où $\omega(x)$ désigne l'ordre de $x$. Montrer que $A$ est non
	vide, et que $x\mapsto x^2$ est une permutation de $A$.
\end{exercise}

\begin{exercise}
	Soit $\sigma\in\Sigma_n$. On note $\theta(\sigma)$ le nombre d'orbite de
	$\sigma$. Montrer que le nombre minimal de transposition dont $\sigma$ est le
	produit est $n-\theta(\sigma)$.
\end{exercise}

\begin{exercise}
	Soit $(n,m)\in(\N^*)^2$. Combien y a-t-il de morphismes de groupe de
	$\Bigl(\Z/n\Z, +\Bigr)\to\Bigl(\Z/m\Z, +\Bigr)$ ?
\end{exercise}

\begin{exercise}
	Soit $(G,\cdot)$ un groupe abélien fini. Soit $P=\prod_{x\in G}x$. Montrer que
	$P=e_{G}$ (élément neutre de $G$) sauf dans un cas très particulier.
\end{exercise}

\begin{exercise}
	Soit $G$ un sous-groupe additif de $\R$. On suppose qu'il existe un nombre
	fini $n$ d'ensembles de la forme $(x+G)_{x\in\R}$ avec $x+G=\{x+y,~y\in G\}$.
	Montrer que $G=\R$.
\end{exercise}

\begin{exercise}
	Soit $n\in\N^*$. Combien y a-t-il d'automorphismes de $\Bigl(\Z/n\Z, +\Bigr)$?
\end{exercise}

\begin{exercise}
	Soit $(G,\cdot)$ un groupe fini et $\varphi$ un morphisme de $G\to G$. Montrer
	que $\vert G\vert=\vert\im\varphi\vert\times\vert\ker\varphi\vert$. En déduire
	que $\ker\varphi=\ker\varphi^2$ si et seulement si $\im\varphi=\im\varphi^2$.
	
\end{exercise}

\begin{exercise}
	Soit $(G,\cdot)$ un groupe fini d'ordre $n$, et $m\in\N$ tel que $n\wedge
	m=1$. Montrer que pour tout $y\in G$, il existe un unique $x\in G$ tel que
	$x^m=y$.
\end{exercise}

\begin{exercise}
	Soit $(G,\cdot)$ un groupe fini. Pour $g\in G$, on note 
	$$C(g)=\{hgh^{-1},~h\in G\}$$ et 
	$$S_{g}=\{x\in G,~xg=gx\}$$
	\begin{enumerate}
		\item
		Montrer que $S_{g}$ est un sous-groupe de $G$.
		\item
		Montrer que $\vert G\vert=\vert S_{g}\vert\times\vert C(g)\vert$.
		\item
		On note $Z(G)=\{x\in G,~\forall y\in G,~xy=yx\}$. Montrer que $Z(G)$ est
		un sous-groupe de $G$, et que pour tout $g\in G$, $Z(G)\subset S_{g}$.
		\item
		On suppose que $\vert G\vert=p^{\alpha}$ où $p$ est premier et
		$\alpha\geqslant1$. Montrer que $\vert Z(G)\vert\neq1$. On pourra utiliser
		le fait que $x\mathcal{R}y$ si et seulement si il existe $h\in G$ tel que
		$y=hxh^{-1}$ est une relation d'équivalence.
		\item
		On suppose que $\vert G\vert=p^{2}$. Montrer que $G$ est abélien et qu'il
		est isomorphe à $\Z/p^2\Z$ ou à $\Bigl(\Z/p\Z\Bigr)^2$.
	\end{enumerate}
\end{exercise}

\begin{exercise}
	Trouver tous les morphismes de $(\Z,+)$ (respectivement $(\Q,+)$) dans
	$(\Q_{+}^*,\times)$. On pourra poser, pour $p$ premier et $n\in\Z$,
	$\nu_{p}(n)$ la puissance de $p$ dans la décomposition en produit de facteurs
	premiers de $n$.
\end{exercise}

\begin{exercise}
	Soit $G$ un groupe engendré par deux éléments $x,y\neq e_{G}$ tels que
	$x^5=e_{G}$ et $xy=y^2x$. Montrer que $\vert G\vert=155=5\times31$ et qu'il
	est unique à un isomorphisme près.
\end{exercise}

\begin{exercise}
	Soit $(G,\cdot)$ un groupe abélien fini. On note $N=\vee_{x\in G}\omega(x)$
	(ppcm des ordres des éléments de $G$) appelé exposant de $G$, caractérise par
	$\forall k\in\Z, (\forall x\in G,x^{k}=e)$ si et seulement si $(\forall x\in
	G,~\omega(x)\mid k)$ si et seulement si $(N\vert k)$. En particulier,
	$N\mid\vert G\vert$.

	On pose $N=p_{1}^{\alpha_{1}}\dots p_{r}^{\alpha_{r}}$ la décomposition en
	nombres premiers de $N$.
	\begin{enumerate}
		\item
		Soit $i\in\{1,\dots,r\}$. Justifier qu'il existe $y_{i}\in G$, tel que
		$p_{i}^{\alpha_{i}}\mid \omega(y_{i})$.
		\item
		Soit $i\in\{1,\dots,r\}$. Justifier qu'il existe $x_{i}\in G$, tel que
		$\omega(x_{i})=p_{i}^{\alpha_{i}}$.
		\item
		Montrer qu'il existe $x\in G$ tel que $\omega(x)=N$.
	\end{enumerate}
\end{exercise}

\begin{exercise}
	Soit $\K$ un corps fini commutatif, $(\K^*,\times)$ est un groupe abélien
	fini. Soit $N=\vee_{x\in \K^*}\omega(x)$ (ordre multiplicatif). On sait
	d'après l'exercice précédent qu'il existe $x_{0}\in\K^*$ tel que
	$\omega(x_{0})=N$. En étudiant le polynôme $X^{N}-1_{K}$, montrer que
	$(\K^*,\times)$ est cyclique.
	
	En exemple, soit $\Bigl(\Z/13\Z,+,\times\Bigr)$ (c'est un corps).\\
	Trouver un générateur du groupe $\Bigl(\Z/13\Z^*,\times\Bigr)$.
\end{exercise}

\begin{exercise}
	Soit $(G,\cdot)$ un groupe tel que $\forall x\in G,~x^2=e_{G}$.
	\begin{enumerate}
		\item
		Montrer que $G$ est abélien.
		\item
		Montrer que si $G$ est fini, il existe $n\in\N$ tel que $G$ soit isomorphe
		à $\Bigl(\bigl(\Z/2\Z\bigr)^n,+\Bigr)$. On pourra considérer une famille
		génératrice minimale.
	\end{enumerate}
\end{exercise}

\begin{exercise}
	Soit $(G,\cdot)$ un groupe, on appelle groupe dérivé de $G$ et on note
	$$D(G)=\{xyx^{-1}y^{-1},~(x,y)\in G^{2}\}$$.
	\begin{enumerate}
		\item
		Si $G$ est abélien, que vaut $D(G)$?
		\item
		Montrer que pour $n\geqslant3$, les 3-cycles engendrent $\mathcal{A}_{n}$
		(groupe des permutations de signature égale à 1).
		\item
		Montrer que deux 3-cycles $(a_{1},a_{2},a_{3})$ et $(b_{1},b_{2},b_{3})$
		sont conjugués dans $\Sigma_{n}$ (c'est-à-dire qu'il existe
		$\sigma\in\Sigma_{n}$ telle que
		$(b_{1},b_{2},b_{3})=\sigma\circ(a_{1},a_{2},a_{3})\circ\sigma^{-1})$.
		Est-ce encore vrai dans $\mathcal{A}_{n}$?
		\item
		En déduire $D(\Sigma_{n})$.
	\end{enumerate}
\end{exercise}

\begin{exercise}
	Soit $(G,\cdot)$ un groupe fini de cardinal $n$.
	\begin{enumerate}
		\item
		Soit $g\in G$ et \function{\tau_g}{G}{G}{x}{g\cdot x} Montrer que
		\function{\tau}{G}{\Sigma(G)}{g}{\tau_g}
		(où $\Sigma(G)$ est le groupe des permutations de $G$) est un morphisme
		injectif. En déduire que $G$ est isomorphe à un sous-groupe de
		$(\Sigma_{n},\circ)$.
		\item
		Montrer que $G$ est isomorphe à un sous-groupe de $(GL_{n}(\C),\times)$.
	\end{enumerate}
\end{exercise}

\begin{exercise}
	Montrer qu'il n'existe pas $(x,y,z,t,n)\in \N^{5}$ tel que
	$x^{2}+y^{2}+z^{2}=(8t+7)\times 4^{n}$.
\end{exercise}

\begin{exercise}
	Montrer que $10^{10^{n}}\equiv 4 [7]$ pour tout $n\in\N^{*}$.
\end{exercise}

\begin{exercise}
	Pour $n\in\N$, on pose $F_{n}=2^{2^{n}}+1$.
	\begin{enumerate}
		\item
		Montrer que pour tout $n\geqslant1$, $F_{n}=2+\prod_{k=0}^{n-1}F_{k}$.
		\item
		En déduire qu'il existe une infinité de nombres premiers.
	\end{enumerate}
\end{exercise}

\begin{exercise}
	Soit $U$ le groupe des inversibles de $\Z/32\Z$.
	\begin{enumerate}
		\item
		Quel est l'ordre de $\bar{5}$?
		\item
		Montrer que $U=gr\{\bar{-1},\bar{5}\}$ (groupe engendré) et qu'il est
		isomorphe à un groupe produit.
	\end{enumerate}
\end{exercise}

\begin{exercise}
	On note, pour $n\in\N^{*}$, $G_{n}=\{e^{\frac{2\mathrm{i}k\pi}{n}},~k\wedge
	n=1\}$ l'ensemble des racines $n$-ièmes de l'unité, on définit
	$\mu(n)=\sum_{\xi\in G_{n}}\xi$.
	\begin{enumerate}
		\item
		Montrer que si $n\wedge m=1$, alors $\mu(nm)=\mu(m)\mu(n)$.
		\item
		Calculer $\mu(1)$. Que vaut $\mu(n)$ si $n=p_{1}^{\alpha_{1}}\dots
		p_{r}^{\alpha_{r}}$ (décomposition en nombres premiers)?
		\item
		Soit $\C^{\N^{*}}$ muni de \function{f\star g}{\N^*}{\C}{n}{(f\star
		g)(n)=\sum_{d\mid n}f(d)g(n/d)} Montrer que $\star$ est une loi
		associative et commutative, qu'elle admet un élément neutre noté $e$.
		Déterminer l'inverse de $\mu$ pour $\star$. On pourra calculer, pour
		$n\geqslant2$, $\sum_{d\mid n}\mu(d)$.
		\item
		Que vaut pour $n\in\N^{*}$, $\sum_{d\mid n}d\mu(d/n)$?
	\end{enumerate}
\end{exercise}

\begin{exercise}
	Soit $p$ premier. Montrer que
	$$\sum_{k=0}^{p}\binom{p}{k}\binom{p+k}{k}\equiv 2^{p}+1[p^{2}]$$
\end{exercise}

\begin{exercise}
	\phantom{}
	\begin{enumerate}
		\item
		Montrer que les sous-groupes finis de $(\U,\times)$ sont cycliques (où
		$\U$ est le cercle unité).
		\item
		Quels sont les sous-groupes finis de $SO_{2}(\R)$?
		\item
		Soit $G$ un sous-groupe fini de $SL_{2}(\R)$. Montrer que
		\function{\varphi}{\R^2}{\R}{(X,Y)}{\sum_{M\in G}\langle MX,MY\rangle} où
		$\langle\cdot,\cdot\rangle$ est le produit scalaire canonique de $\R$.
		Montrer que $\varphi$ est un produit scalaire pour lequel les matrices de
		$M$ sont des isométries. En déduire que $G$ est cyclique.
	\end{enumerate}
\end{exercise}

\begin{exercise}
	Soit $E=\{x+y\sqrt{2},~x\in\N^{*},~y\in\Z,\text{et}x^{2}-2y=1\}$.
	\begin{enumerate}
		\item
		Montrer que $E$ est un sous-groupe de $(\R_{+}^{*},\times)$.
		\item
		Montrer que $E=\{(x_{0}+y_{0}\sqrt{2})^{n},~n\in\Z\}$ où
		$x_{0}+y_{0}\sqrt{2}=\min E\cap]1,+\infty[$.
	\end{enumerate}
\end{exercise}

\begin{exercise}
	Déterminer les entiers $n\in\N^{*}$ tels que $7\mid n^{n}-3$.
\end{exercise}

\begin{exercise}
	Soit $p$ premier plus grand que 5. Soit $a\in\N$ tel que
	$1+\frac{1}{2}+\dots+\frac{1}{p-1}=\frac{a}{(p-1)!}$. Montrer que $p^{2}\mid
	a$.
\end{exercise}

\begin{exercise}
	Soit $P\in \R[X]$ tel que $\forall x\in\R$, $P(x)\geqslant0$. Montrer qu'il
	existe $(A,B)\in\R[X]^{2}$ tel que $P=A^{2}+B^{2}$.
\end{exercise}

\begin{exercise}
	\phantom{}
	\begin{enumerate}
		\item
		Soit $\alpha\in\R$ tel que $\frac{\alpha}{\pi}\notin\Q$. Montrer que
		$(\sin(n\alpha))_{n\in\N}$ est dense dans $[-1,1]$.
		\item
		Montrer qu'il y a une infinité de puissance de 2 qui commencent par 7 en
		base 10.
	\end{enumerate}
\end{exercise}

\begin{exercise}
	Soit $A$ un anneau commutatif intègre, on dit que $A$ est euclidien si et
	seulement s'il existe $v:A\setminus\{0\}\to\N$ tels que pour tout $(a,b)\in
	A\times A\setminus\{0\}$, il existe $(q,r)\in A^{2}$ tels que $a=bq+r$ et
	$v(r)<v(b)$ ou $r=0$.
	\begin{enumerate}
		\item
		Montrer que $\Z[\mathrm{i}]=\{a+\mathrm{i}b,~(a,b)\in\Z^{2}\}$ est
		euclidien.
		\item
		Montrer que tout anneau euclidien est principal.
	\end{enumerate}
\end{exercise}

\begin{exercise}
	\phantom{}
	\begin{enumerate}
		\item
		Soit $p$ premier plus grand que 3. Soit
		$\bar{x}\in\Z/p\Z\setminus\{\bar{0}\}$. Montrer que $\bar{x}$ est un carré
		dans $\Z/p\Z$ si et seulement $\bar{x}^{\frac{p-1}{2}}=\bar{1}$.
		\item
		En déduire qu'il existe une infinité de nombres premiers congrus à 1
		modulo 4.
	\end{enumerate}
\end{exercise}

\begin{exercise}
	Soit $P=\sum_{i=0}^{n}r_{i}X^{i}\in\Q[X]\setminus\{0\}$. On pose
	$$c(P)=\prod_{p\in\mathcal{P}}p^{\min\limits_{0\leqslant i\leqslant
	n}(\nu_{p}(r_{i}))}$$ où $\mathcal{P}$ est l'ensemble des nombres premiers. On
	écrit $P=c(P)\times P_{1}$.
	\begin{enumerate}
		\item
		Montrer que $P_{1}\in\Z[X]$, que ses coefficients sont premiers entre eux
		dans leur ensemble et qu'une telle écriture est unique.
		\item
		Soit $(P,Q)\in \Bigl(\Q[X]\setminus\{0\}\Bigr)^{2}$. Montrer que
		$c(PQ)=c(P)c(Q)$. On justifiera en passant dans $\Z/p\Z[X]$ que si $p$
		premier divise tous les coefficients de $P_{1}\times Q_{1}$, alors il
		divise tous les coefficients de $P_{1}$ ou tous ceux que $Q_{1}$ [Lemme de
		Gauss].
		\item
		En déduire que si $P\in\Z[X]$ est irréductible sur $\Z[X]$, alors il l'est
		aussi sur $\Q[X]$. La réciproque est-elle vraie ?
		\item
		Trouver tous les $\theta\in[0,2\pi[$ tels que $\frac{\theta}{\pi}\in\Q$ et
		$\cos(\theta)\in\Q$. Si $\theta\not\equiv0[\pi]$ et si $\theta=2\pi p/q$
		avec $p\wedge q=1$, on appliquera ce qui précède à $A=X^{q}-1$ et
		$P=X^{2}-(2\cos(\theta))X+1$.
	\end{enumerate}
\end{exercise}

\begin{exercise}
	Soit $P\in\R[X]$ scindé sur $\R$.
	\begin{enumerate}
		\item
		Montrer que pour tout $\alpha\in\R$, $P+\alpha P'$ est scindé sur $\R$.
		\item
		Soit $R=\sum_{i=0}^{r}a_{i}X^{i}$ scindé sur $\R$. Montrer que
		$\sum_{i=0}^{r}a_{i}P^{(i)}$ l'est aussi.
	\end{enumerate}
\end{exercise}

\begin{exercise}
	Soit $P\in\R[X]$ de degré $n\geqslant1$, scindé sur $\R$. Montrer que pour
	tout $x\in\R$, $(n-1)(P'^{2})(x)\geqslant nP(x)P''(x)$.
\end{exercise}

\begin{exercise}
	\phantom{}
	\begin{enumerate}
		\item
		Soit $P\in\Q[X]$ irréductible sur $\Q[X]$, montrer que $R$ n'a que des
		racines simples sur $\C$. On pourra évaluer $P\wedge P'$ sur $\Q[X]$.
		\item
		Soit $A\in\Q[X]$ et $\alpha\in\C$ une racine de $A$ de multiplicité
		$m(\alpha)>d(A)/2$ où $d(A)$ est le degré de $A$. Montrer que
		$\alpha\in\Q$.
		\item
		Soit $A\in\Q[X]$ de degré $2m+1$. On suppose que $A$ admet une racine
		complexe de multiplicité plus grande que $m$. Montrer que $A$ possède une
		racine rationnelle.
	\end{enumerate}
\end{exercise}

\begin{exercise}
	Soit $(G,\cdot)$ un groupe et $A$ une partie finie de $G$ stable pour $\cdot$.
	Montrer que $A$ est en fait un sous-groupe de $G$.
\end{exercise}

\begin{exercise}
	Soit $p$ premier plus grand que 3. Montrer que pour tout $\alpha\in\N$,
	$$(1+p)^{p^{\alpha}}\equiv 1+p^{\alpha+1}[p^{\alpha+2}]$$
\end{exercise}

\begin{exercise}
	Soit pour $n\in\N^{*}$, $\mu(n)=\sum_{\substack{k=1\\k\wedge
	n}}^{n}e^{\frac{2\mathrm{i}k\pi}{n}}=\sum_{\xi\in\Xi_{n}}\xi$ où $\Xi_{n}$
	sont les racines primitives $n$-ièmes de l'unité. On a notamment
	$\vert\Xi_{n}\vert=\varphi(n)$ (fonction d'Euler).
	\begin{enumerate}
		\item
		Montrer que si $m\wedge n=1$, $\mu(m\times n)=\mu(m)\times\mu(n)$.
		\item
		Si $n=p_{1}^{\alpha_{1}}\dots p_{r}^{\alpha_{r}}$ (décomposition en
		facteurs premiers), que vaut $\mu(n)$?
	\end{enumerate}
\end{exercise}

\begin{exercise}
	Montrer que pour tout $(x,y)\in\Z^{2}$, $7\neq 2x^{2}-5y^{2}$.
\end{exercise}

\begin{exercise}
	Résoudre $x^{3}=1$ dans $\Z/19\Z$.
\end{exercise}

\begin{exercise}
	Soit $n\geqslant 3$.
	\begin{enumerate}
		\item
		Combien y a-t-il d'inversibles dans $\Bigl(\Z/2^{n}\Z,+,\times\Bigr)$? On
		note $\Bigl(\Z/2^{n}\Z\Bigr)^{\times}$ le groupe (multiplicatif) de ses
		inversibles.
		\item
		Montrer que $5^{2^{n-3}}\equiv 1+2^{n-1}[2^{n}]$.
		\item
		Évaluer l'ordre de 5 dans $\Bigl(\Z/2^{n}\Z\Bigr)^{\times}$.
		\item
		Montrer que $gr\{-1\}\cap gr\{5\}=\{1\}$ où $gr$ indique le groupe
		engendré par l'ensemble. En déduire que
		$\Biggl(\Bigl(\Z/2^{n}\Z\Bigr)^{\times},\times\Biggr)$ est isomorphe à
		$\Bigl(\Z/2\Z\times\Z/2^{n-1}\Z,+\Bigr)$.
	\end{enumerate}
\end{exercise}

\begin{exercise}
	Soit $(G,\cdot)$ un ensemble non vide muni d'une loi interne associative. On
	suppose que
	\begin{enumerate}
		\item
		[(i)] $\exists e\in G,\forall x\in G,~x\cdot e=x$,
		\item
		[(ii)] $\forall x\in G,\exists x'\in G,~x\cdot x'=e$.
	\end{enumerate}
	Montrer que $(G,\cdot)$ est un groupe.
\end{exercise}

\begin{exercise}
	Montrer qu'il existe une infinité de multiples de 21 qui s'écrivent avec
	uniquement des 1 en base 10.
\end{exercise}

\begin{exercise}
	Soit $\K$ un corps commutatif fini. Soit $n=\vert \K^{*}\vert$.
	\begin{enumerate}
		\item
		Soit $d$ un diviseur de $n$, on suppose qu'il existe $x_{0}\in \K^{*}$
		d'ordre (multiplicatif) $d$ dans le groupe $(\K^{*},\times)$. Montrer
		qu'il existe exactement $\varphi(d)$ éléments d'ordre $d$ dans
		$(\K^{*},\times)$ ($\varphi$ indique la fonction d'Euler). On pourra
		s'intéresser au polynôme $X^{d}-1_{\K}$.
		\item
		En utilisant $n=\sum_{d\mid n}\varphi(d)$, montrer que $(\K^{*},\times)$
		est cyclique.
	\end{enumerate}
\end{exercise}

\begin{exercise}
	Soit $p$ premier plus grand que 5. et $M=\Z/p\Z\setminus\{0,1\}$.
	\begin{enumerate}
		\item
		Montrer que \function{f}{M}{M}{x}{1-x^{-1}} est bien définie et calculer
		$f^{3}$.
		\item
		Montrer que -3 est un carré dans $\Z/p\Z$ si et seulement si $f$ admet un
		point fixe.
		\item
		Montrer que -3 est un carré dans $\Z/p\Z$ si et seulement si $p\equiv
		1[3]$ (on pourra décomposer $f$ en produit de cycles de supports
		disjoints).
	\end{enumerate}
\end{exercise}

\begin{exercise}
	Soit $x\in\R$ avec $x=\pm b_{m}b_{m-1}\dots b_{0},a_{1}a_{2}\dots a_{n}\dots$
	(écriture décimale). Montrer que $x\in\Q$ si et seulement si $\exists
	n_{0}\in\N,\exists T\in\N^{*},\forall n\geqslant n_{0},~a_{n+T}=a_{n}$ (la
	suite des décimales et périodique à partir du rang $n_{0}$).
\end{exercise}

\begin{exercise}
	On définit $H_{0}=1$ et pour tout $n\geqslant1$,
	$H_{n}=\frac{X(X-1)\dots(X-n+1)}{n!}$.
	\begin{enumerate}
		\item
		Montrer que $H_{n}(\Z)\subset\Z$.
		\item
		Soit $P\in\C[X]$. Montrer que $P(\Z)\subset\Z$ et et seulement si $\exists
		n\in\N,\exists(a_{0},\dots,a_{n})\in\Z^{n+1}$ avec
		$P=\sum_{k=0}^{n}a_{k}H_{k}$.
	\end{enumerate}
\end{exercise}

\begin{exercise}
	Soit $P\in\Q[X]$ irréductible sur $\Q[X]$, $\alpha\in\C$ racine de $P$.
	Montrer que $\alpha$ est racine simple de $P$. On pourra se demander, si le
	degré de $P$ est $n$ et $P=(X-\alpha)(a_{0}+a_{1}X+\dots+a_{n-1}X^{n-1})$,
	quels sont les coefficients $a_{k}$ de $\Q$ tels que $a_{k}\in\Q$.
\end{exercise}

\begin{exercise}
	Soit $P\in\Q[X]$ de degré 5 tel que $P$ admette une racine complexe $\alpha$
	d'ordre plus grand que 2. Montrer que $P$ admet au moins une racine
	rationnelle. Quels sont les entiers $n\in\N$ tels que si $P\in\Q[X]$ est de
	degré $n$ admette une racine complexe multiple, alors $P$ a une racine
	rationnelle?
\end{exercise}

\begin{exercise}
	On définit $\Z[\mathrm{i}]=\{a+\mathrm{i}b\mid(a,b)\in\Z^{2}\}$.
	\begin{enumerate}
		\item
		Montrer que c'est le plus petit sous-anneau de $\C$ contenant $i$.
		\item
		On définit, pour $z=a+\mathrm{i}b\in\Z[\mathrm{i}]$, $\vert
		z\vert^{2}=a^{2}+b^{2}$. Montrer que $z$ est inverse dans $\Z[\mathrm{i}]$
		si et seulement si $\vert z\vert^{2}=1$. En déduire l'ensemble $U$ des
		inversibles.
		\item
		\begin{enumerate}
					\item
					Montrer que pour tout $z_{0}=x_{0}+\mathrm{i}y_{0}\in\C$, il
					existe $z=a+\mathrm{i}b\in\Z[\mathrm{i}]$, $\vert
					z-z_{0}\vert^{2}\leqslant\frac{1}{2}$.
					\item
					Soit $(z_{1},z_{2})\in\Z[\mathrm{i}]^{2}$ avec $z_{2}\neq 0$.
					Montrer qu'il existe $(q,r)\in\Z[\mathrm{i}]^{2}$ tel que
					$z_{1}=qz_{2}+r$ et $\vert r\vert<\vert z_{1}\vert$. A-t-on
					unicité?
					\item
					En déduire que $\Z[\mathrm{i}]$ est principal.
				\end{enumerate}
		\item
		Montrer que tout élément $z\in\Z[\mathrm{i}]\setminus\{0\}$ peut se
		décomposer en
		$z=u\times\prod_{\rho\in\mathcal{P}_{0}}\rho^{\nu_{\rho}(z)}$ où $u\in U$
		et $\mathcal{P}_{0}$ est un ensemble d'irréductibles tel que tout élément
		de $\mathcal{P}$ (irréductibles de $\Z[\mathrm{i}]$) est associé à un
		unique élément de $\mathcal{P}_{0}$ (on pourra raisonner par récurrence
		sur $\vert z\vert^{2}\in\N)$. Montrer l'unicité de cette décomposition.
	\end{enumerate}
\end{exercise}

\begin{exercise}
	Soit $p$ premier plus grand que 3. On note $\mathbb{F}_{p}$ le corps
	$\Bigl(\Z/p\Z,+,\times\Bigr)$. On dit que $x\in\mathbb{F}_{p}^{*}$ est un
	résidu quadratique si et seulement si il existe $y\in\mathbb{F}_{p}^{*}$ tel
	que $x=y^{2}$. On note $R$ l'ensemble des résidus quadratiques.
	\begin{enumerate}
		\item
		Montrer que $R$ est un sous-groupe de $(\mathbb{F}_{p},\times)$ de
		cardinal $\frac{p-1}{2}$ et $a\in R$ si et seulement si
		$a^{\frac{p-1}{2}}=1$.
		\item
		Montrer que si $p=a^{2}+b^{2}$ avec $(a,b)\in\N^{2}$, alors $p\equiv
		1[4]$.
		\item
		Montrer que, pour $k\in\{1,\dots,p-1\}$, \function{f}{\{0,\dots
		E(\sqrt{p})\}^{2}}{\mathbb{F}_p}{(a,b)}{a-kb} n'est pas injective. En
		déduire qu'il existe $(a_{0},b_{0})\in\{1,\dots E(\sqrt{p})\}^{2}$ tel que
		$k=a_{0}\times b_{0}^{-1}$.
		\item
		Soit $p$ premier tel qe $p\equiv 1[4]$. Montrer que $p$ est somme de deux
		carrés.
	\end{enumerate}
\end{exercise}

\begin{exercise}[Fermat]
	Soit $p$ premier. On sait, d'après l'exercice précédent, que $p$ est somme de
	deux carrés si et seulement si $p=2$ ou $p\equiv 1[4]$. On note
	$A=\{n\in\N^{*}\mid\exists(a,b)\in\N^{2},~n=a^{2}+b^{2}\}$.
	\begin{enumerate}
		\item
		Montrer que $A$ est stable par produit. On note alors $P_{1}=\{p
		\text{ premier}\mid p=2\text{ou}p\equiv 1[4]\}$ et
		$P_{2}=\{p\text{ premier}\mid p\equiv3[4]\}$.
		\item
		Soit $n\in\N^{*}$. On suppose que pour tout $p\in P_{2}$, $\nu_{p}(n)$ est
		pair (où $\nu_{p}(n)$ la puissance de $p$ dans la décomposition en produit
		de facteurs premiers de $n$). Montrer que $n\in A$.
		\item
		Montrer la réciproque (pour $n\in A$, pour $p\in P_{1}\cup P_{2})$ tel que
		$\nu_{p}(n)$ est impair, on montrera que $-1$ est un carré dans
		$\mathbb{F}_{p}$.
	\end{enumerate}
\end{exercise}

\cleardoublepage
\section{Séries numériques et familles sommables}

\begin{exercise}
	Soit la suite définie par $a_{0}=1$ et pour tout $n\geqslant1$,
	$$a_{n}=2a_{\lfloor n/3\rfloor}+3a_{\lfloor n/9\rfloor}$$
	\begin{enumerate}
		\item
		On pose pour $p\in\N$, $b_{p}=a_{3p}$. Calculer $b_{p}$ en fonction de
		$p$.
		\item
		Montrer que si $3^{p}\leqslant n<3^{p+1}$, alors $a_{n}=b_{p}$.
		\item
		Déterminer l'ensemble des valeurs d'adhérence de
		$(\frac{a_{n}}{n})_{n\geqslant 2}$.
	\end{enumerate}
\end{exercise}

\begin{exercise}
	Soit $[a,b]\subset\in\R$ avec $a<b$ et $f:[a,b]\to[a,b]$ continue. Soit
	$x_{0}\in[a,b]$ et pour tout $n\in\N$, $x_{n+1}=f(x_{n})$.
	\begin{enumerate}
		\item
		Montrer que $f$ admet au moins un point fixe $l\in[a,b]$.
		\item
		Si $\lim\limits_{n\to+\infty}x_{n+1}-x_{n}=0$, montrer que l'ensemble des
		valeurs d'adhérence de $(x_{n})_{n\in\N}$ est un segment.
		\item
		En déduire que $(x_{n})_{n\in\N}$ converge si et seulement si
		$\lim\limits_{n\to+\infty}x_{n+1}-x_{n}=0$.
	\end{enumerate}
\end{exercise}

\begin{exercise}
	Soit $\theta\in[0,2\pi[$, on définit $u_{0}=e^{\mathrm{i}\theta}$ et pour tout
	$n\in\N$, $u_{n+1}=u_{n}^{2}$. Peut-on avoir $(u_{n})_{n\in\N}$
	\begin{itemize}
		\item
		stationnaire?
		\item
		convergente?
		\item
		périodique?
		\item
		dense dans $\U$?
	\end{itemize}
	On pourra étudier le développement binaire de
	$\frac{\theta}{2\pi}=\sum_{k=1}^{+\infty}\frac{a_{k}}{2^{k}}$.
\end{exercise}

\begin{exercise}
	Soit $(a,b)\in\R_{+}^{2}$, étudier
	$u_{n}=\Bigl(\frac{\sqrt[n]{a}+\sqrt[n]{b}}{2}\Bigr)^{n^{2}}$.
\end{exercise}

\begin{exercise}
	Soit $(x_{n})_{n\in\N}\in\R_{+}^{\N}$ telle que $\lim\limits_{n\to+\infty}=0$
	et $\sum_{n=0}^{+\infty}x_{n}=+\infty$.
	\begin{enumerate}
		\item
		Montrer qu'il existe $\varphi:\N\to\N$ bijective telle que
		$(x_{\varphi(n)})$ est décroissante.
		\item
		Montrer que pour tout $l\in\overline{\R_{+}}$, pour tout $\varepsilon>0$,
		il existe un sous-ensemble $I\subset\N$ fini tel que
		$$\Bigl\vert\sum_{k\in I}x_{k}-l\Bigr\vert\leqslant\varepsilon$$ ou si
		$l=+\infty$: $\forall A>0$, il existe un sous-ensemble $I$ fini tel que
		$\sum_{k\in I}x_{k}\geqslant A$.
	\end{enumerate}
\end{exercise}

\begin{exercise}
	Soit $(u_{n})_{n\in\N}\in\R_{+}^{\N}$ telle que
	$\lim\limits_{n\to+\infty}u_{n}\times\sum_{k=0}^{n}u_{k}^{2}=1$. Montrer que
	$u_{n}\sim\frac{1}{\sqrt[3]{3n}}$. Une telle suite existe-t-elle?
\end{exercise}

\begin{exercise}
	Étudier $x_{n}=n-\sum_{k=1}^{n}\cosh(\frac{1}{\sqrt[]{k+n}})$.
\end{exercise}

\begin{exercise}
	Soit $(a_{n})_{n\in\N},(b_{n})_{n\in\N},(c_{n})_{n\in\N}$ des suites réelles
	telles que 
	\begin{enumerate}
		\item
		[(i)] $\lim\limits_{n\to+\infty}a_{n}+b_{n}+c_{n}=0$,
		\item [(ii)]
		$\lim\limits_{n\to+\infty}e^{a_{n}}+e^{b_{n}}+e^{c_{n}}=3$.
	\end{enumerate}
	Montrer que
	$\lim\limits_{n\to+\infty}a_{n}=\lim\limits_{n\to+\infty}b_{n}=\lim\limits_{n\to+\infty}c_{n}=0$.
	On pourra étudier $\varphi:x\mapsto e^{x}-x-1$.
\end{exercise}

\begin{exercise}
	Soit $u_{0}\in]0,1[$ et pour $n\in\N$, $u_{n+1}=u_{n}-u_{n}^{2}$. On pose
	$v_{n}=\frac{1}{u_{n}}$.
	\begin{enumerate}
		\item
		Montrer que $(v_{n})_{n\in\N}$ est bien définie.
		\item
		Montrer que $v_{n}=n+\ln(n)+O(1)$, en déduire un développement de $u_{n}$.
	\end{enumerate}
\end{exercise}

\begin{exercise}
	\phantom{}
	\begin{enumerate}
		\item
		Montrer que pour tout $n\geqslant 2$, il existe un unique $u_{n}\in\R_{+}$
		tel que $u_{n}^{n}=u_{n}+n$.
		\item
		Montrer que $(u_{n})_{n\geqslant 2}$ converge vers $\lambda\in\R_{+}$.
		\item
		Donner un développement asymptotique à deux termes de $x_{n}-\lambda$.
	\end{enumerate}
\end{exercise}

\begin{exercise}
	Soit $(u_{n})_{n\in\N}$ une suite de réels positifs non tous nuls. On suppose
	que
	
	$$u_n=o\Biggl(\sum_{k=0}^{n}u_{k}\Biggr)$$ Soit $(a_{n})_{n\in\N}\in\C^{\N}$
	de limite $a$. En cas d'existence, évaluer
	$$\lim\limits_{n\to+\infty}\frac{u_{n}a_{0}+u_{n-1}a_{1}+\dots+u_{0}a_{n}}{u_{0}+\dots+u_{n}}$$
\end{exercise}

\begin{exercise}
	\phantom{}
	\begin{enumerate}
		\item
		Soit $x\in[0,1[$, montrer qu'il existe une unique suite
		$(a_{n})_{n\geqslant 2}$ d'entiers naturels telle que 
		\begin{enumerate}
			\item
			[(i)] $0\leqslant a_{n}\leqslant n-1$ pour tout $n\geqslant2$,
			\item
			[(ii)] il existe $m\geqslant n$ tel que $a_{m}<m-1$ pour tout
			$n\geqslant2$,
			\item
			[(iii)] $x=\sum_{n=2}^{+\infty}\frac{a_{n}}{n}$.
		\end{enumerate}
		\item
		Donner une condition nécessaire et suffisante sur $(a_{n})_{n\geqslant2}$
		pour que $x\in\Q$.
		\item
		Soit $l\in[-1,1]$, montrer qu'il existe $x\in[0,1[$ tel que
		$\lim\limits_{n\to+\infty}sin(n!2\pi x)=l$.
	\end{enumerate}
\end{exercise}

\begin{exercise}
	Soit $u_0>0,u_1>0$ et pour tout $n\geqslant 1$,
	$$u_{n+1}=\ln(1+u_{n})+\ln(1+u_{n-1})$$ Étudier la suite $(u_{n})$. On pourra
	poser $M_{n}=\max(u_{n},u_{n-1},l)$, $m_{n}=\min(u_{n},u_{n-1},l)$ où
	$l=2\ln(1+l)$ et $l>0$.
\end{exercise}

\begin{exercise}
	Soit $(p,q)\in(\R^{*})^{2}$ avec $p/q\in\R\setminus\Q$. Soit
	$(x_{n})_{n\in\N}$ une suite réelle bornée. On suppose que
	$(e^{\mathrm{i}px_{n}})_{n\in\N}$ et $(e^{\mathrm{i}qx_{n}})_{n\in\N}$
	convergent. Montrer que $(x_{n})_{n\in\N}$ converge. Et si $(x_{n})_{n\in\N}$
	n'est pas bornée ?
\end{exercise}

\begin{exercise}
	\phantom{}
	\begin{enumerate}
		\item
		Montrer que pour tout $n\geqslant1$, pour tout $k\in\{0,\dots,n\}$,
		$\binom{n}{k}\leqslant\frac{n^{k}}{k!}$.
		\item
		Soit $z\in\C$, montrer que 
		$$\Biggl\vert\sum_{k=0}^{n}\frac{z^{k}}{k!}-\Bigl(1+\frac{z}{n}^{n}\Bigr)\Biggr\vert\leqslant\sum_{k=0}^{n}\frac{\vert
		z\vert^{k}}{k!}-\Bigl(1+\frac{\vert z\vert}{n}\Bigr)^{n}$$
		\item
		En déduire $\lim\limits_{n\to+\infty}\Bigl(1+\frac{z}{n}\Bigr)^{n}$.
	\end{enumerate}
\end{exercise}

\begin{exercise}
	Soit $u_{n}=\prod_{k=2}^{n}\frac{\sqrt{k}-1}{\sqrt{k}+1}$ pour $n\geqslant 2$.
	Quelle est la limite de cette suite? Quelle est la nature de la série
	$\sum_{n\geqslant 2}u_{n}^{\alpha}$ pour $\alpha\in\R$?
\end{exercise}

\begin{exercise}
	Soit $(u_{n})_{n\in\N}\in\R_{+}^{\N}$ décroissante de limite nulle. Montrer
	que si $\sum u_{n}$ converge, alors $u_{n}=o\Bigl(\frac{1}{n}\Bigr)$. On
	pourra minorer $u_{n+1}+\dots+u_{2n}$. Montrer ensuite que si $\{p\in\N,
	pu_{p}\geqslant1\}$ est infini, alors $\sum u_{n}$ diverge.
\end{exercise}

\begin{exercise}
	Nature de $\sum u_{n}$ où $u_{n}=$
	\begin{enumerate}
		\item
		$n^{-1-\frac{1}{n}}$
		\item
		$\int_{0}^{\frac{\pi}{2}}t^{n}\sin(t)dt$
		\item
		$\sin(2\pi\frac{n!}{e})$
		\item
		$\frac{(-1)^{n}}{n^{\alpha}+(-1)^{n}\ln(n)}$ où $\alpha\in\R$
	\end{enumerate}
\end{exercise}

\begin{exercise}
	Montrer la convergence et calculer la somme des différentes séries suivantes:
	\begin{enumerate}
		\item
		$\sum_{n\geqslant1}\sum_{k\geqslant n}\frac{(-1)^{k}}{k}$
		\item
		$\sum_{n\geqslant0}\frac{1}{(3n)!}$
		\item
		$\sum_{n\geqslant1}\frac{E(n^{\frac{1}{3}})-E(n-1)^{\frac{1}{3}}}{4n-n^{\frac{1}{3}}}$
		où $E$ désigne la partie entière.
	\end{enumerate}
\end{exercise}

\begin{exercise}
	Soit $f:[1,+\infty[\to\R_{+}^{*}$ de classe $\mathcal{C}^{2}$ et telle que
	$\lim\limits_{x\to+\infty}\frac{f'(x)}{f(x)}=a<0$. Montrer la convergence de
	$\sum_{n\geqslant1}f(n)$. Donner un équivalent de $R_{n}=\sum_{k=n}^{+\infty}
	f(k)$.
\end{exercise}

\begin{exercise}
	Donner un équivalent de $S_{n}=\sum_{k=1}^{n}\frac{e^{k}}{k}$.
\end{exercise}

\begin{exercise}
	Donner la nature de $\sum u_{n}$ quand $u_{n}$ vaut
	\begin{enumerate}
		\item
		$\Bigl(1-\frac{1}{n}\Bigr)^{n^{\alpha}}$ où $\alpha\in\R$
		\item
		$\frac{1}{\sum_{k=1}^{n}\bigl(\frac{1}{k}\bigr)^{\frac{1}{k}}}$
		\item
		$\frac{\sin(n!\pi e)}{\ln(n)}$
	\end{enumerate}
\end{exercise}

\begin{exercise}
	Montrer la convergence et calculer la somme de $\sum u_{n}$ où $u_{n}$ vaut
	\begin{enumerate}
		\item
		$a\ln(n)+b\ln(n+1)+c\ln(n+2)$ pour $n\geqslant1$ (on cherchera d'abord une
		condition nécessaire et suffisante de convergence).
		\item
		$\frac{2^{n}}{3^{2^{n-1}}+1}$ pour $n\geqslant 1$.
		\item
		$\frac{k-n\lfloor\frac{k}{n}\rfloor}{k(k+1)}$.
		\item
		$\arctan(\frac{1}{n^{2}+n+1})$ pour $n\geqslant0$.
	\end{enumerate}
\end{exercise}

\begin{exercise}
	Soit $(u_{n})_{n\geqslant1}\in\R^{\N}$ et $v_{n}=n(u_{n}-u_{n+1})$. Montrer
	que $\sum u_{n}$ et $\sum v_{n}$ ont même nature lorsque
	\begin{enumerate}
		\item
		[(i)] $(nu_{n})_{n\geqslant1}$ converge vers 0 OU
		\item
		[(ii)] $(u_{n})_{n\geqslant 1}$ décroît et tend vers 0.
	\end{enumerate}
	Comparer alors les sommes respectives. En déduire, pour $p\geqslant1$ fixé,
	$$\sum_{n=1}^{+\infty}\frac{1}{n(n+1)\dots(n+p)}$$
\end{exercise}

\begin{exercise}
	Soit $q\in\Z$ et $v_{n}=\frac{1}{(n+q)!}\sum_{k=1}^{n}k!$. Donner la nature de
	$\sum v_{n}$. En cas de divergence, donner un équivalent des sommes
	partielles.
\end{exercise}

\begin{exercise}
	Soit $(a,b,c)\in (\N^{*})^{3}$, $z\in\C$, $\vert z\vert<1$. Montrer, en
	justifiant l'existence:
	$$\sum_{n=0}^{+\infty}\frac{z^{nb}}{1+z^{na+c}}=\sum_{n=0}^{+\infty}\frac{(-1)^{n}z^{nc}}{1-z^{na+b}}$$
\end{exercise}

\begin{exercise}
	Soit $\sum_{n\geqslant1} a_{n}$ une série complexe absolument convergente. On
	pose pour $q\in\N^{*}$, $b_q=\frac{1}{q(q+1)}(a_{1}+2a_{2}+\dots+qa_{q})$.
	Montrer que $\sum_{q\geqslant1}b_{q}$ converge et évaluer sa somme en fonction
	de $\sum_{n=1}^{+\infty}a_{n}$. On pourra poser
	$u_{n,q}=\frac{na_{n}}{q(q+1)}$ si $n\leqslant q$ et 0 sinon.
\end{exercise}

\begin{exercise}
	Soit $(u_{n})_{n\geqslant1}\in\R_{+}^{\N}$ telle que $\sum u_{n}<+\infty$. On
	pose $v_{n}=\frac{1}{n(n+1)}(u_{1}+\dots+nu_{n})$ et $w_{n}=\sqrt[n]{u_1\times
	u_2\times\dots\times u_n}$. On admet que pour tout $n\in\N^{*}$, pour tout
	$(a_{1},\dots,a_{n})\in\R_{+}^{n}$, on a l'inégalité entre la moyenne
	géométrique et arithmétique:
	$$\sqrt[n]{a_{1}\dots a_{n}}\leqslant\frac{1}{n}(a_{1}+\dots+a_{n})$$ avec
	égalité si et seulement si $a_{1}=\dots=a_{n}$.

	Montrer que $\sum w_{n}$ converge et que $\sum_{n=1}^{+\infty}w_{n}\leqslant
	e\sum_{n=1}^{+\infty}u_{n}$. On pourra utiliser l'exercice précédent. Montrer
	que $e$ est la "meilleure" constante possible, c'est-à-dire que si $\forall
	(u_{n})_{n\geqslant1}\in(\R_{+}^{*})^{\N^{*}}$ telle que $\sum u_{n}$
	converge, on a $\sum w_{n}\leqslant C\sum u_{n}$ alors $C\geqslant e$.
\end{exercise}

\begin{exercise}
	\phantom{}
	\begin{enumerate}
		\item
		Trouver une condition nécessaire et suffisante sur $\alpha\in\R$ pour que
		$\Bigl(\frac{1}{(p+q)^{\alpha}}\Bigr)_{(p,q)\in\N^{2}\setminus\{(0,0)\}}$
		soit sommable et exprimer alors la somme en fonction de la fonction
		$\zeta$ de Riemann.
		\item
		Trouver une condition nécessaire et suffisante sur $\alpha\in\R$ pour que
		$\Bigl(\frac{1}{(p^{2}+q^{2})^{\alpha}}\Bigr)_{(p,q)\in\N^{2}\setminus\{(0,0)\}}$
		soit sommable.
	\end{enumerate}
\end{exercise}

\begin{exercise}
	Étudier la sommabilité de
	$\Bigl(\frac{1}{(m+n^{2})(m+n^{2}+1)}\Bigr)_{(m,n)\in\N^{2}}$.\\
	En déduire la valeur de $\sum_{n=1}^{+\infty}\frac{E(\sqrt{n})}{n(n+1)}$.
\end{exercise}

\begin{exercise}
	\phantom{}
	\begin{enumerate}
		\item
		Montrer que pour tout $s\in]1,+\infty[$, le produit infini
		$\prod_{k=1}^{+\infty}\frac{1}{1-\frac{1}{p_{k}^{s}}}$ converge (où les
		$p_{k}$ sont les nombres premiers). Donner sa valeur en fonction de
		$\zeta(s)$.
		\item
		Généraliser ce résultat à $s\in\C$ avec $\Re(s)>1$.
	\end{enumerate}
\end{exercise}

\begin{exercise}
	On note $\varphi(n)=\vert\{k\in\{1,\dots,n\},~k\wedge n=1\}\vert$ (fonction
	d'Euler). Pour quelles valeurs de $\alpha\in\R$ la somme $\sum
	\frac{\varphi(n)}{n^{\alpha}}$ converge-t-elle? Donner alors sa somme en
	fonction de $\zeta(\alpha)$.
\end{exercise}

\begin{exercise}
	Soit $(z_{n})_{n\in\N}\in(\C^{*})^{\N}$ telle que pour tout $n\neq m$, $\vert
	z_{n}-z_{m}\vert\geqslant1$. Montrer que $\sum_{n\in\N}\frac{1}{z_{n}^{3}}$
	converge.
\end{exercise}

\begin{exercise}
	Donner la nature de $\sum_{n\geqslant1}\frac{(-1)^{E(\sqrt{n})}}{n}$.
\end{exercise}

\begin{exercise}
	Pour $(a,b)\in(\R\setminus\Z^{*})$, on définit
	$u_{n}=\frac{a(a+1)\dots(a+n)}{b(b+1)\dots(b+n)}$....
	\begin{enumerate}
		\item
		Donner une condition nécessaire et suffisante pour que $\sum u_{n}$
		converge.
		\item
		Dans ce cas, calculer sa somme.
		\item
		Faire le cas où $a=-\frac{1}{2}$ et $b=1$.
	\end{enumerate}
\end{exercise}

\begin{exercise}
	Soit $u_{n}=\frac{\ln(n)}{n}$ et $v_{n}=(-1)^{n}u_{n}$ pour $n\geqslant1$.
	\begin{enumerate}
		\item
		Donner la nature de $\sum u_{n}$ et $\sum v_{n}$.
		\item
		Soit $S_{N}=\sum_{n=1}^{N}u_{n}$. Donner un équivalent de $S_{N}$ puis
		développer jusqu'au o(1).
		\item
		Exprimer $\sum_{n=2}^{+\infty}v_{n}$ en fonction de $\gamma$ (constante
		d'Euler) et $\ln(2)$.
	\end{enumerate}
\end{exercise}

\begin{exercise}
	Soit pour $n\in\N^{*}$, $q_{1}(n)$ la nombre de chiffres de l'écriture
	décimale de $n$. On définit par récurrence $q_{k+1}(n)=q_{1}(q_{k}(n))$.
	Étudier la convergence de 
	$$\sum_{n\geqslant1}\frac{1}{nq_{1}(n)q_{2}(n)\dots q_{n}(n)}$$
\end{exercise}

\begin{exercise}
	Soit $P_{n}(X)=\sum_{k=0}^{n}\frac{X^{k}}{k!}$.
	\begin{enumerate}
		\item
		Montrer que pour tout $n\in\N$, $P_{2n}>0$ sur $\R$ et $P_{2n+1}$ s'annule
		une seule fois en $a_{2n+1}<0$.
		\item
		Déterminer $\lim\limits_{n\to+\infty}a_{2n+1}$.
	\end{enumerate}
\end{exercise}

\begin{exercise}
	Montrer qu'il existe un unique $x_{n}\geqslant0$ tel que $e^{x_{n}}=x_{n}+n$.
	Donne un développement asymptotique à deux termes de $x_{n}$ pour
	$n\geqslant1$.
\end{exercise}

\begin{exercise}
	Soit $(u_{n})_{n\in\N}\in(\R_{+}^{*})^{\N}$, on pose
	$S_{n}=\sum_{k=0}^{n}u_{k}$. Soit $\alpha\in\R$ et
	$v_{n}=\frac{u_{n}}{S_{n}^{\alpha}}$.
	\begin{enumerate}
		\item
		On suppose que $\sum u_{n}$ converge, étudier $\sum v_{n}$.
		\item
		On suppose que $\sum u_{n}$ diverge. Pour $\alpha=1$, montrer que pour
		tout $(n,p)\in\N^{2}$, $v_{n+1}+\dots+v_{n+p}\geqslant
		1-\frac{S_{n}}{S_{n+p}}$. En déduire que $\sum v_{n}$ diverge.
		\item
		On suppose que $\sum u_{n}$ diverge. Pour $\alpha>1$, on forme
		$w_{n}=\int_{S_{n-1}}^{S_{n}}\frac{dt}{t^{\alpha}}$. Montrer que $\sum
		v_{n}$ converge. Et si $\alpha<1?$
		\item
		On suppose que $\sum u_{n}$ converge. On pose
		$R_{n}=\sum_{k=n}^{+\infty}u_{k}$ et $w_{n}=\frac{u_{n}}{R_{n}^{\alpha}}$.
		Étudier la nature de $\sum w_{n}$.
	\end{enumerate}
\end{exercise}

\begin{exercise}[Principe des tiroirs de Dirichlet]
	Soit $x\in\R\setminus\Q$.
	\begin{enumerate}
		\item
		Soit $n\in\N^{*}$, montrer qu'il existe $(p,q)\in\Z\times\{1,\dots,n\}$
		tel que $\bigl\vert x-\frac{p}{q}\bigr\vert<\frac{1}{qn}$. On pourra
		étudier les $n+1$ réels $(kx-\lfloor kx\rfloor)=(x_{k})_{0\leqslant
		k\leqslant n}$ et montrer qu'il existe $k\neq k'$ avec $\vert
		x_{k}-x_{k'}\vert<\frac{1}{n}$.
		\item
		Montrer qu'il existe $(p_{n},q_{n})_{n\in\N}\in\Z^{\N}\times(\N^{*})^{\N}$
		telles que $\bigl\vert
		x-\frac{p_{n}}{q_{n}}\bigr\vert<\frac{1}{q_{n}^{2}}$ et
		$\lim\limits_{n\to+\infty}q_{n}=+\infty$.
		\item
		Étudier la convergence de la suite
		$\Bigl(\frac{1}{n\sin(n)}\Bigr)_{n\geqslant1}$ (on admet que
		$\pi\notin\R\setminus\Q)$.
	\end{enumerate}
\end{exercise}

\begin{exercise}
	Soit $(a_{n,p})\in\C^{(\N^{*})^{2}}$ telle que 
	\begin{enumerate}
		\item
		[(i)] pour tout $p\in\N^{*}$, il existe
		$\lim\limits_{n\to+\infty}a_{n,p}=a_{p}\in\C$,
		\item
		[(ii)] il existe une suite de réels positifs $(b_{p})$ donc la série
		converge telle que pour tout $(n,p)\in(\N^{*})^{2}$, $\lvert
		a_{n,p}\rvert\leqslant b_{p}$.
		
	\end{enumerate}
	\begin{enumerate}
		\item
		Évaluer $\lim\limits_{n\to+\infty}\sum_{p=1}^{n}a_{n,p}$.
		\item
		Calculer
		$\lim\limits_{n\to+\infty}\Bigl(\bigl(\frac{1}{n}\bigr)^{n}+\bigl(\frac{2}{n}\bigr)^{n}+\dots+\bigl(\frac{n-1}{n}\bigr)^{n}\Bigr)$.
	\end{enumerate}
\end{exercise}

\begin{exercise}
	Soit $\sum_{n\geqslant1}u_{n}$ une série complexe absolument convergente.
	\begin{enumerate}
		\item
		Montrer que pour tout $k\geqslant1$, on peut définir
		$S_{k}=\sum_{n=1}^{+\infty}u_{kn}$.
		\item
		On suppose que pour tout $k\geqslant1$, $S_{k}=0$. Montrer que pour tout
		$n\geqslant1$, $u_{n}=0$.
	\end{enumerate}
\end{exercise}

\begin{exercise}
	Soit $f:\R\to\R$ telle que pour toute suite $(u_{n})_{n\in\N}\in\R^{\N}$, si
$\sum u_{n}$ converge, alors $\sum f(u_{n})$ converge.
\begin{enumerate}
	\item
	Montrer que $f(0)=0$ et que $f$ est continue en 0.
	\item
	Montrer qu'il existe $\alpha>0$, $\forall x\in]-\alpha,\alpha[$, $f(x)=-f(x)$
	($f$ est impaire au voisinage de 0).
	\item
	Montrer qu'il existe $\beta>0$ $\forall(x,y)\in]-\beta,\beta[^{2}$,
	$f(x+y)=f(x)+f(y)$ ($f$ est linéaire au voisinage de 0).
	\item
	Montrer qu'il existe $\lambda\in\R$ et $\gamma>0$ tels que $\forall
	x\in]-\gamma,\gamma[$, $f(x)=\lambda x$ ($f$ est une homothétie au voisinage
	de 0).
\end{enumerate}
\end{exercise}

\cleardoublepage
\section{Probabilités sur un univers dénombrable}

\begin{exercise}
	On lance une seule fois une pièce équilibrée, puis on effectue des tirages
	avec remise dans une urne contenant initialement 1 boule noire et 1 boule
	blanche: si la pièce a donné pile (respectivement face), on rajoute à chaque
	fois une boule blanche (respectivement noire).
	\begin{enumerate}
		\item
		Quelle est la probabilité de tirer une boule blanche au $k$-ième tirage?
		\item
		Sachant que l'on a tiré une boule blanche au $k$-ième tirage, quelle est
		la probabilité $p_{k}$ d'avoir obtenu pile au lancer initial de la pièce ?
		\item
		Quelle est la probabilité d'obtenir $k$ boules blanches au cours des $k$
		premiers tirages?
		\item
		On note $B_{k}$ l'évènement où la $k$-ième boule est blanche. $B_{k}$ et
		$B_{k+1}$ sont-ils indépendants?
	\end{enumerate}
\end{exercise}

\begin{exercise}
	$A$ et $B$ s'affrontent dans une partie de pile ou face:
	$\P(P)=p,~\P(F)=1-p=q$ avec $p\in]0,1[$. Au départ, ils possèdent un total de
	$N$ euros. Après chaque lancer, le perdant donne un euro au gagnant. $A$ gagne
	si pile, perd si face. Le jeu s'arrête lorsqu'un des joueurs est ruiné (ou
	s'ils avaient 0 au départ).

	On note $p_{a}$ (respectivement $q_{a}$) la probabilité que $A$
	(respectivement $B$) soit ruiné (en temps fini) si $A$ a $a$ euros au départ.
	\begin{enumerate}
		\item
		Évaluer $p_{0},p_{N},q_{0},q_{N}$.
		\item
		Montrer que pour tout $a\in\{1,\dots,N-1\}$, $p_{a}=pp_{a+1}+qp_{a-1}$. En
		déduire l'expression de $p_{a}$.
		\item
		Calculer de même $q_{a}$, puis $p_{a}+q_{a}$. Qu'en déduit-on?
	\end{enumerate}
\end{exercise}

\begin{exercise}
	Deux archers tirent alternativement sur une cible, jusqu'à ce que l'un des
	deux la touche. $A$ commence. Il touche la cible avec une probabilité
	$a\in]0,1[$. $B$ touche la cible avec une probabilité $b\in]0,1[$. On note
	$G_{A}$ (respectivement $G_{B}$) l'évènement où $A$ (respectivement $B$)
	l'emporte.
	\begin{enumerate}
		\item
		Soit $n\in\N$, quelle est la probabilité pour que $A$ (respectivement $B$)
		l'emporte au rang $2n+1$ (respectivement $2n+2$). On note $A_{n}$
		(respectivement $B_{n}$) l'évènement correspondant.
		\item
		En déduire $\P(G_{A})$ et $\P(G_{B})$. Que vaut $\P(G_{A})+\P(G_{B})$?
		\item A quelle condition a-t-on $\P(G_{A})=\P(G_{B})$?
	\end{enumerate}
\end{exercise}

\begin{exercise}
	Un joueur lance une pièce équilibrée jusqu'à l'obtention du premier pile. S'il
	lui a fallu $n$ lancers pour l'obtenir, on lui fait tirer un billet de loterie
	parmi $n$ (un seul billet gagnant).
	\begin{enumerate}
		\item
		Quelle est la probabilité pour que le joueur gagne ?
		\item
		Sachant qu'il a gagné, quelle est la probabilité qu'il ait obtenu pile au
		$n$-ième lancer? Et qu'il ait obtenu pile ?
	\end{enumerate}
\end{exercise}

\begin{exercise}
	Dans une famille donnée, la probabilité pour qu'il y ait $k$ enfants est
	$p_{k}$ ($k\in\N$) avec $p_{0}=p_{1}=\alpha\in]0,\frac{1}{2}[$ et pour tout
	$k\geqslant2$, $p_{k}=\frac{1-2\alpha}{2^{k-1}}$. La probabilité qu'il y ait
	un garçon ou une fille est la même.
	\begin{enumerate}
		\item
		Vérifier que c'est une probabilité sur $\N$.
		\item
		Quelle est la probabilité qu'une famille ait exactement deux garçons?
		\item
		Quelle est la probabilité qu'une famille au au moins deux filles sachant
		qu'elle a au moins deux garçons?
	\end{enumerate}
\end{exercise}

\begin{exercise}
	Deux joueurs jouent avec deux dés non pipés. $A$ (respectivement) gagne s'il
	obtient un total de 6 (respectivement 7). $A$ commence. On s'arrête lorsqu'un
	des deux joueurs gagne. Quelle est la probabilité de succès des deux joueurs ?
\end{exercise}

\begin{exercise}
	Une urne contient $a$ boules blanches et $b$ noires. On en tire successivement
	$n$ au hasard avec remise. Quelle est la probabilité pour que le nombre de
	boules blanches tirées soit pair ?
\end{exercise}

\begin{exercise}
	Soit $n\in\N^{*}$, calculer la probabilité $p_{n}$ pour qu'une bijection de
	$\{1,\dots,n\}$ possède au moins un point fixe. Donner la limite de $p_{n}$
	quand $n\to+\infty$.
\end{exercise}

\begin{exercise}
	Soit $N\in\N^{*}$, $p\in]0,1[$ et $q=1-p$. On joue à pile ou face avec une
	probabilité $p$ d'obtenir pile. On gagne 1 euro si on tombe sur face, on perd
	un euro sinon. Le jeu s'arrête lorsque l'on a 0 ou $N$ euros. Pour
	$n\in\{0,\dots,N\}$, on note $p_{N}(n)$ la probabilité de gagner
	(respectivement de perdre) si on dispose au départ de $n$ euros.
	\begin{enumerate}
		\item
		Calculer $p_{N}(0)$ et $p_{N}(N)$.
		\item
		Calculer $p_{N}(n)$, $q_{N}(n)$ et $\lim\limits_{N\to+\infty}p_{N}(n)$ à
		$n$ fixé.
	\end{enumerate}
\end{exercise}

\begin{exercise}
	Soit $c\in\N^{*}$. Soit une urne contenant initialement une boule blanche et
	une noire. Après tirage, la boule tirée est remise avec $c$ autres boules de
	sa couleur. Soit, pour $n\geqslant1$, $p_{n}$ la probabilité pour que la
	première boule blanche apparaisse au $n$-ième tirage. Calculer $p_{n}$ et
	$\sum_{n\geqslant1}p_{n}$ lorsque 
	\begin{enumerate}
		\item
		$c=1$
		\item
		$c$ quelconque.
	\end{enumerate}
\end{exercise}

\begin{exercise}
	Soit $p\in]0,1[$ et $q=1-p$. Une bactérie vit un seul jour. \`A l'issue de
	cette journée, elle peut se diviser en deux avec la probabilité $p$ ou bien
	disparaître tristement sans laisser de traces avec la probabilité $q$. Pour
	$n\in\N$, on désigne par $U_{n}$ l'évènement indiquant que la lignée d'une
	bactérie donnée est éteinte au $n$-ième jour. On note $u_{n}=\P(U_{n})$.
	Montrer que $\lim\limits_{n\to+\infty}u_{n}=\min(1,\frac{p}{q})$. Interpréter.
	Dans le cas $p=\frac{1}{2}$, chercher un développement asymptotique en
	$o\bigl(\frac{1}{n}\bigr)$ de $u_{n}$.
\end{exercise}

\begin{exercise}
	Une puce se déplace sur une droite. Elle part de 0 et fait des sauts
	successifs de longueur 1. A chaque saut, elle avec avec la probabilité
	$p\in]0,1[$ et recule avec la probabilité $q=1-p$. Quelle est la probabilité
	pour que la puce repasse en 0 ? Montrer que si $p\neq\frac{1}{2}$, le nombre
	de retours à l'origine est presque sûrement fini.
\end{exercise}

\begin{exercise}
	Soit un lancer infini d'une pièce donnant pile avec la probabilité $p\in]0,1[$
	et face avec la probabilité $q=1-p$. On désigne, pour $n\in\N^{*}$, $A_{n}$
	l'évènement qu'après le $n$-ième lancer, on a obtenu pour la première fois
	deux piles consécutifs (donc au $n-1$-ième et au $n$-ième). On note
	$a_{n}=\P(A_{n})$.
	
	\begin{enumerate}
		\item
		Calculer $a_{1},a_{2},a_{3}$.
		\item
		Calculer $a_{n}$ et $\sum_{n=1}^{+\infty}a_{n}$. Interpréter.
	\end{enumerate}
\end{exercise}

\begin{exercise}
	On dispose de $N+1$ urnes numérotées de $0$ à $N$. Pour tout
	$k\in\{0,\dots,N\}$, la $k$-ième urne possède $k$ boules blanches et $N-k$
	boules noires. On choisit une urne au hasard et on tire $n$ fois une boule
	avec remise.
	\begin{enumerate}
		\item
		Quelle est la probabilité qu'au $n+1$-ième tirage, on ait obtenu une
		blanche sachant qu'au cours des $n$ premiers lancers on a obtenu des
		blanches? On note cette probabilité $\P_{N}(n)$.
		\item
		Que vaut $\lim\limits_{N\to+\infty}\P_{N}(n)$ à $n$ fixé ?
	\end{enumerate}
\end{exercise}

\begin{exercise}
	Quelle est la probabilité pour que deux entiers naturels soient premiers entre
	eux ?
\end{exercise}

\begin{exercise}
	$A$ écrit à $B$ avec une probabilité $p_{1}$ s'il lui a écrit la veille,
	$p_{2}$ sinon (avec $(p_{1},p_{2})\in[0,1]^{2}$). Soit, pour $n\in\N^{*}$,
	$X_{n}$ qui vaut 1 si $A$ a écrit à $B$ le jour $n$ et 0 sinon. Déterminer la
	loi et l'espérance (sous réserve d'existence) de $X_{n}$.
\end{exercise}

\begin{exercise}
	On lance deux dés non pipés. On note $D_{i}$ le résultat du $i$-ième dé,
	$X=\max(D_{1},D_{2})$ et $Y=\min(D_{1},D_{2})$.
	\begin{enumerate}
		\item
		Déterminer les lois de $X$ et $Y$ ainsi que leurs espérances et variances.
		\item
		$X$ et $Y$ sont-elles indépendantes?
		\item
		Et si $\P(D_{i}=k)=p_{k,i}$ avec $\sum_{k=1}^{6}p_{k,i}=1$?
	\end{enumerate}
\end{exercise}

\begin{exercise}
	Soit $(a,b)\in]0,1[\times\R_{+}^{*}$, $X$ et $Y$ deux variables aléatoires
	discrètes à valeurs dans $\N$ dont la loi conjointe est
	$$
	p_{i,j}=
	\left\{
		\begin{array}{cc}
			0 & \text{si }i<j\\
			\frac{b^{i}e^{-b}a^{j}(1-a)^{i-j}}{j!(i-j)!} & \text{si }i\geqslant j
		\end{array}
	\right.
	$$
	\begin{enumerate}
		\item
		Vérifier que cette définition est cohérente.
		\item
		Déterminer les lois marginales. $X$ et $Y$ sont-elles indépendantes?
		\item
		Déterminer la loi de $Z=X-Y$. $Y$ et $Z$ sont-elles indépendantes?
	\end{enumerate}
\end{exercise}

\begin{exercise}
	Soit $x\in]0,1[$. Soit une succession d'épreuves de Bernoulli, indépendantes,
	de probabilité d'échec $x$. On définit deux suites de variables aléatoires:
	pour tout $n\in\N^{*}$, $S_{n}$ est la nombre d'épreuves nécessaires pour
	obtenir le $n$-ième succès, et $T_{1}=S_{1}$ et pour $n\geqslant2$, $T_{n}$
	est le nombre d'épreuves séparant le $n$-ième succès du $n-1$-ième.
	\begin{enumerate}
		\item
		Exprimer, pour tout $n\geqslant1$, $S_{n}$ en fonction des
		$(T_{i})_{i\geqslant1}$.
		\item
		Déterminer la loi, l'espérance et la variance de $T_{n}$.
		\item
		Déterminer la loi, l'espérance et la variance de $S_{n}$.
		\item
		Montrer que pour tout $x\in]0,1[$, pour tout $n\in\N^{*}$,
		$$\sum_{k=n}^{+\infty}\binom{k-1}{n-1}x^{k}=\frac{x^{n}}{(1-x)^{n}}$$
	\end{enumerate}
\end{exercise}

\begin{exercise}
Soit $X$ une variable aléatoire à valeur dans $\N$. Montrer que $X$ possède une
espérance si et seulement si $\sum_{n\in\N}\P(X>n)$ converge, et qu'on a alors
$$\E(X)=\sum_{n=0}^{+\infty}\P(X>n)$$
\end{exercise}

\begin{exercise}
	Soit $X\sim\mathcal{P}(\lambda)$ (loi de Poisson) avec $\lambda>0$. Quelle est
	la valeur que prend $X$ avec la plus grande probabilité? Évaluer la limite de
	cette valeur quand $\lambda\to+\infty$.
\end{exercise}

\begin{exercise}
	Un veilleur de nuit doit ouvrir une porte, ils possède un trousseau de 10 clés
	indiscernables dont une seule ouvre la porte. S'il est sobre, après chaque
	échec, il met de côté la mauvaise clé et poursuit avec les autres. S'il est
	ivre, il remet la clé dans le trousseau après chaque échec. Soit $X$
	(respectivement $Y$) la nombre d'essais au bout desquels il ouvre la porte
	s'il est sobre (respectivement ivre).
	\begin{enumerate}
		\item
		Donner les lois de $X$ et $Y$, leurs espérances et variances (si
		définies).
		\item
		Le gardien est ivre un jour sur 3. Sachant qu'un jour il a essayé au moins
		9 clés, quelle est la probabilité qu'il ait été sobre ce jour-là?
	\end{enumerate}
\end{exercise}

\begin{exercise}
	Soit $(n,N)\in(\N^{*})^{2}$. Une urne contient $N$ jetons à deux faces. L'une
	porte un numéro bleu, l'autre un rouge. Pour tout $1\leqslant j\leqslant
	i\leqslant n$, un seul jeton porte $i$ bleu et $j$ rouge. On tire au hasard un
	jeton. On note $B$ (respectivement $R$) le numéro bleu (respectivement rouge)
	tiré et $G=B-R$.
	\begin{enumerate}
		\item
		Déterminer $N$ en fonction de $n$.
		\item
		Donner les lois conjoints et marginales de $B$ et $R$.
		\item
		Déterminer les espérances et variances de $B$, $R$ et $G$.
	\end{enumerate}
\end{exercise}

\begin{exercise}
	Un mobile se déplace sur les points à coordonnées entières (naturelles) d'un
	axe d'origine 0. Au départ, le mobile est en 0. S'il est au point d'abscisse
	$k$ à l'instant $n$, à l'instant $n+1$ il est au point $k+1$ avec une
	probabilité $\frac{k+1}{k+2}$ et 0 avec la probabilité $\frac{1}{k+2}$. On
	note $X_{n}$ l'abscisse du mobile à l'instant $n$ et $u_{n}=\P(X_{n}=0)$.
	\begin{enumerate}
		\item
		Montrer que pour tout $n\in\N$, pour tout $k\in\{1,\dots,n+1\}$,
		$\P(X_{n+1}=k)=\frac{k}{k+1}\P(X_{n}=k-1)$.
		\item
		En déduire que pour tout $n\in\N$, pour tout $k\in\{0,\dots,n\}$,
		$\P(x_{n}=k)=\frac{1}{k+1}u_{n-k}$.
		\item
		Montrer que pour tout $n\in\N$, $\sum_{j=0}^{n}\frac{u_{n}}{n-j+1}=1$.
		Calculer $u_{0},u_{1},u_{2},u_{3}$.
		\item
		Prouver que pour tout $n\in\N$, $\E(X_{n+1})=\E(X_{n})+u_{n+1}$. En
		déduire $\E(X_{n})$ en fonction des $(u_{n})_{n\in\N}$.
		\item
		On note $T$ l'instant auquel le mobile revient pour la première fois à
		l'origine ($T=0$ s'il ne repasse pas par l'origine). Montrer que pour tout
		$n\in\N^{*}$, $\P(T=n)=\frac{1}{n(n+1)}$. En déduire $\P(T=0)$.
		\item
		$T$ admet-elle une espérance?
	\end{enumerate}
\end{exercise}

\begin{exercise}
	Le nombre $N$ de clients arrivant dans un magasin au cours d'une journée suit
	une loi de Poisson de paramètre $\lambda$. Ces clients se répartissent de
	manière équiprobable entre les $m$ caisses du magasin ($m\in\N^{*}$). Soit
	$X_{1}$ le nombre de clients qui arrivent à la caisse 1 au cours d'une
	journée. 
	\begin{enumerate}
		\item
		Soit $n\in\N^{*}$n déterminer la loi conditionnelle de $X_{1}$ sachant
		$N=n$.
		\item
		En déduire la loi de $X_{1}$.
	\end{enumerate}
\end{exercise}

\begin{exercise}
	Soit $X,Y$ deux variables aléatoires indépendantes discrètes suivant la même
	loi de Bernoulli de paramètre $p\in]0,1[$. On pose $U=X+Y$ et $V=X-Y$.
	\begin{enumerate}
		\item
		Quelle est la conjointe de $(U,V)$?
		\item
		Déterminer la covariance de $X,Y$ ?
		\item
		$U$ et $V$ sont-elles indépendantes?
	\end{enumerate}
\end{exercise}

\begin{exercise}
	On tire indéfiniment à pile ou face, avec une probabilité $p\in]0,1[$
	d'obtenir pile, $q=1-p$ d'obtenir face. On note $P$ (respectivement $F$) le
	rang d'apparition du premier pile (respectivement du premier face) et on note
	$X$ (respectivement $Y$) la longueur de la première (respectivement deuxième)
	suite de tirages égaux (ex: (pile, pile, face, pile,...) donne $X=2$ et
	$Y=1$).
	\begin{enumerate}
		\item
		Donner les lois de $P$ et $F$ et leurs espérances et variances.
		\item
		$P$ et $F$ sont-elles indépendantes?
		\item
		Donner les lois conjointes et marginales de $X$ et $Y$. Sont-elles
		indépendantes?
		\item
		Montrer que $\E(X)\geqslant2$.
		\item
		Quelle est la probabilité que $X=Y$ ?
		\item
		Donner la loi de $X+Y$ lorsque $p=\frac{1}{2}$.
	\end{enumerate}
\end{exercise}

\begin{exercise}
	Soit $(X_{n})_{n\geqslant1}$ une suite de variables aléatoires indépendantes
	discrètes de même loi, centrées à valeurs dans $[-1,1]$.
	\begin{enumerate}
		\item
		Montrer que pour tout $\lambda\geqslant0,\forall x\in[-1,1],~e^{\lambda
		x}\leqslant e^{\frac{\lambda^{2}}{2}}+x\sinh(\lambda)$.
		\item
		En déduire que si $X$ est une variable aléatoire discrète centrée à
		valeurs dans $[-1,1]$, alors pour tout $\lambda\geqslant0$, $\E(e^{\lambda
		X})\leqslant e^{\frac{\lambda^{2}}{2}}$ et $\E(e^{-\lambda X})\leqslant
		e^{\frac{\lambda^{2}}{2}}$.
		\item
		Montrer que pour tout $a\in\R$, pour tout $\lambda>0$, $\P(X\geqslant
		a)\leqslant e^{-\lambda a}\E(e^{\lambda X})$.
		\item
		Montrer que pour tout $n\geqslant1$, pour tout $a\geqslant0$,
		$\P\Bigl(\bigl\vert\frac{1}{n}\sum_{i=1}^{n}X_{i}\bigr\vert\geqslant
		a\Bigr)\leqslant2e^{-\frac{a^{2}}{2}}$.
	\end{enumerate}
\end{exercise}

\begin{exercise}
	\phantom{}
	\begin{enumerate}
		\item
		Soient $X$ et $Y$ deux variables aléatoires discrètes à valeurs dans $\N$
		d'espérances finies. Pour $k\in\N$, on définit
		$$\E_{(X=k)}(Y)=\sum_{l=0}^{+\infty}l\P_{(X=k)}(Y=l)$$ Montrer que
		$$\E(Y)=\sum_{k=0}^{+\infty}\E_{(X=k)}(Y)\times\P(X=k)$$
		\item
		Soit $\lambda\geqslant0$, on suppose que le nombre de descendants poules à
		la première génération d'une poule donnée suit une loi de Poisson de
		paramètre $\lambda$. Soit $X_{n}$ le nombre de poules à la $n$-ième
		génération avec $X_{0}=N\in\N^{*}$. Déterminer $\E(X_{n})$.
	\end{enumerate}
\end{exercise}

\begin{exercise}
	Au cours de sa vie, une poule pond $N$ \oe ufs où $N\sim\mathcal{P}(\lambda)$.
	Chaque \oe uf éclot avec une probabilité $p\in]0,1[$. On note $K$ le nombre de
	poussins. Donner la loi et l'espérance de $K$.
\end{exercise}

\begin{exercise}
	Soit $(A_{n})_{n\geqslant1}$ une suite d'évènements indépendants de $\Omega$
	(où $(\Omega,\mathcal{A},\P)$ est un espace probabilisé) telle que pour tout
	$n\geqslant1$, on a $\P(A_{n})=\frac{1}{n}$. Soit
	$S_{n}=\sum_{k=1}^{n}\chi_{A_{k}}$ (indicatrice de l'ensemble).
	\begin{enumerate}
		\item
		Évaluer l'espérance et variance de $S_{n}$, et en donner des équivalents.
		\item
		Soit $\varepsilon>0$ fixé, montrer que 
		$$\lim\limits_{n\to+\infty}\P\Biggl(\Bigl\vert\frac{S_{n}}{\ln(n)}-1\Bigr\vert\geqslant\varepsilon\Biggr)=0$$
	\end{enumerate}
\end{exercise}

\begin{exercise}
	Soit $n\geqslant1$, $(X_{1},\dots,X_{n})$ des variables aléatoires
	indépendantes discrètes telles que $X_{i}\sim\mathcal{G}(p)$ (loi géométrique)
	avec $p\in]0,1[$. On note $q=1-p$ et $U=\min_{1\leqslant i\leqslant n}X_{i}$
	et $V=\max_{1\leqslant i\leqslant n}X_{i}$.
	\begin{enumerate}
		\item
		Donner la loi de $U$ et son espérance (dont on justifiera l'existence).
		\item
		Donner la loi de $V$, montrer que
		$$\E(V)=\sum_{i=1}^{n}\binom{n}{i}(-1)^{i+1}\frac{1}{1-q^{i}}$$
	\end{enumerate}
\end{exercise}

\begin{exercise}
	Un joueur arrive au casino avec une fortune $k\in\{0,\dots,N\}$ où $N$ est la
	"banque", c'est-à-dire la limite que paiera le casino. A chaque étape, il
	gagne $1$ avec une probabilité $p\in]0,1[$ et perd $1$ avec une probabilité
	$q=1-p$. Il s'arrête lorsqu'il a 0 (ruine) ou $N$ (banque). On sait que
	l'arrêt en temps fini est presque sûr. On note $t_{k}$ le temps d'arrêt du
	joueur (le nombre de fois où il joue avant de s'arrêter).
	\begin{enumerate}
		\item
		Montrer que pour tout $n\in\N$,
		$$\P(t_{k}>N(n+1))\leqslant\P(t_{k}>Nn)\times(1-p^{N})$$ En déduire que
		$t_{k}$ possède une espérance notée $T_{k}$.
		\item
		Montrer que pour tout $k\in\{1,\dots,N-1\}$,
		$T_{k}=p(1+T_{k+1})+q(1+T_{k-1})$.
		\item
		En déduire que 
		$$T_{k}= \left\{
			\begin{array}[]{cc}
				\dfrac{1}{q-p}\Biggl[k-N\Biggl(\dfrac{1-\Bigl(\dfrac{q}{p}\Bigr)^{k}}{1-\Bigl(\dfrac{q}{p}\Bigr)^{n}}\Biggr)\Biggr]
& \text{si}p\neq\dfrac{1}{2}\\
				k(N-k) & \text{si}p=\dfrac{1}{2}
			\end{array}
		\right.$$
	\end{enumerate}
\end{exercise}

\begin{exercise}
	Soit $>1$. On munit $\N^{*}$ de la probabilité:
	$$\forall n\geqslant1,~\P(\{n\})=\frac{1}{\zeta(s)n^{s}}$$
	\begin{enumerate}
		\item
		Vérifier que c'est une probabilité sur $\N^{*}$. On forme, pour
		$n\geqslant 1$, $A_{n}=n\N^{*}$. Calculer $\P(A_{n})$.
		\item
		Montrer que si $\mathcal{P}$ est l'ensemble des nombres premiers, alors
		$(A_{p})_{p\in\mathcal{P}}$ sont indépendants.
		\item
		En déduire que 
		$$\P(\{1\})=\prod_{p\in\mathcal{P}}\Bigl(1-\frac{1}{p^{s}}\Bigr)$$ puis
		que 
		$$\zeta(s)=\Biggl(\prod_{p\in\mathcal{P}}\Bigl(1-\frac{1}{p^{s}}\Bigr)\Biggr)^{-1}$$
	\end{enumerate}
\end{exercise}

\begin{exercise}
	On dispose de deux pièces de monnaie. La pièce $A$ (respectivement $B$) amène
	pile avec la probabilité $a\in]0,1[$ (respectivement $b\in]0,1[$). On choisit
	au départ de façon équiprobable une des deux pièces et on effectue le premier
	lancer avec celle là. Si on obtient pile, on garde la même, sinon on change.
	Soit, pour $n\geqslant1$, $E_{n}$ l'évènement désignant le fait que l'on
	utilise pour la première fois la pièce $A$ au $n$-ième lancer. On indique de
	même par $U_{n}$ l'évènement désignant le fait que l'on obtient $n$ piles au
	cours des $n$ premiers lancers.

	Calculer $\P(E_{n}),\P(U_{n}),\P(\cup_{n\in\N}E_{n}),\P(\cap_{n\in\N}U_{n})$.
\end{exercise}

\begin{exercise}
	Soit $N$ pièces de monnaie, dont chacune amène pile avec la probabilité
	$p\in]0,1[$. A chaque lancer, on laisse de côté les pièces tombées sur pile,
	et on relance les autres jusqu'à que toutes arrivent sur pile.
	\begin{enumerate}
		\item
		Pour $i\in\{1,\dots,N\}$ et $n\in\N^{*}$. Calculer la probabilité de
		l'évènement $A_{i,n}$ désignant le fait que la pièce numéro $i$ est lancée
		au plus $n$ fois.
		\item
		Soit $B_{n}$ l'évènement indiquant que l'on effectue au plus $n$ relances.
		Calculer $\P(B_{n})$ et $\lim\limits_{n\to+\infty}\P(B_{n})$.
		\item
		Pour $n\geqslant1$, quelle est la probabilité d'effectuer exactement $n$
		relances?
	\end{enumerate}
\end{exercise}

\begin{exercise}
	Soit $p$ premier et $\K=\Z/p\Z$ (en tant que corps). Soit $n\geqslant1$. On
	définit $\K_{<n}[X]=\{Q\in\K[X]\mid \deg(Q)<n\}$ et
	$\K_{=n}[X]=\{P\in\K[X]\mid \deg(P)=n\}$. Ces espaces sont munis de la
	probabilité uniforme. On forme $\Omega=\K_{<n}[X]\times\K_{=x}[X]$ muni de la
	probabilité uniforme.
	\begin{enumerate}
		\item
		Quelle est la loi de $\deg(Q)$?
		\item
		Quelle est la probabilité pour que $Q\mid P$ ?
		\item
		Soit $R\in\K_{<n-1}[X]$, on pose pour $(Q,P)\in\Omega$ avec $Q\neq0$,
		$R_{1}$ le reste de la division euclidienne de $P$ par $Q$. Calculer
		$\P(R_{1}=R)$ et $\P_{\deg(Q)=k}(R_{1}=R)$.
	\end{enumerate}
\end{exercise}

\begin{exercise}
	On effectue deux tirages sans remise dans une urne contenant $n$ jetons
	numérotés de $1$ à $n$. On note $X_{1}$ le premier tirage et $X_{2}$ le
	deuxième.
	\begin{enumerate}
		\item
		Donner la loi conjointe de $(X_{1},X_{2})$.
		\item
		Donner la loi de $X_{2}$ puis la loi de $X_{{1}_{\mid X_{2}=j}}$ pour
		$j\in\{1,\dots,n\}$.
		\item
		$X_{1}$ et $X_{2}$ sont-elles indépendantes?
		\item
		Généraliser à $k$ tirages sans remise pour $1\leqslant k\leqslant n$.
	\end{enumerate}
\end{exercise}

\cleardoublepage
\section{Calcul matriciel}

\begin{exercise}
	Soit $M=\bigl(\omega^{(k-1)(l-1)}\bigr)_{1\leqslant k,l\leqslant n}$ où
	$\omega=e^{\frac{2\mathrm{i}\pi}{n}}$. Montrer que $M\in GL_{n}(\C)$ et
	calculer $M^{-1}$. Que vaut $\det(M)$ ?
\end{exercise}

\begin{exercise}
	On dit que $A\in\M_{n,p}(\R)$ est positive et on note $A\geqslant0$ si et
	seulement si tous ses coefficients le sont.
	\begin{enumerate}
		\item
		Soit $A\in\M_{n}(\R)$. Montrer que $A\geqslant0$ si et seulement si pour
		tout $X\in\M_{n,1}(\R)$, si $X\geqslant0$ alors $AX\geqslant0$.
		\item
		Quelles sont les matrices $A\in GL_{n}(\R)$ telles que $A\geqslant0$ et
		$A^{-1}\geqslant0$?
	\end{enumerate}
\end{exercise}

\begin{exercise}
	Soit $A=\Bigl(\binom{j-1}{i-1}\Bigr)_{1\leqslant i,j\leqslant n}$. Calculer
	$A^{-1}$ et $A^{k}$ pour $k\in\Z$.
\end{exercise}

\begin{exercise}
	Soit $\K$ un corps de caractéristique non nulle ($\Q,\R,\C,\dots$). Soit
	$A\in\M_{n}(\K)$ avec $\Tr(A)=0$.
	\begin{enumerate}
		\item
		Montrer que $A$ est semblable à une matrice dont tous les coefficients
		diagonaux sont nuls. On pourra procéder par récurrence, en distinguant
		selon qu'il existe $\lambda\in\K,~A=\lambda I_{n}$ ou non. Dans le
		deuxième cas, pour $u\in\L(\K^{n})$ canoniquement associée à $A$, on
		montrera qu'il existe $e_{1}\in\K^{n}$ telle que $(e_{1},u(e_{1}))$ est
		libre.
		\item
		Montrer qu'il existe $(X,Y)\in\M_{n}(\K)^{2}$ tel que $A=[X,Y]=XY-YX$. On
		pourra considérer \function{\varphi}{\M_{n}(\K)}{\M_{n}(\K)}{M}{DM-MD}
		avec $D=\diag(1,2,\dots,n)$ et déterminer $\ker(\varphi)$.
	\end{enumerate}
\end{exercise}

\begin{exercise}
	Soit $(X,Y)\in\M_{n,1}(\K)^{2}$.
	\begin{enumerate}
		\item
		Pour quelles valeurs de $\lambda$, $I_{n}+\lambda XY^\mathsf{T}$ est
		inversible?
		\item
		Soit $A\in GL_{n}(\R)$. A quelle condition nécessaire et suffisante
		$A+\lambda XY^\mathsf{T}\in GL_{n}(\R)$? Montrer alors que 
		$$(A+\lambda XY^\mathsf{T})^{-1}=A^{-1}-\frac{\lambda}{1+\lambda
		Y^{\mathsf{T}}A^{-1}X}A^{-1}XY^{\mathsf{T}}A^{-1}$$
	\end{enumerate}
\end{exercise}

\begin{exercise}
	Soit $n\geqslant1$ et pour tout $j\in\{0,\dots,n\}$, $S_{j}=X^{j}(1-X)^{n-j}$.
	Montrer que c'est une base de $\R_{n}[X]$ et exprimer $(1,X,\dots,X^{n})$ en
	fonction de $(S_{0},\dots,S_{n})$.
\end{exercise}

\begin{exercise}
	Soit $\K$ un corps et $H$ un hyperplan de $\M_{n}(\K)$. Montrer que $H\cap
	GL_{n}(\K)\neq 0$ pour $n\geqslant2$.
\end{exercise}

\begin{exercise}
	Soit $N:\M_{n}(\C)\to\R_{+}$ telle que:
	\begin{enumerate}
		\item
		[(i)] $\forall \lambda\geqslant0,\forall A\in\M_{n}(\C),~N(\lambda
		A)=\lambda N(A)$,
		\item
		[(ii)] $\forall (A,B)\in\M_{n}(\C)^{2},~N(A+B)\leqslant N(A)+N(B)$,
		\item
		[(iii)] $\forall (A,B)\in\M_{n}(\C)^{2},~N(AB)=N(BA)$.
	\end{enumerate}
	\begin{enumerate}
		\item
		Calculer $N(0)$.
		\item
		Évaluer $N(E_{i,j})$ pour $i\neq j$ (matrice élémentaire de la base
		canonique de $\M_{n}(\C)$).
		\item
		Montrer que si $A\in\M_{n}(\C)$ est telle que $\Tr(A)=0$, alors $A$ est
		semblable à une matrice dont tous les coefficients diagonaux sont nuls. On
		pourra procéder par récurrence, en distinguant selon qu'il existe
		$\lambda\in\K,~A=\lambda I_{n}$ ou non. Dans le deuxième cas, pour
		$u\in\L(\K^{n})$ canoniquement associée à $A$, on montrera qu'il existe
		$e_{1}\in\K^{n}$ telle que $(e_{1},u(e_{1}))$ est libre.
		\item
		En déduire $N(A)$ si $\Tr(A)=0$.
		\item
		Montrer qu'il existe $a\in\R^{+}$ telle que pour tout $A\in\M_{n}(\C)$,
		$N(A)=a\vert\Tr(A)\vert$.
	\end{enumerate}
\end{exercise}

\begin{exercise}
	Soit $E$ un espace vectoriel de dimension $n$, $f\in GL(E)$ et $g$ un
	endomorphisme de $E$ de rang 1. Montrer que $f+g\in GL(E)$ si et seulement
	$\Tr(g\circ f^{-1})\neq 1$.
\end{exercise}

\begin{exercise}
	On considère un carré dans $\Z^{2}$. Pour $n\in\N$, quel est le nombre de
	chemins de longueur $n$ qui relient un sommet à un autre ? Généraliser à un
	cube dans $\Z^{3}$.
\end{exercise}

\begin{exercise}[Matrice à diagonale strictement dominante]
	Soit $A=(a_{i,j})_{1\leqslant i,j\leqslant n}\in\M_{n}(\C)$ telle que pour
	tout $i\in\{1,\dots,n\}$$\vert a_{i,i}\vert>\sum_{j\neq i}\vert a_{i,j}\vert$.
	On dit que $A$ est à diagonale strictement dominante. Montrer que $A\in
	GL_{n}(\C)$. Est-ce encore vrai si on a seulement l'inégalité large ?
\end{exercise}

\begin{exercise}
	Calculer, pour $n\geqslant1$, $\det\Bigl((i\wedge j)\Bigr)_{1\leqslant
	i,j\leqslant n}$. On pourra utiliser, pour tout $n\in\N^{*},~m=\sum_{k\mid
	n}\varphi(k)$.
\end{exercise}

\begin{exercise}
	Soit $A=(a_{i,j})_{1\leqslant i,j\leqslant n}\in\M_{n}(\C)$. On pose, pour
	$k\in\{1,\dots, n\}$,$A_{k}=(a_{i,j})_{1\leqslant i,j\leqslant k}$. On suppose
	que pour tout $k\in\{1,\dots,n\},~A_{k}\in GL_{k}(\C)$. Montrer qu'il existe
	une unique décomposition
	$(L,U)\in\mathcal{T}_{n}^{-}(\C)\times\mathcal{T}_{n}^{+}(\C)$ (matrices
	triangulaires inférieures et supérieures) où $L$ a des $1$ sur la diagonale et
	$A=LU$.
\end{exercise}

\begin{exercise}
	Soit $n\in\N$ et $(a_{1},\dots,a_{2n+1})\in\R^{2n+1}$ tel que pour tout
	$i\in\{1,\dots,2n+1\}$, il existe des parties disjointes $A_{i}$ et $B_{i}$ de
	$\{1,\dots,2n+1\}\setminus\{i\}$ avec $\vert A_{i}\vert=\vert B_{i}\vert=n$ et
	$\sum_{k\in A_{i}}a_{k}=\sum_{k\in B_{i}}a_{k}$.

	Monter que $a_{1}=\dots=a_{2n+1}$.
\end{exercise}

\begin{exercise}
	Soit $M\in GL_{n}(\C)$, montrer qu'il existe une unique permutation
	$\sigma\in\Sigma_{n}$ et il existe $(T,T')\in(\mathcal{T}_{n}^{+})^{2}$ telles
	que $M=TP_{\sigma}T'$ où $P_{\sigma}=(\delta_{i,\sigma(j)})_{1\leqslant
	i,j\leqslant n}$ et $\delta$ est le symbole de Kronecker. Cette décomposition
	est-elle unique ?
\end{exercise}

\begin{exercise}
	Soit $\K$ un sous-corps de $\C$ et $J$ un idéal non nul de $M_{n}(\K)$.
	\begin{enumerate}
		\item
		Montrer que si $J\cap GL_{n}(\K)\neq\emptyset$, alors $J=M_{n}(\K)$.
		\item
		Montrer que $J$ contient une matrice de rang 1.
		\item
		Montrer que $J=\M_{n}(\K)$.
	\end{enumerate}
\end{exercise}

\begin{exercise}
	Soit $(A,B)\in\M_{n}(\C)$ et $\lambda\neq 0$ avec $\lambda AB+A+B=0$, montrer
	que $A$ et $B$ commutent.
\end{exercise}

\begin{exercise}
	Soit $(a_{2},\dots,a_{n})\in\R^{n-1}$. Inverser, si possible,
	$$
	A=
	\begin{pmatrix}
		1 		& -a_{2}	& \dots		& -a_{n}\\
		a_{2} 	& \ddots 	& 0			& 0\\
		\vdots 	& 0			& \ddots 	& 0\\
		a_{n}	& 0			& 0			& 1
	\end{pmatrix}
	$$
\end{exercise}

\begin{exercise}
	Soit $A=(a_{i,j})_{1\leqslant i,j\leqslant n}\in\M_{n}(\R)$ telle que pour
	tout $i\in\{1,\dots,n\}$, $a_{i,i}=0$ et pour tout $i\neq j$,
	$a_{i,j}+a_{j,i}=1$. Soit $u\in\L(\R^{n})$ canoniquement associé à $A$. Soit
	$H=\{(x_{1},\dots,x_{n})\in\R^{n}\mid\sum_{i=1}^{n}x_{i}=0\}$.
	\begin{enumerate}
		\item
		Déterminer $\ker(u)\cap H$. En déduire que $\rg(A)\in\{n-1,n\}$.
		\item
		Est-il possible que toutes les matrices $A$ vérifiant ces conditions
		soient de rang $n-1$ ?
		\item
		Même question avec $n$.
	\end{enumerate}
\end{exercise}

\begin{exercise}
	Soit $(M,N)\in\M_{n}(\C)^{2}$ tel que $\rg(M)=\rg(N)=1$. Montrer que $M$ et
	$N$ sont semblables si et seulement si $\Tr(M)=\Tr(N)$.
\end{exercise}

\begin{exercise}
	Soit $F$ un sous-espace vectoriel de $M_{n}(\K)$, on note $r=\max\{\rg(M)\mid
	M\in F \}$.
	\begin{enumerate}
		\item
		Montrer qu'il existe $(P_{0},Q_{0})\in GL_{n}(\R)$ telle que 
		$$
		P_{0}^{-1}
		\underbrace{
			\left(
				\begin{array}{@{}c|c@{}}
					I_{r}
					& 0_{r,n-r} \\
					\hline
					0_{n-r,r} &
					O_{n-r,n-r}
				\end{array}\right)
		}_{\displaystyle J_{r}}
		Q_{0}\in F
		$$
		On note $F_{0}=\{P_{0}MQ_{0}^{-1}\mid M\in F\}$.
		\item
		Montrer que $F_{0}$ est un sous-espace vectoriel de $M_{n}(\R)$ isomorphe
		à $F$, et que \\$r=\max\{\rg(M_{0})\mid M_{0}\in F_{0}\}$.
		\item
		On définit
		
		$$G_{0}= \left(
				\begin{array}{@{}c|c@{}}
					0_{r}
					& B^{\mathsf{T}} \\
					\hline
					B & C \end{array}\right)
		$$
		où $B\in\M_{n-r,r}(\R)$ et $C\in\M_{n-r}(\R)$. 
		Quelle est la dimension de l'espace vectoriel engendré par $G_{0}$ ?
		\item Soit $M_{0}\in G_{0}\cap F_{0}$ avec 
		$$
			M_{0}= \left(
				\begin{array}{@{}c|c@{}}
					0_{r}
					& B^{\mathsf{T}} \\
					\hline
					B & C
				\end{array}
			\right)
			\in
			F_{0}
		$$
		Montrer que pour tout $\lambda\in\R$,
		$$
			\left(
				\begin{array}{@{}c|c@{}}
				\lambda
				I_{r} & B^{\mathsf{T}} \\
					\hline
					B & C
				\end{array}
			\right)
			\in
			F_{0}
		$$
		En déduire que pour tout $(i,j)\in\{1,\dots, n-r\}^{2}$, pour tout $\lambda\neq0$,
		$$
		\det\left(
				\begin{array}{@{}c|c@{}}
				\lambda
				I_{r} &
				\begin{matrix}
				b_{j,1}\\
						\vdots\\
						b_{j,r}
						\end{matrix}
						\\
					\hline
					\begin{matrix}
						b_{i,1} &
						\dots
						& b_{i,r}
						\end{matrix}
						& c_{i,j}
				\end{array}
			\right)=0
		$$
		\item Montrer que $C=0$, puis que $B=0$.
		\item Conclure.
		\item Si $\dim(F)\geqslant n^{2}-n+1$, montrer que $F\cap GL_{n}(\R)\neq\emptyset$.
		\item Et sur $M_{n}(\C)$ ?
	\end{enumerate}
\end{exercise}

\begin{exercise}
	Soit $f:\M_{n}(\C)\to\C$ non constante telle que pour tout $(A,B)\in\M_{n}(\C)^{2}$, $f(AB)=f(A)\times f(B)$. 
	Montrer que $f(M)\neq0$ si et seulement si $M\in GL_{n}(\C)$.
\end{exercise}

\cleardoublepage
\section{Réduction des endomorphismes}

\begin{exercise}
	Soit $E=\M_{n}(\C)$ et pour $(A,B)\in E^{2}$ et \function{f}{E}{E}{M}{AM} et \function{g}{E}{E}{M}{MB} et $h=f\circ g$.
	\begin{enumerate}
		\item Montrer que $f$ (respectivement $g$) est diagonalisable si et seulement si $A$ (respectivement $B$) l'est.
		\item Soient $(X_{1},\dots,X_{n})$ et $(Y_{1},\dots,Y_{n})$ deux bases de $\C^{n}=\M_{n,1}(\C)$. 
		Montrer que $(X_{i}Y_{i}^{\mathsf{T}})_{1\leqslant i,j\leqslant n}$ est une base de $E$.
		\item On suppose que $A$ et $B$ sont diagonalisables. Montrer que $h$ l'est. A-t-on la réciproque ?
	\end{enumerate}
\end{exercise}

\begin{exercise}
	Soit $(P,Q)\in\K[X]^{2}$ unitaires. Soient $D=P\wedge Q$, $M=P\vee Q$ et $f\in\L(E)$ où $E$ est un $\K$-espace vectoriel.
	Montrer les différentes assertions suivantes:
	\begin{enumerate}
		\item $\ker D(f)=\ker P(f)\cap \ker Q(f)$.
		\item $\ker M(f)=\ker P(f)+ \ker Q(f)$.
		\item $\im D(f)=\im P(f)+\im Q(f)$.
		\item $\im M(f)=\im P(f)\cap\im Q(f)$.
	\end{enumerate}
\end{exercise}

\begin{exercise}
	Soit $A\in\M_{n}(\R)$ telle que $A^{2}-4A+5I_{n}=0$. $A$ est-elle inversible ? Que dire de $A$ ? Que dire de $n$ ?
	Calculer les puissances de $A$ ? 
\end{exercise}

\begin{exercise}
	Soit $A=(a_{i,j})_{1\leqslant i,j\leqslant n}\in\M_{n}(\R)$ est dite stochastique si et seulement si
	$$
	\left\{
		\begin{array}[]{c}
			\forall(i,j)\in\{1,\dots,n\}^{2},~a_{i,j}\geqslant0\\
			\forall
			i\in\{1,\dots,n\},~\sum_{j=1}^{n}a_{i,j}=1
		\end{array}
	\right.
	$$
	\begin{enumerate}
		\item Montrer que $1\in\Sp_{\R}(A)$.
		\item Soit $\lambda\in\Sp_{\C}(A)$, montrer que $\vert\lambda\vert\leqslant1$.
		\item Soit $\lambda\in\Sp_{\C}(A)$ et $x$ un vecteur propre associé.\\
		Montrer que si pour tout $i\in\{1,\dots,n\}, a_{i,i}>0$ alors $\lambda=1$.
		\item Soit $\lambda\in\Sp_{\C}(A)$ telle que $\vert \lambda\vert=1$. Montrer que $\lambda$ est une racine de l'unité.
		\item Reconnaître les matrices stochastiques dont toutes les valeurs sont de module 1.
	\end{enumerate}
\end{exercise}

\begin{exercise}
	Soit $(A,B)\in\M_{n}(\C)^{2}$ et \function{\Phi_{A,B}}{\M_n(\C)}{\M_n(\C)}{M}{AM-MB}
	\begin{enumerate}
		\item Déterminer $\Sp(\Phi_{A,B})$ en fonction de $\Sp(A)$ et $\Sp(B)$.
		\item Montrer que si $A$ et $B$ sont diagonalisables, $\Phi_{A,B}$ l'est aussi.
	\end{enumerate}
\end{exercise}

\begin{exercise}
	Soit $A\in\M_{n}(\C)$ et $\theta\in\C$. Soit $F=\{M\in\M_{n}(\C)\mid AM=\theta MA\}$.
	\begin{enumerate}
		\item Montrer que pour tout $P\in\C[X]$, pour tout $M\in F$, on a $P(A)M=MP(\theta A)$. Établir une relation analogue portant sur $P(M)$.
		\item On suppose $A$ diagonalisable. Quelle est l'action de $F$ sur les sous-espaces propres de $A$ ? Donner une condition nécessaire et suffisante sur $\Sp_{\C}(A)$ pour que $F=\{0\}$.
	\end{enumerate}
\end{exercise}

\begin{exercise}
	Réduire sur $\C$
	$$
	A=
	\begin{pmatrix}
		1 & 1 & 0 & 1\\
		1 & 1 & 1 & 0\\
		1 & 0 & 1 & 1\\
		0 & 1 & 1 & 1
	\end{pmatrix}
	$$
\end{exercise}

\begin{exercise}
	Soit $0<a_{1}<\dots<a_{n}$ et $A=(a_{i,j})_{1\leqslant i,j\leqslant n}\in\M_{n}(\R)$ telle que pour tout $i\in\{1,\dots,n\}$, $a_{i,i}=0$ et si $i\neq j$, $a_{i,j}=a_{j}$.
	\begin{enumerate}
		\item Montrer que $\lambda\in\Sp_{\R}(A)$ si et seulement si 
		$$\sum_{k=i}^{n}\frac{a_{k}}{\lambda+a_{k}}=1$$
		\item $A$ est-elle diagonalisable ?
	\end{enumerate}
\end{exercise}

\begin{exercise}
	Soit $G$ le sous-groupe de $GL_{n}(\R)$ engendré par les matrices diagonalisables inversibles. Montrer que $G=GL_{n}(\R)$/ 
\end{exercise}

\begin{exercise}
	Soit $u\in\L(\C)$ et $p\geqslant2$. Montrer que $u^{p}$ est diagonalisable si et seulement si $u$ est diagonalisable et $\ker(u)=\ker(u^{2})$.
\end{exercise}

\begin{exercise}[Matrice circulante]
	Soit $n\geqslant 1$, $(a_{0},\dots,a_{n-1})\in\C^{n}$ et 
	$$
	A(a_{0},\dots,a_{n-1})=
	\begin{pmatrix}
		a_{0} & a_{1} & \dots & \dots & a_{n-1}\\
		a_{n-1} & \ddots & \ddots & \ddots & a_{n-2}\\
		\vdots & \ddots & \ddots & \ddots & \vdots\\
		a_{1} & \dots & \dots & a_{n-1} & a_{0}
	\end{pmatrix}
	$$
	Donner les éléments propres de $A(a_{0},\dots,a_{n-1})$. Est-elle diagonalisable ? Calculer son déterminant.
\end{exercise}

\begin{exercise}
	Soit $E$ un $\K$-espace vectoriel de dimension $n\geqslant1$. Soit $f\in\L(E)$ nilpotent tel que $\dim(\ker(f))=1$. Montrer que $f^{n-1}\neq 0$ et qu'il existe $x\in E\setminus\{0\}$, $(x,f(x),\dots,f^{n-1}(x))$ est une base de $E$.
\end{exercise}

\begin{exercise}[Endomorphisme cyclique]
	Soit $V$ un $\C$-espace vectoriel de dimension finie $n$. Soit $u\in\L(V)$. Montrer qu'il existe $x\in V$ tel que $(x,u(x),\dots,u^{n-1}(x))$ soit une base de $V$ si et seulement si les sous-espaces propres de $u$ sont de dimension 1.
\end{exercise}

\begin{exercise}
	Soit $V$ un $\C$-espace vectoriel de dimension $d$.
	\begin{enumerate}
		\item Pour $f\in\L(V)$, montrer qu'il existe $r(f)=\lim\limits_{n\to+\infty}\rg(f^{n})$.
		\item Si $f$ et $g$ commutent, montrer que $r(f+g)\leqslant r(f)+r(g)$. Et si $f$ et $g$ ne commutent pas ?
		\item Exprimer $r(f)$ en fonction du degré du polynôme caractéristique de $f$.
	\end{enumerate}
\end{exercise}

\begin{exercise}
	Soit $q\in\N^{*}$ et $\mathcal{G}_{q}=\{A\in\M_{n}(\C)\mid A^{q}=I_{n}\}$. Quels sont les points isolés de $\mathcal{G}_{q}$ ?
\end{exercise}

\begin{exercise}
	Soit 
	$$
	M=
	\begin{pmatrix}
		1 & -1 & 0\\
		-1 & 2 & 1\\
		1& 0 & 1
	\end{pmatrix}
	$$
	Soit $u\in\L(R^{3})$ canoniquement associée à $M$. Trouver tous les sous-espaces de $\R^{3}$ stables par $u$.
\end{exercise}

\begin{exercise}
	Soit 
	$$
	A=
	\left(
		\begin{array}{@{}c|c@{}}
		I_{n} &
		\begin{matrix}
			a_{1}\\
			\vdots\\
			a_{n}
			\end{matrix}
			\\
		\hline
		\begin{matrix}
			a_{1} &
			\dots
			& a_{n}
			\end{matrix}
			& 0
		\end{array}
		\right)
	$$
	\begin{enumerate}
		\item $A$ est-elle diagonalisable ?
		\item Donner ses éléments propres.
	\end{enumerate}
\end{exercise}

\begin{exercise}
	Soit $G$ un sous-groupe borné de $\M_{n}(\C)$. Montrer que pour tout $M\in G$, $\Sp_{\C}(M)\subset \U$ et $M$ est diagonalisable. Montrer qu'il existe $\alpha>0$ tel que si $\vertiii{M-I_{n}}<\alpha$ alors $G=\{I_{n}\}$.
\end{exercise}

\begin{exercise}
	Soit 
	$$
	A=
	\begin{pmatrix}
		2 & 1 & 0\\
		-3 & -1 & 1\\
		1 & 0 & -1
	\end{pmatrix}
	$$
	et $u\in\L(\R^{3})$ canoniquement associée à $A$.
	\begin{enumerate}
		\item Trouver tous les sous-espaces de $\R^{3}$ stables par $u$.
		\item Existe-t-il $B\in\M_{3}(\R)$ telle que $B^{2}=A$ ?
	\end{enumerate}
\end{exercise}

\begin{exercise}
	Soit $A\in\M_{3}(\R)$ tel que $A^{3}+A^{2}+A+I_{3}=0$ et $A\neq -I_{3}$. Montrer que $A$ est semblable à
	$$
	\begin{pmatrix}
		0 & -1 & 0\\
		1 & 0 & 0\\
		0 & 0 & -1
	\end{pmatrix}
	$$
\end{exercise}

\begin{exercise}
	Soit $E$ un $\K$-espace vectoriel de dimension $n$, $f\in\L(E)$ et $x\in E$.
	\begin{enumerate}
		\item Montrer qu'il existe un unique $P_{x}$ unitaire tel que pour tout $A\in P_{x}\K$, $A(f)(x)=0$.
		\item Montrer que $\mu_{f}$ (polynôme minimal de $f$) est égal à 
		$$\mu_{f}=\underset{x\in E}{\vee}P_{x}$$
		\item Soit $(x,y)\in E^{2}$, montrer que si $P_{x}\vee P_{y}=1$ alors $P_{x+y}=P_{x}P_{y}$.
		\item Montrer qu'il existe $x\in E$ tel que $P_{x}=\mu_{f}$.
		\item Montrer qu'il existe $v\in E$, tel que $(v,f(v),\dots f^{n-1}(v))$ est une base de $E$ si et seulement si $\deg(\mu_{f})=n$ (donc le polynôme minimal est égal au polynôme caractéristique).
	\end{enumerate}
\end{exercise}

\begin{exercise}
	Soit $S=\R^{\N^{*}}$, pour $s=(s_{n})_{n\geqslant1}\in S$, on définit 
	$$s^{*}=\Biggl(\frac{1}{n}\sum_{k=1}^{n}s_{k}\Biggr)_{n\geqslant1}$$
	\begin{enumerate}
		\item Montrer que \function{\varphi}{S}{S}{s}{s^{*}} est un automorphisme.
		\item Déterminer les éléments propres de $\varphi$.
	\end{enumerate}
\end{exercise}

\begin{exercise}[Disques de Gershgorin]
	Soit $A=(a_{i,j})_{1\leqslant i,j\leqslant n}\in\M_{n}(\C)$. On note, pour tout $1\leqslant i,j\leqslant n$, $L_{i}=\sum_{k\neq i}\vert a_{i,k}\vert$ et $C_{j}=\sum_{k\neq j}\vert a_{k,j}\vert$. Soit $D_{i}=\{z\in\C\mid\vert z-a_{i,i}\vert\leqslant L_{i}\}$ et $S_{j}=\{z\in\C\mid\vert z-a_{j,j}\vert\leqslant C_{j}\}$.
	\begin{enumerate}
		\item Montrer que $\Sp_{\C}(A)\subset\Biggl[\biggl(\bigcup_{i=1}^{n}D_{i}\biggr)\bigcap \biggl(\bigcup_{j=1}^{n}S_{j}\biggr)\Biggr]$
		\item Montrer que si $\lambda\in\Sp_{\C}(A)$, il existe $i_{1}\neq i_{2}\in\{1,\dots,n\}^{2}$ tels que 
		$$\vert\lambda-a_{i_{1},i_{1}}\vert\times\vert\lambda-a_{i_{2},i_{2}}\vert\leqslant L_{i_{1}}\times L_{i_{2}}$$
	\end{enumerate}
\end{exercise}

\begin{exercise}
	Soit 
	$$
	A=
	\left(
		\begin{array}{@{}c|c@{}}
		0_{n} &
		\begin{matrix}
			a_{1}\\
			\vdots\\
			a_{n}
			\end{matrix}
			\\
		\hline
		\begin{matrix}
			a_{1} &
			\dots
			& a_{n}
			\end{matrix}
			& 0
		\end{array}
		\right)\in \M_{n+1}(\R)
	$$
	Réduire $A$.
\end{exercise}

\begin{exercise}
	Soient $f$ et $g$ dans $\L(\K^{n})$ diagonalisables. Montrer que $f$ et $g$ ont les mêmes sous-espaces propres si et seulement s'il existe $(P,Q)\in\K_{n-1}[X]$ tels que $f=P(g)$ et $g=Q(f)$.
\end{exercise}

\begin{exercise}
	Soit $G$ un sous-groupe fini abélien de $GL_{2}(\Z)$. Montrer que $\vert G\vert\in\{1,2,3,4,6\}$ et donner un exemple d'un tel sous-groupe dans chaque cas.
\end{exercise}

\begin{exercise}
	Soit $E$ un $\C$-espace vectoriel de dimension finie et $u\in\L(E)$. Montrer que $u$ est diagonalisable si et seulement si tout sous-espace stable admet un supplémentaire stable.
\end{exercise}

\begin{exercise}
	Soit 
	$$
	A=
	\begin{pmatrix}
		1 & -1\\
		1 & -1
	\end{pmatrix}
	$$
	et 
	$$
	M=
	\begin{pmatrix}
		0 & 0 & A\\
		0 & A & 0\\
		A & 0 & 0
	\end{pmatrix}
	$$
	$M$ est-elle diagonalisable ?
\end{exercise}

\begin{exercise}
	Soit
	$$
	A=
	\left(
		\begin{array}{@{}c|c@{}}
		I_{n} &
		\begin{matrix}
			a_{1}\\
			\vdots\\
			a_{n}
			\end{matrix}
			\\
		\hline
		\begin{matrix}
			a_{1} &
			\dots
			& a_{n}
			\end{matrix}
			& 1
		\end{array}
		\right)
	$$
\end{exercise}

\begin{exercise}
	Soit $n\geqslant1$ et $(x_{1},\dots,x_{n})\in\C^{n}$.
	\begin{enumerate}
		\item
		$$
		A=
		\begin{pmatrix}
			0 & \dots & 0 & x_{n}\\
			\vdots &  & x_{n-1} & 0\\
			0 & \reflectbox{$\ddots$} & & \vdots\\
			x_{1}& 0 &\dots & 0
		\end{pmatrix}
		$$
		est-elle diagonalisable ?
		\item La suite $(A^{p})_{p\in\N}$ converger-t-elle ?
	\end{enumerate}
\end{exercise}

\begin{exercise}
	\phantom{}
	\begin{enumerate}
		\item Donner les valeurs propres et vecteurs propres de \function{\varphi}{\R_n[X]}{\R_n[X]}{P}{XP'-nP}
		\item Donner les valeurs propres et vecteurs propres de \function{\varphi}{\R_n[X]}{\R_n[X]}{P}{XP'-nP''}
	\end{enumerate}
\end{exercise}

\begin{exercise}
	Soit 
	$$
	A=
	\begin{pmatrix}
		a & b & c\\
		c & a & b\\
		b & c & a
	\end{pmatrix}
	$$
	avec $a+b+c=1$ pour $(a,b,c)\in\R_{+}^{3}$.
	\begin{enumerate}
		\item Donner le $\Sp_{\C}(A)$.
		\item La suite $(A^{n})_{n\in\N}$ converge-t-elle ?
	\end{enumerate}
\end{exercise}

\begin{exercise}
	Soit $(n_{1},n_{2})\in(\N^{*})^{2}$ et
	$$
	A=
	\begin{pmatrix}
		B & C &\\
		0_{n_{2},n_{1}} & D
	\end{pmatrix}
	$$
	avec $B\in\M_{n_{1}}(\C)$, $C\in\M_{n_{1},n_{2}}\in\M_{n}(\C)^{2}$ et $D\in\M_{n_{2}}(\C)$.
	\begin{enumerate}
		\item Donner une formule pour $A^{p}$ pour $p\in\N$.
		\item Comparer, du point de vue de la divisibilité $\mu_{A}$, $\mu_{B}\vee\mu_{D}$ et $\mu_{B}\times\mu_{D}$ (polynômes minimaux).
		\item Que dire si $C=0$ ?
		\item Que dire si $B=D$ et $C=I_{n_{1}}$ ?
		\item Trouver une matrice $A$ telle que $\mu_{A}\neq \mu_{B}\vee\mu_{D}$ et $\mu_{A}\neq \mu_{B}\times\mu_{D}$.
	\end{enumerate}
\end{exercise}

\begin{exercise}
	Soit $E$ un $\C$-espace vectoriel de dimension finie $n\geqslant1$. Soit $f\in\L(E)$ et \function{\Phi_f}{\L(E)}{\L(E)}{g}{f\circ g-g\circ f}
	\begin{enumerate}
		\item Montrer que si $f$ est nilpotente, $\Phi_{f}$ l'est aussi. A-t-on la réciproque ?
		\item Montrer que $f$ est diagonalisable si et seulement si $\Phi_{f}$ l'est.
		\item Déterminer $\Sp(\Phi_{f})$ en fonction de celui de $f$.
		\item Montrer que si $g$ est vecteur propre de $\Phi_{f}$, il existe une base $\mathcal{B}$ de $E$ telle que $\mat(f,\mathcal{B})$ et $\mat(g,\mathcal{B})$ sont triangulaires supérieures.
	\end{enumerate}
\end{exercise}

\begin{exercise}
	Soit $E$ un $\C$-espace vectoriel de dimension quelconque et $f\in\L(E)$ et $P\in\C[X]$. On pose $g=P(f)$. Soit $\lambda\in\C$ tel que $g-\lambda id_{E}$ n'est pas inversible. Montrer qu'il existe $\mu\in\C$ tel que $\lambda=P(\mu)$ et $f-\mu id_{E}$ n'est pas inversible. Si $\lambda\in\Sp(g)$, montrer qu'il existe $\mu\in\C$ tel que $\lambda=P(\mu)$ et $\mu\in\Sp(f)$.
\end{exercise}

\begin{exercise}
	Soit $E$ un $\K$-espace vectoriel de dimension finie, $V$ un sous-espace vectoriel de $\L(E)$ tel que $V\{0\}\subset GL(E)$.
	\begin{enumerate}
		\item Montrer que $\dim(V)\leqslant\dim(E)$.
		\item Trouver tous les $V$ possibles pour $\K=\C$.
		\item Trouver tous les $V$ possibles pour $\K=\R$ et $E=\R^{2}$.
		\item Si $\K=\R$ et $\dim(V)\geqslant2$, montrer qu'il existe $(f,g)\in V^{2}$ tel que si $\mathcal{B}$ est une base de $E$, $A=\mat(f,\mathcal{B})$ et $B=\mat(g,\mathcal{B})$ alors $i\in\Sp_{\C}(AB^{-1})$.
	\end{enumerate}
\end{exercise}

\begin{exercise}
	Soit $GL_{2}(\Z)=\{M\in\M_2(\Z)\text{ inversibles }\mid M^{-1}\in\M_{2}(\Z)\}$.
	\begin{enumerate}
		\item Soit $M\in\M_{2}(\Z)$, montrer que $M\in GL_{2}(\Z)$ si et seulement si $\det(M)=\pm1$.
		\item Soit $G$ un sous-groupe abélien fini de $GL_{2}(\Z)$, montrer que $\vert G\vert\in\{1,2,3,4,6\}$ et décrire $G$.
	\end{enumerate}
\end{exercise}

\begin{exercise}
	Soit $\K$ un corps quelconque, $n\geqslant1$ et $A\in\M_{n}(\K)$. On a $\chi_{A}=a_{0}+a_{1}X+\dots+a_{n-1}X^{n-1}+X^{n}$ (polynôme caractéristique).
	\begin{enumerate}
		\item Montrer qu'il existe $(M_{0},\dots,M_{n-1})\in\M_{n}(\K)^{n}$ tel que pour tout $\lambda\in\K$, $\com(\lambda I_{n}-A)^{\mathsf{T}}=M_{0}+\lambda M_{1}+\dots+\lambda^{n-1}M_{n-1}$ (où $\com$ indique la comatrice.)
		\item En formant $(\lambda I_{n}-A)\com(\lambda I_{n}-A)^{\mathsf{T}}$, calculer $(M_{0},\dots,M_{n-1})$ en fonction de $A,A^{2},\dots,A^{n}$. En déduire le théorème de Cayley-Hamilton.
		\item Pour les questions suivantes, on suppose que la caractéristique de $\K$ est 0. Montrer que pour tout $\lambda\in\K$, $\chi_{A}'(\lambda)=\Tr(\com(\lambda I_{n}-A)^{\mathsf{T}})$.
		\item En déduire qu'il existe $f:\K^{n}\to\K^{n}$ telle que pour tout $A\in\M_{n}(\K)$, $(a_{0},\dots,a_{n-1}) = f(\Tr(A),\dots,\Tr(A^{n}))$.
		\item Soit $B\in\M_{n}(\K)$, montrer que si pour tout $k\in\{1,\dots,n\}$ $\Tr(A^{k})=\Tr(B^{k})$ alors $\chi_{A}=\chi_{B}$.
	\end{enumerate}
\end{exercise}

\begin{exercise}
	On admet que si $P\in\K[X]$ (avec $\K$ un corps), il existe $\mathbb{L}$ sur-corps de $\K$ tel que $P$ soit scindé sur $\L$ avec la caractéristique de $\mathbb{L}$ égale à la caractéristique de $\K$.

	Soit $A\in\M_{n}(\Z)$ et $p$ premier, montrer que $\Tr(A^{p})\equiv\Tr(A)[p]$.
\end{exercise}

\begin{exercise}
	Soit $M\in\M_{n}(\C)$. Montrer que $M$ est nilpotente si et seulement s'il existe $(M_{p})_{p\in\N}\in(\M_{n}(\C))^{\N}$ telle que pour tout $p\in\N$, $M_{p}$ est semblable à $M$ et $M_{p}\xrightarrow[p\to+\infty]{}0$.
\end{exercise}

\begin{exercise}
	Montrer que $M\in\M_{n}(\C)$ est diagonalisable si et seulement si $S_{M}=\{P^{-1}MP\bigm| P\in GL_{n}(\C)\}$ est fermé. Pour le sens indirect, on pourra utiliser la décomposition de Dunford et l'exercice précédent.
\end{exercise}

\cleardoublepage
\section{Espaces vectoriels normés}

\begin{exercise}
	On définit \function{N}{\R^{2}}{\R}{(x,y)}{\sup\limits_{t\in\R}\vert x\cos(t)+y\sin(2t)\vert}
	\begin{enumerate}
		\item Montrer que $N$ est une norme.
		\item Montrer que $$\overline{B_{\Vert\cdot\Vert_{1}}(0,1)}\subset \overline{B_{N}(0,1)}\subset\overline{B_{\Vert\cdot\Vert_{\infty}}(0,1)}$$
		\item Montrer que $$S_{N}(0,1)\bigcap(\R_{+})^{2}=\Biggl\{(x,y)\in\R^{2}\Biggm|\exists t\in[0,\frac{\pi}{4}]\colon x\cos(t)+y\sin(2t)=1\Biggr\}$$
		En déduire que $$S_{N}(0,1)\bigcap(\R_{+})^{2}=\Biggl\{\Biggl(\frac{\cos(2t)}{\cos(t)^{3}},\frac{\sin(t)}{2\cos(t)^{3}}\Biggr)\Biggm| t\in\Bigl[0,\frac{\pi}{4}\Bigr]\Biggr\}$$
	\end{enumerate}
\end{exercise}

\begin{exercise}
	Soit $E=\mathcal{C}([0,1],\R)$ et \function{N}{E}{\R}{f}{\sqrt{f(0)^{2}+\int_{0}^{1}f'^{2}}}
	\begin{enumerate}
		\item Montrer que $N$ est une norme sur $E$ et que $\Vert\cdot\Vert_{\infty}\leqslant\sqrt{2}N$.
		\item $N$ et $\Vert\cdot\Vert_{\infty}$ sont-elles équivalentes ?
	\end{enumerate}
\end{exercise}

\begin{exercise}
	Soit $n\geqslant p$ et $f\in\L(\R^{n},\R^{p})$. Montrer que $f$ est ouverte, c'est-à-dire que pour tout $\Theta$ ouvert de $\R^{n}$, $f(\Theta)$ est un ouvert de $\R^{p}$, si et seulement si $f$ est surjective.
\end{exercise}

\begin{exercise}
	Soit $E=\Bigl\{\text{fonctions lipschitziennes}\colon [0,1]\to\R\Bigr\}$. Pour $f\in E$, on pose 
	$$\kappa(f)=\sup\Biggl\{\Biggl\lvert\frac{f(x)-f(y)}{x-y}\Biggr\rvert\Bigm|(x,y)\in[0,1]^{2},x\neq y\Biggr\}$$
	\begin{enumerate}
		\item Montrer que $N(f)=\vert f(0)\vert+\kappa(f)$ est une norme sur $E$.
		\item Montrer que $N$ et $N_{\infty}$ ne sont pas équivalentes.
		\item Montrer que $N'=N_{\infty}+\kappa$ est équivalente à $N$.
	\end{enumerate}
\end{exercise}

\begin{exercise}
	Soit $G$ un sous-groupe de $GL_{n}(\C)$ tel que $G\in\mathcal{V}(I_{n})$ où $\mathcal{V}$ un voisinage de $I_{n}$, muni de la norme 
	$$\Vert (a_{i,j})_{1\leqslant i,j\leqslant n}\Vert=\max\limits_{1\leqslant i,j\leqslant n}\vert a_{i,j}\vert$$. Montrer que $G=GL_{n}(\C)$.
\end{exercise}

\begin{exercise}
	Soit $(E,\Vert\cdot\Vert)$ un $\K$-espace vectoriel normé et $f:E\to E$ une fonction telle que 
	\begin{enumerate}
		\item [(i)] $\forall(x,y)\in E^{2}$, $f(x+y)=f(x)+f(y)$
		\item [(ii)] $\exists M\geqslant0,\forall x\in B_{\Vert\cdot\Vert}(0,1),\Vert f(x)\vert\leqslant M$.
	\end{enumerate}
	Montrer que $f$ est continue et linéaire.
\end{exercise}

\begin{exercise}
	Soit $E$ un espace vectoriel normé. Pour $A\subset E$, on pose $\alpha(A)=\mathring{\overline{A}}$.
	\begin{enumerate}
		\item Montrer que $\alpha(\alpha(A))=\alpha(A)$.
		\item Combien au plus de parties différentes obtient-on à partir de $A$ par itérations d'intérieur et d'adhérence ?
	\end{enumerate}
\end{exercise}

\begin{exercise}
	Soit $A$ une partie non vide d'un $\R$-espace vectoriel normé $E$. On définit \function{d_A}{E}{\R}{x}{d(x,A)=\inf\{\Vert x-a\Vert\big| a\in A\}} avec $d_{\emptyset}(x)=+\infty$ pour tout $x\in E$.
	\begin{enumerate}
		\item Soit $A,B\subset E$. Montrer que $\overline{A}=\overline{B}$ si et seulement si $d_{A}=d_{B}$.
		\item On pose $\rho(A,B)=\sup\limits_{x\in E}\vert d_{A}(x)-d_{B}(x)\vert$ (vaut $+\infty$ si non borné). Montrer que 
		$$\rho(A,B)=\max\Bigl(\sup\limits_{x\in A}d_{B}(x),\sup\limits_{y\in B}d_{A}(y)\Bigr)\overset{def}{=}\alpha(A,B)$$
	\end{enumerate}
\end{exercise}

\begin{exercise}
	Soit $P\in\C[X]$ non constant.
	\begin{enumerate}
		\item Montrer que si $F$ est un fermé de $\C$, alors $P(F)$ est un fermé de $\C$.
		\item Si $\Theta$ est un ouvert non vide de $\C$, montrer que $P(\Theta)$ est un ouvert de $\C$.
	\end{enumerate}
\end{exercise}

\begin{exercise}
	On définit $F=\{P\in\R_n[X]\bigm| P\text{ unitaire et }\deg(P)=n\}$. $F$ est fermé dans $\R_{n}[X]$. Notons $\mathcal{S}=\{P\in F\bigm| P\text{ est scindé sur }\R\}$.
	\begin{enumerate}
		\item Montrer que $P\in\mathcal{S}$ si et seulement si pour tout $z\in\C$, $\vert P(z)\vert\geqslant \vert\Im(z)\vert^{n}$.
		\item En déduire que $\mathcal{S}$ est fermé.
		\item Montrer que $\{M\in\M_{n}(\R)\bigm| M\text{ trigonalisable sur }\R\}$ est fermé.
	\end{enumerate}
\end{exercise}

\begin{exercise}
	Soit $(n,m)\in(\N^{*})^{2}$ et $A=\sum_{i=1}^{n}a_{i}X^{i}$, $B=\sum_{i=1}^{m}b_{i}X^{i}$ avec $a_{n}\neq 0$, $b_{m}\neq0$.
	\begin{enumerate}
		\item Montrer que \function{\varphi}{\K_{m-1}[X]\times\K_{n-1}[X]}{\K_{n+m-1}[X]}{(U,V)}{AU+BV}
		est bijective si et seulement si $A\wedge B=1$.

		On note $M_{A,B}$ la matrice de $\varphi$ dans la base canonique de $\K_{m-1}[X]\times\K_{n-1}[X]$ et on définit le résultant $R_{A,B}=\det(M_{A,B})$.
		\item Pour $\K=\R$ ou $\C$, et $p\in\N$ fixé, on munit $\K_{p}[X]$ d'une norme quelconque. Montrer que \function{\Phi_{A,B}}{\K_{m-1}[X]\times\K_{n-1}[X]}{\K}{(A,B)}{R_{A,B}}
		est continue.
		\item En déduire que $\Delta=\{P\in\C_{p}[X]\bigm| P\text{ scindé à racines simples sur }\C\}$ est ouvert. Et sur $\R$ ?
	\end{enumerate}
\end{exercise}

\begin{exercise}
	Soit $F=\{M\in\M_{n}(\R)\bigm|M^{n}=0\}$. $F$ est donc l'ensemble des matrices nilpotentes.
	\begin{enumerate}
		\item Déterminer $\overline{F}$ et $\mathring{F}$.
		\item On munit $\M_{n}(\R)$ de $\Vert M\Vert=\sqrt{\Tr(M^{\mathsf{T}}M)}$. Vérifier que c'est une norme et calculer $d(I_{n},F)$.
	\end{enumerate}
\end{exercise}

\begin{exercise}
	\phantom{}
	\begin{enumerate}
		\item Soit $\K=\R$ ou $\C$. Montrer que $GL_{n}(\K)$ est un ouvert dense dans $M_{n}(\K)$.
		\item En déduire que pour tout $(A,B)\in\M_{n}(\K),\chi_{AB}=\chi_{BA}$.
	\end{enumerate}
\end{exercise}

\begin{exercise}
	Soit $E$ un $\K$-espace vectoriel de dimension finie, $u\in\L(E)$ telle que $(u^{p})_{p\in\N}$ est bornée (pour une norme quelconque). On pose $v_{p}=\frac{1}{p}\sum_{k=0}^{p-1}u^{p}$.
	\begin{enumerate}
		\item Montrer que 
		$$E=\ker(u-id_{E})\oplus\im(u-id_{E})$$
		On pourra évaluer $v_{p}\circ(id_{E}-u)=(id_{E}-u)$ et faire tendre $p$ vers $+\infty$.
		\item Montrer que $(v_{p})_{p\in\N}$ converge vers $\Pi$, le projecteur sur $\ker(u-id_{E})$ parallèlement à $\im(u-id_{E})$.
	\end{enumerate}
\end{exercise}

\begin{exercise}
	Soit $A$ compact convexe non vide d'un espace vectoriel normé, $f:A\to A$ 1-lipschitzienne.
	\begin{enumerate}
		\item Soit $x_{0}\in A$, et pour $n\geqslant1,\forall x\in A$, $f_{n}(x)=\frac{1}{n}f(x_{0})+(1-\frac{1}{n})f(x_{0})$. Montrer que $f$ possède un unique point fixe $x_{n}$.
		\item Montrer que $f$ possède au moins un point fixe.
		\item Si l'espace est euclidien, montrer que $F=\{x\in A\bigm| f(x)=x\}$ est convexe.
		\item Contre-exemple dans le cas général.
	\end{enumerate}
\end{exercise}

\begin{exercise}
	Soient $E$ et $F$ deux $\R$-espaces vectoriels normés avec $\dim(F)<+\infty$. Soit $f:E\to F$ continue telle qu'il existe $M\geqslant0$, pour tout $(x,y)\in E^{2}$, on a $\Vert f(x+y)-f(x)-f(y)\Vert\leqslant M$.
	\begin{enumerate}
		\item Si $M=0$, montrer que $f$ est linéaire (continue). Est-ce encore vrai si $\K=\C$ ?
		\item On suppose $M>0$, soit pour tout $n\in\N$, \function{v_n}{E}{F}{x}{\frac{1}{2^{n}}f(2^{n}x)}
		Montrer que pour tout $x\in E$, $(v_{n}(x))_{n\in\N}$ converge. On note $g(x)=\lim\limits_{n\to+\infty}v_{n}(x)$.
		\item Montrer que $g$ est l'unique application linéaire continue telle que $g-f$ soit bornée.
	\end{enumerate}
\end{exercise}

\begin{exercise}
	Soit $E$ un $\R$-espace vectoriel normé de dimension plus grande que 2 et $f:E\to\R$ continue telle que pour tout $t\in\R,f^{-1}(\{t\})$ est compact. Montrer que $f$ atteint son maximum ou son minimum sur $E$.
\end{exercise}

\begin{exercise}
	Soit $n\geqslant2$. Existe-t-il $f$ continue injective de $\R^{n}$ dans $\R$ ?
\end{exercise}

\begin{exercise}
	Soit $\varphi\colon l^{1}\to\R$ forme linéaire continue. On pose $K_{n}=\varphi(e_{n})\in\R$ où $e_{n}$ est la base canonique de $l^{1}$.
	\begin{enumerate}
		\item Montrer que $(K_{n})_{n\in\N}$ est bornée et que $\vertiii{\varphi}=\sup\limits_{n\in\N}\vert K_{n}\vert=\Vert(K_{n})_{n\in\N}\Vert_{\infty}$.
		\item Montrer que \function{F}{\L_c(l^{1},\R)}{l^{\infty}}{\varphi}{(\varphi(e_{n}))_{n\in\N}}
		est une isométrie bijective.
	\end{enumerate}
\end{exercise}

\begin{exercise}
	Soit $E$ un $\R$-espace vectoriel normé et $H$ un hyperplan de $E$.
	\begin{enumerate}
		\item Montrer que si $H$ est dense, alors $E\setminus H$ est connexe par arc.
		\item Et si $H$ est fermé ?
		\item Et pour un $\C$-espace vectoriel normé ?
	\end{enumerate}
\end{exercise}

\begin{exercise}
	Soit $\Gamma=\{(x,\sin(\frac{1}{x}))\bigm| x>0\}\subset\R^{2}$. Montrer que $\Gamma$ est connexe par arcs mais que $\overline{\Gamma}$ ne l'est pas.
\end{exercise}

\begin{exercise}
	Soit $K$ compact convexe non vide d'un espace vectoriel normé $E$. Soit $T\in\L_{c}(E)$ tel que $T(K)\subset K$.
	\begin{enumerate}
		\item Soit $a\in K$ et pour tout $n\in\N$, $u_{n}=\frac{1}{n+1}\sum_{k=0}^{n}T^{k}(a)$. Montrer que $T$ admet au moins un point fixe dans $K$.
		\item Soit $U\in\L_{c}(E)$ qui commute avec $T$ et tel que $U(K)\subset K$. Montrer que $U$ et $T$ ont un point fixe commun.
	\end{enumerate}
\end{exercise}

\begin{exercise}[Théorème de Carathéodory]
	Soit $E$ un $\R$-espace vectoriel normé de dimension $n$.
	\begin{enumerate}
		\item Soit $p\geqslant n+2$ et $(x_{1},\dots,x_{p})\in E^{p}$. Soit $(\lambda_{1},\dots,\lambda_{p})\in\R_{-}^{p}$ tel que $\sum_{i=1}^{p}\lambda_{i}=1$ et $x=\sum_{i=1}^{n}\lambda_{i}x_{i}$. Soit \function{u}{\R^p}{E}{(\alpha_{1},\dots,\alpha_{p})}{\sum_{i=1}^{p}\alpha_{i}x_{i}}
		Montrer que $\dim(\ker(u))\geqslant 2$. En déduire qu'il existe $(\alpha_{1},\dots,\alpha_{p})\in\R^{p}\setminus\{0,\dots,0\}$ tel que $\sum_{i=1}^{p}\alpha_{i}x_{i}=0$ et $\sum_{i=1}^{p}\alpha_{i}=0$.
		
		\item Montrer que pour tout $t\in\R$, $x=\sum_{i=1}^{p}(\lambda_{i}+t\alpha_{i})x_{i}$ et que $\sum_{i=1}^{p}\lambda_{i}+t\alpha_{i}=1$. Prouver que l'on peut choisir $t$ tel que $\min\limits_{1\leqslant i\leqslant p}(\lambda_{i}+t\alpha_{i})=0$.
		
		\item En déduire que $x$ est barycentre à coefficients positifs de $n+1$ éléments $(x_{i},\dots,x_{p})$.
		
		\item Soit $K$ un compact de $E$. Montrer que $\conv(K)$ est compact.
	\end{enumerate}
\end{exercise}

\begin{exercise}
	Soit $E$ un $\C$-espace vectoriel de dimension finie $r\in\N^{*}$. Soit $(\alpha_{1},\dots,\alpha_{r})\in\C^{r}$ distincts et $P=(X-\alpha_{1})\dots(X-\alpha_{r})$. Déterminer les composantes connexes par arcs de $A_{P}\in\{u\in\L(E)\bigm| P(u)=0\}$.
\end{exercise}

\begin{exercise}[Théorème de Perron-Frobenius]
	Soit $A=(a_{i,j})_{1\leqslant i,j\leqslant n}\in\M_{n}(\R)$ tel que pour tout $(i,j)\in\{1,\dots,n\}$, $a_{i,j}>0$. On note alors $A>0$, et on peut définir de même $A\geqslant0$. Pour $X\in\R^{n}$, on note $\Vert X\Vert_{1}=\sum_{i=1}^{n}\vert x_{i}\vert$. On pose, pour $X=(x_{1},\dots,x_{n})^{\mathsf{T}}$, $\vert X\vert=(\vert x_{1}\vert,\dots,\vert x_{n}\vert)^{\mathsf{T}}$. 
	On définit $\rho(A)=\max\limits_{\lambda\in\Sp_{\C}(A)}\vert\lambda\vert$ le rayon spectral de $A$.
	\begin{enumerate}
		\item Montrer que si $X\geqslant0$ et $X\neq0$, on a $AX>0$.
		\item Montrer que pour $X\in\M_{n,1}(\C)$, si $\vert AX\vert=A\vert X\vert$, alors il existe $\theta\in\R$, $e^{\mathrm{i}\theta}X\geqslant0$.
		\item On définit 
		$$K=\{X\in\M_{n,1}(\R)\Bigm| X\geqslant0\text{ et }\Vert X\Vert_{1}=1\}$$
		et pour tout $X\in K$, 
		$$I_{X}=\{t\geqslant0\Bigm|AX-tX\geqslant0\}$$
		Montrer que $I_{X}$ est non vide, fermé et borné. On pose $\theta(X)=\max(I_{X})$.
		\item Montrer que $\theta$ est borné sur $K$. On pose $r_{0}=\sup\limits_{x\in K}\theta(X)$. Établir qu'il existe $X^{+}\in K$ tel que $\theta(X^{+})=r_{0}$.
		\item Montrer que $AX^{+}=r_{0}X^{+}$. On pourra poser $Y=AX^{+}-r_{0}X^{+}$ et on montrera que si $Y\neq0$, il existe $\varepsilon>0$ tel que $A(A^{+})-(r_{0}+\varepsilon)AX^{+}>0$.
		\item Soit $\lambda\in\Sp_{\C}(A)$ et $V\in\M_{n,1}(\C)$ tel que $\Vert V\Vert_{1}=1$ et $AV=\lambda V$. Montrer que $\vert AV\vert\leqslant A\vert V\vert$, en déduire que $\vert\lambda\vert\leqslant r_{0}$.
		\item Montrer que si $\vert\lambda\vert=r_{0}$, alors $A\vert V\vert=r_{0}\vert V\vert=\vert AV\vert$, en déduire que $\lambda=r_{0}$.
		\item Montrer que $\dim(\ker(A-r_{0}I_{n}))=1$.
	\end{enumerate}
\end{exercise}

\begin{exercise}
	Soit $E$ un espace vectoriel normé, $U$ et $V$ deux compacts disjoints. Montrer qu'il existe $U'$ et $V'$ des ouverts disjoints tels que $U\subset U'$ et $V\subset V'$.
\end{exercise}

\begin{exercise}
	Soit $K$ un compact non vide d'un espace vectoriel normé $E$. Soit $f\colon K\to K$ tel que pour tout $x\neq y\in K^{2}$, $\Vert f(x)-v(y)\Vert<\Vert x-y\Vert$.
	\begin{enumerate}
		\item Montrer qu'il existe un unique $a\in K$ tel que $f(a)=a$.
		\item Soit $u_{0}\in K$ et pour tout $n\in\N$, $u_{n+1}=f(u_{n})$. Montrer que $\lim\limits_{n\to+\infty}u_{n}=a$.
		\item Étudier $f(x)=\sqrt{1+x^{2}}$ pour $x\in\R$.
	\end{enumerate}
\end{exercise}

\begin{exercise}
	Soient $K_{1},K_{2},K_{3}$ trois compacts non vides du plan tels qu'il n'existe pas de droite coupant $K_{1},K_{2}$ et $K_{3}$ simultanément. Montrer qu'il existe un cercle de rayon minimal les coupant tous les trois.
\end{exercise}

\begin{exercise}
	Soit $E=\mathcal{C}^{0}([0,1],\R)$ muni de $\Vert\cdot\Vert_{\infty}$. On définit pour tout $f\in E$, pour tout $x\in[0,1]$, 
	$$T(f)(x)=\int_{0}^{x}f(t)dt$$
	\begin{enumerate}
		\item Montrer que $T\in\L_{c}(E)$ et calculer $\vertiii{T}$.
		\item Montrer que $id_{E}-T$ est un homéomorphisme.
	\end{enumerate}
\end{exercise}

\begin{exercise}
	Soit $f:\R^{n}\to\R^{p}$ continue, montrer l'équivalence:
	\begin{enumerate}
		\item [(i)] $\lim\limits_{\Vert x\Vert\to+\infty}\Vert f(x)\Vert=+\infty$,
		\item [(ii)] pour tout compact $K$ de $\R^{p}$, $f^{-1}(K)$ est un compact de $\R^{n}$.
	\end{enumerate}
\end{exercise}

\begin{exercise}
	Soit $E$ un espace vectoriel normé et $K$ un compact non vide de $E$ et $f:K\to K$ tel que pour tout $(x,y)\in K^{2}$, $d(f(x),f(y))\geqslant d(x,y)$.
	\begin{enumerate}
		\item Montrer que pour tout $(x,y)\in K^{2}$, pour tout $\varepsilon>0$, il existe $p\in\N^{*}$,
		$$
		\left\{
			\begin{array}[]{l}
				d(x,f^{p}(x))<\varepsilon\\
				d(y,f^{p}(y))<\varepsilon
			\end{array}
		\right.
		$$
		On pourra former $(f^{n}(x),f^{n}(y))_{n\in\N}$.
		\item Montrer que $f$ est isométrie.
		\item Montrer que $f$ est surjective.
	\end{enumerate}
\end{exercise}

\begin{exercise}
	Soit $(A,B)\in(\R^{2})^{2}$ avec $A\neq B$, et $K$ un compact ne coupant pas $(AB)$. Soit 
	$$F=\{r\geqslant0,\text{ il existe un cercle de centre r, passant par A et B et rencontrant K}\}$$
	Montrer que $F$ est compact.
\end{exercise}

\begin{exercise}
	Soit $E=\R[X]$ et $\tau\in\L(E)\colon\tau(P)(X)=P(X+1)$.
	\begin{enumerate}
		\item Déterminer $\Sp(\tau)$.
		\item Vérifier que $\Vert P=\Vert=\sup\limits_{x\geqslant0}\vert P(x))e^{-x}\vert$ est une norme sur $E$.
		\item Montrer que $\tau$ est continue pour cette norme et vérifie $\vertiii{\tau}\leqslant e$.
		\item Calculer $\vertiii{\tau}$.
	\end{enumerate}
\end{exercise}

\begin{exercise}
	$E=\mathcal{C}^{0}([0,1],\R)$ muni de $\Vert\cdot\Vert_{\infty}$. Soit $\varphi\colon[0,1]\to[0,1]$ continue strictement croissante. Pour $f\in E$ et $x\in[0,1]$, soit 
	$$T(f)(x)=\int_{0}^{1}\min(x,\varphi(t))f(t)dt$$
	\begin{enumerate}
		\item $T$ définit-il un endomorphisme de $E$ ?
		\item Est-il continue ?
		\item Calculer $\vertiii{T}$.
	\end{enumerate}
\end{exercise}

\begin{exercise}
	Soit $E=\C[X]$ muni de $\Vert\sum_{k\in\N} a_{k}X^{k}\Vert_{\infty}=\max\limits_{k\in\N}\vert a_{k}\vert$. Soit \function{\varphi}{E}{\C}{\sum_{k\in\N}a_{k}X^{k}}{\sum_{k\in\N}\frac{a_k}{2^{k}}}
	\phantom{}
	\begin{enumerate}
		\item Montrer que $\ker(\varphi)$ est fermé.
		\item Soit $P=\sum_{k\in\N}a_{k}X^{k}\in\ker(\varphi)$. Montrer que $\Vert P-1\Vert_{\infty}>\frac{1}{2}$.
		\item Évaluer $d(1,\ker(\varphi))$. Cette distance est-elle atteinte ?
	\end{enumerate}
\end{exercise}

\begin{exercise}
	Soit $K$ un compact non vide de $\R^{n}$ (muni de $\Vert\cdot\Vert_{2}$). Montrer qu'il existe une unique boule fermée de rayon minimal contenant $K$.
\end{exercise}

\begin{exercise}
	Soit $E=\mathcal{C}^{0}([0,1],\R)$ muni de $\Vert\cdot\Vert_{\infty}$ et \function{\varphi}{E}{\R}{f}{\int_{0}^{\frac{1}{2}}f-\int_{\frac{1}{2}}^{1}f}
	Montrer que $\varphi$ est une forme linéaire continue. Calculer $\vertiii{\varphi}$. Est-elle atteinte ?
\end{exercise}

\begin{exercise}
	Soit $(E,\Vert\cdot\Vert)$ un espace vectoriel normé et $(u,v)\in\L(E)$. On suppose que $u\circ v-v\circ u=id$.
	\begin{enumerate}
		\item Cette hypothèse sur $u$ et $v$ est-elle possible en dimension finie ?
		\item Montrer que pour tout $n\in\N$, $u\circ v^{n+1}-v^{n+1}\circ u=(n+1)v^{n}$.
		\item En utilisant la norme, mettre en évidente une contradiction.
		\item Soit $E=\R[X]$ et \function{T}{E}{E}{P}{XP(X)} et \function{D}{E}{E}{P}{P'}
		Montrer que $T$ et $D$ ne sont pas simultanément continues pour aucune norme.
	\end{enumerate}
\end{exercise}

\begin{exercise}
	Soit $\Vert\cdot\Vert$ une norme sur $\C^{n}$ et $\vertiii{\cdot}$ la norme subordonnée sur $\M_{n}(\C)$. Soit $(\alpha,\beta)\in[0,1[^{2}$, $A\neq I_{n}$ et $B\in\M_{n}(\C)$ tel que
	$$
	\left\{
		\begin{array}[]{l}
			\vertiii{A-I_{n}}\leqslant\alpha\\
			\vertiii{B-I_{n}}\leqslant\beta
		\end{array}
	\right.
	$$
	\begin{enumerate}
		\item Montrer que $A$ et $B$ sont inversibles et que 
		$$\vertiii{ABA^{-1}B^{-1}-I_{n}}\leqslant\frac{2\vertiii{A-I_{n}}\vertiii{B-I_{n}}}{(1-\alpha)(1-\beta)}$$
		\item Montrer que si $\alpha$ et $\beta$ sont suffisamment petits, 
		$$\vertiii{ABA^{-1}B^{-1}-I_{n}}<\vertiii{A-I_{n}}$$
		\item Soit $G=gr\{A,B\}$ (sous-groupe de $GL_{n}(\C)$ engendré par $A$ et $B$). Montrer que si $G$ est discret, alors il existe $C\in G\setminus\{I_{n}\}$, qui commute avec toutes les matrices de $G$.
	\end{enumerate}
\end{exercise}

\begin{exercise}
	\phantom{}
	\begin{enumerate}
		\item Soit $A\in\M_{n}(\C)$, montrer que $\exp(A)\in\C_{n-1}[A]$ : il existe $P\in\C_{n-1}[X]$ tel que $\exp(A)=P(A)$.
		\item Montrer que $A$ est diagonalisable sur $\C$ si et seulement si $\exp(A)$ l'est.
		\item Résoudre $\exp(A)=I_{n}$.
		\item Le résultat de la question 2 est-il valable sur $\R$ ?
	\end{enumerate}
\end{exercise}

\begin{exercise}
	On pose, pour $n\geqslant1$,
	$$
	\left\{
		\begin{array}[]{l}
			P(X)=X-\frac{X^{2}}{2}+\frac{X^{3}}{3}+\dots+(-1)^{n}\frac{X^{n-1}}{n-1}\\
			Q(Y)=1+Y+\frac{Y^{2}}{2}+\dots+\frac{Y^{n-1}}{(n-1)!}
		\end{array}
	\right.
	$$
	\begin{enumerate}
		\item Montrer qu'il existe $A\in\R[X]$ tel que $Q(P(X))=1+X+X^{n}A(X)$. On pourra écrire les développements limités à l'ordre 1 de $\exp$ et $\ln$.
		\item Soit $N\in\M_{n}(\C)$ nilpotente, montrer que $\exp(P(N))=Q(P(N))=I_{n}+N$.
		\item En déduire que $\exp\colon\M_{n}(\C)\to GL_{n}(\C)$ est surjective.
	\end{enumerate}
\end{exercise}

\cleardoublepage
\section{Fonction d'une variable réelle}

\begin{exercise}
	Soit $\varphi:I=[a,b]\to\R$ continue. On lui associe 
	\function{\omega_{\varphi}}{\R_+^*}{\R}{h}{\sup\{\vert\varphi(x)-\varphi(y)\vert\bigm| (x,y)\in I^2\text{ et }\vert x-y\vert <h\}}
	\begin{enumerate}
		\item Montrer que $\omega_{\varphi}$ est définie et croissante.
		\item Soit $(h,h')\in(\R_{+}^{*})^{2}$, montrer que $\omega_{\varphi}(h+h')\leqslant\omega_{\varphi}(h)+\omega_{\varphi}(h')$.
		\item Soit $(h,\lambda)\in(\R_{+}^{*})^{2}$ et $n\in\N^{*}$, montrer que $\omega_{\varphi}(nh)\leqslant n\omega_{\varphi}(h)$ et $\omega_{\varphi}(\lambda h)\leqslant (1+\lambda)\omega_{\varphi}(h)$.
		\item Montrer que $\lim\limits_{h\to 0}\omega_{\varphi}(h)=0$. En déduire que $\omega_{\varphi}$ est continue.
	\end{enumerate}
\end{exercise}

\cleardoublepage
\section{Suites et séries de fonctions}
\cleardoublepage
\section{Séries entières}
\cleardoublepage
\section{Intégration}
\cleardoublepage
\section{Espaces préhilbertiens}
\cleardoublepage
\section{Espaces euclidiens}
\cleardoublepage
\section{Calcul différentiel}
\cleardoublepage
\section{\'Equation différentielles linéaires}

\end{document}