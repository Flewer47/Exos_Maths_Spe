\section{Séries numériques et familles sommables}

\begin{exercise}
	Soit la suite définie par $a_{0}=1$ et pour tout $n\geqslant1$,
	$$a_{n}=2a_{\lfloor n/3\rfloor}+3a_{\lfloor n/9\rfloor}$$
	\begin{enumerate}
		\item
		On pose pour $p\in\N$, $b_{p}=a_{3^p}$. Calculer $b_{p}$ en fonction de
		$p$.
		\item
		Montrer que si $3^{p}\leqslant n<3^{p+1}$, alors $a_{n}=b_{p}$.
		\item
		Déterminer l'ensemble des valeurs d'adhérence de
		$(\frac{a_{n}}{n})_{n\geqslant 2}$.
	\end{enumerate}
\end{exercise}

\begin{exercise}
	Soit $[a,b]\subset\in\R$ avec $a<b$ et $f:[a,b]\to[a,b]$ continue. Soit
	$x_{0}\in[a,b]$ et pour tout $n\in\N$, $x_{n+1}=f(x_{n})$.
	\begin{enumerate}
		\item
		Montrer que $f$ admet au moins un point fixe $l\in[a,b]$.
		\item
		Si $\lim\limits_{n\to+\infty}x_{n+1}-x_{n}=0$, montrer que l'ensemble des
		valeurs d'adhérence de $(x_{n})_{n\in\N}$ est un segment.
		\item
		En déduire que $(x_{n})_{n\in\N}$ converge si et seulement si
		$\lim\limits_{n\to+\infty}x_{n+1}-x_{n}=0$.
	\end{enumerate}
\end{exercise}

\begin{exercise}
	Soit $\theta\in[0,2\pi[$, on définit $u_{0}=e^{\mathrm{i}\theta}$ et pour tout
	$n\in\N$, $u_{n+1}=u_{n}^{2}$. Peut-on avoir $(u_{n})_{n\in\N}$
	\begin{itemize}
		\item
		stationnaire?
		\item
		convergente?
		\item
		périodique?
		\item
		dense dans $\U$?
	\end{itemize}
	On pourra étudier le développement binaire de
	$\frac{\theta}{2\pi}=\sum_{k=1}^{+\infty}\frac{a_{k}}{2^{k}}$.
\end{exercise}

\begin{exercise}
	Soit $(a,b)\in\R_{+}^{2}$, étudier
	$u_{n}=\Bigl(\frac{\sqrt[n]{a}+\sqrt[n]{b}}{2}\Bigr)^{n^{2}}$.
\end{exercise}

\begin{exercise}
	Soit $(x_{n})_{n\in\N}\in\R_{+}^{\N}$ telle que $\lim\limits_{n\to+\infty}=0$
	et $\sum_{n=0}^{+\infty}x_{n}=+\infty$.
	\begin{enumerate}
		\item
		Montrer qu'il existe $\varphi:\N\to\N$ bijective telle que
		$(x_{\varphi(n)})$ est décroissante.
		\item
		Montrer que pour tout $l\in\overline{\R_{+}}$, pour tout $\varepsilon>0$,
		il existe un sous-ensemble $I\subset\N$ fini tel que
		$$\Bigl\vert\sum_{k\in I}x_{k}-l\Bigr\vert\leqslant\varepsilon$$ ou si
		$l=+\infty$: $\forall A>0$, il existe un sous-ensemble $I$ fini tel que
		$\sum_{k\in I}x_{k}\geqslant A$.
	\end{enumerate}
\end{exercise}

\begin{exercise}
	Soit $(u_{n})_{n\in\N}\in\R_{+}^{\N}$ telle que
	$\lim\limits_{n\to+\infty}u_{n}\times\sum_{k=0}^{n}u_{k}^{2}=1$. Montrer que
	$u_{n}\sim\frac{1}{\sqrt[3]{3n}}$. Une telle suite existe-t-elle?
\end{exercise}

\begin{exercise}
	Étudier $x_{n}=n-\sum_{k=1}^{n}\cosh(\frac{1}{\sqrt[]{k+n}})$.
\end{exercise}

\begin{exercise}
	Soit $(a_{n})_{n\in\N},(b_{n})_{n\in\N},(c_{n})_{n\in\N}$ des suites réelles
	telles que 
	\begin{enumerate}
		\item
		[(i)] $\lim\limits_{n\to+\infty}a_{n}+b_{n}+c_{n}=0$,
		\item [(ii)]
		$\lim\limits_{n\to+\infty}e^{a_{n}}+e^{b_{n}}+e^{c_{n}}=3$.
	\end{enumerate}
	Montrer que
	$\lim\limits_{n\to+\infty}a_{n}=\lim\limits_{n\to+\infty}b_{n}=\lim\limits_{n\to+\infty}c_{n}=0$.
	On pourra étudier $\varphi:x\mapsto e^{x}-x-1$.
\end{exercise}

\begin{exercise}
	Soit $u_{0}\in]0,1[$ et pour $n\in\N$, $u_{n+1}=u_{n}-u_{n}^{2}$. On pose
	$v_{n}=\frac{1}{u_{n}}$.
	\begin{enumerate}
		\item
		Montrer que $(v_{n})_{n\in\N}$ est bien définie.
		\item
		Montrer que $v_{n}=n+\ln(n)+O(1)$, en déduire un développement de $u_{n}$.
	\end{enumerate}
\end{exercise}

\begin{exercise}
	\phantom{}
	\begin{enumerate}
		\item
		Montrer que pour tout $n\geqslant 2$, il existe un unique $u_{n}\in\R_{+}$
		tel que $u_{n}^{n}=u_{n}+n$.
		\item
		Montrer que $(u_{n})_{n\geqslant 2}$ converge vers $\lambda\in\R_{+}$.
		\item
		Donner un développement asymptotique à deux termes de $x_{n}-\lambda$.
	\end{enumerate}
\end{exercise}

\begin{exercise}
	Soit $(u_{n})_{n\in\N}$ une suite de réels positifs non tous nuls. On suppose
	que
	
	$$u_n=o\Biggl(\sum_{k=0}^{n}u_{k}\Biggr)$$ Soit $(a_{n})_{n\in\N}\in\C^{\N}$
	de limite $a$. En cas d'existence, évaluer
	$$\lim\limits_{n\to+\infty}\frac{u_{n}a_{0}+u_{n-1}a_{1}+\dots+u_{0}a_{n}}{u_{0}+\dots+u_{n}}$$
\end{exercise}

\begin{exercise}
	\phantom{}
	\begin{enumerate}
		\item
		Soit $x\in[0,1[$, montrer qu'il existe une unique suite
		$(a_{n})_{n\geqslant 2}$ d'entiers naturels telle que 
		\begin{enumerate}
			\item
			[(i)] $0\leqslant a_{n}\leqslant n-1$ pour tout $n\geqslant2$,
			\item
			[(ii)] il existe $m\geqslant n$ tel que $a_{m}<m-1$ pour tout
			$n\geqslant2$,
			\item
			[(iii)] $x=\sum_{n=2}^{+\infty}\frac{a_{n}}{n!}$.
		\end{enumerate}
		\item
		Donner une condition nécessaire et suffisante sur $(a_{n})_{n\geqslant2}$
		pour que $x\in\Q$.
		\item
		Soit $l\in[-1,1]$, montrer qu'il existe $x\in[0,1[$ tel que
		$\lim\limits_{n\to+\infty}sin(n!2\pi x)=l$.
	\end{enumerate}
\end{exercise}

\begin{exercise}
	Soit $u_0>0,u_1>0$ et pour tout $n\geqslant 1$,
	$$u_{n+1}=\ln(1+u_{n})+\ln(1+u_{n-1})$$ Étudier la suite $(u_{n})$. On pourra
	poser $M_{n}=\max(u_{n},u_{n-1},l)$, $m_{n}=\min(u_{n},u_{n-1},l)$ où
	$l=2\ln(1+l)$ et $l>0$.
\end{exercise}

\begin{exercise}
	Soit $(p,q)\in(\R^{*})^{2}$ avec $\frac{p}{q}\in\R\setminus\Q$. Soit
	$(x_{n})_{n\in\N}$ une suite réelle bornée. On suppose que
	$(e^{\mathrm{i}px_{n}})_{n\in\N}$ et $(e^{\mathrm{i}qx_{n}})_{n\in\N}$
	convergent. Montrer que $(x_{n})_{n\in\N}$ converge. Et si $(x_{n})_{n\in\N}$
	n'est pas bornée ?
\end{exercise}

\begin{exercise}
	\phantom{}
	\begin{enumerate}
		\item
		Montrer que pour tout $n\geqslant1$, pour tout $k\in\{0,\dots,n\}$,
		$\binom{n}{k}\leqslant\frac{n^{k}}{k!}$.
		\item
		Soit $z\in\C$, montrer que 
		$$\Biggl\vert\sum_{k=0}^{n}\frac{z^{k}}{k!}-\Bigl(1+\frac{z}{n}^{n}\Bigr)\Biggr\vert\leqslant\sum_{k=0}^{n}\frac{\vert
		z\vert^{k}}{k!}-\Bigl(1+\frac{\vert z\vert}{n}\Bigr)^{n}$$
		\item
		En déduire $\lim\limits_{n\to+\infty}\Bigl(1+\frac{z}{n}\Bigr)^{n}$.
	\end{enumerate}
\end{exercise}

\begin{exercise}
	Soit $u_{n}=\prod_{k=2}^{n}\frac{\sqrt{k}-1}{\sqrt{k}+1}$ pour $n\geqslant 2$.
	Quelle est la limite de cette suite? Quelle est la nature de la série
	$\sum_{n\geqslant 2}u_{n}^{\alpha}$ pour $\alpha\in\R$?
\end{exercise}

\begin{exercise}
	Soit $(u_{n})_{n\in\N}\in\R_{+}^{\N}$ décroissante de limite nulle. Montrer
	que si $\sum u_{n}$ converge, alors $u_{n}=o\Bigl(\frac{1}{n}\Bigr)$. On
	pourra minorer $u_{n+1}+\dots+u_{2n}$. Montrer ensuite que si $\{p\in\N,
	pu_{p}\geqslant1\}$ est infini, alors $\sum u_{n}$ diverge.
\end{exercise}

\begin{exercise}
	Nature de $\sum u_{n}$ où $u_{n}=$
	\begin{enumerate}
		\item
		$n^{-1-\frac{1}{n}}$
		\item
		$\int_{0}^{\frac{\pi}{2}}t^{n}\sin(t)dt$
		\item
		$\sin\left(2\pi\frac{n!}{e}\right)$
		\item
		$\dfrac{(-1)^{n}}{n^{\alpha}+(-1)^{n}\ln(n)}$ où $\alpha\in\R$
	\end{enumerate}
\end{exercise}

\begin{exercise}
	Montrer la convergence et calculer la somme des différentes séries suivantes:
	\begin{enumerate}
		\item
		$\sum_{n\geqslant1}\sum_{k\geqslant n}\frac{(-1)^{k}}{k}$
		\item
		$\sum_{n\geqslant0}\frac{1}{(3n)!}$
		\item
		$\sum_{n\geqslant1}\frac{E\left(n^{\frac{1}{3}}\right)-E\left(\left(n-1\right)^{\frac{1}{3}}\right)}{4n-n^{\frac{1}{3}}}$
		où $E$ désigne la partie entière.
	\end{enumerate}
\end{exercise}

\begin{exercise}
	Soit $f:[1,+\infty[\to\R_{+}^{*}$ de classe $\mathcal{C}^{2}$ et telle que
	$\lim\limits_{x\to+\infty}\frac{f'(x)}{f(x)}=a<0$. Montrer la convergence de
	$\sum_{n\geqslant1}f(n)$. Donner un équivalent de $R_{n}=\sum_{k=n}^{+\infty}
	f(k)$.
\end{exercise}

\begin{exercise}
	Donner un équivalent de $S_{n}=\sum_{k=1}^{n}\frac{e^{k}}{k}$.
\end{exercise}

\begin{exercise}
	Donner la nature de $\sum u_{n}$ quand $u_{n}$ vaut
	\begin{enumerate}
		\item
		$\Bigl(1-\frac{1}{n}\Bigr)^{n^{\alpha}}$ où $\alpha\in\R$
		\item
		$\frac{1}{\sum_{k=1}^{n}\bigl(\frac{1}{k}\bigr)^{\frac{1}{k}}}$
		\item
		$\frac{\sin(n!\pi e)}{\ln(n)}$
	\end{enumerate}
\end{exercise}

\begin{exercise}
	Montrer la convergence et calculer la somme de $\sum u_{n}$ où $u_{n}$ vaut
	\begin{enumerate}
		\item
		$a\ln(n)+b\ln(n+1)+c\ln(n+2)$ pour $n\geqslant1$.
		\item
		$\frac{2^{n}}{3^{2^{n-1}}+1}$ pour $n\geqslant 1$.
		\item
		$\frac{n-k\lfloor\frac{n}{k}\rfloor}{n(n+1)}$ avec $k\in\N^{*}$ est fixé.
		\item
		$\arctan(\frac{1}{n^{2}+n+1})$ pour $n\geqslant0$.
	\end{enumerate}
\end{exercise}

\begin{exercise}
	Soit $(u_{n})_{n\geqslant1}\in\R^{\N}$ et $v_{n}=n(u_{n}-u_{n+1})$. Montrer
	que $\sum u_{n}$ et $\sum v_{n}$ ont même nature lorsque
	\begin{enumerate}
		\item
		[(i)] $(nu_{n})_{n\geqslant1}$ converge vers 0 OU
		\item
		[(ii)] $(u_{n})_{n\geqslant 1}$ décroît et tend vers 0.
	\end{enumerate}
	Comparer alors les sommes respectives. En déduire, pour $p\geqslant1$ fixé,
	$$\sum_{n=1}^{+\infty}\frac{1}{n(n+1)\dots(n+p)}$$
\end{exercise}

\begin{exercise}
	Soit $q\geqslant2$ et $v_{n}=\frac{1}{(n+q)!}\sum_{k=1}^{n}k!$. Donner la nature de
	$\sum v_{n}$. En cas de divergence, donner un équivalent des sommes
	partielles.
\end{exercise}

\begin{exercise}
	Soit $(a,b,c)\in (\N^{*})^{3}$, $z\in\C$, $\vert z\vert<1$. Montrer, en
	justifiant l'existence:
	$$\sum_{n=0}^{+\infty}\frac{z^{nb}}{1+z^{na+c}}=\sum_{n=0}^{+\infty}\frac{(-1)^{n}z^{nc}}{1-z^{na+b}}$$
\end{exercise}

\begin{exercise}
	Soit $\sum_{n\geqslant1} a_{n}$ une série complexe absolument convergente. On
	pose pour $q\in\N^{*}$, $b_q=\frac{1}{q(q+1)}(a_{1}+2a_{2}+\dots+qa_{q})$.
	Montrer que $\sum_{q\geqslant1}b_{q}$ converge et évaluer sa somme en fonction
	de $\sum_{n=1}^{+\infty}a_{n}$. On pourra poser
	$u_{n,q}=\frac{na_{n}}{q(q+1)}$ si $n\leqslant q$ et 0 sinon.
\end{exercise}

\begin{exercise}[Inégalité de Carleman]
	Soit $(u_{n})_{n\geqslant1}\in\R_{+}^{\N}$ telle que $\sum u_{n}<+\infty$. On
	pose $v_{n}=\frac{1}{n(n+1)}(u_{1}+\dots+nu_{n})$ et $w_{n}=\sqrt[n]{u_1\times
	u_2\times\dots\times u_n}$. On admet que pour tout $n\in\N^{*}$, pour tout
	$(a_{1},\dots,a_{n})\in\R_{+}^{n}$, on a l'inégalité entre la moyenne
	géométrique et arithmétique:
	$$\sqrt[n]{a_{1}\dots a_{n}}\leqslant\frac{1}{n}(a_{1}+\dots+a_{n})$$ avec
	égalité si et seulement si $a_{1}=\dots=a_{n}$.

	Montrer que $\sum w_{n}$ converge et que $\sum_{n=1}^{+\infty}w_{n}\leqslant
	e\sum_{n=1}^{+\infty}u_{n}$. On pourra utiliser l'exercice précédent. Montrer
	que $e$ est la "meilleure" constante possible, c'est-à-dire que si $\forall
	(u_{n})_{n\geqslant1}\in(\R_{+}^{*})^{\N^{*}}$ telle que $\sum u_{n}$
	converge, on a $\sum w_{n}\leqslant C\sum u_{n}$ alors $C\geqslant e$.
\end{exercise}

\begin{exercise}
	\phantom{}
	\begin{enumerate}
		\item
		Trouver une condition nécessaire et suffisante sur $\alpha\in\R$ pour que
		$\Bigl(\frac{1}{(p+q)^{\alpha}}\Bigr)_{(p,q)\in\N^{2}\setminus\{(0,0)\}}$
		soit sommable et exprimer alors la somme en fonction de la fonction
		$\zeta$ de Riemann.
		\item
		Trouver une condition nécessaire et suffisante sur $\alpha\in\R$ pour que
		$\Bigl(\frac{1}{(p^{2}+q^{2})^{\alpha}}\Bigr)_{(p,q)\in\N^{2}\setminus\{(0,0)\}}$
		soit sommable.
	\end{enumerate}
\end{exercise}

\begin{exercise}
	Étudier la sommabilité de
	$\Bigl(\frac{1}{(m+n^{2})(m+n^{2}+1)}\Bigr)_{(m,n)\in\N^{2}}$.\\
	En déduire la valeur de $\sum_{n=1}^{+\infty}\frac{E(\sqrt{n})}{n(n+1)}$.
\end{exercise}

\begin{exercise}
	\phantom{}
	\begin{enumerate}
		\item
		Montrer que pour tout $s\in]1,+\infty[$, le produit infini
		$\prod_{k=1}^{+\infty}\frac{1}{1-\frac{1}{p_{k}^{s}}}$ converge (où les
		$p_{k}$ sont les nombres premiers). Donner sa valeur en fonction de
		$\zeta(s)$.
		\item
		Généraliser ce résultat à $s\in\C$ avec $\Re(s)>1$.
	\end{enumerate}
\end{exercise}

\begin{exercise}
	On note $\varphi(n)=\vert\{k\in\{1,\dots,n\},~k\wedge n=1\}\vert$ (fonction
	d'Euler). Pour quelles valeurs de $\alpha\in\R$ la somme $\sum
	\frac{\varphi(n)}{n^{\alpha}}$ converge-t-elle? Donner alors sa somme en
	fonction de $\zeta(\alpha)$.
\end{exercise}

\begin{exercise}
	Soit $(z_{n})_{n\in\N}\in(\C^{*})^{\N}$ telle que pour tout $n\neq m$, $\vert
	z_{n}-z_{m}\vert\geqslant1$. Montrer que $\sum_{n\in\N}\frac{1}{z_{n}^{3}}$
	converge.
\end{exercise}

\begin{exercise}
	Donner la nature de $\sum_{n\geqslant1}\frac{(-1)^{E(\sqrt{n})}}{n}$.
\end{exercise}

\begin{exercise}
	Pour $(a,b)\in(\R\setminus\Z^{*})$, on définit
	$u_{n}=\frac{a(a+1)\dots(a+n)}{b(b+1)\dots(b+n)}$....
	\begin{enumerate}
		\item
		Donner une condition nécessaire et suffisante pour que $\sum u_{n}$
		converge.
		\item
		Dans ce cas, calculer sa somme.
		\item
		Faire le cas où $a=-\frac{1}{2}$ et $b=1$.
	\end{enumerate}
\end{exercise}

\begin{exercise}
	Soit $u_{n}=\frac{\ln(n)}{n}$ et $v_{n}=(-1)^{n}u_{n}$ pour $n\geqslant1$.
	\begin{enumerate}
		\item
		Donner la nature de $\sum u_{n}$ et $\sum v_{n}$.
		\item
		Soit $S_{N}=\sum_{n=1}^{N}u_{n}$. Donner un équivalent de $S_{N}$ puis
		développer jusqu'au o(1).
		\item
		Exprimer $\sum_{n=2}^{+\infty}v_{n}$ en fonction de $\gamma$ (constante
		d'Euler) et $\ln(2)$.
	\end{enumerate}
\end{exercise}

\begin{exercise}
	Soit pour $n\in\N^{*}$, $q_{1}(n)$ la nombre de chiffres de l'écriture
	décimale de $n$. On définit par récurrence $q_{k+1}(n)=q_{1}(q_{k}(n))$.
	Étudier la convergence de 
	$$\sum_{n\geqslant1}\frac{1}{nq_{1}(n)q_{2}(n)\dots q_{n}(n)}$$
\end{exercise}

\begin{exercise}
	Soit $P_{n}(X)=\sum_{k=0}^{n}\frac{X^{k}}{k!}$.
	\begin{enumerate}
		\item
		Montrer que pour tout $n\in\N$, $P_{2n}>0$ sur $\R$ et $P_{2n+1}$ s'annule
		une seule fois en $a_{2n+1}<0$.
		\item
		Déterminer $\lim\limits_{n\to+\infty}a_{2n+1}$.
	\end{enumerate}
\end{exercise}

\begin{exercise}
	Montrer qu'il existe un unique $x_{n}\geqslant0$ tel que $e^{x_{n}}=x_{n}+n$.
	Donne un développement asymptotique à deux termes de $x_{n}$ pour
	$n\geqslant1$.
\end{exercise}

\begin{exercise}
	Soit $(u_{n})_{n\in\N}\in(\R_{+}^{*})^{\N}$, on pose
	$S_{n}=\sum_{k=0}^{n}u_{k}$. Soit $\alpha\in\R$ et
	$v_{n}=\frac{u_{n}}{S_{n}^{\alpha}}$.
	\begin{enumerate}
		\item
		On suppose que $\sum u_{n}$ converge, étudier $\sum v_{n}$.
		\item
		On suppose que $\sum u_{n}$ diverge. Pour $\alpha=1$, montrer que pour
		tout $(n,p)\in\N^{2}$, $v_{n+1}+\dots+v_{n+p}\geqslant
		1-\frac{S_{n}}{S_{n+p}}$. En déduire que $\sum v_{n}$ diverge.
		\item
		On suppose que $\sum u_{n}$ diverge. Pour $\alpha>1$, on forme
		$w_{n}=\int_{S_{n-1}}^{S_{n}}\frac{dt}{t^{\alpha}}$. Montrer que $\sum
		v_{n}$ converge. Et si $\alpha<1?$
		\item
		On suppose que $\sum u_{n}$ converge. On pose
		$R_{n}=\sum_{k=n}^{+\infty}u_{k}$ et $w_{n}=\frac{u_{n}}{R_{n}^{\alpha}}$.
		Étudier la nature de $\sum w_{n}$.
	\end{enumerate}
\end{exercise}

\begin{exercise}[Principe des tiroirs de Dirichlet]
	Soit $x\in\R\setminus\Q$.
	\begin{enumerate}
		\item
		Soit $n\in\N^{*}$, montrer qu'il existe $(p,q)\in\Z\times\{1,\dots,n\}$
		tel que $\bigl\vert x-\frac{p}{q}\bigr\vert<\frac{1}{qn}$. On pourra
		étudier les $n+1$ réels $(kx-\lfloor kx\rfloor)=(x_{k})_{0\leqslant
		k\leqslant n}$ et montrer qu'il existe $k\neq k'$ avec $\vert
		x_{k}-x_{k'}\vert<\frac{1}{n}$.
		\item
		Montrer qu'il existe $(p_{n},q_{n})_{n\in\N}\in\Z^{\N}\times(\N^{*})^{\N}$
		telles que $\bigl\vert
		x-\frac{p_{n}}{q_{n}}\bigr\vert<\frac{1}{q_{n}^{2}}$ et
		$\lim\limits_{n\to+\infty}q_{n}=+\infty$.
		\item
		Étudier la convergence de la suite
		$\Bigl(\frac{1}{n\sin(n)}\Bigr)_{n\geqslant1}$ (on admet que
		$\pi\notin\R\setminus\Q)$.
	\end{enumerate}
\end{exercise}

\begin{exercise}
	Soit $(a_{n,p})\in\C^{(\N^{*})^{2}}$ telle que 
	\begin{enumerate}
		\item
		[(i)] pour tout $p\in\N^{*}$, il existe
		$\lim\limits_{n\to+\infty}a_{n,p}=a_{p}\in\C$,
		\item
		[(ii)] il existe une suite de réels positifs $(b_{p})$ donc la série
		converge telle que pour tout $(n,p)\in(\N^{*})^{2}$, $\lvert
		a_{n,p}\rvert\leqslant b_{p}$.
		
	\end{enumerate}
	\begin{enumerate}
		\item
		Évaluer $\lim\limits_{n\to+\infty}\sum_{p=1}^{n}a_{n,p}$.
		\item
		Calculer
		$\lim\limits_{n\to+\infty}\Bigl(\bigl(\frac{1}{n}\bigr)^{n}+\bigl(\frac{2}{n}\bigr)^{n}+\dots+\bigl(\frac{n-1}{n}\bigr)^{n}\Bigr)$.
	\end{enumerate}
\end{exercise}

\begin{exercise}
	Soit $\sum_{n\geqslant1}u_{n}$ une série complexe absolument convergente.
	\begin{enumerate}
		\item
		Montrer que pour tout $k\geqslant1$, on peut définir
		$S_{k}=\sum_{n=1}^{+\infty}u_{kn}$.
		\item
		On suppose que pour tout $k\geqslant1$, $S_{k}=0$. Montrer que pour tout
		$n\geqslant1$, $u_{n}=0$.
	\end{enumerate}
\end{exercise}

\begin{exercise}
	Soit $f:\R\to\R$ telle que pour toute suite $(u_{n})_{n\in\N}\in\R^{\N}$, si
$\sum u_{n}$ converge, alors $\sum f(u_{n})$ converge.
\begin{enumerate}
	\item
	Montrer que $f(0)=0$ et que $f$ est continue en 0.
	\item
	Montrer qu'il existe $\alpha>0$, $\forall x\in]-\alpha,\alpha[$, $f(x)=-f(x)$
	($f$ est impaire au voisinage de 0).
	\item
	Montrer qu'il existe $\beta>0$ $\forall(x,y)\in]-\beta,\beta[^{2}$,
	$f(x+y)=f(x)+f(y)$ ($f$ est linéaire au voisinage de 0).
	\item
	Montrer qu'il existe $\lambda\in\R$ et $\gamma>0$ tels que $\forall
	x\in]-\gamma,\gamma[$, $f(x)=\lambda x$ ($f$ est une homothétie au voisinage
	de 0).
\end{enumerate}
\end{exercise}