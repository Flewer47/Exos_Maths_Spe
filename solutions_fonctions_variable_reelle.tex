\section{Fonction d'une variable réelle}

\begin{solution}
	Tout d'abord, $\deg(L_{n})=n$ et  son coefficient dominant et $\frac{(2n)!}{2^{n}(n!)^{2}}$.
	\begin{enumerate}
		\item Soit $f\in\mathcal{C}^{0}([0,1],\R)$. $-1$ et $1$ sont racines d'ordre $n$ de $P_{n}$ donc pour tout $k\in\{0,\dots,n-1\}$ $P_{n}^{(k)}(-1)=P_{n}^{(k)}(-1)=0$. Ainsi, on a par intégrations par parties successives:
		$$(f|L_{n})=(-1)^{n}\int_{-1}^{1}f^{(n)}(t)P_{n}(t)dt$$
		Notamment, si $P\in\R_{n-1}[X]$, $P^{(n)}=0$ et $(P|L_{n})=0$. En particulier, pour tout $m<n$, $\deg(L_{m})\leqslant n-1$ et $(L_{m}|L_{n})=0$ donc $(L_{n})_{n\in\N}$ est orthogonale. Notons dès maintenant que l'on peut calculer la norme de $L_{n}$ grâce aux intégrales de Wallis:
		
		\begin{align*}
			\Vert L_{n}\Vert_{2}^{2}
			&=(L_{n}|L_{n})\\
			&=(-1)^{n}\int_{-1}^{1}L_{n}^{(n)}(t^{2}-1)^{n}dt\\
			&=\frac{(2n)!}{2^{2n}(n!)^{2}}\int_{-1}^{1}(1-t^{2})^{n}dt
		\end{align*}
		On pose $t=\cos(\theta)$ d'où $dt=-\sin(\theta)d\theta$, d'où
		\begin{align*}
			\int_{-1}^{1}(1-t^{2})^{n}dt
			&=\int_{0}^{\pi}\sin(\theta)^{2n+1}d\theta\\
			&=2I_{2n+1}\text{ [Wallis] }
		\end{align*}
		On a classiquement $I_{n+2}=\frac{n+1}{n+2}I_{n}$.
		D'où
		\begin{align*}
			I_{2n+1}
			&=\frac{2n}{2n+1}\times\frac{2n-2}{2n-1}\times\dots\times\frac{2}{3}\times \underbrace{I_{1}}{=1}\\
			&=\frac{2^{2n}(n!)^{2}}{(2n+1)!}
		\end{align*}
		d'où
		$$\Vert L_{n}\Vert_{2}^{2}=\frac{(2n)!}{2^{2n}(n!)^{2}}\times 2\times \frac{2^{2n}(n!)^{2}}{(2n+1)!}=\frac{2}{2n+1}$$

		\item On utilise la formule de Leibniz en écrivant $X^{2}-1=(X+1)(X-1)$.
		\item On montre le résultat par récurrence sur $k\in\{0,\dots,n\}$ en invoquant le théorème de Rolle. On trouve donc que $L_{n}=P_{n}^{(n)}$ s'annule au moins $n$ fois sur $]-1,1[$. Or $\deg(L_{n})=n$, donc ces zéros sont simples et ce sont les seuls.
		\item $(L_{0},\dots,L_{n})$ est une base de $\R_{n}[X]$ (étagée en degré). Donc il existe $(\alpha_{n,0},\dots,\alpha_{n,k})\in\R^{k+1}$ tel que $XL_{n-1}=\sum_{k=0}^{n}\alpha_{n,k}L_{k}$. Si $k\leqslant n-3$, on a
		$$(XL_{n-1} L_{k})=\alpha_{n,k}\Vert L_{k}\Vert_{2}^{2}=(L_{n-1}XL_{k})=0$$
		car $\deg(XL_{k})=k+1\leqslant n-2$. Donc 
		$$XL_{n-1}=\alpha_{n,n-2}L_{n-2}+\alpha_{n,n-1}L_{n-1}+\alpha_{n,n}L_{n}$$
		Pour calculer les coefficients, on fait tout simplement les produits scalaires:
		$$(Xl_{n-1}|L_{n-1})=\int_{-1}^{1}tL_{n-1}(t)^{2}dt$$
		Or $P_{n}$ est paire, donc $L_{n}$ est de la parité de $n$ et donc $L_{n}^{2}$ est paire puis $XL_{n}^{2}$ est impaire. Donc $\alpha_{n,n-1}=0$.

		\begin{align*}
			(XL_{n-1}|L_{n-2})
			&=\alpha_{n,n-2}\underbrace{\Vert L_{n-2}\Vert_{2}^{2}}_{=\frac{2}{2n-3}}\\
			&=(-1)^{n}\int_{-1}^{1}P_{n-1}(t)\underbrace{(XL_{n-2})^{(n-1)}(t)}_{\frac{(2n-4)!(n-1)}{2^{n-2}(n-2)!}}
		\end{align*}
		Par ailleurs,
		\begin{align*}
			(-1)^{n-1}\int_{-1}^{1}P_{n-1}(t)dt
			&=\frac{1}{2^{n-1}(n-1)!}\underbrace{\int_{-1}^{1}(1-t^{2})^{n-1}dt}_{2I_{2n-1}}\\
			&=\frac{1}{2^{n-1}(n-1)!}\times 2\times\frac{2^{2n-2}(n-1)!)^{2}}{(2n-1)!}\\
			&=\frac{2^{n}(n-1)!}{(2n-1)!}
		\end{align*}
		donc $\frac{\alpha_{n,n-2}}{\alpha_{n,n}}=\frac{n-1}{n}$. D'où le résultat.
	\end{enumerate}
\end{solution}

\begin{solution}
	On forme \function{g}{[a,b]}{\R}{x}{\underbrace{\Delta f(x_{0},\dots,x_{n-1},x)}_{\varphi(x)}-\underbrace{\prod_{i=0}^{n-1}(x-x_{i})A}_{P(x)}}
	On a $g(x_{n})=0$. On suppose les $(x_{i})_{1\leqslant i\leqslant n}$ distincts, et on pose 
	$$A=\frac{V(x_{0},\dots,x_{n})}{\prod_{i=0}^{n-1}(x_{n}-x_{i})}$$
	$g$ est de classe $\mathcal{C}^{n}$ et pour tout $i\in\{0,\dots,n\}$, on a $g(x_{i})=0$.
	Donc il existe $\xi\in]a,b[$ tel que $g^{(n)}(\xi)=0$ (théorème de Rolle appliqué $n$ fois. $\deg(P)=n$ et son coefficient dominant est $A$ donc $P^{(n)}(\xi)=An!=\varphi^{(n)}(\xi)$.

	On développe maintenant $\varphi(x)$ par rapport à la dernière colonne:
	$$\varphi(x)=f(x)\times V_{n}(x_{0},\dots,x_{n-1})+Q(X)$$
	avec $\deg(Q)\leqslant n-1$ et $V_{n}(x_{0},\dots,x_{n-1})=\prod_{0\leqslant j<i\leqslant n-1}(x_{i}-x_{j})$ (déterminant de Vandermonde). On a donc 
	$$\varphi^{(n)}(x)=f^{(n)}(x)\prod_{0\leqslant j<i\leqslant n-1}(x_{j}-x_{i})$$
	et en reportant, on a 
	$$\frac{f^{(n)}(\xi)}{n!}=\frac{A}{\prod_{0\leqslant i<j\leqslant n-1}(x_{j}-x_{i})}=\Delta f(x_{0},\dots,x_{n})$$
\end{solution}

\begin{solution}
	On utilise le développement de Taylor avec reste intégral.
	$$f(0)=f\Bigl(\frac{1}{2}\Bigr)-\frac{1}{2}f'\Bigl(\frac{1}{2}\Bigr)+\int_{\frac{1}{2}}^{0}-tf''(t)dt$$
	et de même
	$$f(1)=f\Bigl(\frac{1}{2}\Bigr)-\frac{1}{2}f'\Bigl(\frac{1}{2}\Bigr)+\int_{\frac{1}{2}}^{1}(1-t)f''(t)dt$$
	D'où
	\begin{align*}
		A(f)
		&=f(0)-f\Bigl(\frac{1}{2}\Bigr)+f(1)-f\Bigl(\frac{1}{2}\Bigr)\\
		&=\int_{0}^{\frac{1}{2}}tf''(t)dt+\int_{\frac{1}{2}}^{1}(1-t)f''(t)dt\\
		&\leqslant\int_{0}^{\frac{1}{2}}tdt+\int_{\frac{1}{2}}^{1}(1-t)dt\\
		&=\frac{1}{4}
	\end{align*}
	Et c'est atteint pour $f(t)=\frac{t^{2}}{4}$.
\end{solution}

\begin{solution}
	Pour tout $(x,h)\in\R^{2}$, $f(x+h)-f(x-h)=2hf'(x)$ donc 
	\begin{equation}
		\label{eq:7.1}
		f'(x)=\frac{1}{2}(f(x+1)-f(x-1))
	\end{equation}
	donc $f'$ est $\mathcal{C}^{1}$ et donc $f$ est $\mathcal{C}^{2}$. On fixe alors $x$ et on dérive deux fois~\eqref{eq:7.1} en fonction de $h$. On a alors
	$$f''(x+h)=f''(x-h)$$
	pour tout $(x,h)\in\R^{2}$ donc $f''$ est constante et $f$ est polynômiale de degré 2.

	Réciproquement, si $f(x)=ax^{2}+bx+c$, on a bien la relation de l'énoncé.
\end{solution}

\begin{solution}
	\phantom{}
	\begin{enumerate}
		\item Soit $a>0$, \function{\tau_{a}}{\R}{]a,+\infty[}{x}{\frac{f(x)-f(a)}{x-a}}
		est croissante. Donc il existe $l=\lim\limits_{x\to+\infty}\tau_{a}(x)\in\overline{\R}$. On écrit alors 
		$$\frac{f(x)}{x}=\frac{f(x)-f(a)}{x-a}\times \frac{x-a}{x}+\frac{f(a)}{x}\xrightarrow[x\to+\infty]{}l$$

		\item S'il existe $a<b\in(\R_{+}^{*})^{2}$ tel que $f(a)<f(b)$, alors $\tau_{a}(b)>0$. Comme $\tau_{a}$ est croissante, $l\geqslant\tau_{a}(b)>0$. Par contraposée, si $l\geqslant0$, $f$ est décroissante.
		\item Posons pour tout $x\in\R_{+}^{*}$, $\varphi(x)=f(x)-lx$. Pour $x<y$, on a 
		$$\frac{\varphi(y)-\varphi(x)}{y-x}=\frac{f(y)-f(x)}{y-x}-l\leqslant0$$
		Donc $\varphi$ est décroissante et $\lim\limits_{x\to+\infty}\varphi(x)\in\overline{\R}$ existe.
	\end{enumerate}
\end{solution}

\begin{solution}
	\phantom{}
	\begin{enumerate}
		\item On forme \function{g}{[0,1]}{\R}{x}{\frac{1}{\frac{1}{p}+x}}
		Alors 
		$$\sum_{k=0}^{np}\frac{1}{n+k}=\frac{1}{np}\sum_{k=0}^{np}\frac{1}{\frac{1}{p}+\frac{k}{np}}\xrightarrow[n\to+\infty]{}\int_{0}^{1}\frac{dx}{\frac{1}{p}+x}=\ln(p+1)=l_{p}$$

		\item On note $f(x)=f(0)+xf'(0)+x\varepsilon(x)$ avec $\varepsilon(x)\xrightarrow[\varepsilon\to0]{}0$. 
		
		Soit $\varepsilon_{0}>0$. Il existe $\alpha_{0}>0$ tel que si $0<x<\alpha_{0}$, alors $\vert\varepsilon(x_{0})\vert\leqslant\varepsilon_{0}$, et il existe $N_{0}\in\N$ tel que pour tout $n\geqslant N_{0}$, $\frac{1}{n}\leqslant\alpha_{0}$. Alors pour tout $n\geqslant N_{0}$, pour tout $k\in\{0,\dots,np\}$, 
		$$\frac{1}{k+n}\Rightarrow \Biggl\vert\varepsilon\Bigl(\frac{1}{k+n}\Bigr)\Biggr\vert\leqslant\frac{\varepsilon_{0}}{p}$$
		et
		$$\Biggl\vert\sum_{k=0}^{np}\frac{\varepsilon(\frac{1}{k+n})}{k+n}\Biggr\vert\leqslant\sum_{k=0}^{np}\frac{\frac{\varepsilon_{0}}{p}}{k+n}\leqslant\frac{\varepsilon_{0}}{p}\frac{np+1}{n+1}\leqslant\varepsilon_{0}$$

		On a donc
		$$v_{n}=\sum_{k=0}^{np}\frac{1}{n+k}f'(0)+\sum_{k=0}^{np}\frac{\varepsilon(\frac{1}{n+k})}{n+k}\xrightarrow[n\to+\infty]{}\ln(p+1)f'(0)$$

		\item On peut penser à $f\colon x\mapsto\sqrt{x}$ continue et $f(0)=0$. De plus,
		$$\sum_{k=0}^{np}\frac{1}{\sqrt{n+k}}\geqslant\frac{np+1}{\sqrt{n(p+1)}}\xrightarrow[n\to+\infty]{}+\infty$$
		donc $v_{n}$ diverge.

		\item On écrit $f(x)=f(0)+xf'(0)+\frac{x^{2}}{2!}f''(0)+x^{2}\varepsilon(x)$ avec $\varepsilon(x)\xrightarrow[\varepsilon\to+\infty]{}0$. Ainsi, 
		$$v_{n}=\sum_{k=0}^{np}\frac{f''(0)}{2(n+k)^{2}}+\sum_{k=0}^{bp}\frac{\varepsilon(\frac{1}{k+n})}{(k+n)^{2}}$$
		Soit $\varepsilon>0$, il existe $N\in\N$ tel que pour tout $n\geqslant N$, pour tout $k\in\{0,\dots,np\}$, $\vert\varepsilon(\frac{1}{n+k})\vert\leqslant\varepsilon$ et donc 
		$$\Biggl\vert\sum_{k=0}^{np}\frac{\varepsilon(\frac{1}{n+k})}{(n+k)^{2}}\Biggr\vert\leqslant\sum_{k=0}^{np}\frac{\varepsilon}{(n+k)^{2}}$$
		donc 
		$$\sum_{k=0}^{np}\frac{\varepsilon(\frac{1}{n+k})}{(n+k)^{2}}=O\Biggl(\sum_{k=0}^{np}\frac{f''(0)}{2}\times\frac{1}{(n+k)^{2}}\Biggr)$$
		puis
		$$v_{n}\underset{n\to+\infty}{\sim}\sum_{k=0}^{np}\frac{f''(0)}{2(n+k)^{2}}$$
		Or 
		\begin{align*}
			\sum_{k=0}^{np}\frac{1}{(n+k)^{2}}
			&=\frac{1}{(np)^{2}}\sum_{k=0}^{np}\frac{1}{(\frac{1}{p}+\frac{k}{np})^{2}}\\
			&=\frac{1}{np}\times \underbrace{\frac{1}{np}\sum_{k=0}^{np}\frac{1}{(\frac{1}{p}+\frac{k}{np})^{2}}}_{\xrightarrow[n\to+\infty]{}\int_{0}^{1}\frac{dx}{(\frac{1}{p}+x)^{2}}}
		\end{align*}
		donc 
		$$v_{n}\underset{n\to+\infty}{\sim}\frac{f''(0)p}{n(p+1)}$$
	\end{enumerate}
\end{solution}

\begin{solution}
	Supposons que $f'$ ne tend pa vers 0 en $+\infty$: il existe $\varepsilon_{0}>0,\forall A>0,\exists x_{A}\geqslant A,\vert f'(x_{A})\vert\geqslant\varepsilon_{0}>0$. Par continuité uniforme, il existe $\alpha_{0}\geqslant0$, $\forall(x,y)\in(\R_{+})^{2}$, si $\vert x-y\vert\leqslant\alpha_{0}$ alors $\vert f'(x)-f'(y)\vert\leqslant\frac{\varepsilon_{0}}{2}$. Alors pour tout $t\in[x_{A}-\alpha,x_{A}+\alpha]$, on a 
	$$\vert f'(t)\vert\geqslant \vert f'(x_{A})\vert-\vert f'(x_{A})-f'(t)\vert\geqslant\varepsilon_{0}-\frac{\varepsilon_{0}}{2}\geqslant\frac{\varepsilon_{0}}{2}$$
	et pour $A=n$, pour tout $n\in\N,\exists x_{n}\geqslant n,\forall t\in[x_{n}-\alpha,x_{n}+\alpha],\vert f'(t)\vert\geqslant\frac{\varepsilon_{0}}{n}$. D'après le théorème des valeurs intermédiaires, $f'$ est de signe constant sur $[x_{n}-\alpha,x_{n}+\alpha]$. Quitte à changer $f$ en $-f$, on peut supposer qu'il existe une infinité de $n\in\N$ tels que $f'>0$ sur les $[x_{n}-\alpha,x_{n}+\alpha]$. Alors
	$$f(x_{n}+\alpha_{0})-f(x_{n}-\alpha_{0})=\int_{x_{n}-\alpha_{0}}^{x_{n}+\alpha_{0}}f'(t)dt\geqslant\varepsilon_{0}\alpha_{0}>0$$
	mais comme $\lim\limits_{x\to+\infty}f(x)\in\R$, on a 
	$$\lim\limits_{n\to+\infty}f(x_{n}+\alpha_{0})-f(x_{n}-\alpha_{0})=0$$
	d'où la contradiction.

	Si $f\in\mathcal{C}^{1}(\R_{+},\C)$, on applique ce qui précède à $\Im(f)$ et $\Re(f)$. 

	Si $f'$ n'est pas uniformément continue, ce n'est plus valable, par exemple 
	$$f(x)=\frac{\sin(x^{2})}{x}\xrightarrow[x\to+\infty]{}0$$
	car $\vert f(x)\vert\leqslant\frac{1}{x}$ et 
	$$f'(x)=\underbrace{-\frac{1}{x^{2}}\sin(x^{2})}_{\xrightarrow[x\to+\infty]{}0}+\underbrace{\frac{2x\cos(x^{2})}{x}}_{\text{n'a pas de limite en }+\infty}$$
\end{solution}

\begin{solution}
	Soit $x\in\R$ et $h\neq0$, on a 
	$$\frac{f(x+h)-f(x)}{h}=g(x+\frac{h}{2})\xrightarrow[h\to0]{}g(x)$$
	par continuité de $g$. Donc $f$ est dérivable et $f'=g$. Par ailleurs, pour $y=\frac{1}{2}$, on a 
	$$f'(x)=f(x+\frac{1}{2})-f(x-\frac{1}{2})$$
	par récurrence $f$ est $\mathcal{C}^{\infty}$.

	En outre, en fixant $x$ et en dérivant la relation de départ deux fois par rapport à $y$, on a 
	$$f''(x+y)-f''(x-y)=0$$
	Donc $f''$ est constante donc $f$ est un polynôme de degré plus petit que 2.

	Réciproquement, on vérifie que ces fonctions marchent (avec $f'=g$).
\end{solution}

\begin{solution}
	On a 
	$$S_{n}=\sum_{k=1}^{n-1}\frac{1}{2}(f(k)+f(k+1))-\int_{k}^{k+1}f(t)dt$$
	On note $F(x)=\int_{1}^{x}f(t)dt$ de classe $\mathcal{C}^{2}$.

	On a
	$$F(b)=F(a)+F'(a)(b-a)+\int_{a}^{b}F''(t)(b-t)dt$$
	Pour $a=k$ et $b=k+\frac{1}{2}$, on a 
	$$F(k+\frac{1}{2})=F(k)+\frac{1}{2}F'(k)+\int_{k}^{k+\frac{1}{2}}(k+\frac{1}{2}-t)f'(t)dt=F(k)+\frac{1}{2}F'(k)+\int_{0}^{\frac{1}{2}}uf'(k+\frac{1}{2}-u)du$$
	et pour $a=k+1,b=k+\frac{1}{2}$,
	$$F(k+\frac{1}{2})=F(k+1)-\frac{1}{2}F'(k+1)+\int_{k+1}^{k+\frac{1}{2}}(k+\frac{1}{2}-t)f'(t)dt=F(k+1)-\frac{1}{2}F'(k+1)+\int_{0}^{\frac{1}{2}}uf'(k+\frac{1}{2}+u)du$$

	On a donc
	$$\frac{1}{2}(f(k)-f(k+1))-\int_{k}^{k+1}f(t)dt=\int_{0}^{\frac{1}{2}}u(f'(k+\frac{1}{2}+u)-f'(k+\frac{1}{2}-u))du$$
	d'où
	$$S_{n}=\int_{0}^{\frac{1}{2}}u\sum_{k=1}^{n-1}\underbrace{f'(k+\frac{1}{2}+u)-f'(k+\frac{1}{2}-u)}_{\geqslant0\text{ car }u\geqslant0\text{ et }f'\text{ croissante}}du$$
	et 
	$f'(k+\frac{1}{2}+u)-f'(k+\frac{1}{2}-u)\leqslant f'(k+1)-f'(k)$ d'où 
	$$S_{n}\leqslant\underbrace{\int_{0}^{\frac{1}{2}}udu}_{=\frac{1}{8}}(f'(n)-f'(1))$$
\end{solution}

\begin{solution}
	\phantom{}
	\begin{enumerate}
		\item D'après l'inégalité de Taylor-Lagrange, on a 
		$$
		\left\{
			\begin{array}[]{l}
				\Vert A\Vert\leqslant\frac{h^{2}}{2}M_{2}\\
				\Vert B\Vert\leqslant\frac{h^{2}}{2}M_{2}
			\end{array}
		\right.
		$$
		On a $B-A-f(x-h)+f(x+h)=2hf'(x)$ d'où 
		$$\Vert f'(x)\Vert\leqslant\frac{hM_{2}}{2}+\frac{M_{0}}{h}$$
		Donc $f'$ est bornée sur $\R$. On a ensuite un majorant qui dépend de $h$ que l'on peut optimiser, et on trouve la borne demandée.

		\item L'inégalité de Taylor-Lagrange donne à nouveau
		$$\forall k\in\{1,\dots,n-1\},\Vert A_{k}\Vert\leqslant\frac{k^{n}}{n!}M_{n}$$
		On forme alors
		$$
		\begin{pmatrix}
			A_{1}-f(x+1)\\
			\vdots\\
			A_{k}-f(x+k)\\
			\vdots\\
			A_{n}-f(x+n)
		\end{pmatrix}
		=
		\underbrace{
		\begin{pmatrix}
			-1 & -1 & \dots & \frac{-1}{(n-1)!}\\
			\vdots & \vdots & & \vdots\\
			-1 & -k & \dots & \frac{-k^{n-1}}{(n-1)!}\\
			\vdots & \vdots & & \vdots\\
			-1 & -n & \dots & \frac{-n^{n-1}}{(n-1)!}\\
		\end{pmatrix}}_{=M}
		\begin{pmatrix}
			f(x)\\
			\vdots\\
			f^{(k)}(x)\\
			\vdots\\
			f^{(n-1)}(x)
		\end{pmatrix}
		$$

		On a 
		$$\det(M)=\frac{(-1)^{n}}{1!\times 2!\times\dots\times (n-1)!}V(1,\dots,n)$$
		où $V$ est le déterminant de Vandermonde. Donc $\det(M)\neq0$. On peut former les $f^{(j)}(x)$ en fonction des $(A_{i}-f(x+i))_{1\leqslant i\leqslant n}$: il existe $(\alpha_{1},\dots,\alpha_{n})\in\R^{n}$ tel que pour tout $x\in\R$, $f^{(j)}(x)=\sum_{i=1}^{n}\alpha_{i}(A_{i}-f(x+i))$. Donc 
		$$\Vert f^{(j)}(x)\Vert\leqslant\sum_{i=1}^{n}\vert\alpha_{i}\vert\Bigl(\frac{n}{n!}M_{n}+M_{0}\Bigr)$$
		Donc $f^{(j)}$ est bornée pour tout $j\in\{1,\dots,n-1\}$.
	\end{enumerate}
\end{solution}

\begin{solution}
	\phantom{}
	\begin{enumerate}
		\item 
		$$l_{\sigma,\gamma}=\sum_{i=0}^{n-1}\Bigl\Vert\int_{a_{i}}^{a_{i+1}}\gamma'(t)dt\Bigr\Vert\leqslant\sum_{i=0}^{n-1}\int_{a_{i}}^{a_{i+1}}\Vert\gamma'(t)\Vert dt=\int_{a}^{b}\Vert\gamma'(t)\Vert dt$$

		\item On a 
		\begin{align*}
			\Bigl\vert l_{\sigma,\gamma}-\sum_{i=0}^{n-1}\Vert\gamma'(a_{i})\Vert(a_{i+1}-a_{i})\Bigr\vert
			&=\Bigl\vert\sum_{i=0}^{n-1}\Vert\gamma(a_{i+1})-\gamma(a_{i})\Vert-\Vert\underbrace{(a_{i+1}-a_{i})}_{>0}\gamma'(a_{i})\Vert\Bigr\vert\\
			&\leqslant\sum_{i=0}^{n-1}\Vert\gamma(a_{i+1})-\gamma(a_{i})-(a_{i+1}-a_{i})\gamma'(a_{i})\Vert\\
			&\leqslant\sum_{i=0}^{n-1}\int_{a_{i}}^{a_{i+1}}\Vert\gamma'(t)-\gamma'(a_{i})\Vert dt
		\end{align*}

		\item $\Vert\gamma'\Vert$ est continue donc 
		$$\int_{a}^{b}\Vert\gamma'(t)\Vert dt=\lim\limits_{\delta(\sigma)\to0}\sum_{i=0}^{n-1}\Vert\gamma'(a_{i})\Vert(a_{i+1}-a_{i})$$
		Donc $\alpha_{0}$ existe.

		$\gamma'$ est continue sur $[a,b]$ donc uniformément continue sur $[a,b]$, et il existe $\alpha_{1}>0$ tel que pour tout $(x,y)\in[a,b]^{2}$, on a 
		$$\vert x-y\vert\leqslant\alpha_{}\Rightarrow\Vert\gamma'(x)-\gamma'(y)\Vert\leqslant\frac{\varepsilon}{2(b-a)}$$
		Alors si $\delta(\sigma)\leqslant\alpha_{1}$, pour tout $i\in\{0,\dots,n-1\}$, pour tout $t\in[a_{i},a_{i+1}]$, on a
		$$\vert t-a_{i}\vert\leqslant(a_{i+1}-a_{i})\leqslant\alpha_{1}$$
		d'où 
		$$\Vert \gamma'(a_{i})-\gamma'(t)\Vert\leqslant\frac{\varepsilon}{2(b-a)}$$
		et d'après la question 2, on a donc 
		$$\Bigl\vert l_{\sigma,\gamma}-\sum_{i=0}^{n-1}\Vert\gamma'(a_{i})\Vert(a_{i+1}-a_{i})\Bigr\vert\leqslant\frac{\varepsilon}{2}$$

		Finalement, si $@d(\sigma)\leqslant\min(\alpha_{0},\alpha_{1})$, on a 
		$$\Bigl\vert l_{\sigma,\gamma}-\int_{a}^{b}\Vert\gamma'(t)\Vert dt\Bigr\vert\leqslant\varepsilon$$
		Donc 
		$$l(\gamma)=\int_{a}^{b}\Vert\gamma'(t)\Vert dt$$

		\item On a 
		$$\gamma'(t)=\begin{pmatrix}
			-R\sin(t)\\
			R\cos(t)
		\end{pmatrix}$$
		donc $\Vert\gamma'(t)\Vert=R$ et $l(\gamma)=2\pi R$.
	\end{enumerate}
\end{solution}

\begin{solution}
	\phantom{}
	\begin{enumerate}
		\item Pour tout $t\in I$, on a 
		$$\gamma(t)=\vert\gamma(t)\vert e^{\mathrm{i}\theta_{1}(t)}=\vert\gamma(t)\vert e^{\mathrm{i}\theta_{2}(t)}$$
		donc 
		$$e^{\mathrm{i}(\theta_{1}(t)-\theta_{2}(t))}=1$$

		Ainsi, pour tout $t\in I$, il existe $k(t)\in\Z$ telle que $\theta_{2}(t)-\theta_{1}(t)=2k(t)\pi$. On a 
		$$k(t)=\frac{\theta_{2}(t)-\theta_{1}(t)}{2\pi}$$
		qui est continue et à valeurs entières, donc constante égale à $k_{0}$ d'après le théorème des valeurs intermédiaires.

		\item Si $\gamma(t)=x(t)+\mathrm{i}y(t)$, 
		$$\vert\gamma(t)\vert=\sqrt{x(t)^{2}+y(t)^{2}}$$
		Comme $\sqrt{\cdot}$ est $\mathcal{C}^{\infty}$ sur $\R_{+}^{*}$, par composition, $f$ est $\mathcal{C}^{k}$. On a alors
		$$f(t)=e^{\mathrm{i}\theta(t)}\Rightarrow f'(t)=\mathrm{i}\theta'(t)e^{\mathrm{i\theta(t)}}=\mathrm{i}\theta'(t)f(t)$$
		Donc 
		$$\theta(t)=-\mathrm{i}\frac{f'(t)}{f(t)}$$

		De plus, on a 
		$$\theta(t)=\theta(t_{0})-\mathrm{i}\int_{t_{0}}^{t}\frac{f'(u)}{f(u)}du$$
		pour $t_{0}\in I$.

		\item On fixe $t_{0}\in I$. Soit $\theta_{0}$ un argument de $\gamma(t_{0})$, on pose 
		$$\theta(t)=\theta_{0}-\mathrm{i}\int_{t_{0}}^{t}\frac{f'(u)}{f(u)}du$$
		
		Comme $\frac{f'}{f}$ est $\mathcal{C}^{k-1}$, $\theta$ est bien $\mathcal{C}^{k}$. On forme $g(t)=e^{\mathrm{i}\theta(t)}$ qui est de classe $\mathcal{C}^{k}$. On a 
		$$g'(t)=\mathrm{i}\theta'(t)g(t)=\frac{f'(t)}{f(t)}g(t)$$
		donc $\Bigl(\frac{g}{f}\Bigr)'=0$, donc $\frac{g}{f}$ est constante sur $I$ et $g(t_{0})=e^{\mathrm{i}\theta_{0}}=f(t_{0})$ donc $g=f$ sur $I$. Ainsi, pour tout $t\in I$, on a $\vert f(t)\vert=\vert e^{\mathrm{i}\theta(t)}\vert=1$ et si $\theta(t)=a(t)+\mathrm{i}(t)$, on a donc 
		$$e^{\mathrm{i}\theta(t)}=e^{-b(t)}e^{\mathrm{i}a(t)}$$
		donc $b(t)=0$ et $\theta(t)\in\R$.
	\end{enumerate}
\end{solution}