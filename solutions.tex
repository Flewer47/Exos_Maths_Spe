\documentclass[12pt]{article}
\usepackage[french]{babel}
\usepackage[utf8]{inputenc}
\usepackage[T1]{fontenc}
\usepackage{amsmath}
\usepackage{amssymb}
\usepackage{color}
\definecolor{persianplum}{rgb}{0.44, 0.11, 0.11}
\usepackage{url}
\usepackage[breaklinks]{hyperref}
\hypersetup{
	colorlinks=true,
	linkcolor=persianplum,
	filecolor=blue,
	citecolor=black,      
	urlcolor=cyan,
}

\textwidth=18cm \textheight=23cm \oddsidemargin=-1.00cm
\evensidemargin=-1.00cm
\parindent=1cm
\topmargin=-2cm

\setlength{\parindent}{4em}
\setlength{\parskip}{1em}
\renewcommand{\baselinestretch}{1.5}

\usepackage{amsthm}
\newtheorem{solution}{Solution}[section]
\theoremstyle{remark}
\newtheorem{remark}{Remarque}[section]

\newcommand{\K}{\mathbb{K}} \newcommand{\R}{\mathbb{R}}
\newcommand{\C}{\mathbb{C}} \newcommand{\Q}{\mathbb{Q}}
\newcommand{\N}{\mathbb{N}} \newcommand{\Z}{\mathbb{Z}}
\newcommand{\U}{\mathbb{U}} \newcommand{\E}{\mathbb{E}}
\newcommand{\M}{\mathcal{M}} \renewcommand{\L}{\mathcal{L}}
\renewcommand{\P}{\mathbb{P}} \newcommand{\im}{\emph{Im}}
\renewcommand{\i}{\mathrm{i}}
\DeclareMathOperator{\sgn}{sgn} \DeclareMathOperator{\diag}{diag}
\DeclareMathOperator{\rg}{rg} \DeclareMathOperator{\Tr}{Tr}
\DeclareMathOperator{\Sp}{Sp} \DeclareMathOperator{\mat}{mat}
\DeclareMathOperator{\com}{com} \DeclareMathOperator{\conv}{conv}
\DeclareMathOperator{\ppcm}{ppcm}
\newcommand{\vertiii}[1]{{\left\vert\kern-0.25ex\left\vert\kern-0.25ex\left\vert{}#1
\right\vert\kern-0.25ex\right\vert\kern-0.25ex\right\vert}}
\newcommand{\function}[5]{
	$$
	\begin{array}{rccl}
		#1: & #2 & \to & #3 \\
		& #4 & \mapsto & #5
	\end{array}
	$$
}
\numberwithin{equation}{section}

%\includeonly{}

\begin{document}

\begin{titlepage}
	\centering
	\vspace*{\fill}
	\Huge \textit{\textbf{Solutions Exercices MP/MP$^*$}}
	\vspace*{\fill}
\end{titlepage}

\cleardoublepage

\tableofcontents

\cleardoublepage

\section{Algèbre Générale}

\begin{solution}
	Soit $(x,y)\in G^{2}$. On a d'abord
	\begin{align}
		x\cdot y
		&=(x\cdot y)^{p+1}(x\cdot y)^{-p} \notag \\
		&=x^{p+1}\cdot y^{p+1}\cdot y^{-p}\cdot x^{-p}\notag \\
		&=x^{p+1}\cdot y \cdot x^{-p} \label{eq:1.1}
	\end{align}
	On cherche maintenant à montrer que $x^{p+1}$ et $y$ commutent.
	On a
	\begin{align*}
		y^{p+2}\cdot x^{p+2}
		&=(y\cdot x)^{p+2}\\
		&=(y\cdot x)^{p+1}\cdot y\cdot x\\
		&=y^{p+1}\cdot x^{p+1}\cdot y\cdot x
	\end{align*}
	Donc on a $y\cdot x^{p+1}=x^{p+1}\cdot y$. En reportant dans~\eqref{eq:1.1}, on a $x\cdot y=y\cdot x$ et donc G est abélien.
\end{solution}

\begin{remark}
	\phantom{}
	\begin{itemize}
		\item Pour $(\Sigma_{3},\cdot)$, on a $f_{0},f_{1}$ et $f_{6}$ des morphismes mais $\Sigma_{3}$ n'est pas commutatif.
		\item Si $f_{2}$ est un morphisme, alors on a $(x\cdot y)^{2}=x\cdot y\cdot x\cdot y=x^{2}\cdot y^{2}$ d'où $y\cdot x=x\cdot y$.
	\end{itemize}
\end{remark}

\begin{solution}
	$A$ est non vide car $\omega(e_{G})=1$ et $e_{G}\in A$. Soit $x\in A$ tel que $\omega(x)=2p+1$. Soit $k\in\Z$, on a 
	\begin{align*}
		x^{2k}=e_{G}
		&\Leftrightarrow 2p+1\mid 2k\\
		&\Leftrightarrow 2p+1\mid k
	\end{align*}
	d'après le théorème de Gauss.

	Ainsi, $\omega(x^{2})=2p+1$ et $x^{2}\in A$, donc \function{\varphi}{A}{A}{x}{x^{2}} est bien définie. Soit $x\in A$, il existe $p\in\N$ tel que $x^{2p+1}=e_{G}$ donc $x^{2p+2}=x$ d'où $(x^{p+1})^{2}=x$. Il suffit donc de vérifier que $x^{p+1}$ pour montrer que l'application est surjective. Comme $A$ est fini, elle sera bijective.

	On a $gr\{x^{p+1}\}\subset gr\{x\}$ et $(x^{p+1})^{2}=x$ donc $gr\{x\}=gr\{x^{p+1}\}$ donc $\omega(x)=\omega(x^{p+1})=2p+1$ et donc $x^{p+1}\in A$.
\end{solution}

\begin{solution}
	On note $m=\theta(\sigma)$. On suppose que $\sigma$ se décompose en produit de cycle de longueur $l_{1},\dots,l_{m}$ avec $l_{1}+\dots+l_{m}=n$. Comme
	$$(a_{1},\dots,a_{l})=[a_{1},a_{2}]\circ[a_{2},a_{3}]\circ\dots\circ[a_{l-1},a_{l}]$$
	Donc $\sigma$ se décompose en $\sum_{i=1}^{m}(l_{i}-1)=n-m$ transpositions. Montrons par récurrence sur $k$, $\mathcal{H}(k)\colon$ "Un produit de $k$ transpositions possède au moins $n-k$ orbites".

	Pour $k=0$, $\sigma=id$ possède $n$ orbites.

	Pour $k=1$, soit $\tau$ une transposition, on a $\theta(\tau)=n-2+1=n-1$.

	Soit $k\in\N$, supposons $\mathcal{H}_{k}$, soit $\sigma\in\Sigma_{n}$ qui se décompose en produit de $k+1$ transpositions.
	$$\sigma=\underbrace{\tau_{1}\circ\dots\tau_{k}}_{\sigma'}\circ\tau_{k+1}$$
	D'après $\mathcal{H}_{k}$, on a $\theta(\sigma')\geqslant n-k$. Notons $\tau_{k+1}=[a,b]$. 
	
	Si $a$ et $b$ appartiennent à la même orbite. On note $(a_{1},\dots,a_{r})$ le cycle correspondant avec $a_{r}=a$ et $a_{s}=b$ où $s\in\{1,\dots,n-1\}$. On a 
	$$
	\left\{
		\begin{array}[]{ll}
			(a_{1},\dots,a_{r-1},a_{r})\circ[a,b](a_{i})=a_{i+1} &\text{où }i\notin\{r,s\}\\
			(a_{1},\dots,a_{r-1},a_{r})\circ[a,b](a_{r})=a_{s+1}&\\
			(a_{1},\dots,a_{r-1},a_{r})\circ[a,b](a_{s})=a_{1}&
		\end{array}
	\right.
	$$

	On n'a pas perdu d'orbites, donc $\theta(\sigma)\geqslant n-k-1$. 

	Si $a$ et $b$ n'appartiennent pas à la même orbite, notons $(a_{1},\dots,a_{r})$ et $(b_{1},\dots,b_{s})$ ces orbites avec $a=a_{r}$ et $b=b_{s}$. On a 
	$$
	\left\{
		\begin{array}[]{ll}
			\underbrace{(a_{1},\dots,a_{r-1},a_{r})\circ(b_{1},\dots,b_{s})\circ[a_{r},b_{s}]}_{\sigma''}(a_{i})=a_{i+1} &\text{où }i\in\{1,\dots,r-1\}\\
			(a_{1},\dots,a_{r-1},a_{r})\circ(b_{1},\dots,b_{s})\circ[a_{r},b_{s}](b_{j})=b_{j+1} &\text{où }j\in\{1,\dots,s-1\}\\
			(a_{1},\dots,a_{r-1},a_{r})\circ(b_{1},\dots,b_{s})\circ[a_{r},b_{s}](a_{r})=b_{1}&\\
			(a_{1},\dots,a_{r-1},a_{r})\circ(b_{1},\dots,b_{s})\circ[a_{r},b_{s}](b_{s})=a_{1}&
		\end{array}
	\right.
	$$

	Donc 
	$$\sigma''=(a_{1},\dots,a_{r},b_{1},\dots,b_{s})$$
	On a perdu une orbite et donc $\theta(\sigma)\geqslant n-k-1$. D'où le résultat par récurrence sur $k$.
\end{solution}

\begin{solution}
	On note par $\overline{k}$ les éléments de $\Z/n\Z$ et par $\widetilde{l}$ les éléments de $\Z/m\Z$.

	Soit $f$ un morphisme. On pose $f(\overline{1})=\widetilde{x}$ où $x\in\{0,\dots,m-1\}$. On a donc $nf(\overline{1})=f(\overline{0})=\widetilde{0}$.

	On a donc $\widetilde{nx}=\widetilde{0}$ donc $m\mid nx$. On écrit $m=m_{1}(m\wedge n)$ et $n=n_{1}(m\wedge n)$. D'après le théorème de Gauss, on a donc $m_{1}\mid x$. Donc $x=km_{1}$ avec $k\in\{0,\dots,(n\wedge m)-1\}$.

	Réciproquement, soit $k\in\{0,\dots,(n\wedge m)-1\}$. On définit 
	\function{f_k}{\Z/n\Z}{\Z/m\Z}{\overline{l}}{\widetilde{lkm_{1}}}
	Si $\overline{l}=\overline{l'}$, alors $n\mid l-l'$ et donc $nm_{1}\mid (l-l')km_{1}$ puis $n_{1}(n\wedge m)m_{1}\mid (l-l')km_{1}$ donc $m\mid (l-l')km_{1}$ d'où $\widetilde{lkm_{1}}=\widetilde{l'km_{1}}$ donc $f$ est bien définie et c'est évidemment un morphisme. 
	
	Soit $k,k'\in\{0,\dots,n\wedge m-1\}$ avec $k\neq k'$. Si $\widetilde{km_{1}}=\widetilde{k'm_{1}}$ alors $m\mid (k-k')m_{1}$ et donc $n\wedge m\mid k-k'$ et $\vert k-k'\vert< n\wedge m$ donc $k=k'$ ce qui est absurde. Ainsi, les $f_{k}$ sont distincts, on a donc $n\wedge m$ morphismes.
\end{solution}

\begin{remark}
	Exemple pour l'exercice précédent: morphisme de $\Z/4\Z$ dans $\Z/6\Z$. On a $f(\overline{1})=\widetilde{x}$ d'où $\widetilde{4x}=\widetilde{0}$ donc $3\mid x$ d'où $x\in\{0,3\}$. On a donc le morphisme trivial $f_{0}\colon \overline{l}\mapsto\widetilde{0}$ et $f_{1}\colon\overline{l}\mapsto\widetilde{3l}$.
\end{remark}

\begin{solution}
	On considère $H=\{x\in G\bigm| x^{2}=e_{G}\}$. Si $x\notin H$, alors $x^{-1}\neq x$ et donc $P=\prod_{x\in H}x$. $H$ est le noyau du morphisme $x\mapsto x^{2}$ (morphisme car $G$ est abélien) donc $H$ est un sous-groupe. Soit $K$ un sous-groupe de $H$ et $a\in H\setminus K$. Montrons que $K\cup aK$ est un sous-groupe de $H$.
	
	On a $e_{G}\in K\cup aK$. Soit $x\in K\cup aK\subset H$, on a $x^{-1}=x\in K\cup aK$. Soit $(x_{1},x_{2})\in (K\cup aK)^{2}$, si $(x_{1},x_{2})\in K^{2},$, c'est ok. Si $(x_{1},x_{2})\in (aK)^{2}$, on note $x_{1}=a\cdot k_{1}$ et $x_{2}=a\cdot k_{2}$ avec $(k_{1},k_{2})\in K^{2}$. On a $x_{1}\cdot x_{2}=a^{2}\cdot k_{1}\cdot k_{2}=k_{1}\cdot k_{2}\in K$. Si $x_{1}\in K$ et $x_{2}\in aK$, alors $x_{1}\cdot x_{2}=a\cdot k_{1}\cdot k_{2}\in aK$. Donc $K\cup aK$ est un sous-groupe de $H$.

	Soit $x\in K\cap aK$, il existe $(k_{1},k_{2})\in K^{2}$ tel que $k_{1}=a\cdot k_{2}$ et $a\in K$ ce qui est impossible. Donc $K\cap aK=\emptyset$.

	On construit alors par récurrence $K_{n}$: on pose $K_{0}=\{e_{G}\}$ et à l'étape $n$, si $K_{n}=H$ on arrête, sinon il existe $a_{n+1}\in H\setminus K_{n}$ et on pose $K_{n+1}=K_{n}\cup a_{n+1}K$. Alors $\vert K_{n+1}\vert=2\vert K_{n}\vert$. Comme $H$ est fini, il existe $n_{0}\in\N$ tel que $H=K_{n_{0}}$. On a alors $\vert H\vert=2^{n_{0}}$.

	Ainsi, si $n_{0}=0$, on a $H=\{e_{G}\}$ et $P=e_{G}$. Si $n_{0}=1$, on a $H=\{e_{G},a_{1}\}$ et $P=a_{1}\neq e_{G}$. Si $n_{0}\geqslant 2$, comme chaque $a_{k}$ apparaît un nombre pair de fois dans le produit, on a $P=e_{G}$.
\end{solution}

\begin{solution}
	Soit $x_{0}\in\R$. $(\overline{kx_{0}})_{0\leqslant k\leqslant n}$ ne sont pas deux à deux distincts. Donc il existe $l\neq l'\in\{0,\dots,n\}^{2}$ tel que $\overline{lx_{0}}=\overline{l'x_{0}}$ d'où $0<\vert l-l'\vert\leqslant n$. Donc il existe $j\in\{1,\dots, n\}$ avec $jx_{0}\in G$. Ainsi, $n!x_{0}\in G$ (itéré de $jx_{0}$). Ce raisonnement est vrai pour $x=\frac{x_{0}}{n!}$ donc $x_{0}\in G$. Ainsi, $G=\R$.
\end{solution}

\begin{solution}
	Soit $f$ un isomorphisme de $\Z/n\Z$ dans lui-même. Soit $k\in\{0,\dots,n-1\}$, on a $f(\overline{k})=kf\overline{1})$. Par isomorphisme, $\omega(f(\overline{1}))=\omega(\overline{1})=n$. Notons alors $\overline{x}=f(\overline{1})$ avec $x\in\{0,dots,n-1\}$.

	Si $x\wedge n=1$, il existe $(u,v)\in\Z^{2}$ tel que $ux+vn=1$, donc $u\overline{x}=\overline{1}\in gr\{\overline{x}\}$. Ainsi, $Z\/n\Z=gr\{\overline{x}\}$ (car les éléments de $\Z/n\Z$ sont des itérés de $\overline{1}$) donc $\omega(\overline{x})=n$.

	Réciproquement, si $\omega(\overline{x})=n$, $\overline{1}\in gr\{\overline{x}\}$ donc il existe $u\in\Z$ tel que $u\overline{x}=1=\overline{ux}$. Donc $n\mid ux-1$, c'est-à-dire qu'il existe $v\in\Z$ tel que $ux-1=vn$, d'où $ux+vn=1$. D'après Bézout, on a $x\wedge n=1$. Finalement, on a $\omega(\overline{x})=n$ si et seulement si $x\wedge n=1$.

	Ainsi, les isomorphismes sont nécessairement de la forme \function{f_{x}}{\Z/n\Z}{\Z/n\Z}{\overline{k}}{\overline{kx}}
	où $x\in\{0,\dots,n-1\}$ et $x\wedge n=1$.

	Réciproquement, si $x\in\{0,\dots,n-1\}$ est tel que $x\wedge n=1$, $f_{x}$ est évidemment un morphisme. Si $\overline{k}\in\ker(f_{x})$, on a $f_{x}(\overline{k})=\overline{0}$ si et seulement si $\overline{kx}=\overline{0}$ si et seulement si $n\mid kx$ et comme $n\wedge x=1$, d'après le théorème de Gauss, on a $n\mid k$ donc $\overline{k}=\overline{0}$ donc $\ker(f_{x})=\{\overline{0}\}$. Donc $f_{x}$ est injective, donc bijective car $\Bigl\vert\Z/n\Z\Bigr\vert=\Bigl\vert\Z/n\Z\Bigr\vert=n$.
\end{solution}

\begin{solution}
	Si $y\in \im\varphi$, $y$ possède $\vert\ker\varphi\vert$ antécédents. En effet, il existe $x_{0}\in G$ tel que $y=\varphi(x_{0})$. Pour tout $x\in G$, on a $\varphi(x)=y$ si et seulement si $\varphi(x)=\varphi(x_{0})$ si et seulement si $\varphi(x_{0}^{-1}\cdot x)=e_{G}$ si et seulement si $x_{0}^{-1}\cdot x\in\ker\varphi$ si et seulement si $x\in x_{0}\ker\varphi$. Comme \function{g}{\ker\varphi}{x_{0}\ker\varphi}{x}{x\cdot x_{0}}
	est bijective, on a $\vert\ker\varphi\vert=\vert x_{0}\varphi\vert$. Ainsi, on a $\vert G\vert=\vert\im\varphi\vert\times\vert\ker\varphi\vert$.

	Dans tous les cas, on a $\ker\varphi\subset\ker\varphi^{2}$ et $\im\varphi^{2}\subset\im\varphi$. On a ensuite 
	\begin{align*}
		\im\varphi^{2}=\im\varphi
		&\Longleftrightarrow \vert\im\varphi^{2}\vert=\vert\im\varphi\vert\\
		&\Longleftrightarrow \vert\ker\varphi^{2}\vert\vert\im\varphi^{2}\vert=\vert\ker\varphi^{2}\vert\vert\im\varphi\vert=\vert G\vert=\vert\ker\varphi\vert\vert\im\varphi\vert\\
		&\Longleftrightarrow \vert\ker\varphi^{2}\vert=\vert\ker\varphi\vert\\
		&\Longleftrightarrow \ker\varphi^{2}=\ker\varphi
	\end{align*}
\end{solution}

\begin{solution}
	On considère \function{f}{G}{G}{x}{x^{m}}
	l'exercice revient à montrer que $f$ est bijective. D'après le théorème de Bézout, il existe $(a,b)\in\Z^{2}$ tel que $am+bn=1$. Soit $y\in G$, on a 
	$$y^{1}=y=y^{am+bn}=y^{am}\cdot \underbrace{y^{bn}}_{=e_{G}}=y^{am}=(y^{a})^{m}$$
	Donc $f$ est surjective et comme $G$ est fini, $f$ est bijective.
\end{solution}

\begin{solution}
	\phantom{}
	\begin{enumerate}
		\item On a $e_{G}\in S_{g}$, si $(x,y)\in S_{g}^{2}$ alors $x\cdot y\cdot g=x\cdot g\cdot y=g\cdot x\cdot y$ donc $x\cdot y\in S_{g}$ et si $x\in S_{g}$ alors $x\cdot g=g\cdot x$ implique $g\cdot x^{-1}=x^{-1}\cdot g$ en multipliant par l'inverse de $x$ à gauche et à droite donc $x^{-1}\in S_{g}$.
		
		\item Soit $(h,h')\in G^{2}$. On a $h\cdot g\cdot h^{-1}=h'\cdot g\cdot h'^{-1}$ si et seulement si $g\cdot h^{-1}\cdot h'=h^{-1}\cdot h\cdot g$ si et seulement si $h^{-1}\cdot h\in S_{g}$ si et seulement si $h'\in hS_{g}$. Or $\vert hS_{g}\vert=\vert S_{g}\vert$ car \function{I_{h}}{S_{g}}{hS_{g}}{x}{h\cdot x} est bijective de réciproque $I_{h^{-1}}$. Soit la relation d'équivalence $\mathcal{R}_{0}$ sur $G$ définie par $h\mathcal{R}_{0}h'$ si et seulement si $h\cdot g\cdot h^{-1}=h'\cdot g\cdot h'^{-1}$. Chaque classe à $\vert S_{g}\vert$ éléments et il y y a $\vert C(g)\vert$ classes dans $G$ d'où $\vert G\vert=\vert S_{g}\vert\vert C(g)\vert$.
		
		\item On a $Z(G)=\cap_{g\in G}S_{g}$ donc $Z(G)$ est un sous-groupe et pour tout $g\in G$, $Z(G)\subset S_{g}$.
		
		\item Pour $x\in G$, on note $\overline{x}=\{h\cdot x\cdot h^{-1}\bigm| h\in G\}=C(x)$. 
		
		On a $\vert\overline{x}\vert=1$ si et seulement si pour tout $h\in G$, $h\cdot x\cdot h^{-1}=x$ si et seulement si $x\in Z(G)$.
		
		Soit $\mathcal{A}$ une partie de $G$ telle que $(\overline{x})_{x\in\mathcal{A}}$ forme une partition de $G\setminus Z(G)$. On a 
		$$\vert G\vert=p^{\alpha}=\vert Z(G)\vert+\sum_{x\in\mathcal{A}}\vert C(x)\vert$$
		Si $x\in\mathcal{A}$, $x\notin Z(G)$ donc $\vert S_{x}\vert <\vert G\vert$ (car $x\in Z(G)$ si et seulement si $S_{x}=G$) et donc 
		$$\vert C(x)\vert=\frac{\vert G\vert}{\vert S_{x}\vert}$$
		d'après 2. Donc $\vert C(x)\vert=p^{\beta}$ avec $\beta\in\{1,\dots,\alpha\}$ car $\vert C(x)\vert\neq 1$. Donc 
		$$p\Bigm|\sum_{x\in\mathcal{A}}\vert C(x)\vert$$
		d'où 
		$$p\bigm|\vert Z(G)\vert$$
		donc $\vert Z(G)\vert\neq1$.

		\item On a 
		$$p^{2}=\vert Z(G)\vert+\sum_{x\in\mathcal{A}}\vert C(x)\vert$$
		D'après la question 4, on a $\vert Z(G)\vert\neq1$ et $\vert Z(G)\vert\bigm|\vert G\vert$.

		Si $Z(G)\neq G$, alors $\vert Z(G)\vert=p$. Pour $x\in\mathcal{A}$, $Z(G)\subset S_{x}\neq G$ donc $\vert S_{x}\vert= p$ (car $\vert S_{x}\vert\bigm|\vert G\vert$) et donc $Z(G)=S_{x}$. Or $x\in S_{x}$ et $x\notin Z(G)$ ce qui n'est pas possible, donc $\vert Z(G)\vert=p^{2}$ et $Z(G)=G$. Donc $G$ est abélien.

		S'il existe un élément d'ordre $p^{2}$. $G$ est cyclique et est isomorphe à $\Z/p^{2}\Z$. Sinon, pour tout $x\in G\setminus\{e_{G}\}$, on a $\omega(x)=p$. Soit $x_{1}\in G\setminus\{e_{G}\}$ et $x_{2}\in G\setminus gr\{x_{1}\}$.
		Soit \function{f}{\Bigl(\Z/p\Z\Bigr)^{2}}{G}{(\overline{k},\overline{l})}{x_{1}^{k}\cdot x_{2}^{l}}
		$f$ est bien définie car si $\overline{k}=\overline{k'}$ et $\overline{l}=\overline{l'}$, on a $p\mid k-k'$ et $p\mid l-l'$ donc $x_{1}^{k}\cdot x_{2}^{l}=x_{1}^{k'}\cdot x_{2}^{l'}$. Comme $G$ est abélien, $f$ est un morphisme. 
		
		Montrons que $f$ est injective. Soit $(\overline{k},\overline{l})\in\ker(f)$ avec $(k,l)\in\{0,\dots,p-1\}^{2}$, on a $x_{1}^{k}\cdot x_{2}^{l}=e_{G}$ donc $x_{2}^{l}=x_{1}^{-k}$. Si $l\in\{1,\dots,p-1\}$ or $p$ est premier donc $l\wedge p=1$ donc il existe $(u,v)\in\Z^{2}$ tel que $lu+pv=1$. Alors on a 
		$$x_{2}=x_{2}^{lu+pv}=x_{2}^{lu}\cdot x_{2}^{pv}=x_{2}^{lu}=x_{1}^{-k}\in gr\{x_{1}\}$$ ce qui n'est pas possible. Donc $\overline{l}=\overline{0}$ et de même $\overline{k}=\overline{0}$ donc $f$ est injective et ainsi 
		$\vert\Z/p^{2}\Z\vert=\vert G\vert$ donc $f$ est un isomorphisme.
	\end{enumerate}
\end{solution}

\begin{remark}
	Les groupes de cardinal $p^{3}$ ne sont pas nécessairement abélien, par exemple le groupe des isométries du carré $\mathcal{D}_{4}$ de cardinal 8.
\end{remark}

\begin{solution}
	Soit $f$ un morphisme de $(\Z,+)$ dans $(\Q_{+}^{*},\times)$. Pour tout $n\in\Z$, $f(n)=f(1)^{n}$ donc il existe $r_{0}\in\Q_{+}^{*}$ tel que $f(1)=r_{0}$ donc $f\colon n\mapsto r_{0}^{n}$.

	Soit $f$ un morphisme de $(\Q,+)$ dans $(\Q_{+}^{*},\times)$. Pour tout $a\in\N^{*}$, $f(1)=f(\frac{1}{a})^{a}$. Pour tout $p$ premier, on a $\nu_{p}(f(1))=a\nu_{p}(f(\frac{1}{a}))$ donc pour tout $a\in\N^{*}$, $a\mid\nu_{p}(f(1))$ donc $\nu_{p}(f(1))=0$ pour tout $p$ premier, donc $f(1)=1$. Ainsi, pour tout $n\in\Z$, $f(n)=f(1)^{n}=1$ et $f(b\times\frac{a}{b})=f(a)=1=f(\frac{a}{b})^{b}$ donc $f(\frac{a}{b})=1$. Donc $f\colon r\mapsto 1$.
\end{solution}

\begin{solution}
	On a $xy=y^{2}x$, $x^{2}y=xy^{2}x=y^{4}x^{2}$, $x^{3}y=x^{2}y^{2}x=xy^{4}x^{2}=y^{8}x^{3}$, $x^{5}y=y^{32}x^{5}$ donc $y^{31}=e_{G}$ et $\omega(y)=31$. Tout élément de $G$ peut s'écrire $y^{\lambda}x^{\mu}$ avec $(\lambda,\mu)\in\{0,\dots,30\}\times\{0,\times 4\}$. Soit \function{f}{\{0,\dots,30\}\times\{0,\times 4\}}{G}{(\lambda,\mu)}{y^{\lambda}x^{\mu}} est surjective par construction. Soit $((\lambda,\mu),(\lambda',\mu'))\in(\{0,\dots,30\}\times\{0,\times 4\})^{2}$ tel que $y^{\lambda}x^{\mu}=y^{\lambda'}x^{\mu'}$ donc $y^{\lambda-\lambda'}=x^{p'-p}$ d'où $y^{5(\lambda-\lambda')}=x^{5(\mu'-\mu)}=e_{G}$. Or $\omega(y)=31$ donc $31\mid 5(\lambda-\lambda')$ et d'après le théorème de Gauss, $31\mid \lambda-\lambda'$. Or $(\lambda,\lambda')\in\{0,\dots,30\}^{2}$ donc $\lambda=\lambda'$ et de même $\mu=\mu'$ donc $f$ est injective donc bijective et $\vert G\vert=155$. Soit $G'$ un autre tel groupe engendré par $x'$ et $y'$, on forme \function{g}{G}{G}{y^{p}x^{\mu}}{y'^{\lambda}x'^{\mu}}
	et on vérifie que $g$ est un isomorphisme.
\end{solution}

\begin{solution}
	\phantom{}
	\begin{enumerate}
		\item Soit $i\in\{1,\dots,r\}$, il existe nécessairement $y_{i}\in G$ tel que $\nu_{p_{i}}(\omega(y_{i}))=p_{i}^{\alpha_{i}}$ (où $\nu_{p}$ est la valuation $p$-adique), sinon on ne pourrait pas avoir ce terme dans le $\ppcm$. Donc $p_{i}^{\alpha_{i}}\mid \omega(y_{i})$.
		
		\item Il existe $n\in\N$ tel que $\omega(y_{i})=p_{i}^{\alpha_{i}}n$. Posons $x_{i}=y_{i}^{n}\in G$. Alors pour $k\in\N$,
		$$x_{i}^{k}=e_{G}\Longleftrightarrow y_{i}^{nk}=e_{G}\Longleftrightarrow \omega(y_{i})\mid nk\Longleftrightarrow p_{i}^{\alpha_{i}}\mid k$$
		Donc $\omega(x_{i})=p_{i}^{\alpha_{i}}$.

		\item On pose $x=\prod_{i=1}^{r}x_{i}$. Soit $k\in\N$, alors 
		$$x^{k}=e_{G}\Longleftrightarrow \prod_{i=1}^{r}x_{i}^{k}=e_{G}$$
		Pour $i\in\{1,\dots,r\}$, on met le tout à la puissance $M_{i}=\prod_{\substack{j=1\\j\neq i}}^{r}p_{j}^{\alpha_{j}}$. On a alors, pour tout $i\in\{1,\dots,r\}$,
		$$x_{i}^{kM_{i}}=e_{G}\Longleftrightarrow p_{i}^{\alpha_{i}}\mid kM_{i}\Longleftrightarrow p_{i}^{\alpha_{i}}\mid k$$
		la dernière équivalence venant du théorème de Gauss. Donc pour tout $i\in\{1,\dots,r\}$, $p_{i}^{\alpha_{i}}\mid k$, ce qui équivaut donc à $N\mid k$ et donc $\omega(x)=N$.
	\end{enumerate}
\end{solution}

\begin{solution}
	Sur un corps commutatif, un polynôme de degré $n$ admet au plus $n$ racines. Montrons qu'il existe $x_{1}\in\K^{*}$ tel que $\omega(x_{i})=\vert\K^{*}\vert$. Par définition de $N$, pour tout $x\in\K^{*}$, $\omega(x)\mid N$. D'où $x^{N}=1_{\K}$. Donc $x$ est racine de $X^{N}-1$. Ainsi, $\vert\K^{*}\vert\leqslant N$. Par ailleurs, $N\mid\vert\K^{*}\vert$ car pour tout $x\in\K^{*}$, $x^{\vert\K^{*}\vert}=1_{\K^{*}}$. Donc $\vert\K^{*}\vert=N$ et donc $\K^{*}=gr\{x_{1}\}$.

	On a $\vert\Z/13\Z^{*}\vert=12$ donc pour tout $\overline{x}\in(\Z/13\Z)^{*}$, $\omega(\overline{x})\in\{1,2,3,4,6,12\}$. On a $\overline{2}^{2}=\overline{4}$, $\overline{2}^{3}=\overline{8}$, $\overline{2}^{4}=\overline{16}=\overline{3}$, $\overline{2}^{6}=\overline{12}$ donc $\omega(\overline{2})=12$ et 
	$$\Z/13\Z^{*}=gr\{\overline{2}\}=\Bigl\{\overline{2}^{k}\bigm| k\in\{0,\dots,11\}\Bigr\}$$
\end{solution}

\begin{solution}
	\phantom{}
	\begin{enumerate}
		\item Soit $(x,y)\in G^{2}$, on a $(x\cdot y)^{2}=(x\cdot y)\cdot (x\cdot y)=e_{G}$ donc $x\cdot y=y^{-1}\cdot x^{-1}$ et comme $x^{2}=e_{G}$, $x^{-1}=x$ d'où $xy=yx$ et $G$ est abélien.
		
		\item Soit $(x_{1},\dots,x_{n})$ une famille génératrice minimale de $G$: pour tout $x\in G$, il existe$(\varepsilon_{i})\in\{0,1\}^{n}$ tel que $x=\prod_{i=1}^{n}x_{i}^{\varepsilon_{i}}$ (car $G$ est abélien).
		Soit \function{f}{(\Z/2\Z)^{n}}{G}{(\overline{\varepsilon_{1}},\dots,\overline{\varepsilon_{n}})}{\prod_{i=1}^{n}x_{i}^{\varepsilon_{i}}}
		Si pour tout $i\in\{1,\dots,n\}$ on a $\overline{\varepsilon_{i}}=\overline{\varepsilon_{i}'}$, alors $x^{\varepsilon_{i}}=x^{\varepsilon_{i}'}$ car $x_{i}^{2}=e_{G}$ et $2\mid\varepsilon_{i}-\varepsilon_{i}'$. Donc $f$ est bien définie.

		$f$ est clairement un morphisme (car $G$ est abélien). D'après la première question, $f$ est surjective. Montrons que $f$ est injective. Soit $(\overline{\varepsilon_{1}},\dots,\overline{\varepsilon_{n}})$ tel que $\prod_{i=1}^{n}x_{i}^{\varepsilon_{i}}=e_{G}$. Soit $i\in\{1,\dots,n\}$, supposons $\varepsilon_{i}$ impair, on a alors $x_{i}=\varepsilon_{i}=x_{i}$. D'où $x_{i}=\prod_{j=1}^{n}x_{j}^{-\varepsilon_{j}}=\prod_{j=1}^{n}x_{j}^{\varepsilon_{j}}$ car $x^{2}=e_{G}$. Donc $x_{i}\in gr(x_{j},j\in\{1,\dots,n\}, j\neq i)$, ce qui contredit le caractère minimal de $(x_{1},\dots,x_{n})$. Ainsi, $f$ est injective donc est un isomorphisme.
	\end{enumerate}
\end{solution}

\begin{remark}
	En notant $+$ la loi sur $G$, on peut définir \function{f}{\Z/2\Z\times G}{G}{(\varepsilon,x)}{x^{\varepsilon}}. Alors $(G,+,\cdot)$ est un $\Z/2\Z$-espace vectoriel, de dimension finie $n$ car $G$ est fini, et le choix d'une base réalise un isomorphisme de $((\Z/2\Z)^{n},+)$ dans $(G,+)$.
\end{remark}

\begin{remark}
	Par isomorphisme, on a 
	$$\prod_{x\in G}x=f(\sum_{(\overline{\varepsilon_{1}},\dots,\overline{\varepsilon_{n}})\in(\Z/2\Z)^{n}}(\overline{\varepsilon_{1}},\dots,\overline{\varepsilon_{n}}))$$

	Pour $n=1$, on a $\overline{0}+\overline{1}=\overline{1}$, pour $n=2$, on a $(\overline{0},\overline{0})+(\overline{0},\overline{1})+(\overline{1},\overline{0})+(\overline{1},\overline{1})=(\overline{0},\overline{0})$. Pour $n>2$, $\overline{1}$ apparaît $2^{n+1}$ fois sur chaque coordonnée (donc un nombre pair de fois), donc la somme fait $(\overline{0},\dots,\overline{0})$.
\end{remark}

\begin{solution}
	\phantom{}
	\begin{enumerate}
		\item Si $G$ est abélien, on a $D(G)=\{e_{G}\}$.
		\item Soit $\sigma\in\mathcal{A}_{n}$, $\sigma$ se décompose en un produit d'un nombre pair de transpositions. Soient $[a,b]$ et $[c,d]$ deux transpositions.
		\begin{itemize}
			\item Si $\{a,b\}=\{c,d\}$, alors $[a,b]\circ[c,d]=id$.
			\item Si $a\in\{c,d\}$, supposons par exemple $a=c$ et $b\neq d$. On a alors $[a,b]\circ[c,d]=[a,b]\circ[a,d]=[b,a,d]$.
			\item Si $\{a,b\}\cap\{c,d\}=\emptyset$, on a 
			$$[a,b]\circ[c,d]=[a,b]\circ\underbrace{[b,c]\circ[b,c]}_{=id}\circ[c,d]=[a,b,c]\circ[b,c,d]$$
		\end{itemize}
		Donc les $3$-cycles engendrent $\mathcal{A}_{n}$.

		\item On a 
		$$\sigma\circ(a_{1},a_{2},a_{3})\circ\sigma^{-1}=(\sigma(a_{1}),\sigma(a_{2}),\sigma(a_{3}))$$
		On peut trouver $\sigma\colon\{1,\dots,n\}\to\{1,\dots,n\}$ telle que $a_{i}$ soit envoyé sur $b_{i}$ pour $i\in\{1,2,3\}$ et les éléments $\{1,\dots,n\}\setminus\{a_{1},a_{2},a_{3}\}$ dans $\{1,\dots,n\}\setminus\{b_{1},b_{2}b_{3}\}$.
		Donc les 3-cycles sont conjugués dans $\Sigma_{n}$.

		Si $n\geqslant5$ et $\sigma$ impair, soit $(c_{1},c_{2})\in\{1,\dots,n\}\setminus\{a_{1},a_{2},a_{3}\}$. $\sigma'=\sigma\circ[c_{1},c_{2}]$ est pair et $\sigma'(a_{i})=b_{i}$. Donc les trois cycles sont conjugués dans $\mathcal{A}_{n}$ pour $n\geqslant5$. C'est cependant faux pour $n=3$ et $n=4$.

		\item Soit $(\sigma,\sigma')\in\Sigma_{n}^{2}$. En notant $\mathcal{E}$ la signature d'une permutation (morphisme de $(\Sigma_{n},\circ)$ dans $(\{-1,1\},\times)$), on a
		$$\mathcal{E}(\sigma\circ\sigma^{-1}\circ\sigma'\circ\sigma'^{-1})=1$$
		donc $\sigma\circ\sigma^{-1}\circ\sigma'\circ\sigma'^{-1}\in\mathcal{A}_{n}$. Donc $D(\Sigma_{n})\subset\mathcal{A}_{n}$.

		Soit ensuite $(a_{1},a_{2},a_{3})$ un 3-cycle. On a $(a_{1},a_{3},a_{2})^{2}=(a_{1},a_{2},a_{3})$ et $(a_{1},a_{3},a_{2})^{-1}=(a_{1},a_{2},a_{3})$. Ainsi, on a 
		$$\sigma\circ(a_{1},a_{3},a_{2})\circ\sigma^{-1}\circ(a_{1},a_{2},a_{3})=(a_{1},a_{3},a_{2})^{2}=(a_{1},a_{2},a_{3})$$
		On pose $\sigma=[a_{2},a_{3}]$, et alors $(a_{1},a_{2},a_{3})$ est un commutateur. Ainsi, $(a_{1},a_{2},a_{3})\in D(\Sigma_{n})$ et donc $\mathcal{A}_{n}\subset D(\Sigma_{n})$ (d'après la première question).

		Finalement, on a $D(\Sigma_{n})=\mathcal{A}_{n}$.
	\end{enumerate}
\end{solution}

\begin{remark}
	Pour $n\geqslant5$, on a $D(\mathcal{A}_{n})=\mathcal{A}_{n}$.
\end{remark}

\begin{solution}
	\phantom{}
	\begin{enumerate}
		\item Pour $g\in G$, $\tau_{g}$ est bijective de réciproque $\tau_{g^{-1}}$. On a notamment $\tau_{g\cdot g'}=\tau_{g}\circ\tau_{g'}$ donc $\tau$ est un morphisme. Si $g\in G$ est tel que $\tau_{g}=id$, pour tout $x\in G$, on a $gx=x$ donc $g=e_{G}$. Donc $\tau$ est un morphisme injectif et $G$ est isomorphe à $\im\tau=\tau(G)$, sous-groupe de $\Sigma(G)$, lui-même isomorphe à $\Sigma_{n}$.
		
		\item Soit \function{f}{\Sigma_{n}}{GL_{n}(\C)}{\sigma}{(\delta_{i,\sigma(j)})_{1\leqslant i,j\leqslant n}=P_{\sigma}}
		$P_{\sigma}$ est la matrice de permutation associée à $\sigma$. $f$ est un morphisme, et est injectif, donc $G$ est isomorphe à un sous-groupe de $GL_{n}(\C)$.
	\end{enumerate}
\end{solution}

\begin{solution}
	Soit $(x,y,z,t)\in\N^{4}$ tel que $x^{2}+y^{2}+z^{2}=8t+7$. Dans $\Z/8\Z$, on a $\overline{0}^{2}=\overline{0}$, $\overline{1}^{2}=\overline{1}$, $\overline{2}^{2}=\overline{4}$, $\overline{3}^{2}=\overline{1}$, $\overline{4}^{2}=\overline{0}$, $\overline{5}^{2}=\overline{1}$, $\overline{6}^{2}=\overline{4}$ et $\overline{7}^{2}=\overline{1}$. Donc la somme de 3 de ces classes ne donnent pas $\overline{7}$.

	Par récurrence, prouvons la propriété. Soit $(x,y,z,t)\in\N^{4}$ tel que $x^{2}+y^{2}+z^{2}=(8t+7)4^{n+1}$. Parmi $x,y,z$ les trois sont pairs ou deux d'entre eux sont impairs. Si $x,y$ impairs et $z$ pair, on écrit $x=2x'+1,y=2y'+1,z=2z'$, alors $x^{2}+y^{2}+z^{2}\equiv 2[4]$ mais $(8t+7)4^{n+1}\equiv 0[4]$: contradiction. Nécessairement, $x,y$ et $z$ sont pairs. En divisant par $4$, on se ramène donc à l'hypothèse de récurrence.
\end{solution}

\begin{solution}
	On raisonne sur $\Z/7\Z$. On a $\overline{10^{10^{n}}}=\overline{3^{10^{n}}}$. Dans le groupe $((\Z/7\Z)^{*},\times)$, $\overline{3}$ a un ordre qui divise $\vert \Z/7\Z^{*}\vert=6$. On a $\overline{3}^{2}=\overline{2}$, $\overline{3}^{3}=\overline{-1}$ et $\overline{3}^{6}=\overline{1}$. Donc $\overline{3}^{6k}=\overline{1}, \overline{3}^{6k+1}=\overline{3},\overline{3}^{6k+2}=\overline{2},\overline{3}^{6k+3}=\overline{-1}, 3^{6k+4}=\overline{4}$ et $3^{6k+5}=\overline{5}$..

	On se place maintenant dans $\Z/6\Z$: $\overline{10}=\overline{4},\overline{10}^{2}=\overline{4}$ et donc par récurrence sur $n\in\N^{*}$, $\overline{10}^{n}=\overline{4}$. Donc il existe $k\in\Z$ tel que $10^{n}=6k+4$. Ainsi, $\overline{10^{10^{n}}}=\overline{4}$.
\end{solution}

\begin{solution}
	\phantom{}
	\begin{enumerate}
		\item On a $F_{1}=5$ et $2+\prod_{k=0}^{0}F_{k}=2+3=5$. Soit $n\geqslant1$, supposons que $F_{n}=2+\prod_{k=0}^{n-1}F_{k}$. Alors 
		\begin{align*}
			F_{n+1}-2=2^{2^{n+1}}-1
			&=(2^{2^{n}})^{2}-1\\
			&=(2^{2^{n}}+1)(2^{2^{n}}-1)\\
			&=F_{n}(F_{n}-2)\\
			=F_{n}\times\prod_{k=0}^{n-1}F_{k}\\
			&=\prod_{k=0}^{n}F_{k}
		\end{align*}
		d'où le résultat par récurrence.

		\item Soit $p$ un facteur premier de $F_{n}$. S'il existe $k\in\{0,\dots,n-1\}$ tel que $p\mid F_{k}$, alors d'après la première question on a $p\mid F_{n}-\prod_{k=0}^{n-1}F_{k}=2$. Donc $p=2$. Or $F_{n}$ est impair, donc non divisible par deux, ce qui est absurde. Donc $p$ ne divise aucun $F_{k}$ pour $k\in\{0,n-1\}$ et il existe donc une infinité de nombres premiers (car les $F_{n}$ sont tous différents deux à deux).
	\end{enumerate}
\end{solution}

\begin{remark}
	Si $n\neq m$ alors $F_{n}\wedge F_{m}=1$.
\end{remark}

\begin{solution}
	\phantom{}
	\begin{enumerate}
		\item On teste uniquement les puissances qui divisent 32: 2,4,8,16,32. On a $\overline{5}^{2}=\overline{-7},\overline{5}^{4}=\overline{-15},\overline{5}^{8}=\overline{1}$. Donc $\omega(\overline{5})=8$.
		\item On note \function{\psi}{\Z/2\Z\times\Z/8\Z}{U}{(\dot{k},\tilde{l})}{\overline{-1}^{k}\times\overline{5}^{l}}
		\item On a $\omega(\overline{-1})=2$ et $\gamma(\overline{5})=8$ donc $\psi$ est bien définie. $\psi$ est bien un morphisme de groupes. Soit $(\dot{k},\tilde{l})\in\ker(\psi)$, on a $\overline{-1}^{k}\times \overline{5}^{l}=\overline{1}$. Si $\dot{k}=\dot{1}$, alors $\overline{-1}^{k}=\overline{-1}=\overline{5}^{-l}=\overline{5}^{l}\in gr\{\overline{5}\}$. Donc $\overline{5}^{2l}=\overline{1}$ et ainsi $8\mid 2l$ d'où $4\mid l$. Mais alors $l\in\{0,4\}$ ce qui est impossible. Donc $\dot{k}\neq\dot{1}$. De ce fait, $\dot{k}\neq\dot{1}$. Ainsi, $\overline{5}^{l}=\overline{1}$ donc $\tilde{l}=\tilde{0}$. Ainsi, $\ker(\psi)=\{(\dot{0},\tilde{0})\}$ donc $\psi$ est injective, puis bijective car $\vert\Z/2\Z\times\Z/8\Z\vert=\vert U\vert$.
	\end{enumerate}
\end{solution}

\begin{remark}
	$U$ n'est pas cyclique car, par isomorphisme, ses éléments ont un ordre qui divise 8.
\end{remark}

\begin{solution}
	\phantom{}
	\begin{enumerate}
		\item Soit \function{f}{G_{n}\times G_{m}}{U_{nm}}{(\xi,\xi')}{\xi\times\xi'}
		Soit $(\xi,\xi')\in G_{n}\times G_{m}$, Soit $k\in\Z$ tel que $(\xi\times\xi')^{k}=1$. Alors $(\xi\times\xi')^{km}=1$ d'où $\xi^{km}=1$ donc $n\mid km$ et $n\mid k$ d'après le théorème de Gauss. De même pour $n$, on a $m\mid k$ et donc $nm\mid k$. La réciproque est immédiate: $\xi\times\xi'\in G_{nm}$. Donc $f(G_{n}\times G_{m})\subset G_{nm}$ et $\vert G_{n}\times G_{m}\vert=\varphi(n)\times\varphi(m)=\varphi(nm)=\vert G_{nm}\vert$ où $\varphi$ est l'indicatrice d'Euler.

		Montrons que $f$ est injective: soit $(x,y,x',y')\in G_{n}^{2}\times G_{m}^{2}$ tel que $xx'=yy'$. On a alors $x^{m}=y^{m}$ et $x'^{n}=y'^{n}$ d'où $(xy^{-1})^{m}=1$ d'où $\omega(xy^{-1})\mid m$ et $\omega(xy^{-1})\mid n$. Donc $\omega(xy^{-1})=1$ donc $x=y$ et en reportant, on a $x'=y'$. Donc $f$ est injective puis bijective (égalité des cardinaux).

		On a alors 
		\begin{align*}
			\mu(n)\mu(m)
			&=\sum_{\xi\in G_{n}}\xi\times\sum_{\xi'\in G_{m}}\xi'\\
			&=\sum_{(\xi,\xi')\in G_{n}\times G_{m}}\xi\xi'\\
			&=\sum_{\xi\in G_{nm}}\xi\text{ [f est bijective]}\\
			&=\mu(nm)
		\end{align*}

		\item On a $\mu(1)=1$. Soit $p$ premier. On a 
		$$\sum_{k=0}^{p-1}e^{\frac{2\mathrm{i}k\pi}{p}}=0$$ 
		donc 
		$$\mu(p)\sum_{k=1}^{p-1}e^{\frac{2\mathrm{i}k\pi}{p}}=-1$$. Soit alors $\alpha\in\N$ avec $\alpha\geqslant2$, on a 
		$$\mu(p^{\alpha})=\sum_{\substack{k=1\\ k\wedge p=1}}^{p^{\alpha}}e^{\frac{2\mathrm{i}k\pi}{p^{\alpha}}}=\sum_{k=1}^{p^{\alpha}}e^{\frac{2\mathrm{i}k\pi}{p^{\alpha}}}-\sum_{k=1}^{p^{\alpha-1}}e^{\frac{2\mathrm{i}k\pi}{p^{\alpha-1}}}=0$$

		Si $n=p_{1}^{\alpha_{1}}\dots p_{r}^{\alpha_{r}}$, s'il existe $i\in\{1,\dots,r\}$ tel que $\alpha_{i}\geqslant2$ alors $\mu(n)=0$. Sinon, on a 
		$$\mu(n)=\prod_{i=1}^{r}\mu(p_{i})=(-1)^{r}$$

		\item Soit $(f,g)\in(\C^{\N^{*}})^{2}$, on a 
		\begin{align*}
			(f\star g)(n)
			&=\sum_{d_{1}d_{2}=n}f(d_{1})g(d_{2})\\
			&=\sum_{d_{1}d_{2}=n}g(d_{1})f(d_{2})\\
			&=(g\star f)(n)
		\end{align*}
		Donc $\star$ est commutative. 

		Soit $(f,g,h)\in(\C^{\N^{*}})^{3}$, on a 
		\begin{align*}
			(f\star (g\star h))(n)
			&=\sum_{d_{1}d=n}f(d_{1})(g\star h)(d)\\
			&=\sum_{d_{1}d=n}\Biggl[f(d_{1})\times \sum_{d_{2}d_{3}=d}g(d_{2})h(d_{3})\Biggr]\\
			&=\sum_{d_{1}d_{2}d_{3}=n}f(g_{1})g(d_{2})h(d_{3})\\
			&=((f\star g)\star h)(n)
		\end{align*}
		donc $\star$ est associative. 

		On vérifie maintenant que l'élément neutre est $e:\N^{*}\to\C$ qui à $1$ associe $1$ et 0 si $n\geqslant2$.
		Soit \function{\psi}{\N}{\Z}{n}{\sum_{d\mid n}\mu(d)}
		On a $\psi(1)=1$. Soit $n\geqslant2$ avec $n=\prod_{i=1}^{r}p_{i}^{\alpha_{r}}$. Les diviseurs de $n$ sont dans $D=\{\prod_{i=1}^{r}p_{i}^{\beta_{i}}\Bigm| \beta_{i}\leqslant\alpha_{i}\}$. Ainsi, $\psi(n)=\sum_{d\in D}\mu(d)$. Or $\mu(d)$ vaut 0 s'il existe $\beta_{i}\geqslant2$ et $(-1)^{k}$ si $k$ $\beta_{i}$ valent 1 et les autres 0. Il y a $\binom{r}{k}$ choix possibles pour que $k$ $\beta_{i}$ valent $1$. Ainsi,
		$$\psi(n)=\sum_{k=0}^{r}1^{r-k}(-1)^{k}\binom{r}{k}=0$$
		Donc $\mu\star 1=e$, et $\mu^{-1}=1\colon n\mapsto 1$ pour tout $n\in\N$.

		\item On note \function{id}{\N^{*}}{\N^{*}}{n}{n}
		Alors 
		\begin{align*}
			\sum_{d\mid n}d\mu(\frac{n}{d})
			&=(\mu\star id)(n)\\
			&=(id\star \mu)(n)\\
			&=(1\star (\varphi\star \mu))(n)\\
			&=\varphi(n)
		\end{align*}
		la troisième égalité venant du fait que $id=1\star\varphi$ car $n=\sum_{d\mid n}\varphi(d)$.
	\end{enumerate}
\end{solution}

\begin{solution}
	Pour $k\in\{1,\dots,p-1\}$, on a 
	$$\binom{p+k}{k}=\frac{(p+k)\times\dots\times(p+1)}{k\times\dots\times 1}=1+\alpha kp$$
	car $(p+k)\times\dots\times(p+1)=k!+p\times\text{qqchose}$. On a $p\mid\binom{p}{k}$ donc 
	$$\sum_{k=1}^{p-1}\binom{p}{k}\binom{p+k}{k}\equiv\sum_{k=1}^{p-1}\binom{p}{k}[p^{2}]$$

	Pour $k=0$, on a $\binom{p}{0}\binom{p}{0}=1$ et pour $k=p$, on a $\binom{p}{p}\binom{2p}{p}=\binom{2p}{p}$. Et 
	$$\sum_{k=1}^{p-1}\binom{p}{k}=\sum_{k=0}^{p}\binom{p}{k}-2=2^{p}-2$$
	Il reste donc à prouver que $\binom{2p}{p}\equiv 2[p^{2}]$.

	Or 
	$$\binom{2p}{p}=\sum_{k=0}^{p}\binom{p}{k}\binom{p}{p-k}\equiv2[p^{2}]$$
	la première égalité venant de l'égalité du terme en $X^{p}$ dans $(1+X)^{2p}=(1+X)^{p}(1+X)^{p}$, et la deuxième venant du fait que seuls les termes en $k=0$ et $k=p$ ne contiennent pas de $p^{2}$, et valent chacun 1.

	Finalement, on a 
	$$\sum_{k=0}^{p}\binom{p}{k}\binom{p+k}{k}\equiv 2^{p}-2+1+2[p^{2}]\equiv 2^{p}+1[p^{2}]$$
\end{solution}

\begin{solution}
	\phantom{}
	\begin{enumerate}
		\item Soit $G$ un sous-groupe de $(\U,\times)$. On note $\vert G\vert=d$. On a donc $G\subset\U_{d}$ car pour tout $x\in G$, $x^{d}=1$. Donc $G=\U_{d}$ est cyclique.
		
		\item On pose \function{\psi}{SO_{2}(\R)}{(\U,\times)}{R_{\theta}}{e^{\mathrm{i}\theta}}
		qui est un isomorphisme. Donc les sous-groupes de $SO_{2}(\R)$ sont les $G_{n}$ pour $n\geqslant1$ avec 
		$$G_{n}=\{R_{\frac{2k\pi}{n}}\bigm| k\in\{0,\dots,n-1\}\}$$

		\item $\varphi$ est bilinéaire et symétrique. Pour tout $X\in\R^{2}$, on $\varphi(X,X)=\sum_{M\in G}\Vert MX\Vert^{2}\geqslant0$ et si $\varphi(X,X)=0$, on a pour tout $M\in G$, $X=0$. Notamment, $I_{2}\in G$ et donc $X=0$. Donc $\varphi$ est bien un produit scalaire.
		
		Pour tout $(M_{0},X,Y)\in G\times(\R^{2})^{2}$, on a $\varphi(M_{0}X,M_{0}Y)=\sum_{M\in G}\langle MM_{0}X,MM_{0}Y\rangle$
		et $M\mapsto MM_{0}$ est bijective de $G$ dans $G$ donc $\varphi(M_{0}X,M_{0}Y)=\varphi(X,Y)$.

		Soit $\mathcal{B}_{0}$ la base canonique de $\R^{2}$ et $\mathcal{B}_{1}$ une base orthonormée pour $\varphi$. On note $P_{0}=\mat\limits_{\mathcal{B}_{0}\to \mathcal{B}_{1}}$. 
		
		Pour tout $M\in G$, $P_{0}^{-1}MP_{0}$ est la matrice d'une isométrie pour $\varphi$ dans une base orthonormée pour $\varphi$. Donc $P_{0}^{-1}MP_{0}$ est orthogonale, et $\det(P_{0}^{-1}MP_{0})=1$ car pour tout $M\in G$, $\det(M)=1$. Ainsi, $\{P_{0}^{-1}MP_{0}\bigm| M\in G\}$ est un sous-groupe fini de $SO_{2}(\R)$, donc cyclique. Il est isomorphe à $G$ donc $G$ est cyclique.
	\end{enumerate}
\end{solution}

\begin{solution}
	\phantom{}
	\begin{enumerate}
		\item On a $1=1+0\sqrt{2}\in E$. On remarque ensuite que pour tout $s=x+y\sqrt{2}\in E$, on a $ss^{-1}=1$ avec $s^{-1}=x-y\sqrt{1}\in E$. Soit $(s,s')\in E^{2}$ avec $s=x+y\sqrt{2}$ et $s'=x'+y'\sqrt{2}$. Notons déjà que $x+y\sqrt{2}>0$ car $x=\sqrt{1+2y^{2}}>\vert y\vert\sqrt{2}$.
		On a donc
		$$ss'=\underbrace{xx'+2yy'}_{\in\Z}+\sqrt{2}\underbrace{(yx'+y'x)}_{\in\Z}$$
		On a $xx'\in\N$ et $x>\sqrt{2}\vert y\vert\geqslant0$ et $x'>\sqrt{2}\vert y'\vert\geqslant0$ donc $xx'>2\vert yy'\vert$ et ainsi $xx'+2yy'\in\N^{*}$. Enfin, on a 
		\begin{align*}
			(xx'+2yy')^{2}-2(yx'+y'x)^{2}
			&=(xx')^{2}+4(yy')^{2}-2(yx')^{2}2(y'x)^{2}\\
			&=(x^{2}-2y^{2})(x'^{2}-2y'^{2})\\
			&=1
		\end{align*}
		Donc $ss'\in E$. Finalement, $E$ est un sous-groupe de $(\R_{+}^{*},\times)$.

		\item $\ln$ est un isomorphisme de $E$ sur $\ln(E)$, sous-groupe de $(\R,+)$. On sait que si 
		$$\underbrace{\inf(\ln(E)\cap\R_{+})}_{\alpha}>0$$
		alors $\ln(E)=\alpha\Z$ (sous-groupe de $(\R,+)$ dans le cas $\alpha>0$, pour rappel si $\alpha=0$ alors le sous-groupe est dense dans $\R$). On cherche la borne inférieure de $E\cap]1+\infty[$ que l'on note $\beta$. $\beta$ existe car cet ensemble est non vide, par exemple $3+2\sqrt{2}$ y appartient.
		
		Si $\beta=1$, on peut trouver une suite de termes de $E$ strictement décroissante convergeant vers 1. Alors pour tout $n\in\N$, on a 
		$$1<x_{n+1}+y_{n+1}\sqrt{2}<x_{n}+y_{n}\sqrt{2}$$
		On sait que 
		$$x_{n}-y_{n}\sqrt{2}=(x_{n}+y_{n}\sqrt{2})^{-1}<1<x_{n}+y_{n}\sqrt{2}$$
		donc $-y_{n}\sqrt{2}<1-x_{n}<0$ donc $y_{n}>0$. Ainsi, 
		$$y_{n}=\sqrt{\frac{x_{n}^{2}-1}{2}}$$
		Si $x_{n+1}\geqslant x_{n}$, alors $y_{n+1}\geqslant y_{n}$ d'où $x_{n+1}+\sqrt{2}y_{n+1}>x_{n}+\sqrt{2}y_{n}$ ce qui est absurde. Donc $x_{n+1}<x_{n}$ et on obtient une suite strictement décroissante d'entiers naturels ce qui est impossible. Donc $\beta>1$ et $E=\{(x_{0}+y_{0}\sqrt{2})^{n}\bigm| n\in\Z\}$ est monogène.

		On peut identifier $\beta$:
		$$x_{0}=\min\{x\in\N^{*}\setminus\{1\},\exists y\in\Z,x+y\sqrt{2}\in E\cap],+\infty[\}$$
		Donc $\beta=3+2\sqrt{2}$ Finalement, $x^{2}-2y^{2}=1$ avec $x\in\N,y\in\N$ si et seulement s'il existe $n\in\N$ tel que $x_{n}+y_{n}\sqrt{2}=\beta^{n}$.
	\end{enumerate}
\end{solution}

\begin{remark}
	En fait, on a 
	$$
	\left\{
		\begin{array}[]{lcl}
			x_{n} &= &\sum\limits_{k=0}^{\lfloor \frac{n}{2}\rfloor}\binom{n}{2k}2^{2k}3^{n-2k}\\[0.5cm]
			y_{n} &= &\sum\limits_{k=0}^{\lfloor \frac{n-1}{2}\rfloor}\binom{n}{2k+1}2^{2k+1}3^{n-2k-1}
		\end{array}	
	\right.
	$$
\end{remark}

\begin{solution}
	On a $7\mid n^{n}-3$ si et seulement si $\overline{n}^{n}=\overline{3}$ dans $\Z/7\Z$. $(\Z/7\Z^{*},\times)$ est un groupe de cardinal 6. Donc l'ordre de ses éléments divisent 6, et sont donc 1,2,3 ou 6. Notamment, on vérifie que $\omega(\overline{3})=6$ et donc le groupe engendré par $\overline{3}$ est exactement $(\Z/7\Z^{*},\times)$. Ainsi, 
	$$(\Z/7\Z^{*},\times)=\{\overline{3}^{k\bigm| k\in\{0,\dots,5\}}\}$$
	(c'est un groupe cyclique). Les générateurs sont $\{\overline{3}^{k},k\wedge 6=1\}=\{\overline{3},\overline{3}^{5}=\overline{-2}=\overline{5}\}$.
	Donc $\overline{n}=\overline{3}$ ou $\overline{n}=\overline{5}$.

	Si $\overline{n}=3$, $\overline{3}^{n}=\overline{3}$ si et seulement si $n\equiv1[6]$ donc $n\equiv 3[7]$ et $n\equiv1[6]$. D'après le théorème des restes chinois, on vérifie que ceci équivaut à $n\equiv31[42]$. La réciproque est immédiate.

	Si $\overline{n}=5$, $\overline{5}^{n}=\overline{3}$ si et seulement si $n\equiv5[6]$ et $n\equiv5[7]$. D'après le théorème des restes chinois, on vérifie que ceci équivaut à $n\equiv5[42]$.

	Donc les solutions sont $n\in\N^{*}$ tels que $n\equiv31[42]$ ou $n\equiv5[42]$.
\end{solution}

\begin{solution}
	On a 
	\begin{align*}
		\sum_{k=1}^{p-1}\frac{1}{k}+\frac{1}{p-k}=\frac{2a}{(p-1)!}
		&\Longleftrightarrow \sum_{k=1}^{p-1}\frac{p}{k(p-k)}=\frac{2a}{(p-1)!}\\
		&\Longleftrightarrow \sum_{k=1}^{p-1}\frac{p(p-1)!}{k(p-k)}=2a\\
		&\Longleftrightarrow p\underbrace{\sum_{k=1}^{p-1}\frac{(p-1)!^{3}}{k(p-k)}}_{\in\N}=2a\underbrace{(p-1)!^{2}}_{p\wedge (p-1)!^{2}=1}
	\end{align*}
	donc $p\mid a$ d'après le théorème de Gauss.

	On écrit alors $a=p\times b$ avec $b\in\N$. On a alors
	$$\sum_{k=1}^{p-1}\frac{1}{k(p-k)}=\frac{2b}{(p-1)!}$$
	comme $(p-1)!,k$ et $p-k$ ($1\leqslant k\leqslant p$) sont inversibles dans $\Z/p\Z$, on a alors
	$$\sum_{k=1}^{p-1}\overline{-k}^{-2}=\overline{2b}\times\underbrace{\overline{(p-1)!}^{-1}}_{=\overline{-1}\text{ d'après le théorème de Wilson}}$$
	Donc 
	$$\overline{2b}=\sum_{k=1}^{p-1}\overline{k}^{-2}$$
	Comme \function{f}{\Z/p\Z^{*}}{\Z/p\Z^{*}}{\overline{k}}{\overline{k}^{-1}}
	est bijective, on a 
	$$\overline{2}\times\overline{b}=\sum_{k=1}^{p-1}\overline{k}^{2}=\overline{\frac{p(p-1)(2p-1)}{6}}$$
	Or $p\geqslant5$ est premier, donc $p-1$ est pair et $p$ est congru ) 1 ou 2 modulo 3. Donc $p-1\equiv0[3]$ ou $2p-1\equiv0[3]$ donc $\frac{(p-1)(2p-1)}{6}\in\N$. Ainsi, 
	$$\overline{2}\times\overline{b}=\sum_{k=1}^{p-1}\overline{k}^{2}=\overline{p}\times\overline{\frac{(p-1)(2p-1)}{6}}=0$$
	et donc $p\mid b$ par le théorème de Gauss. Donc $p^{2}\mid a$.
\end{solution}

\begin{solution}
	Les racines réelles de $P$ ont une multiplicité paire, le coefficient dominant est positif (car la limite en $+\infty$ est positive) et les racines complexes non réelles sont 2 à 2 conjuguées:
	$$(X-\alpha)(X-\overline{\alpha})=X^{2}-2\Re(\alpha)X+\vert\alpha\vert^{2}=(X-\Re(\alpha))^{2}+\vert\Im(\alpha)\vert^{2}$$
	avec $\Im(\alpha)\neq0$.
	D'où le résultat en décomposant $P$ sur $\C[X]$.
\end{solution}

\begin{solution}
	\phantom{}
	\begin{enumerate}
		\item $G=\Z+\alpha\Z$ est un sous-groupe de $\R$ engendré par $\alpha$ et $1$. S'il existait $a\in\R_{+}^{*}$ tel que $G=a\Z$, alors il existait $(n,m)\in(\Z^{*})^{2}$ tel que $1=na$ et $\alpha=ma$, d'où $\alpha=\frac{m}{n}\in\Q$ ce qui est absurde. Donc $G$ est dense dans $\R$. Le fait que $\Z+\alpha\N$ est dense dans $\R$ est alors immédiate.
		\item Posons $\beta=\frac{\alpha}{2\pi}\notin\Q$. Alors $\Z+\beta\N$ est dense dans $\R$. Soit $c<d\in\R^{2}$. Comme $\frac{c}{2\pi}<\frac{d}{2\pi}$, il existe $x\in\Z+\beta\N\cap]\frac{c}{2\pi},\frac{d}{2\pi}[$ et alors $2\pi x\in2\pi\Z+\alpha\N\cap]c,d[$. On pose $c=\arcsin(a)$ et $d=\arcsin(b)$ avec $a<b$. On a bien $c<d$ car $\arcsin$ est strictement croissante.
		
		Alors il existe $(m,n)\in\Z\times\N$ tel que $2\pi m+\alpha m=2\pi x\in]c,d[$ donc $\sin(2\pi x)=\sin(2\pi m+\alpha n)=\sin(\alpha n)\in]a,b[$.

		Donc $(\sin(n\alpha))_{n\in\N}$ est dense dans $]-1,1[$. En particulier, cela vaut pour $\alpha=1$ car $\pi\notin\Q$. Donc $(\sin(n))_{n\in\N}$ est dense dans $[-1,1]$.

		\item Soit $n\in\N$. $2^{n}$ commence par 7 en base 10 si et seulement s'il existe $p\in\N$ avec 
		\begin{align*}
			7\times10^{p}\leqslant2^{n}<8\times10^{p}
			&\Longleftrightarrow \ln(7)+p\ln(10)\leqslant n\ln(2)<\ln(8)+p\ln(10)\\
			&\Longleftrightarrow \frac{\ln(7)}{\ln(10)}\leqslant\frac{n\ln(2)}{\ln(10)}-p<\frac{\ln(8)}{\ln(10)}
		\end{align*}
		On a alors 
		$$p=\Bigl\lfloor\frac{n\ln(2)}{\ln(10)}\Bigr\rfloor\in\N$$

		On étudie donc $\N\frac{\ln(2)}{\ln(10)}+\Z$. Supposons que $\frac{\ln(2)}{\ln(10)}=\frac{p}{q}\in\Q$. Alors on a $2^{q}=10^{p}$ mais comme $p\neq0$, on a $5\mid 10^{p}$ mais $5\nmid 2^{q}$, donc $\frac{\ln(10)}{\ln(2)}\notin\Q$.

		On sait que 
		$$u_{n}=n\frac{\ln(2)}{\ln(10)}-\Bigl\lfloor\frac{n\ln(2)}{\ln(10)}\Bigr\rfloor\in\Bigl]\frac{\ln(7)}{\ln(10)},\frac{\ln(8)}{\ln(10)}\Bigr[$$
		Par densité, on peut donc construire par récurrance $(u_{n_{p}})_{p\in\N}$ telle que 
		$$\frac{\ln(7)}{\ln(10)}<u_{n_{p+1}}<u_{n_{p}}<\frac{\ln(8)}{\ln(10)}$$
		Donc on a bien une infinité de puissance de 2 commençant par 7 en base 10.
	\end{enumerate}
\end{solution}

\begin{remark}
	$(e^{\mathrm{i}n\alpha})_{n\in\N}$ est de la même façon dense dans $\U$. On peut montrer qu'elle est équirépartie, c'est à dire que pour tout $a<b\in[0,2\pi[^{2}$, on a 
	$$\lim\limits_{N\to+\infty}\Bigl\vert\Bigl\{n\in\{1,\dots,N\}\Bigm| n\alpha-\frac{\lfloor 2\pi n\alpha\rfloor}{2\pi}\in]a,b[\Bigr\}\Bigr\vert\times\frac{1}{N}=\frac{b-a}{2\pi}$$
\end{remark}

\begin{remark}
	Par équirépartition dans $[0,1[$ des 
	$$\Bigl\{n\frac{\ln(2)}{\ln(10)}-\Bigl\lfloor\frac{n\ln(2)}{\ln(10)}\Bigr\rfloor\bigm| n\in\N\Bigr\}$$
	la probabilité pour qu'une puissance de $2$ commence par $k$ en base $10$ est ($k\in\{1,\dots,9\}$)
	$$\frac{\ln(k+1)-\ln(k)}{\ln(10)}=\frac{\ln(1+\frac{1}{k})}{\ln(10)}$$
\end{remark}

\begin{solution}
	\phantom{}
	\begin{enumerate}
		\item Pour $\alpha=a+\i b$, on définit le module au carré: $\vert\alpha\vert^{2}=a^{2}+b^{2}$. Soit $\beta=c+\i d\neq0$. Si $\alpha=\beta q+r$ avec $q,r\in\Z[\i]^{2}$ et $\vert r\vert^{2}<\vert \beta\vert^{2}$, alors $\vert\alpha-\beta q\vert^{2}<\vert\beta\vert^{2}$ et $\beta\neq0$ donc
		$$\Bigl\vert\underbrace{\frac{\alpha}{\beta}}_{\in\C}-\underbrace{q}_{\in\Z[\i]}\Bigr\vert<\vert 1\vert$$
		On pose $\frac{\alpha}{\beta}=x+\i y$. On pose 
		$$
		u_{x}=
		\left\{
			\begin{array}[]{ll}
				\lfloor x\rfloor & \text{si }x\in[\lfloor x\rfloor,\lfloor x\rfloor+\frac{1}{2}[\\
				\lfloor x\rfloor+1 & \text{si }x\in[\lfloor x\rfloor+\frac{1}{2},\lfloor x\rfloor+1[
			\end{array}
		\right.
		$$
		et de même pour $u_{y}$. On a alors $q=u_{x}+\i u_{y}\in\Z[i]$ et 
		$$\Bigl\vert\frac{\alpha}{\beta}-q\Bigr\vert^{2}=\vert x-u_{x}\vert^{2}+\vert y-u_{y}\vert^{2}\leqslant2\times \Bigl(\frac{1}{2}\Bigr)^{2}=\frac{1}{2}<1$$
		On pose donc $r=\alpha-\beta q\in\Z[i]$ et ainsi l'anneau $\Z[i]$ est euclidien.

		\item Soit $A$ un anneau euclidien et $I$ un idéal de $A$ non réduit à $\{0\}$. Il existe $x\in I$ tel que 
		$$v(x_{0})=\min\{v(x)\bigm| x\in I\{0\}\}$$
		On a $x_{0}A\subset I$. Soit $x\in I$. Il existe $q,r\in A$ tel que 
		$$x=x_{0}q+r$$
		avec $v(r)<v(x_{0})$ ou $r=0$. Or $r\in I$ donc $r=0$. Ainsi $x\in x_{0}A$ et donc $I=x_{0}A$. Donc tout anneau euclidien est principal.
	\end{enumerate}
\end{solution}

\begin{remark}
	C'est encore vrai avec $\Z[\i\sqrt{2}]=\{a+\i b\sqrt{2}\bigm|(a,b)\in\Z^{2}\}$.
\end{remark}

\begin{solution}
	\phantom{}
	\begin{enumerate}
		\item Si $\overline{x}=\overline{y}^{2}$ est un carré, d'après le petit théorème de Fermat, on a $\overline{x}^{\frac{p-1}{2}}=\overline{y}^{p-1}=\overline{1}$. Soit \function{f}{\Z/p\Z^{*}}{\Z/p\Z^{*}}{\overline{y}}{\overline{y}^{2}}
		$f$ est un morphisme multiplicatif, $\im(f)$ est un sous-groupe de $\Bigl(\Z/p\Z^{*},\times\Bigr)$.

		Comme $\mathbb{F}_{p}$ est un corps, chaque carré possède exactement deux antécédents. Il y a $p-1$ antécédents, donc il y a $\frac{p-1}{2}$ carrés dans $\Z/p\Z^{*}$. Donc $\vert\im(f)\vert=\frac{p-1}{2}$ et si $\overline{x}$ est un carré, x est racine de $X^{\frac{p-1}{2}}-\overline{1}$. Le polynôme $X^{\frac{p-1}{2}}-\overline{1}$ possède au plus $\frac{p-1}{2}$ racines et tout carré est racine. Donc les racines sont exactement les carrés et $\overline{x}^{\frac{p-1}{2}}=\overline{1}$ si et seulement si $\overline{x}$ est un carré.

		\item On a $p\equiv1[4]$ si et seulement si $\frac{p-1}{2}$ est pair si  et seulement si $(\overline{-1})^{\frac{p-1}{2}}=\overline{1}$ si et seulement si $\overline{-1}$ est un carré dans $\mathbb{F}_{p}$.
		Supposons qu'il y ait un nombre fini de nombres premiers $p_{1},\dots,p_{r}$ tous congrus à 1 modulo 4. 
		On pose $n=(p_{1}\times\dots\times p_{r})^{2}+1$.
		Soit $p$ un facteur premier de $n$, on a $n\equiv 1[n_{i}]$ donc $p\notin\{p_{1},\dots,p_{r}\}$.
		Dans $\Z/p\Z$, on a $\overline{n}=\overline{0}$ donc $\overline{-1}=\overline{p_{1}\times\dots\times p_{r}}^{2}$ donc $p\equiv1[4]$ ce qui est une contradiction.

		Donc il y a une infinité de nombres premiers congrus à 1 modulo 4.
	\end{enumerate}
\end{solution}

\begin{solution}
	\phantom{}
	\begin{enumerate}
		\item On pose $P_{1}=\sum_{i=0}^{n}r'_{i}X^{i}$, et $\nu_{p}(r'_{i})$ est positif par définition de $c(P)$. Donc $P_{1}\in\Z[X]$.
		
		Pour tout $p\in\mathcal{P}$, il existe $i_{0}\in\{1,\dots,n\}$ tel que 
		$$\min\limits_{i\in\{1,\dots,n\}}\nu_{p}(r_{i})=\nu_{p}(r_{i_{0}})$$
		et $\nu_{p}(r'_{i_{0}})=0$ donc $p\nmid r'_{i_{0}}$ donc 
		$$\bigwedge_{i=1}^{n}r'_{i}=1$$

		Si on a $P=\alpha_{1}P_{1}=\alpha_{2}P_{2}$ avec les conditions requises, soit $p\in\mathcal{P}$, si $\nu_{p}(\alpha_{2})>\nu_{p}(\alpha_{1})$, alors $p$ divise tous les coefficients de $P_{1}$ ce qui n'est pas possible, donc $\nu_{p}(\alpha_{2})=\nu_{p}(\alpha_{1})$. Ceci étant vrai pour tout $p\in\mathcal{P}$, on a aussi $\alpha_{1}=\alpha_{2}$ et donc $P_{1}=P_{2}$.

		\item On a $P=c(P)P_{1}$ et $Q=c(Q)Q_{1}$ donc $PQ=c(P)c(Q)P_{1}Q_{1}$ et $P_{1}Q_{1}\in\Z[X]$.
		
		Soit $p\in\mathcal{P}$ divisant tous les coefficients de $P_{1}Q_{1}$. On définit, si $R=\sum_{i\in\N}\gamma_{i}X^{i}\in\Z[X]$, $\overline{R}=\sum_{i\in\N}\overline{\gamma_{i}}X^{i}\in\Z/p\Z[X]$. $R\mapsto\overline{R}$ est un morphisme d'anneaux. Par hypothèse, on a $\overline{P_{1}Q_{1}}=\overline{0}=\overline{P_{1}}\overline{Q_{1}}$ et par intégrité de $\Z/p\Z[X]$, on a $\overline{P_{1}}=\overline{0}$ ou bien $\overline{Q_{1}}=\overline{0}$, ce qui est exclu par les hypothèses. Donc $c(PQ)=c(P)c(Q)$.

		\item 
		Soit alors $P$ irréductible dans $\Z[X]$ (les inversibles de $\Z[X]$ étant -1 et 1). Posons 
		\begin{align*}
			P
			&=QR\in\Q[X]^{2}\\
			&=c(Q)c(R)\underbrace{Q_{1}R_{1}}_{\in\Z[X]}
		\end{align*}
		OR $c(Q)c(R)=c(P)$ c'après le lemme de Gauss et nécessairement, $c(P)=1$. Donc $P=Q_{1}R_{1}$, et alors $Q_{1}=\pm 1$ et $R_{1}=\pm 1$, et $Q$ ou $R$ est constant, donc $P$ est irréductible sur $\Q[X]$.

		Pour la réciproque, on a $2X$ est irréductible sur $\Q[X]$ car de degré 1, mais pas sur $\Z[X]$ car ni 2 ni $X$ ne sont inversibles.

		\item Soit $\theta=\frac{2\pi p}{q}$ avec $p\wedge q=1$ et $\cos(\theta)\in\Q$. Sur $\C[X]$, on a $P=(X-e^{\i\theta})(X-e^{-\i\theta})=X^{2}-2\cos(\theta)X+1\in\Q[X]$.
		
		Et $e^{\i\theta}\neq e^{-\i\theta}$ car $\theta\not\equiv0[\pi]$. On a $\theta=\frac{2\pi p}{q}$ donc $e^{\i\theta}\in \U_{q}$, et $e^{\i\theta}$ et $e^{-\i\theta}$ sont des racines de $A$. Donc, dans $\C[X]$, on a $P\mid A$ et $A\in\Q[X]$, donc il existe $B\in\Q[X]$ tel que 
		$$\underbrace{A}_{\in\Q[X]}=\underbrace{B}_{\in\C[X]}\times\underbrace{P}_{\in\Q[X]}$$
		Or $B$ s'obtient par la division euclidienne de $A$ par $P$, qui est indépendante du corps de référence, il vient $B\in\Q[X]$ et donc $A\mid P$ dans $\Q[X]$.

		On a $c(A)=1=c(B)c(P)$ et $A=c(B)c(P)B_{1}P_{1}=B_{1}P_{1}\in\Z[X]$ et le coefficient dominant de $A$ est donc 1. Donc le coefficient dominant de $B_{1}$ et de $P_{1}$ est aussi 1. En reportant, on a $P=P_{1}\in\Z[X]$.

		Donc $2\cos(\theta)\in\Z\cap[-2,2]$ donc $\cos\{\theta\}\in\{-\frac{1}{2},\frac{1}{2},0\}$ ($-1$ et $1$ ne peuvent y être car on a supposé $\theta\not\equiv0[\pi]$). Les solutions sont donc 
		$$\theta\in\Bigl\{0,\frac{\pi}{3},\frac{\pi}{2},\frac{2\pi}{3},\pi,\frac{4\pi}{3},\frac{3\pi}{2},\frac{5\pi}{3}\Bigr\}$$
		(en rajoutant $\theta=0$ et $\pi$).
	\end{enumerate}
\end{solution}

\begin{remark}
	On a $\frac{\arccos(\frac{1}{3})}{\pi}\notin Q$ car $\cos(\theta)=\frac{1}{3}$ n'est pas dans l'ensemble solutions.
\end{remark}

\begin{solution}
	\phantom{}
	\begin{enumerate}
		\item Soit $P=a\prod_{i=1}^{s}(X-a_{i})^{\alpha_{i}}$ avec les $a_{i}$ distincts et $\alpha_{i}\geqslant1$. $a_{i}$ est racine de $P'$ de multiplicité $\alpha_{i}-1$. Il manque donc $s$ racines. Si $\alpha=0$, le résultat est évident, sinon on pose \function{f}{\R}{\R}{x}{P(x)e^{\frac{x}{\alpha}}}
		et on a pour tout $x\in\R$,
		$$f'(x)=\frac{e^{\frac{x}{\alpha}}}{\alpha}(P(x)+\alpha P'(x))$$
		Comme $P$ est scindé sur $\R$, $P'$ est scindé sur $\R$ (appliquer le théorème de Rolle entre les racines distinctes de $P$), donc $f'$ s'annule $s-1$ fois entre les racines de $P$ donc $P+\alpha P'$ aussi. 

		La dernière racine est réelle car sinon, le conjugué de la racine complexe supposée serait aussi racine.

		\item On pose $R=\mu\prod_{i=0}^{r}(X-\beta_{i})$. On pose \function{\Delta}{\R[X]}{\R[X]}{P}{P'}
		On a alors 
		$$\sum_{i=0}^{r}a_{i}P^{(i)}=\sum_{i=0}^{r}a_{i}\Delta^{i}(P)=R(\Delta)(P)=\mu\prod_{i=0}^{r}(\Delta-\beta_{i}id)(P)$$
		Par récurrence sur $r$, on montre que $\prod_{i=0}^{r}(\Delta-\beta_{i}id)(P)$ est scindé d'après la première question.
	\end{enumerate}
\end{solution}

\begin{remark}
	On a aussi pour tout $\lambda\in\R$, $P'+\lambda P$ est aussi scindé sur $\R$ si $P$ est scindé sur $\R$.
\end{remark}

\begin{solution}
	Soit $F=\frac{P'}{P}$ définie sur $\R\setminus\{a_{1},\dots,a_{n}\}$ où $a_{i}$ sont les racines de $P$. On note $\alpha$ le coefficient dominant de $P$, et on a 
	$$P'=\alpha\sum_{i=1}^{n}\Bigl(\prod_{\substack{j=1\\ j\neq i}}^{n}(X-a_{j})\Bigr)$$
	On a donc $F=\sum_{i=1}^{n}\frac{1}{X-a_{i}}$ et on a 
	$$F'=-\sum_{i=1}^{n}\frac{1}{(X-a_{i})^{2}}=\frac{P''P-P'P'}{P^{2}}$$

	Pour $x\notin\{a_{1},\dots,a_{n}\}$, on a 
	\begin{align*}
		(n-1)(P'^{2}(x))(x)\geqslant nP(x)P''(x)
		&\Longleftrightarrow n(P''(x)P(x)-P'^{2}(x))\leqslant-P'^{2}(x)\\
		&\Longleftrightarrow \frac{P'^{2}(x)}{P^{2}(x)}\leqslant n(P''(x)P(x)-P'^{2}(x))\times\frac{1}{P^{2}(x)}\\
		&\Longleftrightarrow F^{2}(x)\leqslant n(-F'(x))\\
		&\Longleftrightarrow\Bigl(\sum_{i=1}^{n}\frac{1}{(X-a_{i})}\Bigr)^{2}\leqslant n\times\sum_{i=1}^{n}\frac{1}{(X-a_{i})^{2}}
	\end{align*}
	qui est l'inégalité de Cauchy-Schwarz dans $\R^{2}$ avec $(1\dots 1)$ et $(\frac{1}{x-a_{1}}\dots\frac{1}{x-a_{n}})$.
\end{solution}

\begin{remark}
	Si $P=\alpha(X-a_{1})^{m_{1}}(X-a_{r})^{m_{r}}$, alors 
	$$\frac{P'}{P}=\sum_{i=1}^{r}\frac{m_{i}}{X-a_{i}}$$
\end{remark}

\begin{solution}
	\phantom{}
	\begin{enumerate}
		\item $P'\in\C[X]$ et $\deg(P')=\deg(P)-1$. On a $P\wedge P'=1$ car $P$ est irréductible sur $\Q[X]$. Comme le pgcd est obtenu par l'algorithme d'Euclide qui est indépendant du corps de référence, on a $P\wedge P'=1$ sur $\C[X]$ donc $P$ n'a que des racines simples sur $\C$.
		\item Notons $P\in\Q[X]$ le polynôme minimal de $\alpha$ sur $\Q$ (défini car $A(\alpha)=0$ donc $\alpha$ est algébrique). Comme $A(\alpha)=0$, on a $P\mid A$ et $P$ est irréductible sur $\Q[X]$. Si $\alpha\notin\Q$, on a $\deg(P)\geqslant2$, on peut donc décomposer sur $\Q[X]$:
		$$A=P^{r}\times P_{1}^{r_{1}}\times\dots P_{s}^{r_{s}}$$
		avec les $P_{i}$ irréductibles sur $\Q[X]$ non associés.

		$\alpha$ n'est pas racine d'un $P_{i}$ car sinon $P\mid P_{i}$ ce qui est impossible. $\alpha$ est racine simple de $P$ donc $m(\alpha)=r>\frac{\deg(A)}{2}$. Par ailleurs, $\deg(P)^{r}\geqslant2r>\deg(A)$ ce qui est impossible.

		Donc $\alpha\in\Q$.
	\end{enumerate}
\end{solution}

\begin{solution}
	Soit $x\in A$. Il existe $(n,m)\in\N^{2}$ avec $n<m$ tel que $x^{n}=x^{m}$. Alors $x^{m-n}=e_{G}\in A$.
	\function{f}{\N^{*}}{A}{n}{x^{n}}
	n'est pas injective, car $\N^{*}$ est infini et $A$ est fini. Or $m-n\in\N^{*}$ donc
	$$x^{m-n}=e_{G}\Rightarrow x=x\cdot x^{m-n-1}=e_{G}$$
	donc $x^{-1}=x^{m-n-1}\in A$ et ainsi $A$ est un sous-groupe.
\end{solution}

\begin{solution}
	Pour $\alpha=0$, on a $1+p\equiv 1+p[p^{2}]$. Pour $\alpha=1$, on a 
	$$(1+p)^{p}=\sum_{k=0}^{p}\binom{p}{k}p^{k}=1+p^{2}+\binom{p}{2}p^{2}\sum_{k=3}^{p}\binom{p}{k}p^{k}$$
	Or $\binom{p}{2}p^{2}=\frac{p(p-1)p^{2}}{2}\equiv0[p^{3}]$ car $p$ est premier plus grand que trois donc impair, et la somme est aussi congru à 0 modulo $p^{3}$.

	Soit $\alpha\geqslant1$, supposons que l'on ait 
	$$(1+p)^{p}\equiv 1+p^{\alpha+1}[p^{\alpha+2}]$$
	Il existe $l\in\N$ tel que 
	$$(1+p)^{p^{\alpha}}=1+p^{\alpha+1}+lp^{\alpha+2}$$
	Alors 
	$$(1+p)^{p^{\alpha+1}}=(1+\underbrace{p^{\alpha+1}+lp^{\alpha+2}}_{x})^{p}$$
	Or
	$$(1+x)^{p}=\sum_{k=0}^{p}\binom{p}{k}x^{k}=1+px+\sum_{k=2}^{p}\binom{p}{k}x^{k}=1+p^{\alpha+2}+lp^{\alpha+3}+\underbrace{\sum_{k=2}^{p}\binom{p}{k}x^{k}}_{\text{divisible par }x^{2}}$$
	Comme $p^{\alpha+1}\mid x$, $p^{2\alpha+2}\mid x^{2}$ avec $2\alpha+2\geqslant\alpha+3$ ($\alpha\geqslant1)$. D'où 
	$$p^{\alpha+3}\Bigm| x^{2}\Bigm|\sum_{k=2}^{p}\binom{p}{k}x^{k}$$
	et
	$$(1+p)^{p^{\alpha+1}}\equiv1+p^{\alpha+2}[p^{\alpha+3}]$$
\end{solution}

\begin{remark}
	Pour $p=2,\alpha=1$, on a $3^{2}=9\not\equiv 5[8]$.
\end{remark}

\begin{solution}
	Si $7=2x^{2}-5y^{2}$, on a $\overline{0}=2\overline{x}^{2}-5\overline{y}^{2}=\overline{2}(\overline{x}^{2}+\overline{y}^{2})$ dans $\Z/7\Z$. Comme 2 et 7 sont premiers entre eux donc $\overline{2}$ est inversible. Donc $\overline{x}^{2}+\overline{y}^{2}=\overline{0}$. La seule possibilité est $\overline{x}=\overline{0}$ et $\overline{y}=\overline{0}$. Donc $7\mid x$ et $y\mid y$. Si $x=7k$ alors $x^{2}=49k^{2}$ donc $49\mid x^{2}$ et $49\mid y^{2}$ donc $47\mid 2x^{2}-5y^{2}=7$ ce qui est faux.
\end{solution}

\begin{solution}
	$\mathbb{F}_{19}$ est un corps car 19 est premier. On a donc $\overline{x}^{3}=\overline{1}$ si et seulement si $(x-\overline{1})(x^{2}+x-\overline{1})=\overline{0}$. On a donc $x=\overline{1}$ ou $x^{2}+x+\overline{1}=\overline{0}$.
	On a 
	$$x^{2}+x+\overline{1}=(x+\overline{2}^{-1})^{2}+\overline{3}\times\overline{4}^{-1}=(x+\overline{10})^{2}+\overline{3}\times\overline{5}\overline{0}$$
	Donc $(x+\overline{10})^{2}=\overline{4}$ d'où $x=\overline{-8}=\overline{11}$ ou $x=\overline{-12}=\overline{7}$.
\end{solution}

\begin{solution}
	\phantom{}
	\begin{enumerate}
		\item $m$ est inversible si et seulement si $m\wedge 2^{n}=1$ si et seulement si $m\wedge 2=1$ si et seulement si $m$ est impair. IL y a donc $2^{n-1}$ inversibles.
		\item On a $5^{2^{3-3}}=5\equiv1+2^{2}[2^{3}]$. Par récurrence, soit $n\geqslant3$. Il existe $k\in\Z$ avec $5^{2^{n-3}}=1+2^{n-1}+k2^{n}$ donc 
		$$5^{2^{n-1}}=1+2^{n}+k2^{n+1}+2^{2n-2}(1+2k)^{2}\equiv 1+2^{n}[2^{n+1}]$$
		car $2n-2\geqslant n+1$ ($n\geqslant3)$.

		\item On a $5^{2^{n-2}}\equiv 1+2^{n}[2^{n+1}]\equiv 1[2^{n}]$ et $5^{2^{n-3}}\not\equiv1[2^{n}]$. Donc l'ordre de $\overline{5}$ est $2^{n-2}$. 
		
		\item $gr\{\overline{-1}\}=\{\overline{-1},\overline{1}\}$. $\overline{5}$ n'engendre pas $\overline{-1}$ car si $\overline{5}^{k}=\overline{-1}$, on a $\overline{5}^{2k}=\overline{1}$ d'où $2^{n-2}\mid 2k$ donc $2^{n-3}\mid k$. Ainsi, $k\in\{2^{n-3},2^{n-2},2^{n-1}\}$. Mais $\overline{5}^{2^{n-2}}=\overline{1},\overline{5}^{2^{n-3}}=\overline{1+2^{n-1}}\neq\overline{-1}$ donc un tel $k$ n'existe pas.
		
		Posons \function{\varphi}{\Bigl(\Z/2\Z\times\Z/2^{n-2}\Z, +\Bigr)}{\Bigl(\Z/2^{n}\Z^{\times},\times\Bigr)}{(\widetilde{a},\dot{b})}{\overline{-1}^{a}\overline{5}^{b}}
		Elle est bien définie car $\omega(\overline{-1})=2$ et $\omega(\overline{5})=2^{n-2}$. C'est évidemment un morphisme, on a égalité des cardinaux des ensembles de départ et d'arrivée, et on vérifie qu'elle est injective, et donc c'est un isomorphisme.
	\end{enumerate}
\end{solution}

\begin{solution}
	Soit $(x,x')\in G^{2}$ tel que $x\cdot x'=e$. Alors 
	$$e\cdot x=x\cdot x'\cdot x =x\cdot e\cdot x'\cdot x$$
	si et seulement si 
	$$e\cdot x\cdot x'=e=x\cdot e\cdot x'\cdot x\cdot x'=x\cdot e \cdot	x'$$

	Soit $(x,x',x'')\in G^{3}$ tel que $x\cdot x'=e$ et $x'\cdot x''=e$. On a alors 
	$$x\cdot x'\cdot x''=x\cdot e = x = e\cdot x''$$
	Donc $x=e\cdot x''$ et $e=e\cdot x''\cdot x'$. Si on prouve que $e\cdot x''=x''$, alors $x=x''$ et $x'\cdot x=e$.

	Montrons donc que pour tout $x\in G$, $e\cdot x=x$. Notons que s'il existe $e'\in G$ tel que pour tou t$x\in G$, $e'\cdot x=x$, alors $e'\cdot e=e'=e$.
	Il vient donc 
	$$x'\cdot x=x'\cdot e\cdot x''=x'\cdot x''=e$$
	Donc pour tout $x\in G$, l'élément $x'$ est inverse à droite et à gauche: $x\cdot x'=e$.

	Donc 
	$$x\cdot x'\cdot x=e\cdot x =x\cdot x'\cdot x=x\cdot e=x$$
	Et donc $e$ est neutre à gauche. Finalement, $(G,\cdot)$ est un groupe.
\end{solution}

\begin{remark}
	Si $f\colon\R\to\R$ est surjective, on peut définir \function{g}{\R}{\R}{y}{f(x)} pour un certain $x\in\R$. On a $f\circ g=id$. Si $f$ n'est pas injective: s'il existait $h\colon\R\to\R$ telle que $h\circ f=id$, soit $(x,x')\in \R^{2}$ telle que $f(x)=f(x')$. En composant par $h$, on aurait $x=x'$ donc $f$ serait injective ce qui n'est pas. 

	On on peut avoir un inverse à droite mais pas à gauche.
\end{remark}

\begin{solution}
	Soit $n\in\N^{*}$.
	$$\underbrace{1\dots 1}_{\text{n fois en base 10}}=1+10+\dots+10^{n-1}=\frac{10^{n}-1}{9}$$
	On a 
	\begin{align*}
		21\Bigm|\frac{10^{n}-1}{9}
		&\Longleftrightarrow 3\Bigm|\frac{10^{n}-1}{9}\text{ et }7\Bigm|\frac{10^{n}-1}{9}\\
		&\Longleftrightarrow 27\bigm|10^{n}-1\text{ et }7\bigm| 10^{n}-1
	\end{align*}
	car $7\wedge 9=1$.
	Dans $\Z/7\Z$, on a $\overline{10}=\overline{3}$ donc pour tout $k\in\N$, $\overline{10}^{6k}=\overline{1}$ d'après le petit théorème de Fermat. Dans $\Z/27\Z$, $\widetilde{10}$ est inversible car $10\wedge 27=1$. $\Bigl((\Z/27\Z)^{\times},+,\times\Bigr)$ comporte 18 éléments donc pour tout $k'\in\N$, on a $\widetilde{10}^{18k'}=\widetilde{1}$.

	Lorsque $81\mid n$, on a $21\mid 1\dots 1$. 
	
	Cherchons plus précisément les ordres de $\overline{10}$ dans $((\Z/7\Z)^{*},\times)$ et de $\widetilde{10}$ dans $((\Z/27\Z)^{\times},\times)$.
	Dans $(\Z/7\Z)^{*}$, groupe de cardinal 6, on vérifie que l'ordre de 10 est 6. Dans l'autre groupe, on vérifie que l'ordre de $\widetilde{10}$ est 3. Ainsi, $21\mid 1\dots 1$ si et seulement si $6\mid n$.

	Il y a donc une infinité de multiples de 21 qui s'écrivent avec uniquement des 1 en base 10.
\end{solution}

\begin{remark}
	Il suffit de trouver l'ordre de 10 dans les deux ensembles et de prendre le ppcm.
\end{remark}

\begin{solution}
	\phantom{}
	\begin{enumerate}
		\item $X^{d}-1$ a au plus $d$ racines dans $\K$. Pour tout $k\in\{0,\dots,d-1\}$, $x_{0}^{k}$ est racine de $X^{d}-1_{\K}$ car $gr\{x_{0}\}$ a pour cardinal $d$. Donc les racines sont exactement les puissances de $x_{0}$.
		
		Soit $x\in\K^{*}$ d'ordre $d$. On a $x\in gr\{x_{0}\}$ car $x^{d}=1$ (racine du polynôme de $X^{d}=1_{\K}$). Or, dans le groupe cyclique engendré par $x_{0}$, il y a $\varphi(d)$ éléments.

		\item On a ou bien $\varphi(d)$ ou bien aucun élément d'ordre $d$ dans $\K$. Soit $d$ tel que $d\mid n$, on note $H_{d}=\{x\in K\bigm| \omega(x)=d\}$. On a 
		$$\K^{*}=\bigcup_{d\mid n}H_{d}$$
		Alors
		$$n=\sum_{d\mid n}\vert H_{d}\vert\leqslant\sum_{d\mid n}\varphi(d)=n$$

		Alors pour tout $d$ tel que $d\mid n$, on a $\vert H_{d}\vert=\varphi(d)$. En particulier, on a $\vert H_{n}\vert=\varphi(n)\geqslant1$ donc $H_{n}$ est non vide. Donc il existe (au moins) un élément d'ordre $n$, on $(\K^{*},\times)$ est cyclique.
	\end{enumerate}
\end{solution}

\begin{solution}
	\phantom{}
	\begin{enumerate}
		\item Soit $x\in M$. On a $\overline{1}-\overline{x}^{-1}$ si et seulement si $\overline{x}=\overline{1}$ et $\overline{1}-\overline{x}^{-1}=\overline{1}$ si et seulement si $\overline{x}=\overline{0}$, ce qui n'est pas possible pour les deux cas. Donc $f$ est bien définie.
		
		Soit $x\in M$, on a 
		\begin{align*}
			f^{2}(x)
			&=f(\overline{1}-\overline{x}^{-1})\\
			&=\overline{1}-(\overline{1}-\overline{x}^{-1})^{-1}\\
			&=(\overline{1}-\overline{x}^{-1})^{-1}(\overline{1}-\overline{x}^{-1}-\overline{1})\\
			&=-\overline{x}^{-1}(\overline{1}-\overline{x}^{-1})^{-1}
		\end{align*}
		Donc 
		\begin{align*}
			f^{3}(x)
			&=\overline{1}-(\overline{1}-(\overline{1}-\overline{x}^{-1})^{-1})^{-1}\\
			&=\overline{1}-(-x\overline{x}^{-1}(\overline{1}-\overline{x}^{-1})^{-1})^{-1}\\
			&=\overline{1}+\overline{x}(\overline{1}-\overline{x}^{-1})\\
			&=\overline{1}+\overline{x}-\overline{1}\\
			&=\overline{x}
		\end{align*}

		Donc $f^{3}=id_{M}$.

		\item Soit $x\in M$, on a 
		\begin{align*}
			f(x)=x
			&\Longleftrightarrow \overline{1}-\overline{x}^{-1}=x\\
			&\Longleftrightarrow \overline{x}^{2}-\overline{x}+\overline{1}=\overline{0}\\
			&\Longleftrightarrow (\overline{x}-\overline{2}^{-1})^{2}+\overline{3}\times\overline{4}^{-1}=\overline{0}\\
			&\Longleftrightarrow \overline{-3}=(\overline{2}\overline{x}-\overline{1})^{2}
		\end{align*}
		$f$ admet un point fixe si et seulement $\overline{-3}$ est un carré dans $\Z/p\Z$ car $\overline{y}=\overline{2}\overline{x}-\overline{1}$ si et seulement si $\overline{x}=\overline{2}^{-1}(\overline{y}+\overline{1})$.

		\item Comme $p$ est premier plus grand que 5, on a $p\equiv 1\text{ ou }2[3]$ donc $p-2\equiv 0\text{ ou }2[3]$ car $f^{3}=id_{M}$, les longueurs des cycles qui composent $f$ valent 1 ou 3. 
		
		Si $f$ n'a pas de point fixe, tous les cycles sont de longueur 3, donc $3\mid p-2$ donc $p\equiv 2[3]$. Si $p\equiv 2[3]$, alors $3\mid p-2$, le nombre de points fixes est un multiple de $3$ donc aussi du nombre de racine carrés de $\overline{-3}$. Et puisque l'on est dans un corps, il y a au plus 2 racines de $\overline{-3}$. Donc si $p\equiv2[3]$, il n'y a pas de point fixe.
	\end{enumerate}
\end{solution}

\begin{solution}
	Soit $x\in\R$. Supposons que $x$ possède un développement décimal périodique. Alors il existe $(n_{0},T)\in\N\times\N^{*}$ tels que pour tout $n\geqslant n_{0}$, $a_{n+T}=a_{n}$. On a alors 
	$$\vert x\vert=\underbrace{b_{m}\dots b_{0},a_{0}\dots a_{n_{0}-1}}_{\in\Q}+\frac{1}{10^{n_{0}-1}}\underbrace{(0,a_{n_{0}}\dots a_{n_{0}+T-1}a_{n_{0}}\dots)}_{=y}$$
	$$10^{T}y-y=a_{n_{0}}\dots a_{n_{0}+T-1}\in\N$$
	et donc 
	$$y=\frac{a_{n_{0}}\dots a_{n_{0}+T-1}}{10^{T}-1}\in\Q$$
	Donc $x\in Q$.

	Réciproquement, soit $x=\frac{p}{q}\in\Q$ avec $q\in\N^{*}$. Il existe $(a,b)\in\Z\times\N^{*}$ tel que $p=aq+b$ avec $b\in\{0,\dots,q-1\}$. Si $b=0$, on arrête. On a sinon 
	$$x=a+\frac{1}{10^{k}}\frac{10^{k}b}{q}$$
	où $k=\min\{m\geqslant1\bigm| 10^{m}b>q\}$.
	On réitère l'algorithme avec $\frac{10^{k}b}{q}$ car on a $\Bigl\lfloor\frac{10^{k}b}{q}\Bigr\rfloor\in\{1,\dots,9\}$ par définition de $k$.

	Il y a $q$ restes possibles dans la division euclidienne par $q$. Ainsi, au bout d'au plus de $q+1$ itérations, on retrouve un reste précédent. Par unicité de la division euclidienne, on obtient un développement décimal périodique.
\end{solution}

\begin{remark}
	On peut écrire $q=2^{a}5^{b}q'$ avec $q'\wedge 2=q'\wedge 5=1$. On se ramène alors à $q\wedge2=q\wedge5=1$. En reportant dans l'écriture décimale de $x$, on a 
	$$\frac{\alpha}{q}=\frac{\beta}{10^{T}-1}$$
	avec $\alpha\wedge q=1$. On a donc $q\mid 10^{T}-1$ d'après le lemme de Gauss. $T$ revient donc à l'ordre de $\overline{10}$ dans $(\Z/q\Z^{\times},\times)$ qui contient $\varphi(q)$ éléments. Par défaut, on a donc $T=\varphi(q)$.
\end{remark}

\begin{solution}
	\phantom{}
	\begin{enumerate}
		\item Soit $m\in\Z$. Si $m\in\{0,\dots,n-1\}$, on a $H_{n}(m)=0\in\Z$.
		Si $m\geqslant n$, on a $H_{n}(m)=\binom{m}{n}\in\Z$. Si $m<0$, on a 
		$$H_{n}(m)=\frac{m(m-1)\dots(m-n+1)}{n!}=(-1)^{n}\binom{-m+n-1}{-m-1}\in\Z$$
		Donc $H_{n}(\Z)\subset\Z$.

		\item Supposons qu'il existe $n\in\N$ et $(a_{0},\dots,a_{n})\in\Z^{n+1}$ et $P=\sum_{k=0}^{n}a_{k}H_{k}$. On a $H_{k}(\Z)\subset\Z$ donc $P(\Z)\subset\Z$.
		Supposons $P(\Z)\subset\Z$. $(H_{k})_{k\in\N}$ est une base étagée en degré de $\C[X]$. Donc il existe $(a_{0},\dots,a_{n})\in\C^{n+1}$ tel que $P=\sum_{k=0}^{n}a_{k}H_{k}$. Par récurrence, on a $P(0)=a_{0}\in\Z$. Soit $k\in\{0,n-1\}$, supposons $(a_{0},\dots,a_{k})\in\Z^{k+1}$. On a alors 
		$$P(k+1)=\underbrace{\sum_{i=0}^{k}\underbrace{a_{k}}_{\in\Z}H_{k}}_{\in\Z}+a_{k+1}\underbrace{H_{k+1}(k+1)}_{=1}$$
		Donc $a_{k+1}\in\Z$.
	\end{enumerate}
\end{solution}

\begin{remark}
	Les translation $X+\alpha$ sont les seules pour lesquelles on a $(X+\alpha)(\Z)=\Z$. En effet, si $P\in\C[X]$ est tel que $P(\Z)=\Z$, on a $P\in\Q[X]$ d'après ce qui précède. Si $\deg(P)\geqslant2$, quitte à remplacer $P$ par $-P$, on peut supposer le coefficient dominant de $P$ strictement positif. On a alors $\lim\limits_{x\to+\infty}P'(x)=+\infty$ donc il existe $A>0$ tel que $P$ est strictement croissant sur $[A,+\infty[$. De plus, $P(x+1)-P(x)\to+\infty$ quand $x\to+\infty$. Donc il existe $A'>0$ tel que $P(x+1)>P(x)+1$. Pour $n\geqslant\max(A,A')$, on a $P(n+1)\geqslant P(n)+2$ ce qui contredit $P(\Z)=\Z$. Donc le degré de $P$ est inférieur à 1.
\end{remark}

\begin{solution}
	Le coefficient en $X^{k}$ s'écrit $a_{k-1}-\alpha a_{k}\in\Q$. Si $a_{k}\in\Q$, on a donc $a_{k-1}\in\Q$. Il est donc impossible d'avoir deux coefficients consécutifs rationnels. Or $x_{n-1}\in\Q$ car c'est le coefficient dominant de $P$. Donc $\alpha$ est nécessairement racine simple.
\end{solution}

\begin{solution}
	Soit $\Delta=P\wedge P'=\Delta$. On a $\deg(\Delta)\in\{1,2,3,4\}$ car $\Delta\mid P'$.

	Si $\deg(\Delta)=4$, alors $\Delta=P'$ (car associé). Donc il existe $\beta\in\C$ d'où $\underbrace{P}_{\in\Q[X]}=(X-\beta)\underbrace{P'}_{\in\Q[X]}$. Par division euclidienne, $X-B\in\Q[X]$ et $\beta\in\Q$ d'après l'algorithme de la division euclidienne.

	Si $\deg(\Delta)=1$, on a $P=X-\beta$ avec $\beta\in\Q$ racine de $P$.

	Si $\deg(\Delta)=2$, si $\Delta=(X-\beta)^{2}$, on a $\Delta'=2(X-\beta)\in\Q[X]$ donc $\beta\in\Q$ racine de $\Delta$ donc de $P$.
	Si $\Delta=(X-\alpha_{1})(X-\alpha_{2})$ avec $\alpha_{1}\neq\alpha_{2}$. $\alpha_{1}$ et $\alpha_{2}$ sont racines doubles de $P$ donc $P=(X-\beta)\underbrace{(X-\alpha_{1})^{2}(X-\alpha_{2})^{2}}_{=\Delta^{2}\in\Q[X]}$
	Par division euclidienne, $X-\beta\in\Q[X]$ et donc $\beta\in\Q$.

	Si $\deg(\Delta)=3$, si $\Delta=(X-\beta)^{3}$, on a $\Delta^{(2)}=6(X-\beta)\in\Q[X]$ donc $\beta\in\Q$.
	Si $\Delta=(X-\alpha_{1})(X-\alpha_{2})(X-\alpha_{3})$ avec $\alpha_{1},\alpha_{2}$ et $\alpha_{3}$ distinctes. $\alpha_{1},\alpha_{2}$ et $\alpha_{3}$ seraient racines doubles de $P$ ce qui contredit $\deg(P)=5$.
	Si $\Delta=(X-\alpha)^{2}(X-\beta)$, $\alpha$ est racine triple de $P$ et $\beta$ racine double de $P$ donc $P=(X-\alpha)^{3}(X-\beta)^{2}\in\Q[X]$. Par division euclidienne, $(X-\alpha)(X-\beta)\in\Q[X]$ et 
	$$X-\alpha=\frac{\Delta}{(X-\alpha)(X-\beta)}\in\Q[X]$$
	donc $\alpha\in\Q$.
\end{solution}

\begin{solution}
	\phantom{}
	\begin{enumerate}
		\item $1\in\Z[\i],0\in\Z[\i],\i\in\Z[\i]$. Soit $(a,b,a',b')\in\Z^{4}$:
		$$
		\left\{
			\begin{array}[]{l}
				(a+\i b)-(a'+\i b')=(a-a')+\i(b-b')\in\Z[\i]\\
				(a+\i b)\times (aa'-bb')+\i(ab'+ba')\in\Z[\i]
			\end{array}
		\right.
		$$
		Donc $\Z[\i]$ est un sous-anneau de $\C$ contenant $\i$.

		Soit $A$ un sous anneau de $\C$ contenant $\i$. $A$ est stable par $x$ donc $i^{4}=1\in A$. $A$ est stable par + donc $\Z\subset A$, puis $\i\Z\subset A$ donc $\Z[\i]\subset A$. $\Z[\i]$ est donc le plus petit sous anneau de $\C$ contenant $\i$.

		\item Si $\vert z\vert^{2}=1$ c'est-à-dire $a^{2}+b^{2}=1$, alors 
		$$\frac{1}{z}=\frac{a-\i b}{\vert z\vert^{2}}=a-\i b\in\Z[\i]$$
		Si $z$ est inversible dans $\Z[\i]$, il existe $'\in\Z[\i]$ tel que $zz'=1$ donc $\vert z\vert^{2}\vert z'\vert^{2}=1$ donc $\vert z\vert^{2}=1$.

		Soit $(a,b)\in\Z^{2}$. Si $\vert a\vert\geqslant2$ ou $\vert b\geqslant2$, alors $a^{2}+b^{2}\geqslant4$ donc si $\vert z\vert^{2}=1$, alors $a^{2}+b^{2}=1$ et $\Bigl(\vert a\vert=1$ et $\vert b\vert=0\Bigr)$ ou $\Bigl(\vert a\vert=0$ et $\vert b\vert=1\Bigr)$. Donc 
		$$U=\{1,-1,\i,-\i\}$$

		\item 
		\begin{enumerate}
			\item Si $x\in\R$, il existe $n\in\Z$ tel que $\vert x-n\vert\leqslant\frac{1}{2}$ (faire un dessin et le montrer grâce aux parties entières). Soit alors $z_{0}=x_{0}+\i y_{0}\in\C$, on prend un $(a,b)\in\Z^{2}$ tel que $\vert x_{0}-a\vert\leqslant\frac{1}{2},\vert y_{0}-b\vert\leqslant\frac{1}{2}$. Et pour $z=a+\i b\in\Z[\i]$, on a $\vert z-z_{0}\vert^{2}=(x_{0}-a)^{2}+(y_{0}-b)^{2}\leqslant\frac{1}{2}$.
			
			\item Soit $(q,r)\in\Z[\i]^{2}$, on a $z_{1}=qz_{2}+r$ si et seulement si $\frac{z_{1}}{z_{2}}-q=\frac{r}{z_{2}}$. On a $\vert r\vert<\vert z_{1}\vert$ si et seulement si $\bigl\vert\frac{z_{1}}{z_{2}}-q\bigr\vert<1$.
			On a $\frac{z_{1}}{z_{2}}\in\C$ donc d'après 3.(a), il existe $q\in\Z[\i]$ tel que $\bigl\lvert \frac{z_{1}}{z_{2}}-q\bigr\rvert\leqslant\frac{\sqrt{2}}{2}<1$. On pose alors $r=z_{1}-qz_{2}\in\Z[\i]$ par stabilité. Il vient donc $\vert r\vert<\vert z_{2}\vert$

			Il n'y a pas unicité: par exemple $z_{2}=1$ et $z_{1}=\frac{1+\i}{2}$. On peut prendre $q\in\{0,1,\i,1+\i\}$.

			\item Soit $I\neq\{0\}$ un idéal de $\Z[\i]$. On note $n_{0}=\min\{\vert z\vert^{2}\bigm| z\in I\setminus\{0\}\}$ (partie non vide de $\N^{*}$). Soit $z_{0}\in I\setminus\{0\}$ tel que $\vert z_{0}\vert^{2}=n_{0}$. On a directement $z_{0}\Z[\i]\subset I$ ($I$ est un idéal). 
			
			Réciproquement, soit $z\in I$, d'après 3.(b), il existe $(q,r)\in\Z[\i]^{2}$ tel que 
			$$r=\underbrace{z}_{\in I}-\underbrace{z_{0}}_{\in I}\underbrace{q}_{\in\Z[\i]}\in I$$ 
			et $\vert r\vert^{2}<n_{0}$. Nécessairement, $r=0$ et $z=z_{0}q\in z_{0}\Z[\i]$. Donc $I=z_{0}\Z[\i]$. Finalement, $\Z[\i]$ est principal.
		\end{enumerate}

		
		\item Si $\vert z\vert^{2}=1$, alors $z\in U$ donc c'est bon. On travaille ensuite par récurrence sur $n\in\N^{*}$. Supposons que la décomposition existe pour $z\in\Z[\i]$ avec $\vert z\vert^{2}\leqslant n$. Soit $z\in\Z[\i]$ tel que $\vert z\vert^{2}=n+1$. On a $\vert z\vert^{2}\geqslant2$ donc $z\in U$. Si $z$ est irréductible, c'est bon. Sinon, il existe $(z_{1},z_{2})\in\Z[\i]^{2}$ tel que $z=z_{1}z_{2}$ et $z_{1}$ et $z_{2}$ non inversibles. Alors $\vert z_{1}\vert^{2}\geqslant2$ et $\vert z_{2}\vert^{2}\geqslant2$. Or $\vert z\vert^{2}=n+1=\vert z_{1}\vert^{2}\vert z_{2}\vert^{2}$ donc $\vert z_{1}\vert^{2}\leqslant n$ et $\vert z_{2}\vert^{2}\leqslant n$. Par hypothèse de récurrence, on peut décomposer $z_{1}$ et $z_{2}$, donc $z$ est décomposable.
			
		Pour l'unicité, soit $z\in\Z[\i]\setminus\{0\}$ tel que $z=u\prod_{\rho\in\mathcal{P}_{0}}\rho^{\nu_{\rho}(z)}=v\prod_{\rho\in\mathcal{P}_{0}}\rho^{\mu_{\rho}(z)}$. Le théorème de Gauss est valable dans $\Z[\i]$, car c'est un anneau principal. S'il existe $\rho_{0}\in\mathcal{P}_{0}$ tel que $\nu_{\rho_{0}}(z)<\mu_{\rho_{0}}(z)$, alors 
		$$\rho_{0}\Bigm|\prod_{p\in\mathcal{P}_{0}\setminus\{\rho_{0}\}}\rho^{\nu_{\rho}(z)}$$
		ce qui est proscrit par le théorème de Gauss. On a donc pour tout $\rho\in\mathcal{P}_{0}$, $\nu_{\rho}(z)=\mu_{\rho}(z)$. En reportant, on a $u=v$. D'où l'unicité de la décomposition.
	\end{enumerate}
\end{solution}

\begin{solution}
	\phantom{}
	\begin{enumerate}
		\item On a $\overline{1}\in R$. Soit $(\overline{x_{1}},\overline{x_{2}})\in R^{2}$, il existe $(\overline{y_{1}},\overline{y_{2}})\in(\mathbb{F}_{p}^{*})^{2}$ tel que $\overline{x_{1}}=\overline{y_{1}}^{2}$ et $\overline{x_{2}}=\overline{y_{2}}^{2}$. On a  alors 
		$$\overline{x_{1}}\overline{x_{2}}^{-1}=(\overline{y_{1}}\overline{y_{2}}^{-1})^{2}\in R$$ donc $R$ est un sous groupe de $(\mathbb{F}_{p}^{*},\times)$. Soit \function{\varphi}{\mathbb{F}_{p}^{*}}{\mathbb{F}_{p}^{*}}{\overline{y}}{\overline{y}^{2}}
		On a $\im(\varphi)=R$. Comme $\mathbb{F}_{p}$ est un corps, chaque éléments de $R$ a exactement 2 antécédents par $\varphi$. Donc $\vert R\vert=\frac{\vert\mathbb{F}_{p}^{*}\vert}{2}=\frac{p-1}{2}$.

		S'il existe $\overline{y}\in\mathbb{F}_{p}^{*}$ tel que $\overline{a}=\overline{y}^{2}$, on a $\overline{a}^{\frac{p-1}{2}}=\overline{y}^{p-1}=\overline{1}$ par le théorème de Fermat.

		Réciproquement, si $\overline{a}^{\frac{p-1}{2}}=\overline{1}$, $X^{\frac{p-1}{2}}-\overline{1}$ admet au plus $\frac{p-1}{2}$ racines dans $\mathbb{F}_{p}^{*}$. Tous les éléments de $R$ sont racines de ce polynôme, ce sont donc ses seules racines. Donc $a\in R$.

		\item Si $p=a^{2}+b^{2}$, alors $\overline{0}=\overline{a}^{2}+\overline{b}^{2}$. Si $\overline{a}=\overline{b}=\overline{0}$, on a $p\mid a$ et $p\mid b$ donc $p^{2}\mid p$ ce qui est exclu. Par exemple, si $\overline{a}\neq\overline{0}$, on a $\overline{1}=-\overline{b}^{2}\overline{a}^{-2}$ donc $\overline{-1}=(\overline{a}^{-1}\overline{b})^{2}\in R$ d'après 1. On a donc $(\overline{-1})^{\frac{p-1}{2}}=\overline{1}$ si et seulement si $2\bigm|\frac{p-1}{2}$ (car $p$ est premier plus grand que 3) d'où $4\mid p-1$ donc $p\equiv 1[4]$.
		
		\item On a $\vert\mathbb{F}_{p}\vert=p$, $E(\sqrt{p})\leqslant\sqrt{p}<E(\sqrt{p})+1$ et $\vert\{0,\dots,E(\sqrt(p))\}\vert^{2}=(E(\sqrt{p})+1)^{2}>p$ ($p$ est premier, ce n'est pas un carré) donc $f$ n'est pas injective (cardinalité).
		
		Donc il existe 
		$$((a_{1},b_{1}),(a_{2},b_{2}))\in(\{0,\dots,E(\sqrt{p})\}^{2})^{2}$$ avec $(a_{1},b_{1})\neq (a_{2},b_{2})$ et $f(a_{1},b_{1})=f(a_{2},b_{2})$. Donc 
		$$\overline{a_{1}}-\overline{k}\overline{b_{1}}=\overline{a_{2}}-\overline{k}\overline{b_{2}}\Rightarrow \overline{a_{1}}-\overline{a_{2}}=\overline{k}(\overline{b_{1}}-\overline{b_{2}})$$
		
		Si $\overline{b_{1}}=\overline{b_{2}}$, alors $\overline{a_{1}}=\overline{a_{2}}$ donc $p\mid b_{1}-b_{2}$ et $p\mid a_{1}-a_{2}$ donc $(a_{1},b_{1})=(a_{2},b_{2})$ ce qui n'est pas vrai. Donc $\overline{b_{1}}\neq\overline{b_{2}}$. Posons $b_{0}=b_{1}-b_{2}$ et $a_{0}=a_{1}-a_{2}$. On a $\overline{b_{0}}\neq\overline{0}$. Il vient donc $(\vert a_{0}\vert,\vert b_{0}\vert)\in\{1,\dots,E(\sqrt{p})\}^{2}$, $\overline{a_{0}}=\overline{k}\overline{b_{0}}$ donc $\overline{k}=\overline{a_{0}}\overline{b_{0}}^{-1}$.

		\item Si $p\equiv 1[4]$, en remontant les calculs, on a $(\overline{-1})^{\frac{p-1}{2}}=\overline{1}$ donc $\overline{-1}\in R$ et il existe $\overline{k}\in\mathbb{F}_{p}^{*}$ tel que $\overline{-1}=\overline{k}^{2}$. Alors d'après 3., il existe $(a_{0},b_{0})$ tels que $\overline{k}=\overline{a_{0}}\overline{b_{0}}^{-1}$. Il vient alors $\overline{-1}=\overline{a_{0}}^{2}(\overline{b_{0}}^{-1})^{2}$ donc $\overline{-b_{0}}^{2}=\overline{a_{0}}^{2}$. On a 
		$$p\mid a_{0}^{2}+b_{0}^{2}\in\{2,\dots,2E(\sqrt{p})\}^{2}\subset\{2,\dots,2p-1\}$$
		Nécessairement, $a_{0}^{2}+b_{0}^{2}=p$ et $p$ est somme de deux carrés.
	\end{enumerate}
\end{solution}

\begin{solution}
	\phantom{}
	\begin{enumerate}
		\item Soit $(m,n)\in A^{2}$. Il existe $(a,b,c,d)\in\N^{4}$ tel que $m=a^{2}+b^{2}=\vert a+\i b\vert^{2}$ et $n=c^{2}+d^{2}=\vert c+\i d\vert^{2}$. Donc 
		$$m\times n=\vert ac-bd6\i(bc+ad)\vert^{2}=(ac-bd)^{2}+(bc+ad)^{2}\in A$$
		\item On a 
		$$n=\underbrace{\prod_{p\in\mathcal{P}_{1}}p^{\nu_{p}(n)}}_{\in A\text{ car }\mathcal{P}_{1}\subset A}\times\underbrace{\prod_{p\in\mathcal{P}_{2}}p^{\nu_{p}(n)}}_{=\prod_{p\in\mathcal{P}_{2}}p^{2\alpha_{p}}\in A}\in A$$

		\item Soit $n\in A$, il existe $(a,b)\in\N^{2}$ avec $n=a^{2}+b^{2}$. Soit $p\in \mathcal{P}_{1}\cup\mathcal{P}_{2}$, on a $p\mid a^{2}+b^{2}$ donc $\overline{a^{2}+b^{2}}=\overline{0}$ dans $\Z/p\Z$. Si $p\nmid a$ ou $p\nmid b$, alors $\overline{1+\frac{b^{2}}{a^{2}}}=\overline{0}$ donc $\overline{-1}\in R$ (résidus quadratiques, voir exercice précédent). Donc $p=2$ ou $p\equiv 1[4]$.
		
		Si $p\mid a$ et $p\mid b$, $a=p^{k}a', b=p^{l}b'$ avec $p\nmid a'$ et $p\nmid b'$. On suppose $1\leqslant k\leqslant l$ (quitte à échanger $a$ et $b$).
		On a 
		$$a^{2}+b^{2}=p^{2k}(a'^{2}+p^{2(l-k)}b'^{2})=n$$
		donc 
		$$p\Bigm| a'^{2}+p^{2(l-k)b'^{2}}$$
		et $\overline{a'}^{2}+\overline{p^{2(l-k)}}\overline{b'}^{2}=\overline{0}$. Nécessairement, $l=k$. De même $p\in\mathcal{P}_{1}$. Par contraposée, $\nu_{p}$ est pair.
	\end{enumerate}
\end{solution}
\section{Séries numériques et familles sommables}

\begin{solution}
	\phantom{}
	\begin{enumerate}
		\item On a $b_{0}=a_{1}=5,b_{1}=a_{3}=13$ et pour $p\geqslant2$, $b_{p}=2b_{p-1}+3b_{p-2}$.
		
		On a donc l'équation caractéristique $x^{2}-2x-3=0$. Les deux solutions sont 3 et -1. Donc il existe $(\lambda,\mu)\in\R^{2}$, $b_{p}=\lambda 3^{p}+\mu(-1)^{p}$.

		On a alors $b_{0}=5=\lambda+\mu$ et $b_{1}=13=3\lambda-\mu$. On trouve alors $\lambda=\frac{9}{2}$ et $\mu=\frac{1}{2}$.

		\item On le montre par récurrence sur $p\in\N$.
		
		\item Si $3^{p}\leqslant n<3^{p+1}$, on a $a_{n}=b_{p}=\frac{9}{2}3^{p}+\frac{1}{2}(-1)^{p}$.
		Alors 
		$$\frac{3}{2}+\frac{1}{2}(-1)^{p}\frac{1}{3^{p+1}}<\frac{a_{n}}{n}\leqslant\frac{9}{2}+\frac{1}{2}(-1)^{p}\frac{1}{3^{p}}$$
		Soit $\sigma\colon\N\to\N$ strictement croissante telle que 
		$$\frac{a_{\sigma(n)}}{\sigma(n)}\xrightarrow[n\to+\infty]{}\lambda$$
		Soit $p_{n}\in\N$ tel que $3^{p_{n}}\leqslant\sigma(n)<3^{p_{n}+1}$. On a 
		$$p_{n}=\bigl\lfloor\log_{3}(\sigma(n))\bigr\rfloor\xrightarrow[n\to+\infty]{}+\infty$$
		En reportant, on a $\frac{3}{2}\leqslant\lambda\leqslant\frac{9}{2}$.

		Si $\sigma(n)=3^{n}$, on a 
		$$\frac{a_{3^{n}}}{3^{n}}=\frac{b_{n}}{3^{n}}=\frac{9}{2}+\frac{1}{2}\frac{(-1)^{n}}{3^{n}}\xrightarrow[n\to+\infty]{}\frac{9}{2}$$
		Si $\sigma(n)=3^{n+1}-1$, on a 
		$$\frac{a_{3^{n}}}{3^{n}}=\frac{b_{n}}{3^{n+1}-1}\xrightarrow[n\to+\infty]{}\frac{3}{2}$$

		Soit $\mu\in[1,3[$ et $\sigma(n)=\lfloor 3^{n}\mu\rfloor\underset{n\to+\infty}{\sim}3^{n}\mu$. Alors 
		$$\frac{a_{\sigma(n)}}{\sigma(n)}=\frac{b_{n}}{\lfloor3^{n}\mu\rfloor}\underset{n\to+\infty}{\sim}\frac{b_{n}}{3^{n}\mu}=\frac{9}{2\mu}+\frac{1}{2\mu}\frac{(-1)^{n}}{3^{n}}\xrightarrow[n\to+\infty]{}\frac{9}{2\mu}$$
		
		Donc tout réel compris dans $\bigl[\frac{3}{2},\frac{9}{2}\bigr]$ est valeur d'adhérence.
	\end{enumerate}
\end{solution}

\begin{solution}
	\phantom{}
	\begin{enumerate}
		\item \function{g}{[a,b]}{\R}{x}{f(x)-x}
		est continue, $g(a)\geqslant0$ et $g(b)\leqslant0$, donc le théorème des valeurs intermédiaires affirme qu'il existe $l\in[a,b]$ avec $g(l)=0$, d'où $f(l)=l$.

		\item On note $A=\{\lambda\bigm| \lambda\text{ est valeur d'adhérence}\}$.
		Le théorème de Bolzano-Weierstrass indique que $A$ est non vide. De plus, $A$ est borné car $A\subset[a,b]$. Soit $\lambda=\inf(A)$ et $\mu=\sup(A)$. 
		
		Si $\lambda=b$, on a $\mu=b$ et $A=\{b\}=\{\lambda\}=\{\mu\}$.

		Si $\lambda<b$, soit $\varepsilon>0$. Si $\lambda\notin A$, $\{k\in\N\bigm| x_{k}\in]\lambda,\lambda+\varepsilon[\}$ est infini. Par définition, $\lambda$ est valeur d'adhérence. Donc $\lambda\in A$, et de même $\mu\in A$.

		Soit $\nu\in]\lambda,\mu[$ avec $\lambda<\mu$. Si $\nu\notin A$, il existe $\varepsilon_{0}>0$ tel que $\{k\in\N\bigm|\vert x_{k}-\nu\vert<\varepsilon_{0}\}$ est fini. Donc il existe $N_{0}\in\N$ tel que pour tout $n\geqslant N_{0}$, $x_{n}\notin]\nu-\varepsilon_{0},\nu+\varepsilon_{0}[$. Comme $\lim\limits_{n\to+\infty}\vert x_{n+1}-x_{n}=0$, il existe $N_{1}\in\N$ tel que pour tout $n\geqslant N_{1}$, $\vert x_{n+1}-x_{n}\vert<2\varepsilon_{0}$. 
		Soit alors $n\geqslant\max(N_{0},N_{1})$. Si $x_{n}\leqslant\nu-\varepsilon_{0}$, alors $x_{n+1}\leqslant\nu-\varepsilon_{0}$. Si $x_{n}\geqslant\nu+\varepsilon_{0}$, alors $x_{n+1}\geqslant\nu+\varepsilon_{0}$. Ceci contredit que $\lambda$ et $\mu$ sont valeur d'adhérence. Ainsi, $\nu\in A$ et $[\lambda,\mu]$ est le segment des valeurs d'adhérence.

		\item Si $(x_{n})$ converge, alors $\lim\limits_{n\to+\infty}x_{n+1}-x_{n}=0$. Réciproquement, si $\lim\limits_{n\to+\infty}x_{n+1}-x_{n}=0$, d'après 2., on a $A=[\lambda,\mu]$. On suppose $\lambda<\nu$. Ainsi, $\frac{\lambda+\nu}{2}=\alpha$ est valeur d'adhérence. Donc il existe $\sigma\colon\N\to\N$ strictement croissante telle que $x_{\sigma(n)}\xrightarrow[n\to+\infty]{}\alpha$. Alors $\lim\limits_{n\to+\infty}x_{\sigma(n)+1}=f(\alpha)$ par continuité de $f$ et c'est aussi égale à $\lim\limits_{n\to+\infty}x_{\sigma(n)}=\alpha$ car $\lim\limits_{n\to+\infty}x_{n+1}-x_{n}=0$.
		Ainsi, $f(\alpha)=\alpha$.

		Par ailleurs, il existe $n_{0}\in\N$ tel que $x_{n_{0}}\in[\lambda,\mu]$ et $f(x_{n_{0}})=x_{n_{0}}\in A$, alors pour tout $n\geqslant n_{0}$, on a $x_{n}=x_{n_{0}}$. Donc $(x_{n})_{n\in\N}$ converge et $\lambda=\mu$: $(x_{n})_{n\in\N}$ est bornée et a une unique valeur d'adhérence, donc $(x_{n})_{n\in\N}$ converge.
	\end{enumerate}
\end{solution}

\begin{solution}
	On a $u_{n}=e^{\i 2^{n}\theta}$ pour tout $n\in\N$.

	Si $(u_{n})_{n\in\N}$ converge, alors $\lim\limits_{n\to+\infty}u_{n}=1$ car $l=l^{2}$ et $\vert l\vert=1$.

	Si $(u_{n})_{n\in\N}$ est périodique au-delà d'un certain rang, il existe $T\in\N^{*}$, il existe $N_{0}\in\N$ tel que pour tout $n\geqslant N_{0}$, $u_{n+T}=u_{n}$. En particulier, $u_{N_{0}+T}=u_{N_{0}}$. On veut alors $2^{N_{0}+T}\theta\equiv 2^{N_{0}}\theta[2\pi]$. D'où $2^{N_{0}+T}\theta=2\theta+2k\pi$ donc $2^{N_{0}}(2^{T}-1)\theta=2k\pi$. Donc $\frac{\theta}{2\pi}\in\Q$.

	Réciproquement, si $\frac{\theta}{2\pi}\in\Q$, son développement binaire est périodique à partir d'un certain rang, et donc $(u_{n})_{n\in\N}$ l'est aussi.

	Si $(u_{n})_{n\in\N}$ est stationnaire, il existe $N\in\N$ tel que pour tout $n\geqslant N$, $U_{N+1}=U_{N}=U_{N^{2}}$. Comme $\vert U_{N}\vert=1$, alors $2^{n}\theta\in 2\pi\N$ et $\frac{\theta}{2\pi}$ est dyadique. 

	Réciproquement, s'il existe $p\in\N$, $u_{0}\in\N$ tel que $\frac{\theta}{2\pi}=\frac{p}{2^{n_{0}}}$ (nombre dyadique). Alors pour tout $n\geqslant n_{0}$, $2^{n}\theta\in 2\pi\N$ et $u_{n}=u_{n_{0}}=1$.

	Pour la densité, on prend une suite $(a_{n})_{n\in\N}$ en écrivant successivement, pour tout $k\in\N^{*}$, tous les paquets de $k$ entiers sont dans $\{0,1\}^{k}$. Soit $x\in[0,1[$ tel que 
	$$x=\sum_{n=1}^{+\infty}\frac{a_{n}}{2^{n}}$$
	Soit $N\in\N$, il existe $p_{N}\in\N$, 
	$$2^{p_{N}}\theta=2\pi\underbrace{(\dots)}_{\in\N}+2\pi(\frac{a_{1}}{2}+\dots+\frac{a_{N}}{2^{N}}+\underbrace{\dots}_{\in[0,\frac{1}{2^{N}}[})$$
	On a alors 
	$$e^{\i2^{p_{N}}\theta}=e^{\i2\pi(\frac{a_{1}}{2}+\dots+\frac{a_{N}}{2^{N}}+\dots)}$$
	et 
	$$\Bigl\vert\frac{a_{1}}{2}+\dots+\frac{a_{N}}{2^{N}}-x\Bigr\vert\leqslant\frac{1}{2^{N}}$$
	D'où $\lim\limits_{N\to+\infty}u_{p_{N}}=e^{\i2\pi x}$ et $(u_{n})_{n\in\N}$ est dense dans $\U$.
\end{solution}

\begin{solution}
	Si $a=0$ et $b=0$, $u_{n}\xrightarrow[n\to+\infty]{}0$.

	Si $a=0$ et $b\neq0$ (ou inversement), $u_{n}\underset{n\to+\infty}{\sim}\Bigl(\frac{1}{2}\Bigr)^{n^{2}}\xrightarrow[n\to+\infty]{}0$.

	Si $a>0$ ou $b>0$, on a
	\begin{align*}
		u_{n}
		&=\exp\Bigl(n^{2}\ln\Bigl(\frac{e^{\frac{1}{n}\ln(a)}+e^{\frac{1}{n}\ln(b)}}{2}\Bigr)\Bigr)\\
		&=\exp\Bigl(n^{2}\ln\Bigl(1+\frac{1}{2n}\ln(ab)+\frac{1}{4n^2}(\ln(a)^{2}+\ln(b)^{2})\Bigr)+o\Bigl(\frac{1}{n^{2}}\Bigr)\Bigr)\\
		&=\exp\Bigl(\frac{n}{2}\ln(ab)+\frac{1}{4}(\ln(a)^{2}+\ln(b)^{2}+o(1))\Bigr)
	\end{align*}

	Si $ab>1$, on a $\lim\limits_{n\to+\infty} u_{n}=+\infty$.
	
	Si $ab<1$, on a $\lim\limits_{n\to+\infty} u_{n}=0$.

	Si $ab=1$, on a $\lim\limits_{n\to+\infty} u_{n}=e^{\frac{1}{2}\ln(a)^{2}}$.
\end{solution}

\section{Probabilités sur un univers dénombrable}
\section{Calcul matriciel}
\section{Réduction des endomorphismes}

\begin{proof}
	\phantom{}
	\begin{enumerate}
		\item On a \function{f^k}{E}{E}{M}{A^{k}M} donc pour tout polynôme $P$, on a $P(f)=P(A)M$ par combinaison linéaire. Si $P(A)=0$, alors $P(f)=0$. Donc si $A$ est diagonalisable, $f$ l'est aussi. Si $P(f)=0$ alors avec $M=I_{n}$, on a $P(A)=0$ et $A$ est diagonalisable si $f$ l'est.
		
		Même résultat avec $g$ et $B$.

		\item Soit $(\lambda_{i,j})_{1\leqslant i,j\leqslant n}$ tel que $\sum_{(i,j)\in\llbracket1,n\rrbracket^{2}}\lambda_{i,j}X_{i}Y_{j}^{\mathsf{T}}=0$. Alors on a 
		\begin{equation}
			\sum_{j=1}^{n}\left(\sum_{i=1}^{n}\lambda_{i,j}X_{i}\right)Y_{j}^{\mathsf{T}}=0
		\end{equation}

		Soit $k\in\llbracket1,n\rrbracket$, la $k$-ième ligne de notre matrice est 
		\begin{equation}
			\sum_{j=1}^{n}\left(\sum_{i=1}^{n}\lambda_{i,j}X_{i,k}\right)Y_{j}^{\mathsf{T}}=0
		\end{equation}
		Puisque $(Y_{j}^{\mathsf{T}})_{1\leqslant j\leqslant n}$ est libre, on a pour tout $j\in\llbracket1,n\rrbracket$,
		\begin{equation}
			\sum_{i=1}^{n}\lambda_{i,j}X_{i,k}=0
		\end{equation}
		Puisque $(X_{i})_{1\leqslant i\leqslant n}$ est libre, pour tout $(i,j)\in\llbracket1,n\rrbracket^{2}$, $\lambda_{i,j}=0$, d'où le résultat.

		\item Puisque $B$ est diagonalisable, $B^{\mathsf{T}}$ l'est aussi. On prend $(X_{i})_{1\leqslant i\leqslant n}$ une base de vecteurs propres de $A$ avec pour tout $i\in\llbracket1,n\rrbracket$, $AX_{i}=\lambda_{i}X_{i}$. Prenons $(Y_{j})_{1\leqslant j\leqslant n}$ une base de vecteurs propres de $B^{\mathsf{T}}$ avec pour tout $j\in\llbracket1,n\rrbracket$, $B^{\mathsf{T}}Y_{j}=\mu_{j}Y_{j}$ et $Y_{j}B^{\mathsf{T}}=\mu_{j}Y_{j}^{\mathsf{T}}$. Ainsi,
		\begin{equation}
			h\left(X_{i}Y_{j}^{\mathsf{T}}\right)=AX_{i}Y_{j}^{\mathsf{T}}B=\mu_{j}AX_{i}Y_{j}^{\mathsf{T}}=\mu_{j}\lambda_{i}X_{i}Y_{j}^{\mathsf{T}}
		\end{equation}
		et les $(X_{i}Y_{j}^{\mathsf{T}})_{1\leqslant i,j\leqslant n}$ forment une base de $E$ d'après ce qui précède. Donc $h$ est diagonalisable.

		Réciproquement, on a le contre-exemple $A=0$ et $B$ non diagonalisable: $h$ est l'endomorphisme nul.
	\end{enumerate}	
\end{proof}

\begin{remark}
	Généralement, soit $A\in\M_{n}(\K)$ et $B\in\M_{p}(\K)$, on définit \function{h_{A,B}}{\M_{n,p}(\K)}{\M_{n,p}(\K)}{M}{AMB}
	La matrice de $h_{A,B}$ dans la base canonique de $\M_{n,p}(\K)$ s'appelle le produit tensoriel de $A$ et $B$ noté 
	\begin{equation}
		A\otimes B=
		\begin{pmatrix}
			a_{1,1}B & \dots & a_{1,n}B\\
			\vdots & &\vdots\\
			a_{n,1}B&\dots & a_{n,n}B
		\end{pmatrix}
	\end{equation}
	On a toujours 
	\begin{equation}
		\Tr(A\otimes B)=\sum_{i=1}^{n}a_{i,i}\Tr(B)=\Tr(A)\Tr(B)
	\end{equation}
	Si $A$ et $B$ sont diagonalisables, $h_{A,B}$ l'est.
\end{remark}

\begin{proof}
	On pose $P=DP_{1}$ et $Q=DQ_{1}$ avec $P_{1}\wedge Q_{1}=1$. Il existe $(U,V)\in\K[X]^{2}$ telles que $UP_{1}+VQ_{1}=1$. On a $MD=PQ$ donc $M=DP_{1}Q_{1}=PQ_{1}=P_{1}Q$.

	\begin{enumerate}
		\item Soit $x\in\ker(D(f))$. On a 
		\begin{equation}
			P(f)(x)=DP_{1}(f)(x)=P_{1}(f)\circ D(f)(x)=0	
		\end{equation}
		De même pour $Q(f)(x)=0$, donc 
		\begin{equation}
			\ker(D(f))\subset\ker(P(f))\cap\ker(Q(f))
		\end{equation}

		Soit $x\in\ker(P(f))\cap\ker(Q(f))$. On a
		\begin{equation}
			DUP_{1}+DVQ_{1}=0
		\end{equation}
		d'où 
		\begin{equation}
			UP+VQ=0
		\end{equation}
		et
		\begin{equation}
			D(f)(x)=UP(f)(x)+VQ(f)(x)=0
		\end{equation}

		Donc 
		\begin{equation}
			\boxed{\ker(D(f))=\ker(P(f))\cap\ker(M(f))}	
		\end{equation}

		\item On a $P\mid M$ donc $\ker(P(f))\subset\ker(M(f))$. De même, $\ker(Q(f))\subset\ker(M(f))$ donc 
		\begin{equation}
			\ker(P(f))+\ker(Q(f))\subset\ker(M(f))
		\end{equation}

		Si $x\in\ker(M(f))$, on a 
		\begin{equation}
			x=\underbrace{UP_{1}(f)(x)}_{\in\ker(Q(f))}+\underbrace{VQ_{1}(f)(x)}_{\in\ker(P(f))}
		\end{equation}
		car $M=P_{1}Q=Q_{1}P$. Donc 
		\begin{equation}
			\boxed{\ker(M(f))=\ker(P(f))+\ker(Q(f))}
		\end{equation}

		\item Si $i\in\im(P(f))$, il existe $x\in E$ tel que $y=P(f)(x)=D(f)\circ P_{1}(f)(x)\in\im(D(f))$. De même pour $\im(Q(f))\subset\im(D(f))$. Donc 
		\begin{equation}
			\im(P(f))+\im(Q(f))\subset\im(D(f))
		\end{equation}

		Soit $y\in\im(D(f))$, alors il existe $x\in E$ tel que 
		\begin{equation}
			y=D(f)(x)=\underbrace{UP(f)(x)}_{\in\im(P(f))}+\underbrace{VQ(f)(x)}_{\in\im(Q(f))}
		\end{equation}
		Donc 
		\begin{equation}
			\boxed{\im(D(f))=\im(P(f))+\im(Q(f))}
		\end{equation}

		\item On a $P\mid M$ d'où $\im(M(f))\subset\im P(f)$ et $\im(M(f))\subset\im Q(f)$. Ainsi,
		\begin{equation}
			\im(M(f))\subset\im(Q(f))\cap\im\im(Q(f))
		\end{equation}

		Si $y\in\im(P(f))\cap\im(Q(f))$ alors il existe $(x,x')\in E^{2}$ tels que 
		\begin{equation}
			y=P(f)(x)=P(f)(x')
		\end{equation}
		Or $M=P_{1}Q=PQ_{1}$ donc 
		\begin{equation}
			y=UP_{1}(f)(y)+VQ_{1}(f)(y)=UP_{1}Q(f)(x')+VQ_{1}P(f)(x)\in\im(M(f))
		\end{equation}
		donc 
		\begin{equation}
			\boxed{\im(M(f))=\im(P(f))\cap\im(Q(f))}
		\end{equation}
	\end{enumerate}
\end{proof}

\begin{proof}
	On a 
	\begin{equation}
		A\left(\frac{-1}{5}A+\frac{4}{5}I_{n}\right)=I_{n}
	\end{equation}
	donc $A$ est inversible.
	\begin{equation}
		X^{2}-4X+5=(X-2+\i)(X-2-\i)
	\end{equation}
	est scindé à racines simples sur $\C$. Donc $A$ est diagonalisable sur $\C$, semblable sur $\C$ à
	\begin{equation}
		\begin{pmatrix}
			\lambda_{1}I_{n_{1}} &0\\
			0 & \lambda_{2}I_{n_{2}}
		\end{pmatrix}
	\end{equation}
	où $\lambda_{1}=2+\i$ et $\lambda_{2}=2-\i$. $A\in\M_{n}(\R)$ donc $\Tr(A)=n_{1}\lambda_{1}+n_{2}\lambda_{2}\in\R$

	Donc 
	\begin{equation}
		\Im(n_{1}\lambda_{1}+n_{2}\lambda_{2})=0=n_{1}-n_{2}
	\end{equation}

	Ainsi $n_{1}=n_{2}$ donc $n$ est pair.

	$A$ est semblable sur $\C$ à 
	\begin{equation}
		\begin{pmatrix}
			\lambda_{1}&0&\dots&\dots&0\\
			0&\overline{\lambda_{1}}&\ddots&&\vdots\\
			\vdots &\ddots & \ddots &\ddots&\vdots\\
			\vdots & &\ddots & \lambda_{1}&0\\
			0 &\dots&\dots&0&\overline{\lambda_{1}}
		\end{pmatrix}
	\end{equation}

	Soit 
	\begin{equation}
		A_{0}=
		\begin{pmatrix}
			0&-5\\
			1&4
		\end{pmatrix}
	\end{equation}
	On a $\chi_{A_{0}}=X^{2}-4X+5$. $A_{0}$ est diagonalisable sur $\C$ et est semblable à 
	\begin{equation}
		\begin{pmatrix}
			\lambda_{1}&0\\
			0&\overline{\lambda_{1}}
		\end{pmatrix}
	\end{equation}
	Donc $A$ est semblable sur $\C$ à 
	\begin{equation}
		\begin{pmatrix}
			A_{0}&&\\
			&\ddots&\\
			&&A_{0}
		\end{pmatrix}
	\end{equation}
	donc $A$ est semblable sur $\R$ à cette même matrice.

	Soit $l\in\N$, on a 
	\begin{equation}
		X^{l}=Q_{p}(X^{2}-4X+5)+\alpha_{l}X+\beta_{l}
	\end{equation}
	par division euclidienne. Donc 
	\begin{equation}
		A^{l}=\alpha_{l}A+\beta_{l}I_{n}
	\end{equation}
	On a notamment 
	\begin{equation}
		\left\lbrace
			\begin{array}[]{l}
				(2+\i)^{l}=\alpha_{l}(2+\i)+\beta_{l}\\
				(2-\i)^{l}=\alpha_{l}(2-\i)+\beta_{l}
			\end{array}
		\right.
	\end{equation}

	On a donc 
	\begin{equation}
		\boxed{
			\left\lbrace
			\begin{array}[]{l}
				\alpha_{l}=\frac{(2+\i)^{l}-(2-\i)^{l}}{2\i}\\
				\beta_{l}=(2+\i)^{l}-\frac{(2+\i)}{2\i}\left[(2+\i)^{l}-(2-\i)^{l}\right]
			\end{array}
		\right.
		}
	\end{equation}
\end{proof}

\begin{remark}
	On a $2+\i=\sqrt{5}\e^{\i\theta}$ avec $\theta=\arccos\left(\frac{2}{\sqrt{5}}\right)\in]0,\pi[$. Donc $\alpha_{l}=\left(\sqrt{5}\right)^{l}\sin(l\theta)$.
\end{remark}

\begin{remark}
	On a 
	\begin{equation}
		I_{n}-4A^{-1}+5A^{-2}=0
	\end{equation}
	De même, $\left(X-\frac{1}{2-\i}\right)\left(X-\frac{1}{2+\i}\right)$ annule $A^{-1}$ et on a pour tout $l\in-\N^{*}$,
	\begin{equation}
		A^{l}=\alpha_{l}A+\beta_{l}I_{n}
	\end{equation}
\end{remark}

\begin{remark}
	$(A-2I_{n})^{2}=-I_{n}$ donc $\det(-I_{n})=(-1)^{n}>0$ donc $n$ est pair.
\end{remark}

\begin{proof}
	\phantom{}
	\begin{enumerate}
		\item On a 
		\begin{equation}
			A\begin{pmatrix}
				1\\\vdots\\1
			\end{pmatrix}=\begin{pmatrix}
				1\\\vdots\\1
			\end{pmatrix}
		\end{equation}
		et $\begin{pmatrix}
			1&\dots&1
		\end{pmatrix}^{\mathsf{T}}\neq0$ donc 
		\begin{equation}
			\boxed{1\in\Sp_{\R}(A)}
		\end{equation}

		\item Soit $X=\begin{pmatrix}
			x_{1}&\dots&x_{n}
		\end{pmatrix}^{\mathsf{T}}\neq0$ associé à $\lambda$. Pour tout $i\in\llbracket1,n\rrbracket$, on a 
		\begin{equation}
			\lambda x_{i}=\sum_{j=1}^{n}a_{i,j}x_{j}
		\end{equation}
		Soit $i_{0}\in\llbracket1,n\rrbracket$ tel que $\left\lvert x_{i_{0}}\right\rvert=\max\limits_{i\in\llbracket1,n\rrbracket}\left\lvert x_{i}\right\rvert>0$ car $X\neq0$. On a alors 
		\begin{equation}
			\left\lvert\lambda\right\rvert\left\lvert x_{i_{0}}\right\rvert=\left\lvert\sum_{j=1}^{n}a_{i_{0},j}x_{j}\right\rvert\leqslant\sum_{j=1}^{n}a_{i_{0},j}\left\lvert x_{j}\right\rvert\leqslant\left(\sum_{j=1}^{n}a_{i_{0},j}\right)\left\lvert x_{i_{0}}\right\rvert
		\end{equation}
		donc 
		\begin{equation}
			\boxed{\left\lvert\lambda\right\rvert\leqslant1}
		\end{equation}

		\item Soit $J_{i}=\left\lbrace j\in\llbracket1,n\rrbracket\middle| a_{i,j}>0\right\rbrace$. On a 
		\begin{equation}
			\left\lvert\lambda\right\rvert\left\lvert x_{i_{0}}\right\rvert=\left\lvert\sum_{j\in J_{i_{0}}}^{n}a_{i_{0},j}x_{j}\right\rvert\leqslant\sum_{j\in J_{i_{0}}}^{n}a_{i_{0},j}\left\lvert x_{j}\right\rvert\leqslant\left(\sum_{j\in J_{i_{0}}}^{n}a_{i_{0},j}\right)\left\lvert x_{i_{0}}\right\rvert=\left\lvert x_{i_{0}}\right\rvert
		\end{equation}

		On a égalité partout donc pour tout $j\in J_{i_{0}}$, $\left\lvert x_{j}\right\rvert=\left\lvert x_{i_{0}}\right\rvert$ et $x_{j}=\left\lvert x_{i_{0}}\right\rvert\e^{\i \theta}$. En reportant, on a 
		\begin{equation}
			\lambda\left\lvert x_{i_{0}}\right\rvert=\sum_{j\in J_{i_{0}}}a_{i_{0},j}\left\lvert x_{i_{0}}\right\rvert
		\end{equation}
		donc 
		\begin{equation}
			\boxed{\lambda=1}
		\end{equation}

		\item Si $\left\lvert\lambda\right\rvert=1$ et $\lambda\neq1$, on a $i_{0}\notin J_{i_{0}}$ car sinon $\lambda=1$. Donc il existe $i_{1}\in J_{i_{0}}\setminus\lbrace i_{0}\rbrace$ tel que $x_{i_{1}}=\left\lvert x_{i_{0}}\right\rvert\e^{\i\theta}=\lambda x_{i_{0}}$. Ainsi, il existe $i_{2}\neq i_{1}$ tel que $x_{i_{2}}=\lambda x_{i_{1}}$. De proche en proche, il existe $i_{q}\neq i_{q-1}$ tel que $x_{i_{q}}=\lambda x_{i_{q-1}}$ (avec $q\geqslant1$) et $x_{i_{q}}=\lambda^{q}x_{i_{0}}$. Or \function{\varphi}{\N}{\llbracket1,n\rrbracket}{k}{i_{k}} n'est pas injective. Donc il existe $k>l$ tel que $i_{k}=i_{l}$ et $x_{i_{k}}=\lambda^{k-k}x_{i_{k}}$ et $k-l>1$ donc 
		\begin{equation}
			\boxed{
				\lambda\in\U_{k-l}
			}
		\end{equation}

		\item L'identité convient, les matrices de permutation aussi. En effet, si $\sigma\in\Sigma_{n}$, on a $P_{\sigma}^{n!}=I_{n}$ donc les valeurs propres sont racines de $X^{n!}-1$ donc $\Sp_{\C}(P_{\sigma})\subset\U_{n!}$.
		
		Réciproquement, soit $A$ stochastique telle que $\Sp_{\C}(A)\subset(\U)$. Soit $i\in\llbracket1,n\rrbracket$, supposons $\left\lvert J_{i_{0}}\right\rvert\geqslant2$. D'après la décomposition de Dunford, il existe $D$ diagonale et $N$ nilpotente qui commutent telles que $A=D+N$ et $\Sp_{\C}(D)=\Sp_{\C}(A)$. Si $N$ est nilpotente d'indice $r\geqslant2$, on a pour tout $k\in\N^{*}$ avec $k\geqslant r$, on a 
		\begin{equation}
			A^{k}=\sum_{j=1}^{k}\binom{k}{j}N^{j}D^{k-j}=\sum_{j=1}^{r}\binom{k}{j}N^{j}D^{k-j}
		\end{equation}

		Pour tout $j\in\llbracket1,r\rrbracket$, on a 
		\begin{equation}
			\binom{k}{j}=\frac{k(k-1)\dots(k-j+1)}{j!}\underset{k\to+\infty}{\sim}\frac{k^{j}}{j!}
		\end{equation}
		Comme $N^{r-1}\neq0$, on a 
		\begin{equation}
			A^{k}\underset{k\to+\infty}{sim}\frac{k^{r-1}}{(r-1)!}N^{r-1}D^{k-r+1}
		\end{equation}
		et les coefficients de $D^{k-r+1}$ sont bornés car $\Sp(D)\subset\U$.

		Or, notons que si $A$ et $B$ sont stochastiques, $AB$ l'est aussi ($\bm{1}$ est toujours valeur propre). Par récurrence, $A^{k}$ l'est. Donc $A^{k}\in\M_{n}([0,1])$, et l'équivalent est impossible si $r\geqslant2$. Donc $r=1$ donc $N=0$ et $A=D$ est diagonalisable.

		Les valeurs propres de $A$ sont des racines de l'unité, soit $m$ le $\ppcm$ des ordres de ces racines (dans $(\U,\times)$). On a alors 
		\begin{equation}
			A=P\diag(\lambda_{1},\dots,\lambda_{n})P^{-1}
		\end{equation}
		d'où 
		\begin{equation}
			A^{m}=P\diag(\lambda_{1}^{m},\dots,\lambda_{n}^{m})P^{-1}
		\end{equation}

		Notons $M=\max\limits_{j\in J_{i_{0}}}\left\lvert a_{i_{0},j}\right\rvert<1$ (car $\left\lvert J_{i_{0}}\right\rvert\geqslant2$ donc pour tout $j\in J_{i_{0}}$, $a_{i_{0},j}\neq1$). On note $a_{i_{0},i_{0}}^{(m)}$ le coefficient $(i_{0},i_{0})$ de $A^{m}$. On a alors 
		\begin{equation}
			a_{i_{0},i_{0}}^{(m)}=1=\sum_{j\in J_{i_{0}}}a_{i_{0},j}a_{j,i_{0}}^{(m-1)}\leqslant M\sum_{j\in J_{i_{0}}}a_{j,i_{0}}^{(m-1)}\leqslant M\sum_{j=1}^{n}a_{j,i_{0}}^{(m-1)}=M
		\end{equation}
		car $A^{m-1}$ est stochastique. Donc $M=1$ ce qui n'est pas possible (par définition de $M$). Ainsi, pour tout $i\in\llbracket1,n\rrbracket$, on a $\lvert J_{i}\rvert=1$ donc il existe un unique $j_{i}\in\llbracket1,n\rrbracket$ avec $a_{i,j_{i}}=1$ et pour tout $j\neq j_{i}$, $a_{i,j}=0$.

		$i\mapsto j_{i}$ est injective, sinon $\rg(A)\leqslant n-1$ et $0\in\Sp(A)$.
	\end{enumerate}
\end{proof}

\begin{remark}
	On peut avoir $\left\lvert \lambda\right\rvert<1$ pour la question 2, par exemple 
	\begin{equation}
		A=
		\begin{pmatrix}
			\frac{1}{n}&\dots&\frac{1}{n}\\
			\vdots&&\vdots\\
			\frac{1}{n}&\dots&\frac{1}{n}
		\end{pmatrix}
	\end{equation}
	On a $A^{2}=A$ et $\rg(A)=1$, $\Sp(A)=\lbrace0,1\rbrace$.
\end{remark}

\begin{remark}
	Par exemple, pour 4, on a 
	\begin{equation}
		A=
		\begin{pmatrix}
			0&1\\
			1&0
		\end{pmatrix}
	\end{equation}
	On a $\chi_{A}=X^{2}-1$ et $\Sp(A)=\lbrace-1,1\rbrace$.
\end{remark}

\begin{remark}
	Si pour tout $(i,j)\in\llbracket1,n\rrbracket$, $a_{i,j}>0$ (i.e.~pour tout $i\in\llbracket1,n\rrbracket$, $J_{i}=\llbracket1,n\rrbracket$). D'après 3, on a $\Sp_{\C}(A)\cap\U=\lbrace1\rbrace$. De plus, si $X=\begin{pmatrix}
		x_{1}&\dots&x_{n}
	\end{pmatrix}^{\mathsf{T}}\in\M_{n,1}(\C)\setminus\lbrace0\rbrace$ vérifie $AX=X$, d'après ce qui précède, on a $x_{1}=\dots=x_{n}$ et le sous-espace propre associé à 1 est de dimension 1.
\end{remark}

\begin{proof}
	\phantom{}
	\begin{enumerate}
		\item Soit $(\lambda,\mu)\in\Sp_{\C}(A)\times\Sp_{\C}(B)$. On a $\mu\in\Sp_{\C}(B^{\mathsf{T}})$. Soit $(X,Y)\in\M_{n-1}(\C)\setminus\lbrace0\rbrace$ vecteurs propres associés respectivement à $\lambda$ et à $\mu$. On pose $M=XY^{\mathsf{T}}$. Alors
		\begin{equation}
			\Phi_{A,B}(M)=AXY^{\mathsf{T}}-XY^{\mathsf{T}}B=(\lambda-\mu)XY^{\mathsf{T}}=(\lambda-\mu)M
		\end{equation}
		donc 
		\begin{equation}
			\boxed{\lambda-\mu\in\Sp(\Phi_{A,B})}
		\end{equation}

		Réciproquement, soit $\alpha\in\Sp(\Phi_{A,B})$. Il existe $M\in\M_{n}(\C)\setminus\lbrace0\rbrace$ tel que l'on ait $AM-MB=\alpha M$ d'où $AM=M(\alpha I_{n}+B)$. Par récurrence, $A^{k}M=M(\alpha I_{n}+B)^{k}$ et par combinaison linéaire, pour tout $P\in\C[X]$ on a $P(A)M=MP(\alpha I_{n}+B)$. En particulier, on prend $P=\chi_{A}$. D'après le théorème de Cayley-Hamilton, on a 
		\begin{equation}
			0=M\chi_{A}(\alpha I_{n}+B)
		\end{equation}
		On a $M\neq0$ donc $\chi_{A}(\alpha I_{n}+B)$ n'est pas inversible. On écrit 
		\begin{equation}
			\chi_{A}(X)=\prod_{k=1}^{n}(X-\lambda_{k})
		\end{equation}
		d'où 
		\begin{equation}
			\chi_{A}(\alpha I_{n}+B)=\prod_{k=1}^{n}(B+(\alpha-\lambda_{k})I_{n})
		\end{equation}
		donc il existe $k_{0}\in\llbracket1,n\rrbracket$ tel que $B+(\alpha-\lambda_{k_{0}})I_{n}$ est non inversible. Donc $\lambda_{k_{0}}-\alpha\in\Sp(B)$ et donc $\alpha$ est une différence d'un élément de $\Sp(A)$ et de $\Sp(B)$.

		\item On forme \function{f_A}{\M_n(\C)}{\M_n(\C)}{M}{AM} et \function{g_B}{\M_n(\C)}{\M_n(\C)}{M}{MB}
		Toujours par récurrence et combinaison linéaires, pour tout $P\in\C[X]$,
		\begin{equation}
			P(f_{A})M=P(A)M
		\end{equation}
		Si $P(A)=0$, on a $P(f_{A})=0$. Si $P(f_{A})=0$, pour $M=I_{n}$, on a $P(A)=0$. De même pour $B$. Donc $\Pi_{A}=\Pi_{f_{A}}$ (polynômes minimaux) et $A$ est diagonalisable si et seulement si $f_{A}(M)$ est diagonalisable. $f_{A}$ et $g_{B}$ commutent car 
		\begin{equation}
			(f_{A}\circ g_{B})(M)=AMB=(g_{B}\circ f_{A})(M)
		\end{equation}
		Donc $f_{A}$ et $g_{B}$ sont codiagonalisables et donc $\Phi_{A,B}$ l'est.
	\end{enumerate}
\end{proof}

\begin{remark}
	Si $(X_{1},\dots,X_{n})$ (respectivement $(Y_{1},\dots,Y_{n})$) est une base de vecteurs propres de $A$ (respectivement de $B^{\mathsf{T}}$), alors $(X_{i}Y_{j}^{\mathsf{T}})_{1\leqslant i,j\leqslant n}$ est une base de vecteurs propres pour $\Phi_{A,B}$.
\end{remark}

\begin{remark}
	C'est faux sur $\R$, par exemple 
	\begin{equation}
		A=B=
		\begin{pmatrix}
			0 & -1\\
			1 &0	
		\end{pmatrix}
	\end{equation}
	On a $\Sp_{\R}=\emptyset$ et $\Phi_{A,A}(I_{2})=0$ donc $0\in\Sp_{\Phi_{A,A}}$.
\end{remark}

\begin{remark}
	Si $\Phi_{A,B}$ est diagonalisable, soit $(M_{i,j})_{1\leqslant i,j\leqslant n}$ une base de vecteurs propres de $\Phi_{A,B}$. Soit $\lambda\in\Sp_{\C}(B)$ et $X\in\M_{n,1}(\C)\setminus\lbrace0\rbrace$ tel que $BX=\lambda X$. On a 
	\begin{equation}
		AM_{i,j}=M_{i,j}(B+\lambda_{i,j}I_{n})
	\end{equation}
	avec $\Phi_{A,B}(M_{i,j})=\lambda_{i,j}M_{i,j}$. Donc 
	\begin{equation}
		AM_{i,j}X=(\lambda+\lambda_{i,j})M_{i,j}X
	\end{equation}
	Pour tout $X_{0}\in\M_{n,1}(\C)$, il existe $M\in\M_{n}(\C)$ tel que $X_{0}=MX$. $M\in\Vect(M_{i,j})_{1\leqslant i,j\leqslant n}$ donc 
	\begin{equation}
		\Vect(M_{i,j}X)_{1\leqslant i,j\leqslant n}=M_{n,1}(\C)
	\end{equation}
	On peut donc en extraire une base: c'est une base de vecteurs propres de $A$.
\end{remark}

\begin{proof}
	\phantom{}
	\begin{enumerate}
		\item Par récurrence, pour tout $k\in\N$, on a $A^{k}M=\theta^{k}MA^{k}$, or $F$ est un sous-espace vectoriel donc par combinaisons linéaires, pour tout $P\in\K[X]$, on a $P(A)M=MP(\theta A)$.
		
		\item Soit $X\in\ker(A-\lambda I_{n})$. On a $AMX=\theta MAX=\lambda\theta MX$. On a donc $MX\in\ker(A-\lambda\theta I_{n})$.
		
		Si pour tout $\lambda\in\Sp_{\C}(A)$, on a $\theta\lambda\notin\Sp_{\C}(A)$, alors si $\lambda\in\Sp_{\C}(A)$ et $X\in\ker(A-\lambda I_{n})$, alors $\ker(A-\lambda\theta I_{n})=\lbrace0\rbrace$. Donc $MX=0$. Or les vecteurs propres forment une famille génératrice donc $M=0$ et $F=\lbrace0\rbrace$.

		S'il existe $\lambda_{0}\in\Sp_{\C}(A)$ tel que $\theta\lambda_{0}\in\Sp_{\C}(A)$. Soit $X_{1}$ un vecteur propre de $A$ associé à $\lambda_{0}$. On complète $(X_{1})$ en $\mathcal{B}=(X_{1},\dots,X_{n})$ base de $\C^{n}$ formé de vecteurs propres de $A$. On définit $MX_{1}=Y_{1}\in\ker(A-\lambda_{0}\theta I_{n})\setminus\lbrace0\rbrace$ et pour tout $i\in\llbracket2,n\rrbracket$, on a $MX_{i}=0$. Ainsi, pour tout $i\in\llbracket2,n\rrbracket$, on a 
		\begin{equation}
			AMX_{i}=0=\theta MAX_{i}=\theta\lambda_{i}MX_{i}
		\end{equation}
		et 
		\begin{equation}
			AMX_{1}=AY_{1}=\lambda_{0}\theta Y_{1}=\theta MAX_{1}=\theta\lambda_{0}X_{1}
		\end{equation}
		Donc $M\neq0$ et $M\in F$. Finalement, on a $F=\lbrace0\rbrace$ si et seulement si pour tout $\lambda\in\Sp_{\C}(A),\theta\lambda\notin\Sp_{\C}(A)$.

		\item On écrit $\chi_{A}=\prod_{j=1}^{r}(X-\lambda_{j})^{m_{j}}$ avec $\lambda_{j}$ distincts et $m_{j}\geqslant1$. D'après le théorème de Cayley-Hamilton et le lemme des noyaux, on a 
		\begin{equation}
			\C^{n}=\bigotimes_{j=1}^{r}\ker(A-\lambda_{j}I_{n})^{m_{j}}
		\end{equation}

		Supposons $\theta\neq0$. Si $M\in F$ et si $x\in \ker(A-\lambda_{j}I_{n})^{m_{j}}$. On a 
		\begin{equation}
			\left(\left(\frac{X}{\theta}-\lambda_{j}\right)^{m_{j}}\right)(A)(Mx)=M\left(A-\lambda_{j}I_{n}\right)^{m_{j}}(x)=0
		\end{equation}
		Donc 
		\begin{equation}
			Mx\in\ker\left(\frac{1}{\theta}A-\lambda_{j}I_{n}\right)^{m_{j}}=\ker\left(A-\theta\lambda_{j}I_{n}\right)^{m_{j}}
		\end{equation}
		car $\theta\neq0$.

		De plus, $\ker(A-\theta\lambda_{j} I_{n})^{m_{j}}\neq\lbrace0\rbrace0$ si et seulement si $\ker(A-\theta\lambda_{j}I_{n})\neq\lbrace0\rbrace$ car \begin{equation}
			\det\left[(A-\theta\lambda_{j}I_{n})^{m_{j}}\right]=\det\left[(A-\theta\lambda_{j}I_{n})\right]^{m_{j}}
		\end{equation}

		Si pour tout $\lambda\in\Sp_{\C}(A)$, $\lambda\theta\notin\Sp_{\C}(A)$, soit $x\in\ker(A-\lambda_{j}I_{n})^{m_{j}}$. On a 
		\begin{equation}
			Mx\in\ker(A-\theta\lambda_{j}I_{n})^{m_{j}}=\lbrace0\rbrace
		\end{equation}
		donc $M=0$ car $\C^{n}=\bigotimes_{j=1}^{r}\ker(A-\lambda_{j}I_{n})^{m_{j}}$.

		S'il existe $\lambda_{0}\in\Sp_{\C}(A)$ tel que $\lambda_{0}\theta\in\Sp_{\C}(A)$, soit $x_{1}\in\ker(A-\lambda_{0}I_{n})\neq\lbrace0\rbrace$. On pose 
		\begin{equation}
			Mx_{1}=y_{1}\in\ker(A-\lambda_{0}\theta I_{n})\setminus\lbrace0\rbrace
		\end{equation}
		On complète $(x_{1})$ en $\mathcal{B}=(x_{1},\dots,x_{n})$ base de $\C^{n}$ formée de vecteurs appartenant à 
		\begin{equation}
			\bigcup_{j=1}^{r}\ker(A-\lambda_{j}I_{n})^{m_j}	
		\end{equation}
		On a pour tout $i\in\llbracket2,n\rrbracket$, $Mx_{i}=0$. On a $M\neq0$ et 
		\begin{equation}
			AMx_{1}=Ay_{1}=\theta\lambda_{0}y_{1}=\theta\lambda_{0}Mx_{1}
		\end{equation}
		Pour tout $i\in\llbracket2,n\rrbracket$, on a $AMx_{i}=0$ si $x_{i}\in\ker(A-\lambda_{j_{i}}I_{n})^{m_{j_{i}}}$ et si $\lambda_{j_{i}}\neq\lambda_{0}$. On a $Ax_{i}\in\ker(A-\lambda_{j_{i}}I_{n})^{m_{j_{i}}}$ donc 
		\begin{equation}
			Ax_{i}\in\Vect(x_{2},\dots,x_{n})
		\end{equation}
		et $MAx_{i}=0$ donc $AMx_{i}=\theta MAx_{i}$.

		Si $F\neq\lbrace0\rbrace$, il existe $M\neq0$ tel que $AM=\theta MA$. Pour tout $P\in\C[X]$, on a $P(A)M=MP(\theta A)$. En particulier, pour $P=\chi_{A}$, on a
		\begin{equation}
			M\chi_{A}(\theta A)=0
		\end{equation}
		$M\neq0$ et donc $\chi_{A}(\theta A)$ n'est pas inversible. Si $\chi_{A}=\prod_{k=1}^{n}(X-\lambda_{k})$, il existe $k\in\llbracket1,n\rrbracket$, $(\theta A-\lambda_{k}I_{n})$ est non inversible, d'où 
		\begin{equation}
			\boxed{\lambda_{k}\in\Sp_{\C}(A)\cap\Sp_{\C}(\theta A)}
		\end{equation}
	\end{enumerate}
\end{proof}

\begin{proof}
	On a 
	\begin{align}
		\chi_{A}(\lambda)
		&=
		\begin{vmatrix}
			\lambda-1	&-1			&0			&-1\\
			-1			&\lambda-1	&-1			&0\\
			-1			&0			&\lambda-1	&-1\\
			0			&-1			&-1			&\lambda-1
		\end{vmatrix}\\
		&=(\lambda-3)
		\begin{vmatrix}
			1	&1			&1			&1\\
			-1			&\lambda-1	&-1			&0\\
			-1			&0			&\lambda-1	&-1\\
			0			&-1			&-1			&\lambda-1
		\end{vmatrix}\\
		&=(\lambda-3)
		\begin{vmatrix}
			1	&0			&0			&0\\
			-1			&\lambda-1	&-1			&0\\
			-1			&0			&\lambda-1	&-1\\
			0			&-1			&-1			&\lambda-1
		\end{vmatrix}\\
		&=(\lambda-3)
		\begin{vmatrix}
			\lambda &0 &1\\
			1 &\lambda &0\\
			-1 &-1 &\lambda-1
		\end{vmatrix}\\
		&=(\lambda-3)
		\begin{vmatrix}
			\lambda-1 &0 &1\\
			1-\lambda &\lambda &0\\
			1-\lambda &-1 &\lambda-1
		\end{vmatrix}\\
		&=(\lambda-3)(\lambda-1)
		\begin{vmatrix}
			1 &0 &1\\
			-1 &\lambda &0\\
			-1 &-1 &\lambda-1
		\end{vmatrix}\\
		&=(\lambda-3)(\lambda-1)
		\begin{vmatrix}
			1 &0 &1\\
			0 &\lambda &1\\
			0 &-1 &\lambda
		\end{vmatrix}\\
		&=(\lambda-3)(\lambda-1)(\lambda^{2}+1)
	\end{align}
	où l'on a fait successivement les opérations suivantes: $L_{1}\leftarrow L_{1}+L_{2}+L_{3}+L_{4}$, $C_{i}\leftarrow C_{i}-C_{1}$ pour $i\in\lbrace2,3,4\rbrace$, développement selon la première ligne, $C_{1}\leftarrow C_{1}-C_{2}-C_{3}$, $L_{i}\leftarrow L_{i}+L_{1}$ pour $i\in\lbrace2,3\rbrace$, développement selon la première colonne.

	$\chi_{A}$ est scindé à racines simples sur $\C$ donc $A$ est diagonalisable. On trouve ensuite un vecteur propre dans chaque sous-espace propre (qui sont de dimension un).
\end{proof}

\begin{proof}
	\phantom{}
	\begin{enumerate}
		\item On a $\lambda\in\Sp_{\R}(A)$ si et seulement s'il existe $X\in\M_{n,1}(\R)\setminus\lbrace0\rbrace$ telle que $AX=\lambda X$ si et seulement si 
		\begin{equation}
			\left\lbrace
				\begin{array}[]{lll}
					\sum_{i\neq 1}a_{i}x_{i}&=&\lambda x_{1}\\
					\vdots\\
					\sum_{i\neq j}a_{i}x_{i}&=&\lambda x_{1}\\
					\vdots\\
					\sum_{i\neq n}a_{i}x_{i}&=&\lambda x_{1}\\
				\end{array}
			\right.
		\end{equation}

		Soit $S=\sum_{i=1}^{n}a_{i}x_{i}$. Ce système équivaut à 
		\begin{equation}
			S=(\lambda+a_{1})x_{1}=\dots=(\lambda+a_{n})x_{n}
		\end{equation}

		Si $S=0$, pour tout $i\in\llbracket1,n\rrbracket$, on a $\lambda=-a_{i}$ ou $x_{i}=0$ (et $X\neq0$). Les $(a_{i})_{1\leqslant i\leqslant n}$, il existe un unique $i_{0}\in\llbracket1,n\rrbracket$ tel que $\lambda=-a_{i_{0}}$ et pour tout $i\neq i_{0}$, on a $x_{i}=0$. En reportant, on a $S=0=\lambda x_{i_{0}}$ donc $\lambda=0$ ce qui est impossible car $0=\lambda=-a_{i_{0}}>0$.

		Donc $S\neq0$ et pour tout $i\in\llbracket1,n\rrbracket$, $\lambda+a_{i}\neq0$ et pour tout $i\in\llbracket1,n\rrbracket$, $x_{i}=\frac{S}{\lambda+a_{i}}$. On a alors 
		\begin{equation}
			S=\sum_{i=1}^{n}a_{i}x_{i}=\sum_{i=1}^{n}\frac{a_{i}S}{\lambda+a_{i}}
		\end{equation}
		donc 
		\begin{equation}
			\boxed{
				\sum_{i=1}^{n}\frac{a_{i}}{\lambda+a_{i}}=1
			}
		\end{equation}

		Réciproquement, on prend $x_{i}=\frac{1}{\lambda+a_{i}}$ et on a bien $AX=\lambda X$.

		\item On définit \function{f}{\R\setminus\lbrace -a_{n},\dots,-a_{1}\rbrace}{\R}{x}{\sum_{i=1}^{n}\frac{a_{i}}{x+a_{i}}}
		\item Posons $-a_{n+1}=-\infty$ et $-a_{0}=+\infty$. Sur $]-a_{k+1},-a_{k}[$, on a 
		\begin{equation}
			f'(x)=\sum_{i=1}^{n}\frac{-a_{i}}{(x+a_{i})^{2}}	
		\end{equation}

		Les $(a_{i})_{1\leqslant i\leqslant n}$ étant positifs, on a $\lim\limits_{x\to-a_{k+1}^{+}}f(x)=+\infty$ et $\lim\limits_{x\to-a_{k}^{-}}f(x)=-\infty$ (si $k\neq n$) (et $\lim\limits_{x\to-\infty}f(x)=\lim\limits_{x\to+\infty}f(x)=0$).

		D'après le théorème des valeurs intermédiaires, pour tout $k\in\llbracket0,n-1\rrbracket$, il existe un unique $\lambda_{k}\in]-a_{k+1},-a_{k}[$ tel que $f(\lambda_{k})=1$. Donc $A$ admet exactement $n$ valeurs propres réelles distinctes. Donc $A$ est diagonalisable sur $\R$.
	\end{enumerate}
\end{proof}

\begin{remark}
	Soit 
	\begin{equation}
		F(X)=-\sum_{k=1}^{n}\frac{a_{k}}{X+a_{k}}+1=\frac{P(X)}{(X+a_{1}\dots(X+a_{n}))}
	\end{equation}
	avec $P=(X+a_{1})\dots(X+a_{n})-\sum_{k=1}^{n}a_{k}P_{k}$ où $P_{k}=\prod_{\substack{i=1\\i\neq k}}(X+a_{i})$ de degré $n-1$. On a $\deg(P)=n$ et son coefficient dominant est 1. De plus, pour tout $\lambda\in\R$, on a $P(\lambda)=0$ si et seulement si $\sum_{k=1}^{n}\frac{a_{k}}{\lambda+a_{k}}=1$ si et seulement si $\lambda\in\Sp(A)$ donc $P=\chi_{A}$.
\end{remark}

\begin{proof}
	On a 
	\begin{equation}
		\begin{pmatrix}
			1 &0 &\dots&\dots&\dots&\dots&0\\
			0&\frac{1}{2}&\ddots&&&&\vdots\\
			\vdots&\ddots & \ddots &\ddots &&\frac{\lambda}{j}&\vdots\\
			\vdots&& \ddots & \ddots &\ddots&&\vdots\\
			\vdots&&&\ddots&\ddots&\ddots&\vdots\\
			\vdots&&&&\ddots&\ddots&0\\
			0&\dots&\dots&\dots&\dots&0&\frac{1}{n}\\
		\end{pmatrix}
		\times\diag(1,2,\dots,n)
		=
		\begin{pmatrix}
			1 &0 &\dots&\dots&\dots&\dots&0\\
			0&1&\ddots&&&&\vdots\\
			\vdots&\ddots & \ddots &\ddots &&\lambda&\vdots\\
			\vdots&& \ddots & \ddots &\ddots&&\vdots\\
			\vdots&&&\ddots&\ddots&\ddots&\vdots\\
			\vdots&&&&\ddots&\ddots&0\\
			0&\dots&\dots&\dots&\dots&0&1\\
		\end{pmatrix}
	\end{equation}
	où le coefficient est à la $i$-ième ligne et la $j$-ième colonne. La matrice à gauche est diagonalisable car son polynôme caractéristique est scindé à racines simples. Donc les matrices de transvections sont dans $G$. De plus, les matrices de dilatations sont aussi dans $G$. Donc $G=GL_{n}(\R)$.
\end{proof}

\begin{proof}
	Supposons $u$ diagonalisable, il existe un base $\mathcal{B}$ telle que 
	\begin{equation}
		\mat_{\mathcal{B}}(u)=A=\diag(0,\dots,0,\lambda_{1},\dots,\lambda_{r})
	\end{equation}
	avec $\lambda_{i}\neq0$. Donc $\mat_{\mathcal{B}}(u^{p})=A^{p})\diag(0,\dots,0,\lambda_{1}^{p},\dots,\lambda_{r}^{p})$ donc $u^{p}$ est diagonalisable. On a toujours $\ker(u)\subset\ker(u^{2})$ et la forme diagonale implique $\ker(u)=\ker(u^{2})$.

	Supposons $u^{p}$ diagonalisable, on écrit $\Pi_{u^{p}}=(X-\lambda_{0})\dots(X-\lambda_{r})=R$ (avec $\lambda_{k}\neq0$ pour tout $k\geqslant k$) qui est scindé à racines simples. On a 
	\begin{equation}
		P(u^{p})=0=(u^{p}-\lambda_{0}id_{E})\circ\dots\circ(u^{p}-\lambda_{r}id_{E})=Q(u)
	\end{equation}
	avec $Q(X)=P(X^{p})$. 

	Si $\lambda_{0}\neq0$, chaque $\lambda_{k}$ admet $p$ racines $p$-ièmes distinctes et si $\mu_{k}$ est l'une de ses racines, on a 
	\begin{equation}
		X^{p}-\lambda_{k}=\prod_{j=1}^{p}\left(X-\mu_{k}\e^{\i\frac{2j\pi}{p}}\right)
	\end{equation}
	De plus, les racines $p$-ièmes des $(\lambda_{k})_{kk\in\llbracket1,r\rrbracket}$ sont deux à deux distinctes. Donc $Q$ est scindé à racines simples, et donc $u$ est diagonalisable.

	Si $\lambda_{0}=0$, on a $Q=X^{p}A(X)$ avec $A$ scindé à racines simples non nulles et $X^{p}\wedge A=1$. D'après le lemme des noyaux, on a 
	\begin{equation}
		\ker(Q(u))=\C^{n}=\ker(u^{p})\bigotimes\ker(A(u))=\ker(u^{p})\bigotimes_{i\in I}\ker(u-\mu_{i}id)
	\end{equation}
	car $A$ est scindé à racines simples.
	Montrons que $\ker(u)=\ker(u^{p})$. L'inclusion directe est évidente. Réciproquement, montrons que pour tout $k\in\N$, on a $\ker(u^{k})\subset\ker(u^{k+1})$ et si $\ker(u^{k})=\ker(u^{k+1})$, alors $\ker(u^{k+1})=\ker(u^{k+2})$. L'inclusion est évidente, et si on a l'égalité, si $x\in\ker(u^{k+2})$, on a $u(x)\in\ker(u^{k+1})=\ker(u^{k})$ donc $x\in\ker(u^{k+1})$.
	Comme $\ker(u)=\ker(u^{2})$, d'après ce qui précède, par récurrence, on a $\ker(u)=\ker(u^{p})$, donc $u$ est diagonalisable.
\end{proof}

\begin{proof}
	Soit $(e_{1},\dots,e_{n})$ la base canonique de $\C^{n}$, $u$ canoniquement associée à 
	\begin{equation}
		J_{n}=
		\begin{pmatrix}
			0 & 1 & 0&\dots&0\\
			\vdots & \ddots & \ddots &\ddots& \vdots\\
			\vdots&&\ddots&\ddots&0\\
			0 &&&\ddots&1\\
			1 & 0&\dots &\dots&0
		\end{pmatrix}
	\end{equation}. On a 
	\begin{equation}
		\left\lbrace
			\begin{array}[]{lll}
				u(e_{1})&=&e_{n}\\
				u(e_{2})&=&e_{1}\\
				\vdots\\
				u(e_{n})&=&e_{n-1}
			\end{array}
		\right.
	\end{equation}
	d'où 
	\begin{equation}
		\left\lbrace
			\begin{array}[]{lll}
				u^{k}(e_{1})&=&e_{n+1-k}\\
				\vdots
				u^{k}(e_{k-1})&=&e_{n-1}\\
				\vdots\\
				u^{k}(e_{n})&=&e_{n-k}
			\end{array}
		\right.
	\end{equation}
	et donc 
	\begin{equation}
		J_{n}^{k}=
		\begin{pmatrix}
			0 & \dots & \dots&0 &1 &0&\dots &0\\
			\vdots&\ddots&&&\ddots&\ddots&\ddots&\vdots\\
			\vdots&&\ddots&&&\ddots&\ddots&0\\
			0&&&\ddots&&&\ddots&1\\
			1&\ddots&&&\ddots&&&0\\
			0&\ddots&\ddots&&&\ddots&&\vdots\\
			\vdots&\ddots&\ddots&\ddots&&&\ddots&\vdots\\
			0&\dots&0&1&0&\dots&\dots&0
		\end{pmatrix}
	\end{equation}
	où les $1$ commencent à la $k+1$-ième colonne sur la première ligne et à la $n-k+1$-ième ligne sur la première colonne. Notamment, le 1 sur la dernière colonne est à la $n-k$-ième ligne.

	On a $A(a_{0},\dots,a_{n})=\sum_{k=0}^{n-1}a_{k}J_{n}^{k}$. En développant par rapport à la première ligne, on a 
	\begin{equation}
		\chi_{J_{n}}(X)=X
		\begin{vmatrix}
			X&-1&0&\dots&\dots&0\\
			0&\ddots&\ddots&\ddots&&\vdots\\
			\vdots&\ddots&\ddots&\ddots&\ddots&\vdots\\
			\vdots&&\ddots&\ddots&\ddots&0\\
			\vdots&&&\ddots&\ddots&-1\\
			0&\dots&\dots&\dots&0&X
		\end{vmatrix}+
		\begin{vmatrix}
			0&-1&0&\dots&\dots&0\\
			0&X&\ddots&\ddots&&\vdots\\
			\vdots&\ddots&\ddots&\ddots&\ddots&\vdots\\
			\vdots&&\ddots&\ddots&\ddots&0\\
			0&&&\ddots&\ddots&-1\\
			-1 &0&\dots&\dots&0&X
		\end{vmatrix}
	\end{equation}
	Le premier déterminant vaut $X^{n-1}$ et le deuxième vaut $-(-1)^{n}\times(-1)^{n-2}=-1$ donc $\chi_{J_{n}}(X)=X^{n}-1$.
	Ainsi, $\chi_{J_{n}}$ est scindé à racines simples sur $\C$ donc $J_{n}$ est diagonalisable avec des sous-espaces propres de dimension 1. Soit $\omega=\e^{\frac{2\i\pi}{n}}$, on a $\Sp(J_{n})=\left\lbrace\omega^{k},0\leqslant k\leqslant n-1\right\rbrace$. On a $J_{n}X=\omega^{k}X$ si et seulement si 
	\begin{equation}
		\left\lbrace
			\begin{array}[]{lll}
				x_{2}&=&\omega^{k}x_{1}\\
				\vdots\\
				x_{n}&=&=\omega^{k}x_{n-1}\\
				x_{1}&=&\omega^{k}x_{n}
			\end{array}
		\right.
	\end{equation}
	si et seulement si 
	\begin{equation}
		X=x_{1}\begin{pmatrix}
			1\\
			\omega^{k}\\
			\omega^{2k}\\
			\vdots\\
			(\omega^{k})^{n-1}
		\end{pmatrix}=x_{1}X_{k}
	\end{equation}
	avec $X_{k}$ vecteur propre de $J_{n}$ associé à $\omega^{k}$. Posons 
	\begin{equation}
		P=
		\begin{pmatrix}
			1&1&\dots&1\\
			\vdots&\omega&&\omega^{n-1}\\
			\vdots&\vdots&&\vdots\\
			1&\omega^{n-1}&\dots&(\omega^{n-1})^{n-1}
		\end{pmatrix}
	\end{equation}
	et $P^{-1}J_{n}P=\diag(1,\omega,\dots,\omega^{n-1})$. On a donc $P^{-1}A(a_{0},\dots,a_{n})P=\diag(Q(1),Q(\omega),\dots,Q(\omega^{n-1}))$ où $Q=\sum_{k=0}^{n-1}a_{k}X^{k}$.
	Donc $A$ est diagonalisable de valeurs propres $Q(1),\dots,Q(\omega^{n-1})$ et donc
	\begin{equation}
		\boxed{\det(A)=\prod_{k=0}^{n-1}Q(\omega^{k})}
	\end{equation}
\end{proof}

\begin{remark}
	On a 
	\begin{equation}
		\begin{vmatrix}
			a&b&c\\
			c&a&b\\
			b&c&a
		\end{vmatrix}=(a+b+c)(a+\j b+\j^{2}c)(a+\j^{2}b+\j c)=(a+b+c)(a^{2}+b^{2}+c^{2}-ab-bc-ac)
	\end{equation}

	Si $a,b,c\in\R_{+}$ vérifient $a+b+c=1$, on a 
	\begin{equation}
		\left\lvert a+\j b+\j^{2}c\right\rvert=\left\lvert a+\j^{2}b+\j c\right\rvert\leqslant a+b+c=1
	\end{equation}
	si et seulement si $a,\j b,\j^{2}c$ ont même argument si et seulement si $\lbrace a,b,c\rbrace=\lbrace1,0,0\rbrace$.
\end{remark}

\begin{proof}
	On sait que que $f^{n}=0$ d'après le théorème de Cayley-Hamilton et que pour tout $k\in\N$, $\ker(f^{k})\subset\ker(f^{k+1})$ et si $\ker(f^{k})=\ker(f^{k+1})$, alors $\ker(f^{k})=\ker(f^{m})$ pour tout $m\geqslant k$.

	Soit $k\in\llbracket0,n-1\rrbracket$ et \function{u}{\ker(f^{+1})}{\ker(f^{k})}{x}{u(x)} est bien définie car si $x\in\ker(f^{k+1}),f(x)\in\ker(f^{k})$. Comme $\ker(f)\subset\ker(f^{k+1})$, $\ker(u)=\ker(f)$ et $\dim(\ker(u))=1$. D'après le théorème du rang, on a $\dim(\ker(f^{k+1}))=\rg(u)+1\leqslant\dim(\ker(f^{k}))+1$. Par récurrence, on a pour tout $k\in\N$, $\dim(\ker(f^{k}))\leqslant k$ (car on ne peut croître au lus de 1 à chaque itération).

	Si $f^{n-1}=0$, on a $\dim(\ker(f^{n-1}))=n\leqslant n-1$ ce qui est absurde. Donc 
	\begin{equation}
		\boxed{f^{n-1}\neq0}
	\end{equation}

	Soit $x\notin\ker(f^{n-1})$. Soit $(\alpha_{0},\dots,\alpha_{n-1})\in\K^{n}$. Si $\alpha_{0}x+\dots+\alpha_{n-1}f^{n-1}(x)=0$, en appliquant $f^{n-1}$, on a $\alpha_{0}f^{n-1}(x)=0$ donc $\alpha_{0}=0$. Puis on applique $f^{n-2}$, etc. De proche en proche, $\alpha_{0}=\alpha_{1}=\dots=\alpha_{n-1}=0$. Ainsi, $\mathcal{B}=(x,f(x),\dots,f^{n-1}(x))$ est libre en dimension $n$, c'est donc une base et on a 
	\begin{equation}
		\mat_{\mathcal{B}}(f)=
		\begin{pmatrix}
			0&\dots&\dots&\dots&0\\
			1&\ddots&&&\vdots\\
			0&\ddots&\ddots&&\vdots\\
			\vdots&\ddots&\ddots&\ddots&\vdots\\
			0&\dots&0&1&0
		\end{pmatrix}
	\end{equation}
	qui est une matrice nilpotente d'indice $n$. Matriciellement, on a $\ker(f^{k})=\Vect(e_{n-k+1},\dots,e_{n})$.
\end{proof}

\begin{proof}
	Supposons qu'il existe $x\in V$, $(x,u(x),\dots,u^{n-1}(x))$ soit une base de $V$. Notons $u^{n}(x)=a_{0}x+\dots+a_{n-1}u^{n-1}(x)$. Soit $y\in V$ tel que $u(y)=\lambda y$. Pour $y=\sum_{i=0}^{n-1}y_{i}u^{i}(x)$. On a donc 
	\begin{equation}
		u(y)=\sum_{i=0}^{n-1}y_{i}u^{i+1}(x)=\sum_{i=0}^{n-1}\lambda y_{i}u^{i}(x)=\sum_{i=1}^{n-1}y_{i-1}u^{i}(x)+y_{n-1}\sum_{i=0}^{n-1}a_{i}u^{i}(x)
	\end{equation}
	Donc $u(y)=\sum_{i=1}^{n-1}u^{i}(x)(y_{i-1}+y_{n-1}a_{i})+y_{n-1}a_{0}x$ donc 
	\begin{equation}
		\left\lbrace
			\begin{array}[]{lll}
				\lambda y_{0} &= &y_{n-1}a_{0}\\
				\lambda y_{1} &= &y_{0}+a_{1}y_{n-1}\\
				\vdots\\
				\lambda y_{n-2} &= &y_{n-3}+a_{n-2}y_{n-1}\\
				\lambda y_{n-1} &= &y_{n-2}+a_{n-1}y_{n-1}
			\end{array}
		\right.
	\end{equation}
	donc par récurrence 
	\begin{equation}
		\left\lbrace
			\begin{array}[]{lll}
				\lambda y_{n-2} &= &(\lambda -a_{n-1})y_{n-1}\\
				\lambda y_{n-3} &= &(\lambda(\lambda-a_{n-1})-a_{n-2})y_{n-1}\\
				\vdots\\
				\lambda y_{0} &= &(\lambda^{n-1}-a_{n-1}\lambda^{n-2}-\dots-a_{1})y_{n-1}
			\end{array}
		\right.
	\end{equation}
	Donc les sous-espaces propres sont de dimension 1.

	Supposons que les sous-espaces propres de $u$ sont de dimension 1. On écrit $\chi_{u}=\prod_{i=1}^{r}(X-\lambda_{i})^{n_{i}}$. D'après le théorème de Cayley-Hamilton et le lemme des noyaux, on a 
	\begin{equation}
		V=\bigotimes_{i=1}^{r}\underbrace{\ker(u-\lambda_{i}id_{V})^{n_{i}}}_{F_{i}}
	\end{equation}
	et les sous-espaces caractéristiques $F_{i}$ sont stables par $u$. Soit $v_{i}=u_{\mid F_{i}}-\lambda_{i}id_{F_{i}}$. On a $\chi_{u}=\prod_{i=1}^{r}\chi_{u_{\mid F_{i}}}$ (matrice diagonale par blocs dans un base adaptée). $(X-\lambda_{i})^{n}$ annule $u_{\mid F_{i}}$ et $\Sp_{F_{i}}(u_{\mid F_{i}})=\lbrace\lambda_{i}\rbrace$. Alors $\chi_{u_{\mid F_{i}}}=(X-\lambda_{i})^{\dim(F_{i})}$. En reportant, on a $\dim(F_{i})=n_{i}$. De plus, $V_{i}^{n_i}=0$ donc $v_{i}$ est nilpotent. On a donc $\dim(\ker(v_{i}))=\dim(\ker(u-\lambda_{i}id_{E}))=1$. Donc il existe $x_{i}\in F_{i}$ tel que $(x_{i},v_{i}(x_{i}),\dots,v_{i}^{n_{i}-1}(x_{i}))$ soit une base de $F_{i}$.

	On forme $x=\sum_{i=1}^{r}x_{i}$. Soit $(\alpha_{0},\dots,\alpha_{r-1})$ tel que $\sum_{j=0}^{n-1}\alpha_{j}u^{j}(x)=0=\sum_{i=1}^{r}\left(\sum_{j=0}^{n-1}\alpha_{j}u^{j}(x_{i})\right)$. Les $F_{i}$ sont en somme directe donc 
	\begin{equation}
		\sum_{j=0}^{n-1}\alpha_{j}u^{j}(x_{i})=0
	\end{equation}

	Soit $P(X)=\sum_{j=0}^{n-1}\alpha_{j}X^{j}$. $I_{x_{i}}=\left\lbrace A\in\C[X]\middle| A(u)(x_{i})=0\right\rbrace$ est un idéal de $\C[X]$ donc est principal et il existe $\Pi_{i}\in I_{x_{i}}$ minimal et 
	\begin{equation}
		\Pi_{i}\mid P
	\end{equation}
	On a $(X-\lambda_{i})^{n_{i}}(u)(x_{i})=0$ et $(x_{i},u(x_{i}),\dots,u^{n_{i}-1}(x))$ est libre, donc si $P\in I_{x_{i}}$, $\deg(P)\geqslant n_{i}$ donc $\deg(\Pi_{i})=n_{i}$ et $\Pi_{i}=(X-\lambda_{i})^{n_{i}}$. Ainsi, pour tout $i\in\llbracket1,r\rrbracket$, $\Pi_{i}\mid P$ et donc 
	\begin{equation}
		\prod_{i=1}^{r}(X-\lambda_{i})^{n_{i}}\mid P
	\end{equation}
	Mais $P$ est de degré $\leqslant n-1$, nécessairement $P=0$ et $(x,u(x),\dots,u^{n-1}(x))$ est libre.
\end{proof}

\begin{remark}
	Autre méthode pour le sens direct: on a 
	\begin{equation}
		\mat_{(x,u(x),\dots,u^{n-1}(x))}(u)=
		\begin{pmatrix}
			0&\dots&\dots&0&a_{0}\\
			1&\ddots&&\vdots&\vdots\\
			0&\ddots&\ddots&\vdots&\vdots\\
			\vdots&\ddots&\ddots&0&\vdots\\
			0&\dots&0&1&a_{n-1}
		\end{pmatrix}=A
	\end{equation}

	Si $\lambda\in\Sp(u)$, on a 
	\begin{equation}
		A-\lambda I_{n}=
		\mat_{(x,u(x),\dots,u^{n-1}(x))}(u)=
		\begin{pmatrix}
			-\lambda&\dots&\dots&0&a_{0}\\
			1&\ddots&&\vdots&\vdots\\
			0&\ddots&\ddots&\vdots&\vdots\\
			\vdots&\ddots&\ddots&-\lambda&\vdots\\
			0&\dots&0&1&a_{n-1}-\lambda
		\end{pmatrix}
	\end{equation}
	qui est non inversible, mais donc les $(n-1)$ première colonnes sont libres, donc est de rang $n-1$.
\end{remark}

\begin{proof}
	\phantom{}
	\begin{enumerate}
		\item On utilise le fait que pour tout $k\in\N$ tel que $\im(f^{k+1})\subset\im(f^{k})$. S'il existe $k\in\N$, $\im(f^{k+1})=\im(f^{k})$ alors pour tout $l\geqslant k$, $\im(f^{k})=\im(f^{l})$. 

		En effet, si $x=f^{k+1}(x')\in\im(f^{k+1})$,, on a $x=f^{k}(f(x))\in\im(f^{k})$. Si on a égalité des espaces, soit $x=f^{k+1}(x')=f(f^{k}(x'))\in\im(f^{k+1})$. Alors $f^{k}(x')\in\im(f^{k})=\im(f^{k+1})$ donc il existe $x''$ tel que $f^{k}(x')=f^{k+1}(x'')$, mais alors $x=f^{k+2}(x'')\in\im(f^{k+2})$. On a donc le résultat en itérant.

		Ainsi, pour tout $n\geqslant d$, on a $\rg(f^{n})=\rg(f^{d})$ donc $(\rg(f^{n}))_{n\in\N}$ est stationnaire au moins à partir de $d$ et $r(f)=\rg(f^{d})$.

		\item Comme $f$ et $g$ commutent, on a 
		\begin{equation}
			(f+g)^{2d}=\sum_{k=0}^{2d}\binom{2d}{k}f^{k}g^{2d-k}
		\end{equation}
		Pour tot $k\in\llbracket0,2d\rrbracket$, on a $k\geqslant d$ ou $2d-k\geqslant d$ donc 
		\begin{equation}
			\left\lbrace
				\begin{array}[]{l}
					\im(f^{k}g^{2d-k})\subset\im(f^{d})\\
					\text{ou}\\
					\im(f^{k}g^{2d-k})\subset\im(g^{d})
				\end{array}
			\right.
		\end{equation}
		et donc $\im(f^{k}g^{2d-k})\subset\im(f^{d})+\im(g^{d})$. Finalement, $\im(f+g)^{2d}\subset\im(f^{d})+\im(g^{d})$. On a donc 
		\begin{align}
			r(f+g)
			&=\dim(\im(f+g)^{2d})\\
			&\leqslant \dim(\im(f^{d})+\im(g^{d}))\\
			&\leqslant \dim(\im(f^{d}))+\im(g^{d})\\
			&\leqslant r(f)+r(g)
		\end{align}

		Pour un contre-exemple, on utilise $A=\begin{pmatrix}
			0&0\\1&0
		\end{pmatrix}$ et $B=A^{\mathsf{T}}$. On a $A^{2}=B^{2}$ donc $r(A^{2})=r(B^{2})=0$ et $A+B$ inversible donc $r(A+B)=2>r(A)+r(B)$.

		\item On a $\chi_{f}=X^{m_{0}}Q$ avec $\deg(Q)=d-m_{0}$ et $Q(0)=0$. D'après le lemme des noyaux, on a 
		\begin{equation}
			V=\ker(f^{m_{0}})\bigotimes\ker(Q(f))
		\end{equation}
		Dans une base adaptée $\mathcal{B}$, on a $\mat_{\mathcal{B}}(f)=\begin{pmatrix}
			A&0\\0&B
		\end{pmatrix}$ avec $A^{m_{0}}=0$ et $B$ inversible. Alors pour tout $k\geqslant m_{0}$, $\mat_{\mathcal{B}}(f^{k})=\begin{pmatrix}
			0&0\\0&B^{k}
		\end{pmatrix}$ et $\rg(f^{k})=\rg(B^{k})=d-m_{0}=r(f)$.
	\end{enumerate}
\end{proof}


























































\begin{proof}
	Si on a (i), soit $x$ un vecteur propre associé à $\rho(u)=\rho e^{\mathrm{i}\theta}$. On a $\Vert u(x)\Vert=\Vert\rho(u) x\Vert=\rho(u)\Vert x\Vert$ et comme $x\neq0$, on a $\rho(u)\leqslant\vertiii{\rho(u)}<1$ d'où (ii).

	Si (ii), on utilise la décomposition de Dunford $u=n+d$ avec $n$ nilpotent, $d$ diagonalisable et $dn=nd$. Soit $m=\dim(E)$. Pour tout $p\geqslant m$, on a 
	\begin{equation}u^{p}=\sum_{k=0}^{p}\binom{p}{k}n^{k}d^{p-k}=\sum_{k=0}^{m-1}\binom{p}{k}n^{k}\underbrace{d^{p-k}}_{\xrightarrow[p\to+\infty]{}0}\end{equation}
	En effet, on a $k\geqslant m-1$ fixé, il existe une base $\mathcal{B}$ de $E$ telle que 
	\begin{equation}\binom{p}{k}\mat\limits_{\mathcal{B}}(d^{p})=\binom{p}{k}\diag(\lambda_{1}^{p},\dots,\lambda_{m}^{p})\xrightarrow[p\to+\infty]{}0\end{equation}
	car $\vert\lambda_{i}\vert<1$ pour tout $i\in\{1,\dots,m\}$ et 
	\begin{equation}\binom{p}{k}\underset{p\to+\infty}{\sim}\frac{p^{k}}{k!}=\underset{p\to+\infty}{o}\Bigl(\frac{1}{\rho(u)^{p}}\Bigr)\end{equation}
	donc on a (iii).

	Si (iii), soit $x$ un vecteur propré associé à $\lambda\in\C$, on a $u^{p}\xrightarrow[p\to+\infty]{}0$ donc en particulier, $u^{p}(x)=\lambda^{p}\xrightarrow[p\to+\infty]{}0$, donc $\rho(u)^{p}\xrightarrow[p\to+\infty]{}0$ et $\rho(u)\geqslant0$ donc $\rho(u)<1$. Posons encore $u=d+n$ la décomposition de Dunford de $u$. Soit $\varepsilon>0$, il existe $\mathcal{B}_{0}=(e_{1},\dots,e_{n})$ base de $E$ dans laquelle les coefficients de $\mat\limits_{\mathcal{B}_{0}}(n)$ sont en module $\leqslant\varepsilon$. Définissons sur $E$ 
	\begin{equation}\Biggl\Vert\sum_{i=1}^{m}x_{i}e_{i}\Biggr\Vert_{\infty}=\max\limits_{1\leqslant i\leqslant m}\vert x_{i}\vert\end{equation}
	Soit $M=\mat\limits_{\mathcal{B}_{0}}(u)=(m_{i,j})_{1\leqslant i,j\leqslant m}$ triangulaire supérieure avec $m_{ii}=\lambda_{i}$ et pour tout $j\neq i$, $\vert m_{i,j}\vert<\varepsilon$. Soit donc $x=\sum_{i=1}^{m}x_{i}e_{i}\in\C^{m}$, on a 
	\begin{equation}
	\Vert Mx\Vert_{\infty}=\max\limits_{1\leqslant i\leqslant n}\underbrace{\Biggl\vert\sum_{j=1}^{m}m_{i,j}x_{j}\Biggr\vert}_{(\vert\lambda_{i}\vert+(m-1)\varepsilon)\Vert x\Vert_{\infty}}
	\end{equation}
	donc 
	\begin{equation}\vertiii{u}\leqslant\underbrace{\rho(u)}_{<1}+(m-1)\varepsilon\end{equation}
	et on choisit
	\begin{equation}\varepsilon<\frac{1-\rho(u)}{\underbrace{m-1}_{>0}}\end{equation}
	d'où $\vertiii{u}<1$ et donc on a (i) et finalement on a bien l'équivalence.
\end{proof}

\begin{remark}
	$u\mapsto\rho(u)$ n'est pas une norme car pour $u$ nilpotente non nulle, $\rho(u)=0$.
\end{remark}

\begin{proof}
	Supposons (i), soit $Y$ un vecteur propre de $A$ avec $AY=\lambda Y$ pour $\lambda\in\C$. Pour tout $k\in\N,BA^{k}Y=\lambda^{k}BY$ et il existe $k_{0}\in\N$ tel que $\lambda^{k_{0}}BY\neq0$ et $BY\neq0$ donc on a (ii).

	Si (ii), supposons qu'il existe $Y\in\C^{n}\setminus\{0\}$ tel que $\varphi=0$. On note 
	\begin{equation}\chi_{A}=\prod_{i=1}^{r}(X-\lambda_{i})^{m_{i}}\end{equation} avec les $\lambda_{i}$ distincts. Alors $Y=\sum_{i=1}^{r}Y_{i}$ où $Y_{i}\in\ker(A-\lambda_{i}I_{n})$. Il existe $i_{0}\in\{1,\dots,n\}$ tel que $Y_{i_{0}}\neq0$ car $Y\neq0$. On a alors, pour $t\in\R$,
	\begin{equation}B\exp(tA)Y=\sum_{i=1}^{r}B\exp(t\lambda_{i})Y_{i}=0\end{equation}
	Pour tout $k\in\{0,\dots,r-1\}$, on a $\varphi^{(k)}(t)=\sum_{i=1}^{r}B\lambda_{i}^{k}\exp(t\lambda_{i})Y_{i}=0$. Pour $t=0$ on a $\sum_{i=1}^{r}\lambda_{i}^{k}BY_{i}=0$ ce qui, pour $t=0$, donne le système 
	\begin{equation}
	\left\{
		\begin{array}[]{lll}
			BY_{1}+\dots+BY_{r} &= &0\\
			\lambda_{1}BY_{1}+\dots+\lambda_{r}BY_{r} &=& 0\\
			&\vdots&\\
			\lambda_{1}^{r-1}BY_{1}+\dots+\lambda_{r}^{r-1}BY_{r} &= &0
		\end{array}
	\right.
	\end{equation}
	Pour tout $P\in\C_{r-1}[X]$, on a donc $\sum_{i=1}^{r}P(\lambda_{i})BY_{i}=0$. Pour $i\in\{0,\dots,r-1\}$ et $P=\prod_{i\neq j}\frac{(X-\lambda_{j})}{\lambda_{i}-\lambda_{j}}$, on obtient pour tout $i\in\{1,\dots, r\}, BY_{i}=0$. En particulier, $BY_{i_{0}}=0$ et $Y_{i_{0}}$ est un vecteur propre de $A$ car non nul. C'est une contradiction. On a donc (iii).

	\item Soit $Y\in\C^{n}\setminus\{0\}$, supposons que pour tout $k\in\{0,\dots,n-1\}$, $BA^{k}Y=0$. Soit $k\geqslant n$, il existe $(Q_{k},R_{k})\in\C[X]\times\C_{n-1}[X]$ tel que 
	\begin{equation}X^{k}=Q_{k}\chi_{A}+R_{k}\end{equation}
	et le théorème de Cayley-Hamilton donne donc $A^{k}=R_{k}(A)$ d'où $BA^{k}Y=BR_{k}(A)Y=0$. Alors pour tout $t\in\R$,
	\begin{align}
		B\exp(tA)Y
		&=B\sum_{k=0}^{+\infty}\frac{t^{k}A^{k}}{k!}Y\\
		&=\sum_{k=0}^{+\infty}\frac{t^{k}(BA^{k}Y)}{k!}\\
		&=0
	\end{align}
	Par contraposée, on a bien ce qu'il faut, d'où l'équivalence.
\end{proof}
\section{Espaces vectoriels normés}

\begin{proof}
	\phantom{}
	\begin{enumerate}
		\item A $(x,y)\in\R^{2}$ fixé, la fonction \function{\varphi}{\R}{\R}{t}{x\cos(t)+y\sin(2t)}
		est bornée, donc le $\sup$ sur $\R$ existe. Pour la séparation, prendre $t=0$ et $t=\frac{\pi}{4}$. Pour l'inégalité triangulaire, montrer l'inégalité à $t$ fixé puis passer au $\sup$ sur $\R$.
		
		\item Si $\vert x\vert+\vert y\vert\leqslant1$, alors $N(x,y)\leqslant 1$ donc on a la première inclusion. 
		
		Si $N(x,y)\leqslant 1$, utiliser $t=0$ pour avoir $\vert x\vert\leqslant1$ et $t=\frac{\pi}{4}$ puis $t=-\frac{\pi}{4}$ pour pouvoir justifier
		$$\vert 2y\vert\leqslant \Biggl\vert x\frac{\sqrt{2}}{2}+y\Biggr\vert+\Biggl\vert y-x\frac{\sqrt{2}}{2}\Biggr\vert\leqslant 2$$
		et donc $\vert y\vert\leqslant1$. D'où la deuxième inclusion. 

		\item On fixe $(x,y)\in S_{N}(0,1)\cap(\R_{+})^{2}$. $\varphi$ est $2\pi$-périodique, $\varphi(\pi-t)=\varphi(t)$ et $\sup\limits_{t\in\R}\vert\varphi(t)\vert=1$. On peut donc se limite à un intervalle de longueur $2\pi$ pour l'étude de $\varphi$. 
		
		On note que si $t\in[-\pi,0]$, $\cos(t)$ et $\sin(2t)$ sont de signes opposés. Donc
		$$\vert\varphi(t)\vert\leqslant x\vert\cos(t)\vert+y\vert\sin(2t)\vert=\vert\varphi(-t)\vert$$
		et $-t\in[0,\pi]$. Donc le $\sup$ est atteint sur $[0,\pi]$.

		On note maintenant, comme $\vert\varphi(\pi-t)\vert=\vert\varphi(t)\vert$ sur $[0,\frac{\pi}{2}]$, que si $t\in[\frac{\pi}{4},\frac{\pi}{2}]$,
		$$0\leqslant\varphi(t)=x\underbrace{\cos(t)}_{\in[0,\frac{\sqrt{2}}{2}]}+y\sin(2t)\leqslant x\underbrace{\cos(\frac{\pi}{2}-t)}_{\in[\frac{\sqrt{2}}{2},1]}+y\sin(2\times (\frac{\pi}{2}-t))=\varphi(\frac{\pi}{2}-t)$$

		Donc le $\sup$ est atteint sur $[0,\frac{\pi}{4}]$. Soit maintenant $t_{0}\in[0,\frac{\pi}{4}]$ tel que $\varphi(t_{0})$ réalise le $\sup$ (existe car $\varphi$ est continue sur un compact). Comme c'est aussi le $\sup$ sur $\R$ qui est ouvert, on a la condition d'Euler du premier ordre: $\varphi'(t_{0})=0$.

		On a donc $x\cos(t_{0})+y\sin(2t_{0})=1$ et $-x\sin(t_{0})+2y\cos(2t_{0})=0$. On en déduit les valeurs de $x$ et $y$ en fonction de $t_{0}$, en faisant attention que $\cos(t_{0})\neq0$ sinon $\sin(t_{0})=0$ aussi ce qui n'est pas le cas, et au cas où $t_{0}=0$.

		Réciproquement, s'il existe $t_{0}\in[0,\frac{\pi}{4}]$ tel que $x$ et $y$ s'écrivent de la façon demandée, alors $t_{0}$ est l'unique point satisfaisant $\varphi(t_{0})=1$ et $\varphi'(t_{0})=0$. Mais alors le $\sup$ de $\varphi$ sur $[0,\frac{\pi}{4}]$ est atteint en un point $t_{1}$ qui vérifie les mêmes choses, donc $t_{1}=t_{0}$ d'où $N(x,y)=1$.
	\end{enumerate}
\end{proof}

\begin{proof}
	\phantom{}
	\begin{enumerate}
		\item Pour l'inégalité triangulaire, introduire la forme bilinéaire symétrique positive sur $E$ \function{\varphi}{E\times E}{\R}{(f,g)}{f(0)g(0)+\int_{0}^{1}f'(t)g'(t)dt}
		Alors $N(f)=\sqrt{\varphi(f,f)}$ et on utilise l'inégalité de Minkowski.
		\item Pour $x\in[0,1]$, écrire $\vert f(x)\vert=\vert f(0)+f(x)-f(0)\vert$, $f(x)-f(0)=\int_{0}^{x}f'(t)dt$, utiliser Cauchy-Schwarz avec $f'$ et $1$ puis que $\sqrt{a}+\sqrt{b}\leqslant\sqrt{2}\sqrt{a+b}$, pour enfin passer au $\sup$ sur $x$.
		\item Utiliser, pour $n\in\N^{*}$, la fonction \function{f_n}{[0,1]}{\R}{t}{t^n}
	\end{enumerate}
\end{proof}

\begin{proof}
	Si $f$ est ouverte, $f(\R^{n})$ est un sous-espace vectoriel ouvert de $R^{p}$. Donc $f$ est surjective.

	Si $f$ est surjective, on prend $F$ un supplémentaire de $\ker(f)$ dans $\R^{n}$ avec $\dim(\ker(f))=n-p$ et $\dim(F)=p$. Soit $(e_{1},\dots,e_{p})$ une base de $F$ et $(e_{p+1},\dots,e_{n})$ une base de $\ker(f)$. On vérifie que $(f(e_{1},\dots,f(e_{p}))$ est une base de $\R^{p}$. On définit \function{N_1}{\R^n}{\R}{\sum_{i}^{n}x_{i}e_{i}}{\max\limits_{1\leqslant i\leqslant n}\vert x_{i}\vert}
	norme sur $\R^{n}$ et \function{N_2}{\R^p}{\R}{\sum_{i}^{p}y_{i}f(e_{i})}{\max\limits_{1\leqslant i\leqslant p}\vert y_{i}\vert}
	norme sur $\R^{p}$.

	Soit $\Theta$ un ouvert de $\R^{n}$, soit $y_{0}\in f(\Theta)$, il existe $x_{0}\in\Theta\colon y_{0}=f(x_{0})$. Si $x_{0}=\sum_{i=1}^{n}\alpha_{i}e_{i}$, alors $y_{0}=\sum_{i=1}^{p}\alpha_{i}f(e_{i})$. Comme $\Theta$ est un ouvert, il existe $r_{0}>0$ tel que 
	$$B_{N_{1}}(x_{0},r_{0})\subset\Theta$$
	Soit $y=\sum_{i=}^{p}\beta_{i}f(e_{i})\in\R^{p}$, si $N_{2}(y-y_{0})<r_{0}$, pour tout $i\in\{1,\dots,p\}$, $\vert\beta_{i}-\alpha_{i}\vert<r_{0}$ et 
	$$y=f\Biggl(\sum_{i=1}^{p}\beta_{i}e_{i}+\sum_{i=p+1}^{n}\alpha_{i}e_{i}\Biggr)\overset{\text{def}}{=}f(x)$$
	avec $N_{1}(x-x_{0})=\max\limits_{1\leqslant i\leqslant p}\vert\beta_{i}-\alpha_{i}\vert<r_{0}$. Ainsi $x\in\Theta$ et $y\in f(\Theta)$, donc $B_{N_{2}}(y_{0},r_{0})\subset f(\Theta)$ et $f(\Theta)$ est un ouvert.
\end{proof}

\begin{proof}
	\phantom{}
	\begin{enumerate}
		\item Classique.
		\item $$\vert f(x)\vert\leqslant\vert f(0)\vert+\vert f(x)-f(0)\vert\leqslant\vert f(0)\vert+\kappa(f)x\leqslant N(f)$$
		car $x\leqslant 1$, donc $N_{\infty}\leqslant N$. Pour la non-équivalence, prendre \function{f_n}{[0,1]}{\R}{t}{t^n}
		\item On a $\vert f(0)\vert\leqslant N_{\infty}(f)$ donc $N(f)\leqslant N'(f)$. Ensuite, $N_{\infty}\leqslant N$ donne $N'\leqslant N+\kappa\leqslant 2N$. Donc $N$ est $N'$ sont équivalentes.
	\end{enumerate}
\end{proof}

\begin{remark}
	Exemple de normes qui, en dimension infinie, ne se dominent pas mutuellement. On prend $(e_{i})_{i\in I}$ une base (de Hamel), $J=(i_{n})_{n\in\N}\subset I$ dénombrable. Si $x=\sum_{i\in I}x_{i}e_{i}$, on peut vérifier que 
	$$N_{1}(x)=\sum_{n\in\N}\vert x_{i_{n}}\vert+\sum_{i\in I\setminus J}\vert x_{i}\vert$$
	et
	$$N_{2}(x)=\sum_{n\in\N}n\vert x_{i_{2n}}\vert+\sum_{n\in\N}\frac{1}{n+1}\bigl\lvert x_{i_{2n+1}}\bigr\rvert+\sum_{i\in I\setminus J}\vert x_{i}\vert$$
	ne se dominent pas.
\end{remark}

\begin{proof}
	Il existe $\alpha>0$ tel que $B_{\Vert\cdot\Vert_{\infty}}(I_{n},\alpha)\subset G$. Soient $i\neq j$ et $\lambda\in\C$. Il existe $p\in\N^{*}$ tel que $\frac{\vert\lambda\vert}{p}<\alpha$. Alors 
	$$\Biggl\lVert T_{i,j}\Biggl(\frac{\lambda}{p}\Biggr)-I_{n}\Biggr\rVert_{\infty}=\Biggl\lvert\frac{\lambda}{p}\Biggr\rvert<\alpha$$
	donc $T_{i,j}(\lambda)\in G$ ($T_{i,j}$ est la matrice de transvection: $T_{i,j}(\lambda)=I_{n}+\lambda E_{i,j}$).

	Ainsi,
	$$T_{i,j}(\lambda)=\Biggl(T_{i,j}\Biggl(\frac{\lambda}{p}\Biggr)\Biggr)^{p}\in G$$

	Soit $\delta=\rho e^{\mathrm{i}\theta}\in\C^{*}$. On a $\lim\limits_{n\to+\infty}\rho^{\frac{1}{p}}e^{\mathrm{i}\frac{\theta}{p}}=1$ donc il existe $p\in\N^{*}$ tel que $\vert\rho^{\frac{1}{p}}e^{\mathrm{i}\frac{\theta}{p}}-1\vert<\alpha$.
	
	On a alors
	$$\Biggl\lVert D_{n}\Bigl(\rho^{\frac{1}{p}}e^{\mathrm{i}\frac{\theta}{p}}\Bigr)-I_{n}\Biggr\rVert_{\infty}<\alpha$$
	donc $D_{n}(\delta)=D_{n}(\rho^{\frac{1}{p}}e^{\mathrm{i}\frac{\theta}{p}})^{p}\in G$ (matrice de dilatation).

	Comme les matrices de transvection et de dilatation engendrent $GL_{n}(\C)$, on a bien $G=GL_{n}(\C)$.
\end{proof}

\begin{remark}
	C'est faux sur $\R$. Contre-exemple: matrices de déterminant positif.
\end{remark}

\begin{proof}
	Si $f$ n'est pas continue en 0, il existe $\varepsilon_{0}>0$ tel que pour tout $\alpha>0$, il existe $h\in E$ avec $\Vert h\Vert\leqslant\alpha$ et $\Vert f(h)\Vert>\varepsilon_{0}$. On prends $\alpha_{n}=\frac{1}{n+1}$, d'où $\Vert nh_{n}\Vert\leqslant1$ mais $\underbrace{\Vert f(nh_{n})\Vert}_{\leqslant M}>n\varepsilon_{0}\xrightarrow[n\to+\infty]{}+\infty$. Donc $f$ est continue en $0$. Comme $f$ est linéaire, pour tout $x\in E$,
	$$\lim\limits_{\Vert h\Vert\to0}f(x+h)=\lim\limits_{\Vert h\Vert\to0}f(x)+f(h)=f(x)$$
	donc $f$ est continue.

	On a $f(px)=p(fx)$ pour tout $p\in\Z$ puis $qf(\frac{p}{q}x)=f(px)=pf(x)$ pour tout $(p,q)\in\Z\times\N^{*}$ donc pour tout $r\in\Q$, $f(rx)=rf(x)$.
	Soit $\lambda\in\E$, il existe une suite de rationnels telle que $\lim\limits_{n\to+\infty} r_{n}=\lambda$. Comme $f$ est continue, on a 
	\begin{align*}
		f(\lambda x)
		&=\lim\limits_{n\to+\infty}f(r_{n}x)\\
		&=\lim\limits_{n\to+\infty}r_{n}f(x)\\
		&=\lambda f(x)
	\end{align*}
	Donc $f$ est linéaire.
\end{proof}

\begin{remark}
	Soit $e_{0}=1$ et $e_{1}=\sqrt{2}$ et $(e_{i})_{i\in I}$ une $\Q$-base de $\R$ ($0\in I$). On définie 
	$$f\Bigl(\sum_{i\in I}\lambda_{i} e_{i}\Bigr)=\lambda_{0}e_{0}+\sqrt{2}\sum_{i\in I\setminus\{0\}}\lambda_{i}e_{i}$$
	$f$ vérifie $f(x+y)=f(x)+f(y)$, mais si $(r_{n})_{n\in\N}$ est une suite de rationnels tendant vers $\sqrt{2}$, $f(r_{n})=r_{n}\to\sqrt{2}\neq f(\sqrt{2})=2$.
\end{remark}

\begin{proof}
	\phantom{}
	\begin{enumerate}
		\item On a $\alpha(A)\subset \overline{A}$ donc $\overline{\mathring{\overline{A}}}\subset\overline{A}$ donc $\alpha(\alpha(A))\subset\alpha(A)$. Comme $\alpha(A)$ est un ouvert inclus dans $\overline{\mathring{\overline{A}}}\subset\overline{A}$ donc $\alpha(A)\subset\alpha(\alpha(A))$.

		\item Si $\beta(A)=\overline{\mathring{A}}$, on montre aussi que $\beta(\beta(A))=\beta(A)$. On a donc $A,\overline{A},\mathring{A},\overline{\mathring{A}},\mathring{\overline{A}},\overline{\mathring{\overline{A}}}$ et $\mathring{\overline{\mathring{A}}}$ et c'est tout.
	\end{enumerate}
\end{proof}

\begin{proof}
	\phantom{}
	\begin{enumerate}
		\item Si $d_{A}=d_{B}$, 
		$$\overline{A}=\{x\in E\bigm| d_{A}(x)=0\}=\{x\in E\bigm| d_{B}(x)=0\}=\overline{B}$$
		Réciproquement, soit $x\in E$ et $\varepsilon>0$, il existe $a_{1}\in\overline{A}$, $\Vert x-a_{i}\Vert\leqslant d_{\overline{A}}(x)+\frac{\varepsilon}{2}$ (par définition de l'inf). Il existe $a_{2}\in A$, $\Vert a_{1}-a_{2}\Vert\leqslant\frac{\varepsilon}{2}$ (par définition de la fermeture). Ainsi,
		$$d_{A}(x)\leqslant\Vert x-a_{2}\Vert\leqslant\Vert x-a_{1}\Vert+\Vert a_{1}-a_{2}\Vert\leqslant d_{\overline{A}}(x)+\varepsilon$$
		Ceci valant pour tout $\varepsilon>0$, $d_{A}(x)\leqslant d_{\overline{A}}(x)$. Comme $A\subset\overline{A}$, $d_{\overline{A}}\leqslant d_{A}$, on a $d_{A}=d_{\overline{A}}=d_{\overline{B}}=d_{B}$.

		\item Soit $x\in A$, on a $d_{B}(x)=\vert d_{B}(x)-d_{A}(x)\vert\leqslant\rho(A,B)$ donc $\sup\limits_{x\in A}d_{B}(x)\leqslant\rho(A,B)$, de même pour $\sup\limits_{y\in B}d_{A}(y)$ donc on on a un première inégalité.
		
		Réciproquement, soit $x\in E$ et $\varepsilon>0$, il existe $a\in A$ et $b\in B$ tel que $\Vert x-a\Vert\leqslant d_{A}(x)+\varepsilon$ et $\Vert x-b\Vert\leqslant d_{B}(x)+\varepsilon$.
		On a alors
		$$d_{A}(x)\leqslant\Vert x-a\Vert\leqslant\Vert a-b\Vert+\Vert x-b\Vert\leqslant d_{B}(x)+\varepsilon+\alpha(A,B)$$
		Ceci vaut pour tout $\varepsilon>0$, donc $d_{A}(x)\leqslant d_{B}(x)+\alpha(A,B)$. De même, $d_{B}(x)\leqslant d_{A}(x)+\alpha(A,B)$ donc $\rho(A,B)\leqslant\alpha(A,B)$.
	\end{enumerate}
\end{proof}

\begin{proof}
	\phantom{}
	\begin{enumerate}
		\item Soit $(y_{n})_{n\in\N}\in P(F)^{\N}$ qui converge vers $y\in\C$ donc il existe $(x_{n})\in F^{\N}$ telle que l'on ait pour tout $n\in\N$, $P(x_{n})=y_{n}$. $(x_{n})_{n\in\N}$ est bornée car $\lim\limits_{z\to+\infty}\vert P(z)\vert=+\infty$ (car $P$ est non constant), donc on peut extraire (Bolzano-Weierstrass) $x_{\sigma(n)}\to x$ et $x\in F$ car $F$ est fermé. Par continuité de $z\mapsto P(z)$ sur $\C$, on a $y=P(x)\in P(F)$.
		
		\item Soit $\Theta$ un ouvert de $\C$, soit $y\in P(\Theta),\exists x\in\Theta$ tel que $P(x)=y$ et il existe $r>0$, $B(x,r)\subset\Theta$. Soit $y'\in\C$, supposons que pour tout $x'\in\C$ tel que $P(x')=y'$, on a $\vert x-x'\vert>r$. Soit $Q(X)=P(X)-y'=a\prod_{i=1}^{n}(X-x_{i})$ non constant où $a$ est le coefficient dominatrice de $P$. Par hypothèse, pour tout $i\in\{1,\dots,n\}\colon\vert x_{i}-x\vert>r$ (car $P(x_{i})=y'$), ainsi 
		$$\vert Q(x)\vert=\vert y-y'\vert\geqslant\vert a\vert r^{n}$$
		Par contraposée, si $\vert y-y'\vert\leqslant\frac{\vert a\vert r^{n}}{2}$, alors il existe $x'\in\C$ tel que $P(x')=y'$ et $\vert x'-x\vert<r$.Ainsi, $x'\in B(x,r)\subset\Theta$ et $y'\in P(\Theta)$. Donc $B(y,\vert a\vert r^{n})\subset P(\Theta)$ et $P(\Theta)$ est un ouvert.
	\end{enumerate}
\end{proof}

\begin{proof}
	\phantom{}
	\begin{enumerate}
		\item Si $P\notin\mathcal{S}$, il existe $z_{0}\in\C\setminus\R$ tel que $P(z_{0})=0$ et $\vert\Im(z_{0})\vert^{n}>0=P(z_{0})$. Par contraposée, si pour tout $z\in\C$, $\vert P(z)\vert\geqslant\vert\Im(z_{})\vert^{n}$,alors $P\in\mathcal{S}$.

		Réciproquement, si $P=\prod_{i=1}^{n}(X-\lambda_{i})\in\mathcal{S}$ avec $(\lambda_{i})_{1\leqslant i\leqslant n}$ réels, soit $z=a+ib\in\C$. On a
		$$\vert P(z)\vert=\prod_{i=1}^{n}\vert a-\lambda_{i}+ib\vert\geqslant\vert b\vert^{n}$$
		
		\item Soit $(P_{p})_{p\in\N}\in\mathcal{S}^{\N}$ telle que $P_{p}\xrightarrow[p\to+\infty]{}P\in F$. Soit $z\in\C$, on a pour tout $p\in\N$, $\vert P_{p}(z)\vert\geqslant\vert\Im(z)\vert^{n}$ donc quand $p\to+\infty$, $\vert P(z)\vert\geqslant\vert\Im(z)\vert^{n}$ donc $P\in\mathcal{S}$ et $S$ est fermé.
		
		\item Soit $(M_{p})_{p\in\N}$ une suite de matrice trigonalisable sur $\R$ qui converge vers $M\in\M_{n}(\R)$. Ib bite $\chi_{p}$ le polynôme caractéristique de $M_{p}$. Pour tout $p\in\N$, $\chi_{p}\in\mathcal{S}$ et $\chi_{p}\xrightarrow[p\to+\infty]{}\chi_{M}$. Comme $\mathcal{S}$ est fermé, $\chi_{M}\in \mathcal{S}$ et $M$ est trigonalisable sur $\R$.
	\end{enumerate}
\end{proof}

\begin{proof}
	\phantom{}
	\begin{enumerate}
		\item $\varphi$ est linéaire et $\dim(\K_{m-1}[X]\times\K_{n-1}[X])=m+n+=\dim(\K_{n+m-1}[X])$.
		
		Si $\varphi$ est bijective, elle est surjective et il existe $(U,V)\in\K[X]^{2}$ tel que $UA+BV=1$ et d'après le théorème de Bézout, on a $A\wedge B=1$.

		Réciproquement, si $\varphi$ n'est pas surjective, il existe $(U,V)\in(\K_{m-1}[X]\times\K_{n-1}[X])\setminus\{(0,0)\}$ tel que $\varphi(U,V)=0$ d'où $AU=-BV$. Soit $\delta=A\wedge B$, on écrit $A=\delta A_{1}$ et $B=\delta B_{1}$ avec $A_{1}\wedge B_{1}=1$ et on a $A_{1}U=-B_{1}V$. D'après le théorème de Gauss, on a $A_{1}\mid V$ et $B_{1}\mid U$. Si $U=0$, on a $V=0$ et de même si $V=0$, on a $U=0$. On peut donc supposer $U\neq0$ et $V\neq 0$, et on a alors $\deg(A_{1})\leqslant\deg(V)\leqslant n-1<n=\deg(A)$ mais $A=\delta A_{1}$ donc $\deg(\delta)\geqslant1$ et $A\wedge B\neq 1$.

		\item $\Phi$ est continue car $R_{A,B}$ est un polynôme en les coefficients de $A$ et $B$.
		
		\item Comme on est dans $\C$, $\Delta=\{P\in\C_{p}[X]\bigm| P\wedge P'=1\}=\{P\in\C_p[X]\bigm| R_{P,P'}\neq0\}$. $\Phi_{P,P'}$ est continue d'après la question précédente, $\delta=\Phi_{P,P'}^{-1}(\C^{*})$ donc $\Delta$ est ouvert.
		
		Sur $\R$, on n'a pas la caractérisation de scindé à racines simples si et seulement si $P\wedge P'=1$ (contre-exemple: $P=X^{2}+1$). Dans $\R_{3}[X]$, $X$ est scindé à racines simples et $X(1+\varepsilon X)^{2}\xrightarrow[\varepsilon\to0]{}X$ et $-\frac{1}{\varepsilon}$ est racine double, donc $\Delta$ n'est pas ouvert.
	\end{enumerate}
\end{proof}

\begin{remark}
	On peut cependant considérer 
	$$\Delta_{n}=\{P\in\C_{p}[X]\bigm| P\text{ scindé à racines simples sur }\R\text{ et }\deg(P)=n\}$$
	Si $\lambda_{1}<\lambda_{2}<\dots<\lambda_{n}$ sont les racines (distinctes) de $R$ sur $\R$, on choisit $\alpha_{0}\in]-\infty,\lambda_{1}$, $\alpha_{n}\in]\lambda_{n},+\infty[$ et $\alpha_{i}\in]\lambda_{i},\lambda_{i+1}[$ si $i=1,\dots,n-1$. 

	Pour tout $k\in\{0,\dots,n-1\}$, on a $P(\alpha_{k})P(\alpha_{k+1})<0$ (car les racines de $P$ provoquent des changements de signe). Soit \function{\Psi}{\R_n[X]}{\R^n}{Q}{(Q(\alpha_{k})Q(\alpha_{k+1}))_{0\leqslant k\leqslant n-1}}
	$\Psi$ est continue sur $\R_{n}[X]$ et $\Psi(P)\in(\R_{-}^{*})^{n}$ qui est ouvert, donc il existe $r>0$ tel que si $\Vert P-Q\Vert<r$, alors $\Psi(Q)\in(\R_{-}^{*})^{n}$. Donc $Q$ change $n$ fois de signe, et admet au moins $n$ racines. Mais $\deg(Q)=n$, donc $Q$ est scindé à racines simples sur $\R$, donc $\Delta_{n}$ est ouvert dans $\{P\in\R[X]\bigm|\deg(P)=n\}$.
\end{remark}

\begin{remark}
	$$\{M\in\M_{n}(\C)\bigm|M\text{ diagonalisable à racines simples}\}=\{M\in\M_{n}(\C)\bigm|\chi_M\text{ sciné à racines simples}\}$$
	est un ouvert de $\M_{n}(\C)$ car $M\mapsto\chi_{M}$ est continue sur $\M_{n}(\C)$, et c'est aussi vrai sur $\R$.
\end{remark}

\begin{proof}
	\phantom{}
	\begin{enumerate}
		\item Soit \function{f}{\M_n(\R)}{\M_n(\R)}{A}{A^{n}}
		$f$ est continue et $F=f^{-1}(\{0\})$ donc $F=\overline{F}$.

		Soit $M_{0}\in F$, $X^{n}$ annule $M_{0}$ donc $M_{0}$ est trigonalisable: on écrit $M_{0}$ dans une base où les coefficients diagonaux sont tous nuls. Soit alors $M_{\varepsilon}$ la même matrice dans la même base en rajoutant simplement $\varepsilon$ en première position de la diagonale. Alors $M_{\varepsilon}\xrightarrow[\varepsilon\to0]{}M_{0}$ et $M_{\varepsilon}\notin F$ donc $\mathring{F}=\emptyset$. Notons que cela signifie que $F$ est dense.

		\item La norme dérive du produit scalaire $(A|B)\mapsto\Tr(A^{\mathsf{T}}B)$. Soit $M\in F$, on a $\Vert M-I_{n}\Vert^{2}=\Vert M\Vert^{2}+\Vert I_{n}\Vert^{2}-2(M|I_{n})$. On a $(M|I_{n})=\Tr(M)=0$ car $M$ est nilpotente. Donc $\Vert M-I_{n}\Vert^{2}$ est minimale pour $\Vert M\Vert^{2}$ minimale, donc pour $M=0\in F$. Donc $d(I_{n},F)=\Vert I_{n}\Vert=\sqrt{n}$ (et la distance est atteinte pour $0_{\M_n(\R)}$).
	\end{enumerate}
\end{proof}

\begin{proof}
	\phantom{}
	\begin{enumerate}
		\item $A\mapsto\det(A)$ est continue et $GL_{n}(\K)=\det^{-1}(\K^{*})$ est donc ouvert. Si $A\in\M_{n}(\K)$, pour $p\in\N$, on pose $A_{p}=A-\frac{1}{p+1}I_{n}$. Comme $\Sp(A)$ est fini, il existe $N\in\N$, tel que pour tout $p\geqslant N$, $\frac{1}{p+1}\notin\Sp(A)$. Donc pour tout $p\geqslant N$, $A_{p}\in GL_{n}(\K)$, et $A_{p}\xrightarrow[p\to+\infty]{}A$ donc $GL_{n}(\K)$ est dense dans $\M_{n}(\K)$.
		\item On fixe $B\in\M_{n}(\K)$. Soit $A\in GL_{n}(\K)$. On écrit $BA=A^{-1}(AB)A$ donc $AB$ et $BA$ sont semblables donc $\chi_{AB}=\chi_{BA}$. Comme, à $B$ fixé, $A\mapsto\chi_{AB}$ et $A\mapsto\chi_{BA}$ sont continues sur $\M_{n}(\K)$, on a le résultat par densité.
	\end{enumerate}
\end{proof}

\begin{proof}
	\phantom{}
	\begin{enumerate}
		\item On a $v_{p}\circ(id_{E}-u)=(id_{E}-u)\circ v_{p}=\frac{1}{p}(id_{E}-u^{p})$, donc $\Vert v_{p}\circ(id_{E}-u)\Vert\leqslant\frac{1}{p}(\Vert id_{E}\Vert+\Vert u^{p}\Vert)\xrightarrow[p\to+\infty]{}0$.
		
		Soit $x\in\ker(u-id_{E})\cap\im(u-id_{E})$, on a $u(x)=x$ et il existe $y\in E$, $x=(u-id_{E})(y)$. On a $v_{p}(x)=\frac{1}{p}(px)=x$ et $v_{p}(x)=v_{p}\circ(u-id_{E})(y)\xrightarrow[p\to+\infty]{}0$ d'où $x=0$. Le théorème du rang permet de conclure.

		\item Soit $x\in E$, on écrit $x=x_{1}+x_{2}$ avec $\Pi(x)=x_{1}$ et $x_{2}=(u-id_{E})(y_{2})$. Alors $v_{p}(x)=x_{1}+v_{p}\circ(u-id_{E})(y_{2})\xrightarrow[p\to+\infty]{}x_{1}=\Pi(x)$.
	\end{enumerate}
\end{proof}

\begin{proof}
	\phantom{}
	\begin{enumerate}
		\item Pour tout $x\in A$, $f_{n}(x)\in A$ car $A$ est convexe. Soit $(x,y)\in A^{2}$, on a
		$$\Vert f_{n}(x)-f_{n}(y)\Vert=\Bigl(1-\frac{1}{n}\Bigr)\Vert f(x)-f(y)\Vert\leqslant\Bigl(1-\frac{1}{n}\Bigr)\Vert x-y\Vert$$
		Donc $f_{n}$ est $(1-\frac{1}{n})$-lipschitzienne. On forme \function{g_n}{A}{\R}{x}{\Vert f_n(x)-x\Vert}
		qui est continue. Soit $x_{n}\in A$ telle que $g_{n}(x_{n})=\min\limits_{x\in A}g_{n}(x)$ (existe car $A$ est compact et $g_{n}$ continue). On a $x_{n}\in A$, d'où $f_{n}(x_{n})\in A$ et 
		$$g_{n}(f_{n}(x_{n}))=\Vert f_{n}(f_{n}(x_{n}))-f_{n}(x_{n})\Vert\leqslant\Bigl(1-\frac{1}{n}\Bigr)\Vert f_{n}(x_{n})-x_{n}\Vert=\Bigl(1-\frac{1}{n}\Bigr)g_{n}(x_{n})$$
		Si $g_{n}(x_{n})\neq0$, alors on aurait $g_{n}(f(x_{n}))<g_{n}(x_{n})$ ce qui n'est pas possible. Donc $g_{n}(x_{n})=0$ et $f_{n}(x_{n})=x_{n}$.

		Soit $y_{n}$ un autre point fixe, on a 
		$$\Vert f_{n}(x_{n})-f_{n}(y_{n})\Vert=\Vert x_{n}-y_{n}\Vert\leqslant\Bigl(1-\frac{1}{n}\Bigr)\Vert x_{n}-y_{n}\Vert$$
		donc $x_{n}=y_{n}$.

		\item On a $(x_{n})_{n\in\N}\in A^{\N}$ et on extrait (car $A$ est compact) et on a 
		$$x_{\sigma(n)}\xrightarrow[n\to+\infty]{}x\in A$$
		On a 
		$$f_{\sigma(n)}(x_{\sigma(n)})=x_{\sigma(n)}=\underbrace{\frac{1}{\sigma(n)}f(x_{0})}_{\xrightarrow[n\to+\infty]{}0}+\underbrace{\Bigl(1-\frac{1}{\sigma(n)}\Bigr)f(x_{\sigma(n)})}_{\xrightarrow[n\to+\infty]{}f(x)}$$
		par continuité de $f$. Donc $f(x)=x$.

		\item Soit $(x,y)\in A^{2}$, points fixes de $f$, et $t\in[0,1]$, on pose $z=tx+(1-t)y$. On a 
		\begin{align*}
			\Vert x-y\Vert
			&=\Vert f(x)-f(y)\Vert\\
			&\leqslant \Vert f(x)-f(z)\Vert+\Vert f(z)-f(y)\Vert\\
			&\leqslant\Vert x-z\Vert+\Vert z-y\Vert\\
			&=(1-t)\Vert x-y\Vert+t\Vert x-y\Vert\\
			&=\Vert x-y\Vert
		\end{align*}
		On a donc égalité partout: $\Vert f(x)-f(y)\Vert=\Vert f(x)-f(z)\Vert+\Vert f(z)-f(y)\Vert$ et $\Vert f(x)-f(z)\Vert=\Vert x-z\Vert$, $\Vert f(z)-f(y)\Vert=\Vert z-y\Vert$ car $f$ est $1$-lipschitzienne.

		Comme la norme est euclidienne, il existe $\lambda\in\R_{+}$ tel que $f(x)-f(z)=\lambda(f(z)-f(y))$ d'où $f(x)+\lambda f(y)=(\lambda+1)f(z)$ d'où $f(z)=\frac{x+\lambda y}{\lambda+1}=t'x+(1-t')y$ avec $t'=\frac{1}{\lambda+1}\in[0,1]$. En reportant, on a 
		$$\Vert f(x)-f(z))\Vert=\Vert x-t'x-(1-t')y\Vert=(1-t')\Vert x-y\Vert=\Vert x-z\Vert=(1-t)\Vert x-y\Vert$$
		Si $x\neq y$, alors $t=t'$ et $f(z)=tx+(1-t)y=z$.

		\item Soit dans $\R^{2}$, $\overline{B_{\Vert\cdot\Vert}(0,1)}=[-1,1]^{2}=A$. Soit \function{f}{A}{A}{(x,y)}{(x,\vert x\vert)}
		On a 
		\begin{align*}
			\Vert f(x_{1},y_{1})-f(x_{2},y_{2})\Vert_{\infty}
			&= \Vert (x_{1},\vert x_{1}\vert)(x_{2},\vert x_{2}\vert)\Vert_{\infty}\\
			&=\max\{\vert x_{1}-x_{2}\vert, \bigl\vert\vert x_{1}\vert-\vert x_{2}\vert\bigr\vert\}\\
			&=\vert x_{1}-x_{2}\vert\\
			&\leqslant\Vert (x_{1},y_{1})-(x_{2},y_{2})\Vert_{\infty}
		\end{align*}
		Donc $f$ est 1-lipschitzienne, on a $f(x,y)=(y,x)$ si et seulement si $y=\vert x\vert$. Donc ici, $F$ n'est pas convexe.
	\end{enumerate}
\end{proof}

\begin{proof}
	\phantom{}
	\begin{enumerate}
		\item On a pour tout $(x,y)\in E^{2}$, $f(x+y)=f(x)+f(y)$ et par récurrence, pour tout $n\in\Z$, $f(nx)=nf(x)$. Pour $r=\frac{p}{q}\in\Q$, on a $f(qrx)=qf(rx)=f(px)=pf(x)$ donc $f(rx)=rf(x)$. Par densité de $\Q$ dans $\R$ et continuité de $f$, on a pour tout $\lambda\in\R$, $f(\lambda x)=\lambda f(x)$. Donc $f$ est linéaire.
		
		Pour $\K=\C$, cela ne marche pas. Contre-exemple: la conjugaison dans $\C$.

		\item On étudie la série, pour $x$ fixé de terme général 
		$$\Vert v_{n+1}(x)-v_{n}(x)\Vert=\frac{1}{2^{n}}\Vert f(2^{n+1}x)-2f(2^{n}x)\Vert\leqslant\frac{M}{2^{n+1}}$$
		qui est donc convergente. Donc $(v_{n})_{n\in\N}$ converge.

		\item On a $v_{0}(x)=f(x)$, donc $\sum_{n=0}^{+\infty}v_{n+1}(x)-v_{n}(x)=g(x)-f(x)$. $f$ étant continue, $v_{n}$ l'est aussi, et pour tout $n\in\N$, comme pour tout $x\in E$, $\Vert (v_{n+1}-v_{n})(x)\Vert\leqslant\frac{M}{2^{n+1}}$, donc $g$ est continue.
		
		\item On a, pour tout $(x,y)\in E^{2}$,
		$$\Vert v_{n}(x+y)-v_{n}(x)-v_{n}(y)\Vert=\Vert \frac{1}{2^{n}}f(2^{n}(x+y))-\frac{1}{2^{n}}(f(2^{n}x)+f(2^{n}y))\Vert\leqslant\frac{M}{2^{n}}$$
		Donc quand $n\to+\infty$, $g(x+y)=g(x)+g(y)$.

		On a pour tout $x\in E$, 
		$$\Vert g(x)-f(x)\Vert=\Bigl\lVert\sum_{n=0}^{+\infty}v_{n+1}(x)-v_{n}(x)\Bigr\Vert\rVert\leqslant\sum_{n=0}^{+\infty}\Vert v_{n+1}(x)-v_{n}(x)\Vert\leqslant\sum_{n=0}^{\infty}\frac{M}{2^{n}}=M$$

		Soit maintenant $h$ linéaire continue telle que $h-f$ soit bornée, soit $M'=\sup\limits_{x\in E}\Vert h(x)-f(x)\Vert$. On a donc 
		$$\Vert v_{n}(x)-h(x)\Vert=\Bigl\Vert\frac{1}{2^{n}}f(2^{n}x)-\frac{1}{2^{n}}h(2^{n}x)\Bigr\Vert\leqslant\frac{M'}{2^{n}}$$
		car $h$ est linéaire. Donc quand $n\to+\infty$, $g(x)=h(x)$ car $\lim\limits_{n\to+\infty}v_{n}(x)=g(x)$.
	\end{enumerate}
\end{proof}

\begin{proof}
	En particulier, pour $t=f(0)$, $f^{-1}(\{f(0)\})=\{x\in E\bigm| f(x)=f(0)\}$ est borné (car compact). Donc il existe $A$ tel que $f^{-1}(\{f(0)\})\subset\overline{B(0,A)}$. Par contraposée, pour tout $x\in E$, si $\Vert x\Vert>A$, alors $f(x)\neq f(0)$.

	On montre alors que $E\setminus\overline{B(0,A)}$ est connexe par arcs (faire le tour de la boule par l'extérieur).

	$f$ étant continue, d'après le théorème des valeurs intermédiaires, on a soit pour tout $x\in E\setminus\overline{B(0,A)}$, $f(x)>f(0)$ soit $f(x)<f(0)$. Quitte à remplacer $f$ par $-f$, on se place dans le cas $f(x)>f(0)$. Comme on est en dimension finie sur $\overline{B(0,A)}$ compact, $f$ atteint son minimum et ce minimum est plus petit que $f(0)$, c'est donc un minimum global.
\end{proof}

\begin{remark}
	C'est faux pour $n=1$. Contre-exemple: $f=id_{\R}$.
\end{remark}

\begin{proof}
	Si c'était le cas, on prend un cercle $\mathcal{C}$ compact (et connexe par arcs). $f(\mathcal{C})$ est compact connexe par arc dans $\R$. On note $f(\mathcal{C})=[a,b]$ (avec $a<b$ car $f$ injective). Si $x\in\mathcal{C}$ est tel que $f(x)=\frac{a+b}{2}$, on $\underbrace{f(\mathcal{C}\setminus\{x\})}_{\text{connexe par arc}}=\underbrace{[a,b]\setminus\Bigl\{\frac{a+b}{2}\Bigr\}}_{\text{pas connexe par arc}}$ donc une telle fonction n'existe pas.
\end{proof}

\begin{proof}
	\phantom{}
	\begin{enumerate}
		\item Pour tout $n\in\N$, $\Vert e_{n}\Vert_{l^{1}}=1$ et $\vert K_{n}\vert=\vert\varphi(e_{n})\vert\leqslant\vertiii{\varphi}$ donc $(K_{n})_{n\in\N}$ est bornée. On note $M=\sup\vert K_{n}\vert\leqslant\vertiii{\varphi}$.
		
		Soit maintenant $u=(u_{n})_{n\in\N}\in l^{1}$. On a, pour $N\in\N$, 
		$$\Biggl\lVert u-\sum_{n=0}^{N}u_{n}e_{n}\Biggr\rVert_{1}\leqslant\sum_{n=N+1}^{\infty}\vert u_{n}\vert\xrightarrow[N\to+\infty]{}0$$
		(reste d'une série convergente). Par continuité de $\varphi$, on a donc 
		$$\vert \varphi(u)\vert\leqslant\sum_{n=0}^{\infty}\vert u_{n}\vert \vert K_{n}\vert\leqslant M\Vert u\Vert_{1}$$

		Ainsi, $\vertiii{\varphi}\leqslant M$ et donc $\vertiii{\varphi}=M$.

		\item $F$ est linéaire et une isométrie d'après la question précédente, donc injective. 
		
		Soit $(K_{n})_{n\in\N}\in l^{\infty}$. On définit \function{\varphi}{l^1}{\R}{u=(u_n)_{n\in\N}}{\sum_{n=0}^{\infty}u_{n}K_{n}}
		Elle est bien définie car $\sum_{n=0}^{+\infty}\vert u_{n}\vert<+\infty$ et $(K_{n})_{n\in\N}$ est bornée. Elle est linéaire, et continue car $\vert\varphi(u)\vert\leqslant\Vert(K_{n})_{n\in\N}\Vert_{\infty}\Vert u\Vert_{1}$.

		Enfin, pour tout $n\in\N,\varphi(e_{n})=K_{n}$. Donc $F(\varphi)=(K_{n})_{n\in\N}$ et $F$ est surjective. Donc $F$ est une isométrie bijective et le dual topologique de $l^{1}$ est équivalent à $l^{\infty}$.
	\end{enumerate}
\end{proof}

\begin{proof}
	\phantom{}
	\begin{enumerate}
		\item Soit $\varphi$ une forme linéaire non nulle telle que $K=\ker(\varphi)$/ Si $F$ est dense, $\varphi$ est discontinue. Soit $(a,b)\in(E\setminus H)^{2}$ et $(x_{n})_{n\in\N}\in H^{\N}$ qui converge vers $b-a$ (existe car $H$ est dense). La suite $(a+x_{n})_{n\in\N}$ converge vers $b$. Pour $n\in\N$, on a $\varphi(a+x_{n})=\varphi(a)\neq0$, et pour $t\in[0,1]$, $\varphi(t(a+x_{n})+(1-t)(a+x_{n+1}))=\varphi(a)\neq0$. Donc $[a+x_{n},a+x_{n+1}]\subset E\setminus H$.
		
		Soit $\gamma:[0,1]\to E\setminus H$ telle que 
		$$
		\left\{
			\begin{array}[]{rcll}
				\gamma(t) & = &\alpha_{n}t+\beta_{n}\in[a+x_{n},a+x_{n+1}]\subset E\setminus H &\text{si }t\in[1-\frac{1}{n},1-\frac{1}{n+1}]\\
				\gamma(1) & = &b&\\
				\gamma(t) &= & a+tx_{0}&\text{si }t\in[0,\frac{1}{2}]
			\end{array}
		\right.
		$$

		On cherche à définir $\alpha_{n}$ et $\beta_{n}$: on veut $\gamma(1-\frac{1}{n})=a+x_{n}$ et $\gamma(1-\frac{1}{n+1})=a+x_{n+1}$ (pour la continuité en se raccordant au $x_{n}$). En résolvant le système, on trouve $\alpha_{n}=n(n+1)(x_{n}-x_{n+1})$ et $\beta_{n}=a+x_{n}-(n-1)(n+1)(x_{n}-x_{n+1})$.

		Soit alors $\varepsilon>0$, il existe $N\in\N$ tel que pour tout $n\geqslant N\colon\Vert x_{n}+a-b\Vert<\varepsilon$ et pour tout $n\geqslant N$, pour tout $t\in[1-\frac{1}{n},1-\frac{1}{n+1}[$, $\gamma(t)\in[a+x_{n},a+x_{n+1}]\subset B(b,\varepsilon)$ par convexité de la boule. Donc $\lim\limits_{t\to 1}\gamma(t)=b$ et $\gamma$ est continue. Donc $E\setminus H$ est connexe par arcs.

		\item Soit $\varphi$ une forme linéaire telle que $\ker(f)=H$ est fermé. Alors $\varphi$ est continue (à redémontrer). Soit $x\in E\setminus H$, on a $\varphi(x)\varphi(-x)<0$ et d'après le théorème des valeurs intermédiaires, si $E\setminus H$ était connexe par arcs, $\varphi$ s'annulerait sur $E\setminus H$ ce qui n'est pas vrai. Donc $E\setminus H$ n'est pas connexe par arcs.
		
		\item Si $\K=\C$, si $H$ est dense alors $E\setminus H$ est connexe par arc d'après la première question. Si $H$ est fermé, soit $\varphi$ une forme linéaire continue telle que $\ker(f)=H$. Soit $(x_{1},x_{2})\in(E\setminus H)^{2}$. 
		
		\begin{itemize}
			\item Si $\frac{\varphi(x_{1})}{\varphi(x_{2})}\notin\R_{-}^{*}$, alors pour tout $t\in[0,1]$, $\varphi(\underbrace{tx_{1}+(1-t)x_{2}}_{\in E\setminus H})\neq0$ et on peut relier directement $x_{1}$ et $x_{2}$.
			\item Sinon, il existe $\theta\in\R,(\rho,\rho')\in(\R_{+}^{*})^{2}$ avec $\varphi(x_{1})=\rho e^{\mathrm{i}\theta}$ et $\varphi(x_{2})=\rho'e^{\mathrm{i}(\theta+\pi)}$. Alors $x_{3}=ix_{1}$ est tel que $[x_{1},x_{3}]\subset E\setminus H$ et $[x_{2},x_{3}]\subset E\setminus H$ (on contourne l'origine par une rotation de l'angle $\frac{\pi}{2}$). Par conséquent, on peut utiliser $x_{3}$ pour relier $x_{1}$ et $x_{2}$ donc $E\setminus H$ est connexe par arcs.
		\end{itemize}
	\end{enumerate}
\end{proof}

\begin{proof}
	Soit \function{\varphi}{\R_{+}^{*}}{\R}{x}{((x,\sin(\frac{1}{x})))}
	$\varphi$ est continue et $\Gamma)\varphi(\R_{+}^{*})$ est connexe par arcs.

	On a $\overline{\Gamma}=\Gamma\cup\Gamma'$ avec $\Gamma'=\{(0,y)\bigm| y\in[-1,1]\}$. En effet, pour tout $y\in[-1,1]$, on pose $x_{k}=\frac{1}{\arcsin(y)+2k\pi}$. On a $\sin(\frac{1}{x_{k}})=y\xrightarrow[k\to+\infty]{}y$ donc $(0,y)=\lim\limits_{k\to=+\infty}(x_{k},\sin(\frac{1}{x_{k}}))\in\overline{\Gamma}$.

	Réciproquement, si $(x,y)\in\overline{\Gamma}$, il existe $(x_{k})\in(\R_{+}^{*})^{\N}$ avec $x=\lim\limits_{k\to+\infty}x_{k}$ et $y=\lim\limits_{k\to+\infty}\sin(\frac{1}{x_{k}})$. Si $x>0$, par continuité, $y=\sin(\frac{1}{x})$ et $(x,y)\in\Gamma$. Si $x=0$, $y\in[-1,1]$ donc $(x,y)\in\Gamma'$.

	Si $\overline{\Gamma}$ est connexe par arcs, il existe \function{\gamma}{[0,1]}{\overline{\Gamma}}{t}{(x(t),y(t))}
	continue telle que $\gamma(0)=(0,0)$ et $\gamma(1)=(\frac{1}{\pi},0)$. La première projection $t\mapsto x(t)$ est continue avec $x(0)=0$ et $x(1)=\frac{1}{\pi}$. On définit maintenant $t_{1}=\sup\{t\in[0,1]\bigm| x(t)=0\}$. Par continuité, $x(t_{1})=0$ et donc $t_{1}<1$. Donc pour tout $t>t_{1}$, $x(t)>0$ et $\gamma(t)=(x(t),\sin(\frac{1}{x(t)}))$ pour $t>t_{1}$ et $\gamma(t_{1})=(0,y_{1})$ avec $y_{1}\in[-1,1]$.

	Or, -1 et 1 n'appartiennent pas simultanément à $]y_{1}-\frac{1}{2},y_{1}+\frac{1}{2}[$. On peut supposer que $1\notin]y_{1}-\frac{1}{2},y_{1}+\frac{1}{2}[$. Comme $\gamma$ est continue, il existe $t_{2}>t_{1}$ tel que pour tout $t\in]t_{1},t_{2}]$, $\sin(\frac{1}{x(t)})\in]y_{1}-\frac{1}{2},y_{1}+\frac{1}{2}[$. Or $x(t_{2})>0$ et $x(t_{1})=0$ donc il existe $k\in\N^{*}$, $t_{0}\in]t_{1},t_{2}]$ tel que $x(t_{0})=\frac{1}{2k\pi+\frac{\pi}{2}}$ (théorème des valeurs intermédiaires). Mais alors $\sin(\frac{1}{x(t_{0})})=1\notin]y_{1}-\frac{1}{2},y_{1}+\frac{1}{2}[$ ce qui contredit ce qui précède.

	Donc $\overline{\Gamma}$ n'est pas connexe par arcs.
\end{proof}

\begin{proof}
	\phantom{}
	\begin{enumerate}
		\item Pour tout $n\in\N$, $u_{n}\in K$ car $u_{n}$ est le barycentre de $(a,T(a),\dots,T^{n}(a))$ et $K$ est convexe. Comme $K$ est compact, on peut extraire $u_{\sigma(n)}\xrightarrow[n\to+\infty]{}u\in K$. Alors
		$$(id_{E}-T)(u_{\sigma(n)})=\frac{1}{\sigma(n)+1}(id_{E}-T^{\sigma(n)+1})(a)$$
		d'où 
		$$\rVert(id_{E}-T)(u_{\sigma(n)})\lVert\leqslant\frac{1}{\sigma(n)+1}\times 2M\xrightarrow[n\to+\infty]{}0$$
		avec $M=\sup\limits_{x\in K}\Vert x\Vert$ (existe car $K$ est compact donc borné). Par continuité de $T$, on a $T(u)=u$.

		\item Posons $F'=\{u\in K\bigm| T(u)=u\}$ fermé car $K'=K\cap\Bigl((\underbrace{id_{E}-T}_{\text{continu}})^{-1}\{0\}\Bigr)$.
		Donc $K'$ est compact et non vide d'après la première question. De plus, pour tout $(u_{1},u_{2})\in K'^{2}$, pour tout $t\in[0,1]$, par linéarité de $T$, on a 
		$$T(tu_{1}+(1-t)u_{2})=tu_{1}+(1-t)u_{2}$$
		donc $K'$ convexe. De plus, comme $U\circ T=T\circ U$, pour tout $u\in K'$, on a $T(U(u))=U(T(u))=U(u)$ donc $U(u)\in K'$. On applique alors la question 1 à $K'$ est il existe $y\in K'\colon U(y)=y$ et $T(y)=y$.
	\end{enumerate}
\end{proof}

\begin{proof}
	\phantom{}
	\begin{enumerate}
		\item C'est le théorème du rang car $\rg(u)\leqslant n\leqslant p-2$, et $H=\{(\alpha_{1},\dots,\alpha_{p})\bigm|\sum_{i=1}^{p}\alpha_{i}=0\}$ est de dimension $p-1$ donc $H\cap\ker(u)\neq\{0\}$ (formule de Grassmann).
		
		\item On a 
		$$\sum_{i=1}^{p}(\lambda_{i}+t\alpha_{i})x_{i}=\sum_{i=1}^{p}\lambda_{i}x_{i}+t\sum_{i=1}^{p}\alpha_{i}x_{i}=x$$
		et
		$$\sum_{i=1}^{p}(\lambda_{i}+t\alpha_{i})=\sum_{i=1}^{p}\lambda_{i}+t\sum_{i=1}^{p}\alpha_{i}=1$$

		Soit $I_{+}=\{i\in\{1,\dots,p\}\bigm|\alpha_{i}>0\}$ et $I_{-}=\{i\in\{1,\dots,p\}\bigm|\alpha_{i}<0\}$. On a $I_{+}\neq\emptyset$ et $I_{-}\neq\emptyset$ car $\sum_{i=1}^{p}\alpha_{i}=0$ et $(\alpha_{1},\dots,\alpha_{p})\neq(0,\dots,0)$. Soit $t\geqslant0$. Pour tout $i\in I_{+}$, $\lambda_{i}+t\alpha_{i}\geqslant0$. Pour $i\in I_{-}$, $\lambda_{i}+t\underbrace{\alpha_{i}}_{<0}\geqslant 0$ si et seulement si $t\leqslant-\frac{\lambda_{i}}{\alpha_{i}}$. Prenons alors 
		$$t=\min\limits_{i\in I_{-}}\Bigl(-\frac{\lambda_{i}}{\alpha_{i}}\Bigr)$$
		On au aussi pour tout $i\in I_{-}$, $\lambda_{i}+t\alpha_{i}\geqslant 0$ et il existe $i_{0}\in I_{-}$ tel que $\lambda_{i_{0}}+t\alpha_{i_{0}}=0$.

		\item Par récurrence descendante, on se ramène à n+1 points car si $x$ est barycentre de $p$ points avec $p\geqslant n+2$, alors il est barycentre de $p-1$ points.
		
		\item Soit $A=\{(\lambda_{1},\dots,\lambda_{n+1})\in\R_{+}^{n+1}\bigm|\sum_{i=1}^{n+1}\lambda_{i}=1\}$ fermé et borné en dimension finie donc compact. Soit \function{f}{A\times K^{n+1}}{\conv(K)}{((\lambda_{1},\dots,\lambda_{n}),(x_{1},\dots,x_{n+1}))}{\sum_{i=1}^{n+1}\lambda_{i}x_{i}}
		$f$ est surjective et continue, donc $\conv(K)$ est l'image continue d'un compact donc $\conv(K)$ est compact.
	\end{enumerate}
\end{proof}

\begin{proof}
	Pour tout $u\in A_{p}$, $\Sp(u)\subset\{\alpha_{1},\dots,\alpha_{r}\}$ distincts et $u$ est diagonalisable. Réciproquement, si $u$ est diagonalisable et $\Sp(u)\subset\{\alpha_{1},\dots,\alpha_{r}\}$ alors dans une base la matrice de $u$ est diagonale avec des $\alpha_{i}$ (éventuellement plusieurs selon leur multiplicités), donc $u\in A_{p}$.

	Si $u\in A_{p}$, on écrit donc le polynôme caractéristique de $u$
	$$\chi_{u}=\prod_{i=1}^{r}(X-\alpha_{i})^{m_{i}}$$
	avec $0\leqslant m_{i}\leqslant\dim(E)=n$ et $\sum_{i=1}^{r}m_{i}=n$.
	$u\mapsto\chi_{u}$ est continue. Pour $(m_{1},\dots,m_{r})\in\{0,\dots,n\}^{r}$ tel que $\sum_{i=1}^{r}m_{i}=n$, notons 
	$$A_{m_{1},\dots,m_{r}}=\Biggl\{u\in A_{p}\Bigm|\chi_{u}=\prod_{i=1}^{r}(X-\alpha_{i})^{m_{i}}\Biggr\}$$
	et 
	$$\Bigl[u\mapsto\chi_u(A_{p})\Bigr]=\Biggl\{\bigcup_{(m_{1},\dots,m_{r})\in D_{n,r}}\Bigr\{\prod_{i=1}^{r}(X-\alpha_{i})^{m_{i}}\Bigr\}\Biggr\}$$
	où
	$$D_{n,r}=\Bigl\{(m_{1},\dots,m_{r})\in\{0,\dots,n\}^{r}\Bigm|\sum_{i=1}^{r}m_{i}=n\Bigr\}$$

	Donc d'après la contraposée du théorème des valeurs intermédiaires,\\si $(m_{1},\dots,m_{r})\neq(m'_{1},\dots,m'_{r})$, alors $A_{m_{1},\dots,m_{r}}$ et $A_{m'_{1},\dots,m'_{r}}$ ne sont pas dans la même composante connexe par arcs car
	$$\Bigl[u\mapsto\chi_u\Bigl(A_{m_{1},\dots,m_{p}}\bigcup A_{m'_{1},\dots,m'_{r}}\Bigr)\Bigr]=\underbrace{\Biggr\{\prod_{i=1}^{r}(X-\alpha_{i})^{m_{i}}\Bigr\}\Biggr\}\bigcup\Biggr\{\prod_{i=1}^{r}(X-\alpha_{i})^{m'_{i}}\Bigr\}\Biggr\}}_{\text{pas connexe par arcs}}$$
	
	Si $\gamma\colon[0,1]\to A_{p}$ est continue, $t\mapsto\chi_{\gamma(t)}=a_{0}(t)+a_{1}(t)X+\dots+a_{n-1}(t)X^{n-1}+X^{n}$ est continue sur $[0,1]$ et prend un nombre fini de valeurs donc est constante. $a_{i}\colon[0,1]\to\R$ continues et prend un nombre fini de valeurs donc est constante.

	Soit $u_{0}\in A_{m_{1},\dots,m_{r}}$, soit $u\in A_{m_{1},\dots,m_{r}}$, alors il existe une base $\mathcal{B}_{0}$ base de $E$ telle que $\mat\limits_{\mathcal{B}_{0}}(u_{0})=M_{0}$ soit diagonale avec des $\alpha_{1}$ sur les $m_{1}$ premières lignes de la diagonale, $\alpha_{2}$ sur les $m_{2}$ lignes suivantes, etc. Soit $M=\mat\limits_{\mathcal{B}_{0}}(u)$. $M$ est semblable à $M_{0}$ donc il existe $P\in GL_{n}(\C)$ telle que $M=PM_{0}P^{-1}$.

	Or $GL_{n}(\C)$ est connexe par arcs, donc il existe $\varphi\colon[0,1]\to GL_{n}(\C)$ continue telle que $\varphi(0)=P$ et $\varphi(1)=I_{n}$. On pose alors \function{\Phi}{[0,1]}{A_{m_{1},\dots,m_{r}}}{t}{\varphi(t)M_{0}\varphi^{-1}(t)}
	Alors $A_{m_{1},\dots,m_{r}}$ est connexe par arcs.

	Le nombre de composantes est donc égal au cardinal de 
	$$D_{n,r}=\Bigl\{(m_{1},\dots,m_{r})\in\{0,\dots,n\}^{r}\Bigm|\sum_{i=1}^{r}m_{i}=n\Bigr\}$$
	qui vaut $\binom{m+r-1}{r-1}$ possibilités (place $n$ points sur une droite et les séparer avec $r-1$ barres: le nombre de points dans chaque segment donne un $m_{i}$, il y a $m+r-1$ possibilités pour placer les $r-1$ barres).
\end{proof}

\begin{proof}
	\phantom{}
	\begin{enumerate}
		\item Pour tout $i\in\{1,\dots,n\}$, $\vert AX\vert_{i}=\sum_{j=1}^{n}\underbrace{a_{i,j}x_{j}}_{>0}\geqslant0$. Si $\vert AX\vert_{i}=0$ alors pour tout $j\in\{1,\dots,n\}$, $\underbrace{a_{i,j}}_{>0}x_{j}=0$ donc $x_{j}=0$, impossible car $X\neq 0$.
		
		\item Si $\vert AX\vert=A\vert X\vert$. On a pour tout $i\in\{1,\dots,n\}$,
		$$\Bigl\lvert\sum_{j=1}^{n}a_{i,j}x_{j}\Bigr\rvert=\sum_{j=1}^{n}a_{i,j}\lvert x_{j}\rvert$$
		donc les $(a_{i,j}x_{j})_{1\leqslant j\leqslant n}$ ont tous même argument. On prend $\theta=\arg(x_{j})$.

		\item $K$ est fermé et borné en dimension finie: c'est un compact. On a $I_{x}\neq\emptyset$ car $AX\geqslant0$ donc $0\in I_{x}$. Soit $(t_{n})_{n\in\N}\in I_{x}^{\N}$ convergeant vers $t\in\R$. Pour tout $k\in\N$, $AX-t_{k}X\geqslant0$ donc pour tout $i\in\{1,\dots,n\}$, $(AX-t_{k}X)_{i}\geqslant0$ et par passage à la limite, $AX-tX\geqslant0$ donc $I_{x}$ est fermé.
		
		Si $t\in I_{x}$, 
		$$\vert tX\vert_{1}=t=\sum_{i=1}^{n}t\underbrace{x_{i}}_{\geqslant0}\leqslant\sum_{i=1}^{n}\underbrace{\sum_{j=1}^{n}a_{i,j}x_{j}}_{=(AX)_{i}}\leqslant n\max\limits_{1\leqslant i,j\leqslant n}\vert a_{i,j}\vert$$
		car $\sum_{j=1}^{n}x_{j}=1$.
		On note $M=n\max\limits_{1\leqslant i,j\leqslant n}\vert a_{i,j}\vert$.

		\item Pour tout $x\in K$, $\theta(X)\leqslant M$ donc $\theta$ est bien borné sur $K$. Par définition de $r_{0}$, il existe $(X_{k})_{k\in\N}\in K^{\N}$ tel que $\lim\limits_{k\to+\infty}\theta(X_k)=r_{0}$. On note $\theta(X_{k})=t_{k}$. Comme $K$ est compact, il existe $\sigma\colon\N\to\N$ strictement croissante telle que $X_{\sigma(k)}$ converge vers $X^{+}\in K$. A priori, $\theta(X^{+})\leqslant r_{0}$. On a $AX_{\sigma(k)}-t_{\sigma(k)}X_{\sigma(k)}\geqslant0$ pour tout $k\in\N$ donc par passage à la limite, $AX^{+}-r_{0}X^{+}\geqslant0$ et donc $r_{0}\leqslant\theta(X^{+})$ donc $r_{0}=\theta(X^{+})$.
		
		\item Soit $Y=A^{+}-r_{0}X^{+}\geqslant0$. Si $Y\neq0$, alors $AY>0$ d'après la question 1 donc 
		$$AY=A\underbrace{(AX^{+})}_{>0}-r_{0}\underbrace{(AX^{+})_{>0}}>0$$
		On a $AY>\varepsilon AX^{+}$ si et seulement si pour tout $i\in\{1,\dots,n\}$, $\vert AY\vert_{i}>\varepsilon\vert AX^{+}\vert_{i}$ (car $AY>0$). On pose alors 
		$$\varepsilon=\frac{1}{2}\min\limits_{1\leqslant i\leqslant n}\frac{\vert AY\vert_{i}}{\vert AX^{+}\vert_{i}}$$
		On a alors $AY-\varepsilon AX^{+}>0$ d'où 
		$$A\underbrace{\frac{AX^{+}}{\Vert AX^{+}\Vert_{1}}}_{\in K}-(r_{0}+\varepsilon)\frac{AX^{+}}{\Vert AX^{+}\Vert_{1}}>0$$
		donc $r_{0}+\varepsilon\in I_{\frac{AX^{+}}{\Vert AX^{+}\Vert_{1}}}$ c'est-à-dire 
		$$r_{0}+\varepsilon\leqslant\theta\Bigl(\frac{AX^{+}}{\Vert AX^{+}\Vert_{1}}\Bigr)\leqslant r_{0}$$
		ce qui est impossible. Nécessairement $Y=0$.

		\item Pour tout $i\in\{1,\dots,n\}$, on a 
		$$\vert AV\vert_{i}=\Bigl\lvert\sum_{j=1}^{n}a_{i,j}v_{j}\Bigr\rvert\leqslant\sum_{i=1}^{n}a_{i,j}\vert v_{j}\vert=(A\vert V\vert)_{i}$$
		donc $\vert\lambda\vert=\vert AV\vert\leqslant A\vert V\vert$. De plus, $\vert V\vert\in K$ donc $\vert\lambda\vert\leqslant\theta(\vert V\vert)\leqslant r_{0}$. Notons que cela implique que le rayon spectral de $A$ est $\rho(A)$ est plus petit que $r_{0}$ et que l'on a même égalité.

		\item Si $\vert\lambda\vert=r_{0}$, on a $\vert\lambda\vert=\theta(\vert V\vert)=r_{0}$ et d'après la question 5 on a $A\vert V\vert=r_{0}\vert V\vert=\vert AV\vert$.
		
		D'après la question 2, il existe $\theta\in\R$ tel que $V=e^{\mathrm{i}\theta}\vert V\vert$. Or 
		$$AV=\lambda V=e^{\mathrm{i}\theta}A\vert V\vert=e^{\mathrm{i}\theta}r_{0}\vert V\vert$$
		et comme $\vert K\vert\in K, \vert V\vert\neq0$ et on a donc $\lambda=r_{0}$.

		\item Soit $V\in\M_{n,1}(\C)$ tel que $\Vert V\Vert_{1}=1$ et $AV=r_{0}V$. D'après la question précédente, on a $V=e^{\mathrm{i}\theta}\vert V\vert$ et $A\vert V\vert=r_{0}\vert V\vert$. Soit alors $t\in\R$, on a 
		$$A(X^{+}+t\vert V\vert)=r_{0}(X^{+}+t\vert V\vert)$$
		Notons maintenant que si $Y\geqslant0$ avec $Y\neq0$ vérifie $AY=r_{0}Y$, alors $Y>0$. En effet, d'après la première question, $AY>0$. On a $r_{0}\neq0$ car sinon $\Sp_{\C}=\{0\}$ et $A^{n}=0$ ce qui est impossible car ses coefficients sont strictement positifs. D'où $Y>0$.

		Ainsi, par définition de $X^{+}$, on a $X^{+}>0$ et $\vert V\vert>0$. On a alors 
		$$(X^{+})_{i}+t\vert v_{i}\vert\geqslant0$$
		si et seulement si
		$$t\geqslant -\frac{\vert X^{+}\vert_{i}}{\vert v_{i}\vert}$$
		On prend 
		$$t=\min\limits_{1\leqslant i\leqslant n}-\frac{\vert X^{+}\vert_{i}}{\vert v_{i}\vert}$$
		Finalement, on a $X^{+}+t\vert V\vert\geqslant0$ et une de ses coordonnées vaut 0 (car on a pris le minimum sur les $i$). Nécessairement, $X^{+}+t\vert V\vert=0$ (car $A(X^{+}+t\vert V\vert)=r_{0}(X^{+}+t\vert V\vert)$) et donc $\vert V\vert\in\R X^{+}$. Donc $V=e^{\mathrm{i}\theta}\vert V\vert\in\C X^{+}$ et ainsi 
		$$\dim(\ker(A-r_{0}I_{n}))=1$$
	\end{enumerate}
\end{proof}

\begin{proof}
	Soit \function{\varphi}{U\times V}{\R}{(x,y)}{\Vert x-y\Vert}
	On a 
	$$\vert\varphi(x,y)-\varphi(x',y')=\vert\Vert x-y\Vert-\Vert x'-y'\Vert\vert\leqslant\Vert (x-y)-(x'-y')\Vert\leqslant\Vert x-x'\Vert+\Vert y-y'\Vert\leqslant2\Vert(x,y)-(x',y')\Vert_{\infty}$$
	donc $\varphi$ est continue.

	$U\times V$ est compact, donc il existe $(x_{1},y_{1})\in(U\times V)$ telle que $\varphi(x_{1},y_{1})=\min\limits_{(x,y)\in U\times V}\varphi(x,y)$. Comme $U$ et $V$ sont disjoints, $x_{1}\neq y_{1}$ et $\varphi(x_{1},y_{1}))d(U,V)>0$.

	Soit $\alpha=\frac{d(U,V)}{3}$. On pose $U'=\{x\in E\Bigm|d(x,U)<\alpha\}$ et $V'=\{x\in E\Bigm|d(x,V)<\alpha\}$. $x\mapsto\Vert x\Vert$ est continue car $1$-lipschitzienne donc $U'$ est $V'$ sont des ouverts et on a bien $U\subset U'$ et $V\subset V'$. Soit ensuite $x\in U'\cap V'$, on a $d(x,U)<\alpha$ et $d(x,V)<\alpha$ donc il existe $(u,v)\in U\times V$, $d(x,u)<\alpha$ et $d(x,v)<\alpha$. Alors $d(u,v)\leqslant2\alpha$ ce qui est absurde. Donc $U'\cap V'=\emptyset$.
\end{proof}

\begin{proof}
	\phantom{}
	\begin{enumerate}
		\item $f$ est 1-lipschitzienne donc est continue. On forme \function{g}{K}{\R}{x}{\Vert x-f(x)\Vert}
		$g$ est continue, $K$ est compact donc il existe $a\in K$ tel que $g(a)=\min_{x\in K}g(x)$. Si $a\neq f(a)$, alors $\Vert f(a)-f^{2}(a)\Vert=g(f(a))<\Vert a-f(a)\Vert=g(a)$ ce qui est impossible par définition de $a$. Donc $f(a)=a$. S'il existe $a'\neq a$ tel que $f(a')=a'$, alors $\Vert f(a)-f(a')\Vert=\Vert a-a'\Vert<\Vert a-a'\Vert$ ce qui est impossible. Donc $a$ est unique.

		\item S'il existe $n_{0}\in\N$ tel que $u_{n_{0}}=a$ alors pour tout $n\geqslant n_{0}$, $u_{n}=a$ et $\lim\limits_{n\to+\infty}u_{n}=a$. Si pour tout $n\in\N$, $u_{n}\neq a$, alors pour tout $n\in\N$, on a
		$$\Vert u_{n+1}-a\Vert=\Vert f(u_{n})-f(a)\Vert<\Vert u_{n}-a\Vert$$
		donc la suite $(\Vert u_{n}-a\Vert)_{n\in\N}$ est strictement décroissante dans $\R_{+}$ donc elle converge vers $l\geqslant0$. Par compacité de $K$, il existe une extraction $\sigma$ telle que $\lim\limits_{n\to+\infty}u_{\sigma(n)}=\alpha\in K$. Par continuité, $$\lim\limits_{n\to+\infty}\Vert u_{\sigma(n)}-a\Vert=\Vert\alpha-a\Vert=l$$ 
		et
		$$\lim\limits_{n\to+\infty}\Vert \underbrace{u_{\sigma(n)+1}}_{f(u_{\sigma(n)}}-f(a)\Vert=\Vert f(\alpha)-f(a)\Vert=l=\Vert\alpha-a\Vert$$
		par continuité de $f$.
		Ainsi, on a $\alpha=a$ et $l=0$ donc $\lim\limits_{n\to+\infty}u_{n}=a$.

		\item $f$ est $\mathcal{C}^{1}$ sur $\R$. Soit $x<y\in\R^{2}$, il existe $z\in]x,y[$ tel que (égalité des accroissements finis)
		$$\Bigl\lvert\frac{f(x)-f(y)}{x-y}\Bigr\rvert=\vert f'(z)\vert=\Bigl\lvert\frac{z}{\sqrt{z^{2}+1}}\Bigr\rvert<1$$
		donc $f$ vérifie bien l'hypothèse de contraction. Cependant, pour tout $a\in\R$, on a $\sqrt{a^{2}+1}>a$ donc pas de point fixe. La démonstration tombe en défaut car $\R$ n'est pas compact.
	\end{enumerate}
\end{proof}

\begin{proof}
	La condition est équivalente à pour tout $(M_{1},M_{2},M_{3})\in K_{1}\times K_{2}\times K_{3}$, $M_{1},M_{2}$ et $M_{3}$ ne sont pas alignés.\\
	On forme alors \function{f}{K_1\times K_2\times K_3}{\R_+}{(M_1,M_2,M_3)}{R(M_1,M_2,M_3)}
	où $R(M_{1},R_{2},M_{3})$ est le rayon du cercle circonscrit au triangle formé par $M_{1},M_{2}$ et $M_{3}$.

	On note $M_{i}=(x_{i},y_{i})$ et $\Delta_{i}$ la médiatrice de $[M_{j}M_{k}]$. Établissons une équation de $\Delta_{i}$. On a $M=(x,y)\in\Delta_{i}$ si et seulement si $\Vert \vec{MM_{j}}\Vert_{2}^{2}=\Vert\vec{MM_{k}}\Vert_{2}^{2}$ si et seulement si $(\vec{MM_{j}}+\vec{MM_{k}}\bigm|\vec{MM_{j}}-\vec{MM_{k}})=0$ (produit scalaire), si et seulement si $(\vec{MC_{i}}\bigm|\vec{M_{j}M_{k}})=0$ où $C_{i}$ est le milieu de $[M_{j}M_{k}]$, si et seulement si (calculer le produit scalaire)
	$$\Bigl(\frac{x_{j}+x_{k}}{2}-x\Bigr)(x_{k}-x_{j})+\Bigl(\frac{y_{j}+y_{k}}{2}-y\Bigr)(y_{k}-y_{j})=0$$
	Soit alors $M_{0}=(x_{0},y_{0})$ le centre du cercle circonscrit. $M_{0}\in\Delta_{i}\cap\Delta_{j}$ avec $i\neq j$. Par exemple, $M_{0}\in\Delta_{3}\cap\Delta_{1}$ si et seulement si
	$$
	\left\{
		\begin{array}[]{rcl}
			\Bigl(\dfrac{x_{2}+x_{1}}{2}-x_{0}\Bigr)(x_{2}-x_{1})+\Bigl(\dfrac{y_{2}+y_{1}}{2}-y_{0}\Bigr)(y_{2}-y_{1}) &= &0\\[0.5cm]
			\Bigl(\dfrac{x_{3}+x_{2}}{2}-x_{0}\Bigr)(x_{3}-x_{2})+\Bigl(\dfrac{y_{3}+y_{2}}{2}-y_{0}\Bigr)(y_{3}-y_{2}) &= &0
		\end{array}	
	\right.
	$$
	si et seulement si ($L_{2}\leftarrow L_{1}(x_{3}-x_{2})+L_{2}(x_{1}-x_{2})$)
	$$
	\left\{
		\begin{array}[]{rcl}
			x_{0}(x_{1}-x_{2})+y_{0}(y_{1}-y_{2})&=&\dfrac{x_{1}^{2}-x_{2}^{2}+y_{1}^{2}-y_{2}^{2}}{2}\\[0.5cm]
			x_{0}(x_{2}-x_{3})+y_{0}(y_{2}-y_{3})&=&\dfrac{x_{2}^{2}-x_{3}^{2}+y_{2}^{2}-y_{3}^{2}}{2}
		\end{array}	
	\right.
	$$
	si et seulement si ($L_{1}\leftarrow L_{2}(y_{2}-y_{1})+L_{1}(y_{2}-y_{3})$)
	$$
	\left\{
		\begin{array}[]{rcl}
			x_{0} &= & \dfrac{\frac{x_{1}^{2}-x_{2}^{2}+y_{1}^{2}-y_{2}^{2}}{2}(y_{2}-y_{3})-(y_{1}-y_{2})\frac{x_{2}^{2}-x_{3}^{2}+y_{2}^{2}-y_{3}^{2}}{2}}{(x_{1}-x_{2})(y_{2}-y_{3})-(x_{2}-x_{3})(y_{1}-y_{2})}\\[0.5cm]
			y_{0} &= & \dfrac{\frac{x_{2}^{2}-x_{3}^{2}+y_{2}^{2}-y_{3}^{2}}{2}(x_{1}-x_{2})-(x_{2}-x_{3})\frac{x_{1}^{2}-x_{2}^{2}+y_{1}^{2}-y_{2}^{2}}{2}}{(x_{1}-x_{2})(y_{2}-y_{3})-(x_{2}-x_{3})(y_{1}-y_{2})}
		\end{array}	
	\right.
	$$
	et $R(M_{1},M_{2},M_{3})=\sqrt{(x_{0}-x_{3})^{2}+(y_{0}-y_{3})^{2}}$. En reportant, $f$ est continue sur $K_{1}\times K_{2}\times K_{3}$ compact donc $f$ atteint son minimum.
\end{proof}

\begin{proof}
	\phantom{}
	\begin{enumerate}
		\item Pour tout $f\in E$, $T(f)$ est $\mathcal{C}^{1}$ et $(T(f))'=f$, $T(f)(0)=0$. $T$ est clairement linéaire, soit ensuite $x\in[0,1]$, on a 
		$$\vert T(f)(x)\vert=\Bigl\lvert\int_{0}^{x}f(t)dt\Bigr\rvert\leqslant\int_{0}^{x}\vert f(t)\vert dt\leqslant x\Vert f\Vert_{\infty}\leqslant\Vert f\Vert_{\infty}$$
		Donc $\Vert T(f)\Vert_{\infty}\leqslant\Vert f\Vert_{\infty}$ donc $T$ est continue et $\vertiii{T}\leqslant1$. Pour $f=1$, on a $\Vert f\Vert_{\infty}=1$ et pour tout $x\in[0,1]$, $T(f)(x)=x$ donc $\Vert T(1)\Vert_{\infty}=1$. Ainsi, $\vertiii{T}=1$.

		\item $id_{E}-T$ est continue. Soit $(f,g)\in E^{2}$, on a $g=f-T(f)$ si et seulement si $g=y'-y$ et $y(0)=0$. 
		On a $g(x)e^{-x}=\underbrace{e^{-x}(y'(x)-y(x))}_{(e^{-x}y(x))'}$ donc en intégrant de 0 à $x$ on a 
		$$y(x)=e^{x}\int_{0}^{x}e^{-t}g(t)dt$$
		Donc $T(f)$ vérifie le problème de Cauchy si et seulement si pour tout $x\in\R$, $T(f)(x)=e^{x}\int_{0}^{x}e^{-t}g(t)dt$ si et seulement si pour tout $x\in[0,1]$, 
		$$f(x)=g(x)+e^{x}\int_{0}^{x}e^{-t}g(t)dt$$
		Donc $id_{E}-T$ est bijective. 
		Enfin, on a pour tout $x\in[0,1]$, 
		$$\vert f(x)\vert\leqslant\vert g(x)\vert+\Bigl\lvert\int_{0}^{x}g(t)e^{x-t}dt\Bigr\rvert\leqslant\Vert g\Vert_{\infty}(1+xe^{x})\leqslant\Vert g\Vert_{\infty}(1+e)$$
		Ainsi, 
		$$\Vert f\Vert_{\infty}=\Vert(id_{E}-T)^{-1}(g)\Vert_{\infty}\leqslant\Vert g\Vert_{\infty}(1+e)$$
		donc $(id_{E}-T)^{-1}$ est continue. Ainsi, $id_{E}-T$ est un homéomorphisme.
	\end{enumerate}
\end{proof}

\begin{proof}
	\phantom{}
	\begin{enumerate}
		\item [(i) $\Rightarrow$ (ii)] $f^{-1}(K)$ est fermé car $f$ est continue. $K$ est borné, donc il existe $M>0$, tel que pour tout $y\in K$, $\Vert y\Vert\leqslant M$. Donc pour tout $x\in f^{-1}(K)$, $\Vert f(x)\Vert\leqslant M$. Par contraposée de (i) pour $A=M+1$, il existe $B>0$ tel que $\Vert f(x)\Vert<A\Rightarrow\Vert x\Vert<B$. Donc pour $x\in f^{-1}(K)$, $\Vert x\Vert<B$ donc $f^{-1}(K)$ est borné. C'est donc un compact.
		\item [(ii) $\Rightarrow$ (i)] Soit $A\geqslant0$. Soit $K=\overline{B(0,A)}$ compact car fermé et borné en dimension finie. D'après (ii), $f^{-1}(K)$ est compact donc borné: il existe $B>0$ tel que pour tout $x\in f^{-1}(K)$, $\Vert x\Vert\leqslant B$. Par contraposée, si $\Vert x\Vert>B$ alors $x\notin f^{-1}(K)$ et $f(x)\notin K$ donc $\Vert f(x)\Vert >A$. Ainsi, $\lim\limits_{\Vert x\Vert\to+\infty}\Vert f(x)\Vert=+\infty$.
	\end{enumerate}
\end{proof}

\begin{remark}
	Exemple pour l'exercice précédent: les fonctions polynômiales non constantes. Contre-exemple: l'exponentielle, cf $\exp([0,1])=\R_{-}$ non compact.
\end{remark}

\begin{proof}
	\phantom{}
	\begin{enumerate}
		\item 
		Soit $(x,y)\in K^{2}$ compact. Soit $\sigma$ un extraction telle que 
		$$(f^{\sigma(n)}(x),f^{\sigma(n)}(y))\xrightarrow[n\to+\infty]{}(l,l')\in K^{2}$$
		On a 
		$$f^{\sigma(n+1)}(x)-f^{\sigma(n)}(x)\xrightarrow[n\to+\infty]{}0$$
		de même pour $y$. Soit $\varepsilon>0$,
		$$
		\left\{
			\begin{array}[]{l}
				\exists N_{1}\in\N,\forall n\geqslant N_{1},\Vert f^{\sigma(n+1)}(x)-f^{\sigma(n)}(x)\Vert\leqslant\varepsilon\\
			\exists N_{1}\in\N,\forall n\geqslant N_{1},\Vert f^{\sigma(n+1)}(y)-f^{\sigma(n)}(y)\Vert\leqslant\varepsilon
		\end{array}
		\right.
		$$
		Pour $N=\max(N_{1},N_{2})$ et $p=\sigma(N+1)-\sigma(N)\in\N^{*}$, on a 
		\begin{equation*}
			d(x,f^{p}(x))\leqslant d(f^{\sigma(n+1)}(x),f^{\sigma(n)}(x))\leqslant\varepsilon
		\end{equation*}
		et de même pour $y$ avec le même $p$.
	
		\item On a 
		\begin{align*}
			d(x,y)
			&\leqslant d(f(x),f(y))\\
			&\leqslant d(f^{p}(x),f^{p}(y))\\
			&\leqslant d(f^{p}(x),x)+d(x,y)+d(y,f^{p}(y))\\
			&\leqslant 2\varepsilon+d(x,y)
		\end{align*}

		Ceci valant pour tout $\varepsilon>0$, on a égalité tout du long. On a donc notamment, $\Vert x-y\Vert=\Vert f(x)-f(y)\Vert$ et donc $f$ est une isométrie.

		\item $f$ est 1-lipschitzienne donc continue. Donc $f(K)$ est compact donc fermé. Il suffit donc de montrer que $f(K)$ est dense dans $K$. Soit $x\in K$ et $\varepsilon>0$, il existe $p\in\N^{*}$ tel que $\Vert x-\underbrace{f^{p}(x)}_{\in f(K)}\Vert\leqslant\varepsilon$ d'après la première question. Donc $f(K)$ est dense dans $K$ et $f(K)=\overline{f(K)}=K$.
	\end{enumerate}
\end{proof}

\begin{remark}
	Exemple pour l'exercice précédent: une rotation sur la sphère unité.
\end{remark}

\begin{proof}
	Soit \function{f}{K}{\R}{M}{f(M)=\text{rayon du cercle circonscrit au triangle MAB}}
	On a $F=f(K)$. Soit $(C,i,j)$ un repère orthonormé où $C$ est le milieu de $[AB]$ et $A(-\alpha,0)$ et $B(\alpha,0)$ avec $\alpha>0$. La médiatrice $\Delta$ de $[A,B]$ a pour équation $x=0$. Si $M(x,y)$, soit $\varphi(M)$ le centre du cercle circonscrit. On a $\varphi(M)\in\Delta$ donc $\varphi(M)(0,y_{1})$ et $\varphi(M)$ appartient à la médiatrice de $[MA]$. On a $y_{1}\neq0$ car $M\notin(AB)$.

	Notons $M'$ le milieu de $[MA]$. On a $M'(\frac{x-\alpha}{2},\frac{y}{2})$ d'où $\vec{M'\varphi(M)}\cdot\vec{MA}=0$ d'où (en développant le produit scalaire),
	$$y_{1}=\Bigl((\alpha+x)\Bigl(\frac{\alpha-x}{2}\Bigr)-\frac{y^{2}}{2}\Bigr)\Bigl(-\frac{1}{y}\Bigr)$$
	$\varphi$ est donc continue donc $f$ également et $f(K)=F$ est compact.
\end{proof}

\begin{proof}
	\phantom{}
	\begin{enumerate}
		\item Soit $\lambda\in\Sp(\tau)$ et $P\in\R[X]\setminus\{0\}$ avec $\tau(P)=\lambda P$. Si $P$ n'est pas constant, notons $\alpha\in\C$ alors $P(\alpha)=0$. Alors $P(\alpha+1)=0$. En itérant, pour tout $n\in\N$, $P(\alpha+n)=0$, impossible car $P$ n'est pas constant donc pas nul. Finalement, $P$ est constant et $\lambda=1$: $\Sp(\tau)=\{1\}$.
		\item $f\colon x\mapsto P(x)e^{-x}$ est continue et $\lim\limits_{x\to+\infty}f(x)=0$ donc le $\sup$ est bien défini. Il est ensuite facile de vérifier que $\Vert P\Vert$ est une norme.
		\item On a 
		$$\Vert\tau(P)\Vert=\sup\limits_{x\geqslant0}\vert P(x+1)e^{-x}\vert=\sup\limits_{x'\geqslant1}\vert P(x')e^{-x'}e\vert\leqslant\sup\limits_{x'\geqslant0}\vert P(x')e^{-x'}e\vert\leqslant e\Vert P\Vert$$
		\item Utiliser $P=X$.
	\end{enumerate}
\end{proof}

\begin{proof}
	\phantom{}
	\begin{enumerate}
		\item Pour $x$ fixé, $\min(x,\varphi(t))=\frac{x+\varphi(t)-\vert x-\varphi(t)\vert}{2}$ est continue. Donc $T(f)$ est définie.
		
		Si $x\leqslant\varphi(0)$,
		$$T(f)(x)=\int_{0}^{1}xf(t)dt=x\int_{0}^{1}f(t)dt$$
		et si $x\geqslant\varphi(1)$,
		$$T(f)(x)=\int_{0}^{1}\varphi(t)f(t)dt$$ 
		et si $\varphi(0)\leqslant x\leqslant\varphi(1)$, il existe un unique $t_{1}=\varphi^{-1}(x)$ (car $\varphi$ induit un homéomorphisme de $[0,1]$ dans $\varphi([0,1])$). 
		
		Si $t\leqslant t_{1}$, on a $\varphi(t)\leqslant x$, donc $\min(x,\varphi(t))=\varphi(t)$. Si $t\geqslant t_{1}$, on a $\min(x,\varphi(t))=x$. On a donc 
		\begin{align*}
			T(f)(x)
			&=\int_{0}^{t_{1}}\varphi(t)f(t)dt+\int_{t_{1}}^{1}xf(t)dt\\
			&=\underbrace{\int_{0}^{\varphi^{-1}(x)}\varphi(t)f(t)dt}_{=F_{1}(\varphi^{-1}(x))}+x\underbrace{\int_{\varphi^{-1}(x)}^{1}f(t)dt}_{=F_{2}(\varphi^{-1}(x))}
		\end{align*}
		et $f$ et $\varphi$ étant continues, $F_{1}$ et $F_{2}$ sont continues.

		Donc $T(f)$ continue et $T$ linéaire, c'est un endomorphisme de $E$.

		\item On a 
		\begin{equation*}
			\vert T(f)(x)\vert\leqslant\Vert f\Vert_{\infty}\underbrace{\int_{0}^{1}\min(x,\varphi(t))dt}_{=A(x)}
		\end{equation*}
		donc 
		$$\Vert T(f)\Vert_{\infty}\leqslant\Vert f\Vert_{\infty}\Vert A\Vert_{\infty}$$
		donc $T$ est continue et $\vertiii{T}\leqslant\Vert A\Vert_{\infty}$. De plus pour $f=1$, on a $\vertiii{T}=\Vert A\Vert_{\infty}$.

		\item On a 
		$$
		A(x)=\int_{0}^{1}\min(x,\varphi(t))dt=
		\left\{
			\begin{array}[]{lll}
				x & \text{si} & x\leqslant\varphi(0)\\
				\int_{0}^{1}\varphi(t)dt & \text{si} & x\geqslant\varphi(1)
			\end{array}
		\right.
		$$
		Dans tous les cas, 
		$$\Vert A\Vert_{\infty}\leqslant\int_{0}^{1}\varphi(t)dt$$
		donc 
		$$\Vert A\Vert_{\infty}=\int_{0}^{1}\varphi(t)dt$$
	\end{enumerate}
\end{proof}

\begin{proof}
	\phantom{}
	\begin{enumerate}
		\item $\varphi$ est une forme linéaire. et on a 
		$$\vert\varphi(P)\vert\leqslant\sum_{k\in\N}\Bigl\vert\frac{a_k}{2^{k}}\Bigr\vert\leqslant2\Vert P\vert_{\infty}$$
		donc $\varphi$ est continue et $\vertiii{\varphi}\leqslant2$. Pour $p\neq0$, $\vert\varphi(P)\vert<2\Vert P\Vert_{\infty}$ : pour avoir égalité, il faudrait pour tout $k\in\N$, $a_{k}=\text{constante}\neq0$ ce qui n'est pas possible. Pour $P_{n}=\sum_{k=0}^{n}X^{k}$, on a $\Vert P_{n}\Vert_{\infty}=1$ et $\lim\limits_{n\to+\infty}\vert\varphi(P_{n})\vert\xrightarrow[n\to+\infty]{}2$ donc $\vertiii{\varphi}=2$. De plus, $\ker(\varphi)=\varphi^{-1}(\{0\})$ est fermé.

		\item Soit $P=\sum_{k\in\N}a_{k}X^{k}\in\ker(\varphi)$. On a $\varphi(P)=0$ d'où $a_{0}=-\sum_{k=1}^{+\infty}\frac{a_{k}}{2^{k}}$ (et il existe $N_{0}\in\N,\forall n\geqslant N_{0},a_{n}=0$). On a donc 
		$$P(X)-1=(a_{0}-1)+\sum_{k\in\N^{*}}a_{k}X^{k}$$
		et si $\Vert P-1\Vert_{\infty}\leqslant\frac{1}{2}$, on a 
		$$
		\left\{
			\begin{array}[]{l}
				\vert a_{0}-1\vert\leqslant\frac{1}{2}\\
				\forall k\in\N^{*},\vert a_{k}\vert\leqslant\frac{1}{2}
			\end{array}
		\right.
		$$
		et 
		$$\vert a_{0}\vert=\Biggl\vert\sum_{k=1}^{+\infty}\frac{a_{k}}{2^{k}}\Biggr\vert\leqslant\sum_{k=1}^{+\infty}\frac{\vert a_{k}\vert}{2^{k}}\leqslant\sum_{k=1}^{+\infty}\frac{1}{2^{k+1}}=\frac{1}{2}$$

		Et $\frac{1}{2}\leqslant 1-\vert a_{0}\vert\leqslant\vert 1-a_{0}\vert\leqslant\frac{1}{2}$. Donc $\vert a_{0}\vert=\frac{1}{2}$ et $\vert 1-a_{0}\vert=\frac{1}{2}$.
		\begin{align*}
			a_{0}=\frac{1}{2}e^{\mathrm{i}\theta}
			&\Rightarrow \Bigl\vert 1-\frac{1}{2}e^{\mathrm{i}\theta}\Bigr\vert^{2}=\frac{1}{4}\\
			&\Rightarrow \Bigl(1-\frac{1}{2}\cos(\theta)\Bigr)^{2}+\Bigl(\frac{1}{2}\sin(\theta)\Bigr)^{2}=\frac{1}{4}\\
			&\Rightarrow 1-\cos(\theta)+\frac{1}{4}=\frac{1}{4}\\
			&\Rightarrow \cos(\theta)=1
		\end{align*}
		et donc $a_{0}=\frac{1}{2}$.

		Par ailleurs, on a 
		$$\frac{1}{2}=\sum_{k=1}^{+\infty}\frac{\vert a_{k}\vert}{2^{k}}=\sum_{k=1}^{+\infty}\frac{1}{2^{k+1}}$$
		Donc pour tout $k\in\N$, $\vert a_{k}\vert=\frac{1}{2}$, impossible car $P\in\C[X]$, ainsi $\Vert P-1\Vert_{\infty}>\frac{1}{2}$.

		\item On définit, pour $n\geqslant1$, $P_{n}=\frac{1}{2}+\sum_{k=1}^{n}(-\frac{1}{2}+\varepsilon_{n})X^{k}$ avec $\varepsilon_{n}\in\R$ tel que $P_{n}\in\ker(\varphi)$. On a 
		\begin{align*}
			P_{n}\in\ker(\varphi)
			&\Rightarrow\frac{1}{2}+\sum_{k=1}^{n}\Bigl(-\frac{1}{2}+\varepsilon_{n}\Bigr)\frac{1}{2^{k}}=0\\
			&\Rightarrow\varepsilon_{n}=-\frac{1}{2^{n+1}}\times \frac{1}{1-\frac{1}{2^{n}}}
		\end{align*}
		et donc $\varepsilon_{n}\xrightarrow[n\to+\infty]{}0$ (et $\varepsilon_{n}<0$). On a donc $\Vert P_{n}-1\Vert_{\infty}=\frac{1}{2}-\varepsilon_{n}\xrightarrow[n\to+\infty]{}\frac{1}{2}$.

		Donc $d(1,\ker(\varphi))=\frac{1}{2}$ et cette distance n'est pas atteinte.
	\end{enumerate}
\end{proof}

\begin{proof}
	Prouvons d'abord l'existence. Soit $M\in\R^{n}$, on définit $r(M)=\sup\{\Vert M-A\Vert\bigm| A\in K\}$ et $\varphi\colon A\mapsto\Vert M-A\Vert$ est continue sur $K$ compact donc le sup est en fait un max. On a notamment $r(M)=\{R>0\bigm| K\subset B(M,R)\}$. Soit \function{r}{\R^n}{\R}{M}{r(M)}
	Soit $(M,M')\in(\R^{n})^{2}$. Pour tout $A\in K$, on a 
	$$\Vert M-A\Vert\leqslant\Vert M-M'\Vert+\Vert M'-A\Vert\leqslant\Vert M-M'\Vert +r(M')$$
	En particulier, on a
	$$r(M)\leqslant\Vert M-M'\Vert+r(M')$$
	et en échangeant $M$ et $M'$, on a $\vert r(M)-r(M')\vert\leqslant\Vert M-M'\Vert$. Donc $r$ est 1-lipschitzienne donc continue. Soit $A_{0}\in K$, $R(M)\geqslant\Vert M-A_{0}\Vert\geqslant\Vert M\Vert-\Vert A_{0}\Vert\xrightarrow[\Vert M\Vert\to+\infty]{}+\infty$. Donc il existe $M_{0}\in\R^{n}$ tel que $r(M_{0})=\min\limits_{M\in\R^{n}}r(M)=r_{0}$, d'où l'existence d'une boule fermée de rayon minimal.

	Pour l'unicité, soit $(M_{1},M_{2})\in(\R^{n})^{2}$ tel que $r(M_{1})=r(M_{2})=r_{0}$. On suppose que $\Vert M_{1}-M_{2}\Vert=\varepsilon>0$. Soit $M_{3}$ le milieu de $[M_{1}M_{2}]$. On a $K\subset B_{M_{1},r_{0}}\cap B_{M_{2},r_{0}}$. On prend $r^{2}+\bigl(\frac{\varepsilon}{2}\bigr)^{2}=r_{0}^{2}$ d'où 
	$$r=\sqrt{r_{0}^{2}-\frac{\varepsilon^{2}}{4}}<r_{0}$$
	Soit $M\in B(M_{1},r_{0})\cap B(M_{2},r_{0})$, on a 
	\begin{align*}
		\Vert M-M_{3}\Vert^{2}
		&=\frac{1}{4}\Bigl(\Vert M-M_{1}+M-M_{2}\Vert^{2}\Bigr)\\
		&=\frac{1}{4}\Bigl(2\Vert M-M_{1}\Vert^{2}+2\Vert M-M_{2}\Vert^{1}-\underbrace{\Vert M_{1}-M_{2}\Vert^{2}}_{=\varepsilon^{2}}\Bigr)\\
		&\leqslant\frac{1}{4}(2r_{0}^{2}+2r_{0}^{2}-\varepsilon^{2})\\
		&\leqslant r_{0}^{2}-\frac{\varepsilon^{2}}{4}=r^{2}
	\end{align*}
	Donc $B_{1}\cap B_{2}\subset\overline{B(M_{3},r)}$ d'où $K\subset\overline{B(M_{3},r)}$, ce qui est absurde car $r<r_{0}$. Donc $M_{1}=M_{2}$.
\end{proof}

\begin{proof}
	$\varphi$ est évidemment définie et linéaire. Soit $f\in\mathcal{C}^{0}([0,1],\R)$.
	\begin{align*}
		\vert\varphi(f)\vert
		&=\Biggl\vert\int_{0}^{\frac{1}{2}}f-\int_{\frac{1}{2}}^{1}f\Biggr\vert\\
		&\leqslant\Biggl\vert\int_{0}^{\frac{1}{2}}f\Biggr\vert+\Biggl\vert\int_{\frac{1}{2}}^{1}f\Biggr\vert\\
		&\leqslant\int_{0}^{\frac{1}{2}}\vert f\vert+\int_{\frac{1}{2}}^{1}\vert f\vert\\
		&\leqslant\int_{0}^{1}\Vert f\Vert_{\infty}=\Vert f\Vert_{\infty}
	\end{align*}
\end{proof}

Donc $\varphi$ est continue et $\vertiii{\varphi}\leqslant1$. Notons que si l'on a $\vert\varphi(f)\vert=\Vert f\Vert_{\infty}$, alors on a égalité partout au-dessus et pour tout $t\in[0,1]$, $\vert f(t)\vert=\Vert f\Vert_{\infty}$ et comme $\Bigl\vert\int f\Bigr\vert=\int\vert f\vert$ implique que $f$ est de signe constant sur l'intervalle d'intégration, si l'on a $\vert\varphi(f)\vert=\Vert f\Vert_{\infty}$, alors $f$ est de signe constant sur $[0,\frac{1}{2}]$ et sur $[\frac{1}{2},1]$.  Or $\vert\int_{0}^{\frac{1}{2}}f-\int_{\frac{1}{2}}^{1}f\vert=\vert\int_{0}^{\frac{1}{2}}f\vert+\vert\int_{\frac{1}{2}}^{1}f\vert$, $f$ est de signe opposé sur les deux segments. Or $f$ est continue en $\frac{1}{2}$, donc $f$ est nulle. Donc pour $f$ non nulle, on a $\vert\varphi(f)\vert<\Vert f\Vert_{\infty}$ donc la norme triple n'est pas atteinte. Enfin, pour montrer que $\vertiii{\varphi}=1$, on utilise pour $n\geqslant1$,
$$
f_{n}(t)=
\left\{
	\begin{array}[]{lll}
		1 & \text{si} & t\in[0,\frac{1}{2}-\frac{1}{n}]\\[0.3cm]
		(\frac{1}{2}-t)n & \text{si} & t\in[\frac{1}{2}-\frac{1}{n},\frac{1}{2}+\frac{1}{n}]\\[0.3cm]
		-1 & \text{si} & t\in[\frac{1}{2}+\frac{1}{n},1]
	\end{array}
\right.
$$
On a bien $\Vert f_{n}\Vert_{\infty}=1$.

\begin{proof}
	\phantom{}
	\begin{enumerate}
		\item Non car on applique l'application trace.
		\item On a le résultat par récurrence.
		\item On a 
		$$(n+1)\vertiii{v^{n}}=\vertiii{u\circ v^{n}\circ v-v^{n}\circ v\circ r}\leqslant 2\vertiii{u}\vertiii{v}\vertiii{v^{n}}$$
		Si pour tout $n\in\N$, on a $v^{n}=0$, alors pour tout $n\in\N$,
		$$n+1\leqslant 2\vertiii{u}\vertiii{v}$$
		ce qui est impossible. Donc il existe $n\in\N^{*}$ tel que $v^{n}=0$. Alors $u\circ v^{n}-v^{n}\circ u=nv^{n-1}=0$ donc $v^{n-1}=0$ et de proche en proche $v=0$: contradiction.
		\item Pour tout $P\in\R[X]$, 
		$$(D\circ T-T\circ D)(P)=(XP)'-XP'=P$$
		donc $D\circ T-T\circ D=id$. D'après ce qui précède, $T$ et $D$ ne peuvent pas être continus simultanément.
	\end{enumerate}
\end{proof}

\begin{proof}
	\phantom{}
	\begin{enumerate}
		\item $\sum_{k\geqslant0}(A-I_{n})^{k}$ converge absolument car $\vertiii{A-I_{n}}^{k}\leqslant\alpha_{k}$ et $\alpha<$.
		
		Si $AX=0$, $\Vert (A-I_{n})X\Vert=\Vert X\Vert\leqslant\alpha\Vert X\Vert$ donc $\Vert X\Vert=0$ et $X=0$ donc $A\in GL_{n}(\C)$, idem pour $B$. On a alors
		\begin{equation*}
			A\sum_{k=0}^{+\infty}(I_{n}-A)^{k}=((A-I_{n})+I_{n})\sum_{k=0}^{+\infty}(I_{n}-A)^{k}=I_{n}
		\end{equation*}
		par téléscopage. Donc 
		$$A^{-1}=\sum_{k=0}^{+\infty}(I_{n}-A)^{k}$$
		et
		$$\vertiii{A^{-1}}\leqslant\sum_{k=0}^{+\infty}\alpha^{k}=\frac{1}{1-\alpha}$$
		et de même pour $B$. On écrit alors
		$$ABA^{-1}B^{-1}-I_{n}=(AB-BA)A^{-1}B^{-1}=((A-I_{n})(B-I_{n})-(B-I_{n})(A-I_{n}))A^{-1}B^{-1})$$
		d'où
		$$\vertiii{ABA^{-1}B^{-1}-I_{n}}\leqslant\frac{2\vertiii{A-I_{n}}\vertiii{B-I_{n}}}{(1-\alpha)(1-\beta)}$$

		\item On prend $\alpha=\beta=\frac{1}{4}$.
		\item Pour tout $M\in G$, il existe $r>0$ tel que $B(M,r)\cap G=\{M\}$. Montrons que $G$ est discret si et seulement si $I_{n}$ est isolé. En effet, si $I_{n}$ est isolé, il existe $r_{0}>0$ tel que $B(I_{n},r_{0})\cap G=\{I_{n}\}$. Soit $M\in G$, alors pour tout $M'\in G$, $M-M'=M(I_{n}-M^{-1}M')$ d'où $I_{n}-M^{-1}M'=M^{-1}(M-M')$. Si 
		$$\vertiii{M-M'}<\frac{r_{0}}{\vertiii{M^{-1}}}$$
		on a $\vertiii{I_{n}-M^{-1}M'}<r_{0}$ et donc $M'=M$ et $M$ est isolé. Ainsi $G$ est isolé. La réciproque est évidente.

		$C$ est dans le commutant si et seulement si $C$ commute avec $A$ et $B$ si et seulement si
		$$
		\left\{
			\begin{array}[]{l}
				ACA^{-1}C^{-1}=I_{n}\\
				BCB^{-1}C^{-1}=I_{n}
			\end{array}
		\right.
		$$

		Notons maintenant que 
		$$\overline{B_{\Vert\cdot\Vert}(I_{n},\frac{1}{4})}\cap G=\mathcal{A}$$
		est fini. En effet, si cet ensemble était infini, il existerait $(M_{p})_{p\in\N}$ une suite injective dans $\mathcal{A}$. La suite étant bornée, on peut extraite $(M_{\sigma(p)})_{p\in\N}$ qui converge et alors pour tout $p\in I_{n}$
		$$\underbrace{M_{\sigma(p)}M_{\sigma(p+1)}^{-1}}_{\xrightarrow[pto+\infty]{}I_{n}}\in G\setminus\{I_{n}\}$$
		ce qui est impossible car $I_{n}$ est isolé.

		Comme $A\in \mathcal{A}\setminus\{I_{n}\}$, il existe $C\in\mathcal{A}\setminus\{I_{n}\}$ telle que $\vertiii{C-I_{n}}$ soit minimale et $\vertiii{c-I_{n}}\leqslant\frac{1}{4}$. D'après la question 2 on a 
		$$\vertiii{ACA^{-1}C^{-1}-I_{n}}<\vertiii{C-I_{n}}$$
		et même chose pour $B$. Donc nécessairement, $ACA^{-1}C^{-1}=I_{n}$ et de même pour $B$. Ainsi, $C$ commute avec toutes les matrices de $G$.
	\end{enumerate}
\end{proof}

\begin{proof}
	\phantom{}
	\begin{enumerate}
		\item $\C_{n-1}[A]$ est un sous-espace vectoriel de dimension finie donc c'est un fermé. Par division euclidienne par $\chi_{A}$, d'après le théorème de Cayley-Hamilton, $\C[A]=\C_{n-1}[A]$. Comme 
		$$\exp(A)=\lim\limits_{n\to+\infty}\sum_{k=0}^{n}\frac{A^{k}}{k!}$$
		$\exp(A)\in \C[A]=\C_{n-1}[A]$.

		\item Si $A$ est diagonalisable, il existe $P\in GL_{n}(\C)$ tel que 
		$$A=P^{-1}\diag(\lambda_{1},\dots,\lambda_{n})P$$
		et donc 
		$$\exp(A)=P^{-1}\diag(e^{\lambda_{1}},\dots,e^{\lambda_{n}})P$$
		et $\exp(A)$ est diagonalisable.

		Si $\exp(A)$ est diagonalisable, on utilise la décomposition de Dunford: $A=D+N$ avec $DN=ND$, $D$ diagonalisable et $N$ nilpotente. On a donc 
		$$\exp(A)=\exp(D)\underbrace{\exp(N)}_{=\sum_{k=0}^{n-1}\frac{N^{k}}{k!}}=\exp(D)+\exp(D)\Bigl(\sum_{k=1}^{n-1}\frac{N^{k}}{k!}\Bigr)=\exp(D)+N'$$
		avec $N'$ nilpotente et $\exp(D)$ est diagonalisable d'après le sens direct. $N'$ commute avec $\exp(D)$. Par unicité de la décomposition de Dunford, $\exp(A)$ étant diagonalisable, on a $N'=0$. Comme $\exp(D)$ est inversible, 
		$$N\times\underbrace{\sum_{k=1}^{n-1}\frac{N^{k-1}}{k!}}_{=I_{n}+N''}=0$$
		avec $N''$ nilpotente. $I_{n}+N''$ est donc inversible et ainsi $N=0$ et $A$ est diagonalisable.

		\item D'après ce qui précède, $\exp(A)=I_{n}$ est diagonalisable et 
		$$\Sp_{\C}(\exp(A))=\{e^{\lambda}\bigm|\lambda\in\Sp_{\lambda}(\C)\}=\{I_{n}\}$$
		Donc $\Sp_{\C}(A)\subset 2i\pi\Z$.

		Réciproquement, si $A$ est diagonalisable avec $\Sp(A)\subset 2i\pi\Z$, en diagonalisant, on a bien $\exp(A)=I_{n}$.

		\item Sur $\R$, si $A$ est diagonalisable, $\exp(A)$ l'est aussi. Cependant, la réciproque n'est pas vrai, par exemple
		$$M=\begin{pmatrix}
			2\mathrm{i}\pi & 0\\
			0 & -2\mathrm{i}\pi
		\end{pmatrix}\text{  semblable à }
		\begin{pmatrix}
			0 & -4\pi^{2}\\
			1 & 0
		\end{pmatrix}=A$$
		On a $\chi_{M}=X^{2}+4\pi^{2}$, $\exp(A)=I_{2}$ et $A$ n'est pas diagonalisable sur $\C$.
	\end{enumerate}
\end{proof}

\begin{proof}
	\phantom{}
	\begin{enumerate}
		\item On a $\ln(1-x)=P(x)+x^{2}O(1)$ et $\exp(y)=Q(y)+y^{n}O(1)$ d'où 
		$$\exp(\ln(1+x))=1+x=Q(\ln(1+x))+\underbrace{\ln(1+x)^{n}O(1)}_{O(x^{n})}$$
		alors $1+x=Q(P(x)+O(x^{n}))+O(x^{n})=Q(P(x))+O(x^{n})$. Soit $B(X)=Q(P(X))+O(x^{n})\in\R[X]$, on a $\frac{B(x)}{x^{n}}=O(1)$ donc $X^{n}\mid B$ et $$Q(P(X))=1+X+B(X)=1+X+X^{n}A(X)$$

		\item On a $N^{n}=0$ donc $P(N)$ est aussi nilpotente et on a 
		$$\exp(P(N))=\sum_{k=0}^{n-1}\frac{P(N)^{k}}{k!}=Q(P(N))=I_{n}+N+0$$

		\item Soit $M\in GL_{n}(\C)$ et sa décomposition de Dunford: $M=D+N$ avec $D$ diagonalisable, $N$ nilpotente et $DN=ND$. On a $\Sp(D)=\Sp(M)\subset\C^{*}$ et on écrit
		$$M=D\underbrace{(I_{n}+\underbrace{D^{-1}N}_{\text{nilpotente}})}_{=\exp(P(D^{-1}N))}$$
		si $D=P_{1}\diag(\lambda_{1},\dots,\lambda_{n})P_{1}^{-1}$, pour tout $k\in\{1,\dots,n\}$ il existe $\mu_{k}\in\C$ tel que $\lambda_{k}=\exp(\mu_{k})$ (car $\exp$ est surjectif sur $\C^{*}$). Alors 
		$$
		D=\exp(P_{1}\diag(\mu_{1},\dots,\mu_{n})P_{1}^{-1})\in\C[D]
		$$
		puis 
		\begin{align*}
			M
			&=\exp\Bigl(P_{1}\diag(\mu_{1},\dots,\mu_{n})P_{1}^{-1}\Bigr)\exp\Bigl(P(D^{-1}N)\Bigr)\\
			&=\exp\Bigl(P_{1}\diag(\mu_{1},\dots,\mu_{n})P_{1}^{-1}+P(D^{-1}N)\Bigr)
		\end{align*}
		car les matrices commutent.

		Donc $\exp$ est surjective.
	\end{enumerate}
\end{proof}

\begin{proof}
	On a $A\subset\overline{A}$, $0=\lim\limits_{n\to+\infty}(\frac{2}{n})^{2n}\in\overline{A}$ et $e=\lim\limits_{n\to+\infty}(1+\frac{1}{n})^{n+1}\in\overline{A}$.

	Si $n\geqslant2$ et $p\geqslant2$, $(\frac{1}{n}+\frac{1}{p})^{n+p}\leqslant1$. Donc si $(\frac{1}{n}+\frac{1}{p})^{n+p}\geqslant1$, alors $n=1$ ou $p=1$.

	Si $x>e$, à partir d'un certain rang, on a $(1+\frac{1}{n})^{n+1}\leqslant\frac{e+x}{2}$ et si $x\notin A$, $x\notin\overline{A}$.
	Si $1\leqslant x<e$, à partir d'un certain rang, on a $(1+\frac{1}{n})^{n+1}>x$ donc si $x\notin A$, $x\notin\overline{A}$.

	Soit $x<1$, si $n\geqslant2$ et $p\geqslant3$ ou $n\geqslant3$ et $p\geqslant2$, on a $\frac{1}{n}+\frac{1}{p}\leqslant\frac{5}{6}$ et 
	\begin{align*}
		\Biggl(\frac{1}{n}+\frac{1}{p}\Biggr)^{n+p}
		&=\exp\Biggl((n+p)\ln\Bigl(\frac{1}{n}+\frac{1}{p}\Bigr)\Biggr)\\
		&\leqslant\exp\Biggl((n+p)\ln\Bigl(\frac{5}{6}\Bigr)\Biggr)\\
		&\leqslant\max\Biggl(\underbrace{\Bigl(\frac{5}{6}\Bigr)^{n}}_{\xrightarrow[n\to+\infty]{}0},\underbrace{\Bigl(\frac{5}{6}\Bigr)^{p}}_{\xrightarrow[p\to+\infty]{}0}\Biggr)
	\end{align*}
	Il existe $N_{0}$ tel que pour tout $n\geqslant N_{0}$, $(\frac{5}{6})^{n}\leqslant\frac{x}{2}$. Si $n$ ou $p$ est plus grand que $N_{0}$, on a donc 
	$$\Biggl(\frac{1}{n}+\frac{1}{p}\Biggr)^{n+p}\leqslant\frac{x}{2}$$
	Donc il n'y a qu'un nombre fini d'éléments de $A$ plus grand que $\frac{x}{2}$. Ainsi,
	$$\overline{A}=A\cup\{e,0\}$$
\end{proof}

\begin{proof}
	On note 
	$$\mathbb{V}=\bigcup_{m\geqslant1}\U_{m}=\Biggl\{e^{\frac{2\mathrm{i}k\pi}{m}}\Biggm| m\geqslant1,k\in\{0,\dots,m-1\}\Biggr\}$$
	Soit $M\in H$. $X^{m}-1$ est scindé à racines simples sur $\C$ donc $M$ est diagonalisable sur $\C$ avec ses valeurs propres dans $\mathbb{V}$. Réciproquement, si $M$ est diagonalisable sur $\C$ et $\Sp_{\C}(M)\subset\mathbb{V}$. Alors pour tout $\lambda\in\Sp_{\C}(M),\exists m_{\lambda}\in\N^{*},\lambda\in\U_{m_{\lambda}}$ et soit $m=\ppcm\limits_{\lambda\in\Sp_{\C}(M)}(m_{\lambda})$. Alors $M^{m}=I_{n}$.

	Soit $A\in\overline{H}$, il existe $(M_{p})_{p\in\N}\in H^{\N}$ telle que $\lim\limits_{p\to+\infty}M_{p}=A$. Comme le polynôme caractéristique est une fonction continue des coefficients, pour tout $\lambda\in\Sp_{\C}(A)$, on a 
	$$\lim\limits_{p\to+\infty}\chi_{M_{p}}(\lambda)=\chi_{A}(\lambda)=0$$
	Or 
	$$\vert\chi_{M_{p}}(\lambda)\vert=\vert\lambda-\lambda_{1,p}\vert\dots\vert\lambda-\lambda_{n,p}\vert\geqslant d(\lambda,\U)^{n}$$
	avec $\lambda_{i,p}\in\mathbb{V}$ pour tout $i\in\{1,\dots,n\}$. Donc $d(\lambda,\U)=0$ et comme $\U$ est fermé, $\lambda\in\U$.

	Réciproquement, soit $A\in\M_{n}(\C)$ tel que $\Sp_{\C}(A)\subset\U$. Soit 
	$$\bigl\{e^{\mathrm{i}\theta_1},\dots,e^{\mathrm{i}\theta_r}\bigr\}$$
	les valeurs propres distinctes de $A$ de multiplicités $m_{1},\dots,m_{r}$. Il existe $Q\in GL_{n}(\C)$ tel que 
	$$A=Q\diag(\underbrace{e^{\mathrm{i}\theta_{1}},\dots,e^{\mathrm{i}\theta_1}}_{m_{1}\text{ fois}},\dots,\underbrace{e^{\mathrm{i}\theta_r},\dots,e^{\mathrm{i}\theta_r}}_{m_{r}\text{ fois}})Q^{-1}$$
	On a 
	$$\theta=\lim\limits_{k\to+\infty}\frac{2\pi}{k}\lfloor k\frac{\theta}{2\pi}\rfloor$$
	donc on peut former, pour $p\in\N^{*}$,
	$$A=Q\diag(\underbrace{e^{\mathrm{i}\theta_{1,p}},\dots,e^{\mathrm{i}\theta_{1,p}}}_{m_{1}\text{ fois}},\dots,\underbrace{e^{\mathrm{i}\theta_{r,p}},\dots,e^{\mathrm{i}\theta_{r,p}}}_{m_{r}\text{ fois}})Q^{-1}$$
	avec $\theta_{i,p}=\frac{2\pi}{p}\lfloor p\frac{\theta_{j}}{2\pi}\rfloor+\frac{2 j\pi}{p}$. Pour $p$ suffisamment gand, les $(\theta_{j,p})$ sont deux à deux distincts donc $A_{p}$ est diagonalisable et $A_{p}\in H$, et donc $A\in \overline{H}$.
\end{proof}

\begin{proof}
	\phantom{}
	\begin{enumerate}
		\item On a l'inégalité triangulaire et l'homogénéité. On a cependant $N_{a}(X^{k})=\vert a_{k}\vert$ et pour tout $k\in\N$, $X^{k}\neq0$. Donc $N_{a}$ est une norme implique que $a$ ne s'annule pas sur $\N$. Réciproquement, si pour tout $k\in\N$, $a_{k}\neq0$, si $P\neq0$, il existe $k\in\N$ avec $p_{k}$ et donc $N_{a}(P)>0$. Donc $N_{a}$ est une norme si et seulement si pour tout $k\in\N$, $a_{k}\neq0$.
		
		\item Si $N_{a}$ et $N_{b}$ sont équivalentes, alors il existe $(\alpha,\beta)\in(\R_{+}^{*})^{2}$ tel que pour tout $k\in\N$,
		$$\beta N_{b}(X^{k})\leqslant N_{a}(X^{k})\leqslant\alpha N_{b}(X^{k})$$
		d'où
		$$\beta \vert b_{k}\vert\leqslant N_{a}(X^{k})\leqslant\alpha \vert b_{k}\vert$$
		Donc $a=O(b)$ et $b=O(a)$.

		Réciproquement, si $a=O(b)$ et $b=O(a)$, alors on a l'inégalité précédente sur les $a_{k}$ et $b_{k}$, d'où
		$$\beta\sum_{k=0}^{+\infty}\vert p_{k}b_{k}\vert\leqslant\sum_{k=0}^{+\infty}\vert p_{k}a_{k}\vert\leqslant\alpha\sum_{k=0}^{+\infty}\vert p_{k} b_{k}\vert$$
		et donc pour tout $P\in\C[X]$
		$$\beta N_{b}(P)\leqslant N_{a}(P)\leqslant\alpha N_{b}(P)$$
		et $N_{a}$ et $N_{b}$ sont équivalentes.

		\item $\Delta$ est continue pour $N_{a}$ si et seulement s'il existe $c\geqslant0$ tel que pour tout $P\in\C[X]$, $N_{a}(\Delta P)\leqslant CN_{a}(P)$. Si $\Delta$ est continue alors il existe $c\geqslant0$ tel que $N_{a}(kX^{k})\leqslant cN_{a}(X^{k})$ alors pour tout $k\in\N^{*}$,
		\begin{equation}
			\label{eq:6.1}
			\vert ka_{k-1}\vert\leqslant c\vert a_{k}\vert
		\end{equation}
		Réciproquement, si on a \eqref{eq:6.1}, pour tout $P\in\C[X]=N_{a}(\Delta P)\leqslant cN_{a}(P)$. Pour tout $k\in\N,a_{k}=k!$, \eqref{eq:6.1} est vérifiée pour $c=1$. Si $b_{k}=1$ pour tout $k\in\N$, \eqref{eq:6.1} n'est pas vérifiée donc $\Delta$ n'est pas continue pour $N_{b}$.
	\end{enumerate}
\end{proof}

\begin{proof}
	\phantom{}
	\begin{enumerate}
		\item On a $d(x,A)=0$ si et seulement si $\inf\limits_{a\in A}\Vert x-a\Vert=0$ si et seulement si $\varepsilon>0,\exists a\in A\colon\Vert x-a\Vert<\varepsilon$ si et seulement si $x\in\overline{A}$.

		On a $A\subset\overline{A}$ donc $d(x,\overline{A})\leqslant d(x,A)$. Soit $\varepsilon>0$, il existe $a'\in \overline{A}$ tel que $\Vert x-a'\Vert<d(x,\overline{A})+\varepsilon$ et il existe $a\in A$ tel que $\Vert a-a'\Vert<\varepsilon$. Ainsi, 
		$$d(x,A)\leqslant\Vert x-a\Vert\leqslant d(x,\overline{A})+2\varepsilon$$
		Ceci calant pour tout $\varepsilon>0$, on a $d(x,A)\leqslant d(x,\overline{A})$ et donc on a égalité.

		\item $A\times B\subset\overline{A}\times\overline{B}$ donc $d(A,B)\geqslant d(\overline{A},\overline{B})$. De plus, pour tout $\varepsilon>0$, il existe $(a',b')\in\overline{A}\times\overline{B}$ tel que $\Vert a'-b'\Vert<d(\overline{A},\overline{B})+\varepsilon$ et il existe $(a,b)\in A\times B$ tel que $\Vert a-a'\Vert<\varepsilon$ et $\Vert b-b'\Vert\varepsilon$. En utilisant l'inégalité triangulaire, on a donc 
		$$d(A,B)\leqslant\Vert a-b\Vert<d(\overline{A},\overline{B})+3\varepsilon$$
		Ceci valant pour tout $\varepsilon>0$, on a bien l'égalité.
	\end{enumerate}
\end{proof}

\begin{proof}
	$\varphi_{x_{0}}$ est une forme linéaire. Elle est continue si et seulement $C>0$ tel que pour tout $P\in\C[X]$,
	$$\vert P(x_{0})\vert\leqslant C\Vert P\Vert_{\infty}$$
	Si $P=\sum_{k=0}^{n}a_{k}X^{k}$, on a 
	$$\vert P(x_{0})\vert\leqslant \Vert P\Vert_{\infty}\sum_{k=0}^{n}\vert x_{0}\vert^{k}$$
	Si $\vert x_{0}\vert<1$, on a 
	$$\vert P(x_{0})\vert\leqslant \Vert P\Vert_{\infty}\frac{1}{1-\vert x_{0}\vert}$$
	donc $\varphi_{x_{0}}$ est continue et si $x_{0}=\vert x_{0}\vert e^{\mathrm{i}\theta_{0}}$, soit $n\in\N$ et $P_{n}=\sum_{k=0}^{n}e^{-\mathrm{i}k\theta_{0}}X^{k}$, on a $\Vert P_{n}\Vert_{\infty}=1$ et 
	$$\vert \varphi_{x_{0}}(P_{n})=\sum_{k=0}^{n}\vert x_{0}\vert^{k}\xrightarrow[n\to+\infty]{}\frac{1}{1-\vert x_{0}\vert}$$
	donc $\vertiii{\varphi_{x_{0}}}=\frac{1}{1-\vert x_{0}\vert}$.

	Si $\vert x_{0}\vert\geqslant1$, 
	$$\vert\varphi_{x_0}(P_{n})\vert=\sum_{k=0}^{n}\vert x_{0}\vert^{k}\xrightarrow[n\to+\infty]{}+\infty$$
	donc $\varphi_{x_{0}}$ n'est pas continue.
\end{proof}

\begin{proof}
	Pour le sens indirect, soit $\lambda\in\Sp_{\C}(M)$. Pour tout $p\in\N$, $\lambda\in\Sp_{\C}(M_{p})$ donc $\det(M_{p}-\lambda I_{n})=0$. Par continuité du déterminant, on a $0=\det(M_{p}-\lambda I_{n})\xrightarrow[p\to+\infty]{}\det(-\lambda I_{n})$. Donc $\lambda=0$ et $\Sp_{\C}(M)=\{0\}$ donc $M$ est nilpotente.

	Pour le sens direct, soit $u\in\L(\C^{n})$ canoniquement associée à $M$. On trigonalise $u$ sur une base $\mathcal{B}=(\varepsilon_{1},\dots,\varepsilon_{n})$ avec $u(\varepsilon_{1})=0,u(\varepsilon_{2})=a_{1,2}\varepsilon_{1},\dots,u(\varepsilon_{n})=a_{1,n}\varepsilon_{1}+\dots+a_{n-1,n}\varepsilon_{n-1}$. Posons pour $i\in\{1,\dots,n\}$, $\varepsilon_{i,p}=\frac{\varepsilon_{i}}{p^{i-1}}$. On pose $\mathcal{B}_{p}=(\varepsilon_{1,p},\dots,\varepsilon_{n,p})$ et $M_{p}=\mat\limits_{B_{p}}(u)$, semblable à $M$ et $M_{p}\xrightarrow[p\to+\infty]{}0$ car $\Vert M_{p}\Vert\leqslant\frac{1}{p}\Vert M_{1}\Vert$.
\end{proof}

\begin{proof}
	On pose $u\in\L(\C^{n})$ canoniquement associée à $M$. 

	Pour le sens indirect, si $M$ n'est pas diagonalisable, il existe une base $B=(\varepsilon_{1},\dots,\varepsilon_{n})$ de $\C^{n}$ telle que 
	$$\mat\limits_{\mathcal{B}}(u)=D+N$$
	où $D$ est diagonale et $N$ est nilpotente (décomposition de Dunford). En reprenant les bases $\mathcal{B}_{p}$ définies à l'exercice précédent, on a
	$$\mat\limits_{\mathcal{B}_{p}}(u)=D+N_{p}\xrightarrow[p\to+\infty]{}D$$
	Si $D\in S_{M}$, alors $M$ est diagonalisable ce qui est exclu par hypothèse. Donc $S_{M}$ n'est pas fermé.

	Pour le sens direct, si $M$ est diagonalisable, soit $(M_{p})_{p\in\N}\in(S_{M})^{\N}$ avec $M_{p}\xrightarrow[p\to+\infty]{}M'$. Soit $\lambda\in\C$. On a $\chi_{M_{p}}(\lambda)=\det(\lambda I_{n}-M_{p})=\chi_{M}(\lambda)$ car $M$ et $M_{p}$ sont semblables. Par continuité du déterminant, on a $\chi_{M'}(\lambda)=\chi_{M}(\lambda)$, donc $\chi_{M'}=\chi_{M}$. De plus, $A\mapsto\Pi_{M}(A)$ (polynôme minimal) est continue sur $\M_{n}(\C)$ et pour tout $p\in\N$, on a $\Pi_{M}(M_{p})=0$ donc $\Pi_{M}(M')=0$. $M'$ est donc annulée par $\Pi_{M}$, donc $M'$ est diagonalisable et comme $\chi_{M}=\chi_{M'}$, $M$ et $M'$ ont les mêmes valeurs propres avec les mêmes multiplicités. Donc $M'\in S_{M}$.
\end{proof}

\begin{remark}
	Le polynôme caractéristique est une fonction continue de la matrice, mais c'est faux pour le polynôme minimal, par exemple pour 
	$$M_{p}=\begin{pmatrix}
		\frac{1}{p} &0\\
		0 & \frac{2}{p}
	\end{pmatrix}$$
	On a $M_{p}\xrightarrow[p\to+\infty]{}0$ et $\Pi_{M_{p}}=(X-\frac{1}{p})(X-\frac{2}{p})\xrightarrow[p\to+\infty]{} X^{2}\neq X=\Pi_{M_{\infty}}$ donc $\lim\limits_{p\to+\infty}\Pi_{M_p}\neq\Pi_{\lim\limits_{p\to+\infty}M_{p}}$.
\end{remark}

\begin{proof}
	On note $A_{h}=\{\vert\varphi(x)-\varphi(y)\vert\bigm|(x,y)\in I^{2}\text{ et }\vert x-y\vert\leqslant h\}$.
	\begin{enumerate}
		\item $\omega_{\varphi}$ est bien défini car $\vert\varphi(x)-\varphi(y)\vert\leqslant 2\Vert\varphi\Vert_{\infty}$). Si $0<h\leqslant h'$, alors $A_{h}\subset A_{h'}$ donc $\sup(A_{h})\leqslant\sup(A_{h'})$ donc $\omega_{\varphi}(h)\leqslant\omega_{\varphi}(h')$.
		\item Soit $(h,h')\in(\R_{+}^{*})^{2}$, soit $(x,y)\in I^{2}$ tel que $\vert x-y\vert\leqslant h+h'$ (où on peut supposer que $x\leqslant y$).
		\begin{itemize}
			\item Si $y\in[x,x+h]$, alors $\vert x-y\vert\leqslant h$ donc $\vert\varphi(x)-\varphi(y)\vert\leqslant\omega_{\varphi}(h)\leqslant\omega_\varphi(h)+\omega_{\varphi}(h')$
			\item Si $y\in[x+h,x+h+h']$, $\vert\varphi(x)-\varphi(y)\vert\leqslant\vert\varphi(x)-\varphi(x+h)\vert+\vert\varphi(x+h)-\varphi(y)\vert\leqslant\omega_\varphi(h)+\omega_{\varphi}(h')$ car $\vert x-(x+h)\vert\leqslant h$ et $\vert x+h-y\vert\leqslant h'$.
		\end{itemize}
		Donc $\omega_{\varphi}(h+h')\leqslant\omega_\varphi(h)+\omega_\varphi(h')$.
		\item Par récurrence sur $n\in\N$, on a $\omega_\varphi(nh)=n\omega_\varphi(h)$. Si $\lambda\in\R_{+}^{*}$, on a $\lambda h\leqslant(\lfloor \lambda\rfloor+1)h$ et par croissance et ce qui précède, on a 
		$$\omega_\varphi(\lambda h)\leqslant(\lfloor\lambda\rfloor+1)\omega_\varphi(h)\leqslant(\lambda+1)\omega_\varphi(h)$$
		\item Soit $\varepsilon>0$. $\varphi$ étant uniformément continue, il existe $\alpha>0$ tel que pour tout $(x,y)\in I^{2}$, si $\vert x-y\vert\alpha$ on a $\vert\varphi(x)-\varphi(y)\vert\leqslant\varepsilon$ et on a pour $h\leqslant\alpha$, $\omega_\varphi(h)\leqslant\varepsilon$ d'où $\lim\limits_{h\to0}\omega_\varphi(h)=0$.
		
		Soit alors $h_{0}>0$ fixé et $h>0$,
		\begin{itemize}
			\item si $h_{0}\leqslant h$, on a $0\leqslant\omega_\varphi(h)-\omega_\varphi(h_0)\leqslant\omega_\varphi(h-h_0)$.
			\item si $h\leqslant h_{0}$, on a $0\leqslant\omega_\varphi(h_0)-\omega_\varphi(h)\leqslant\omega_\varphi(h_0-h)$.
		\end{itemize}
		Dans tous les cas, on a $\vert\omega_\varphi(h)-\omega_\varphi(h_{0})\vert\leqslant\omega_\varphi(\vert h_{0}-h\vert)$. Donc on a bien $\lim\limits_{h\to h_{0}}\omega_\varphi(h)=\omega_\varphi(h_{0})$. Donc $\omega_{\varphi}$ est continue (et même uniformément).
	\end{enumerate}
\end{proof}

\begin{proof}
	$G$ est borné car si $M\in G$, $\vertiii{M}\leqslant \vertiii{I_{n}}+\mu=1+\mu$. Montrons donc que si $G_{0}$ est un sous-groupe borné de $GL_{n}(\C)$, alors les valeurs propres de ses éléments sont de module 1, et ceux-ci sont diagonalisables.

	En effet, soit $M\in G$ et $\lambda\in\Sp(M)$, soit $X$ un vecteur propre associé. On a 
	$\Vert MX\Vert=\vert\lambda\vert\Vert X\Vert\leqslant\vertiii{M}\Vert X\Vert$ donc $\vert\lambda\vert\leqslant\vertiii{M}\leqslant\sup\limits_{M\in G}\vertiii{M}$. Pour tout $k\in\Z$, $M^{k}\in G$ et $\lambda^{k}\in\Sp(M^{k})$, donc si $\vert\lambda\vert>1$, on a $\lim\limits_{k\to+\infty}\vert\lambda\vert^{k}=+\infty$, et si $\vert\lambda\vert^{\lambda}<1$, on a $\lim\limits_{k\to-\infty}\vert\lambda\vert^{k}=+\infty$. Comme 
	G est borné, $\vert\lambda\vert=1$.

	On utilise ensuite la décomposition de Dunford pour $M$: $M=D+N$ avec $DN=ND$, $D$ diagonalisable et $N$ nilpotente. Grâce au binôme de Newton, pour $k\geqslant r$ p* $r$ est l'indice de nilpotence de $N$, on a
	$$M^{k}=\sum_{p=0}^{k}\binom{k}{p}N^{p}D^{k-p}=\underbrace{D^{k}}_{\text{borné}}+kND+\sum_{p=2}^{r-1}\underbrace{\binom{k}{p}}_{\underset{k\to+\infty}{\sim}\frac{k^{p}}{p!}}N^{p}\underbrace{D^{k-p}}_{\text{borné car }\Sp(D)\subset\U}$$
	Donc
	$$M^{k}\underset{k\to+\infty}{\sim}\underbrace{\frac{k^{r-1}}{(r-1)!}\underbrace{N^{r-1}}_{\neq0}D^{k-r+1}}_{\text{non borné si }N\neq0}$$
	Donc $N=0$ et $M=D$ est diagonalisable.

	Revenons donc à l'exercice. Soit $M\in G$ et $\lambda=e^{\mathrm{i}\theta}\in\Sp(M)$ avec $\theta\in]-\pi,pi]$. Si $X$ est un vecteur propre associé à $\lambda$, on a 
	$$(\lambda-1)\Vert X\Vert=\Vert(M-I_{n})X\Vert\leqslant\mu\Vert X\Vert$$
	donc $\vert\lambda-1\vert=2\vert\underbrace{\sin(\frac{\theta}{2})}_{\geqslant0}\vert\leqslant\mu$.
	Donc $\theta\in[-\theta_{0},\theta_{0}]$ où $\theta_{0}=\arcsin(\frac{\mu}{2})\in[0,\pi[$.

	Si $\frac{\theta}{\pi}\notin\Q$, $e^{\mathrm{i}k\pi}\in\Sp(M^{k})$, $\vert e^{\mathrm{i}k\theta}-1\vert\leqslant\mu$. Alors $\{k\theta+2l\pi\bigm| (k,l)\in\Z^{2}\}$ est un sous-groupe de $(\R,+)$ non monogène et donc dense, et alors $(e^{\mathrm{i}k\theta})_{k\in\Z}$ est dense dans $\U$, donc il existe $k_{0}\in\Z$ tel que $\vert e^{\mathrm{i}k_{0}\theta}+1\vert=\vert 2-(1-e^{\mathrm{i}k_{0}\theta_{0}})\vert<2-\mu$, ce qui est impossible car $\vert 2-(1-e^{\mathrm{i}k_{0\theta}})\vert\geqslant2-\vert 1-e^{\mathrm{i}k_{0}\theta_{0}}\vert\geqslant2-\mu$.

	Ainsi, $\frac{\theta}{\pi}\in\Q$ et il existe $m\in\N^{*}$ tel que $\lambda=e^{\mathrm{i}\theta}\in\U_{m}$. Ce n'est pas forcément le même $m$ pour tout les M dans G. Notons alors pour 
	$$\lambda\in\bigcup_{M\in G}\Sp(M)=\mathcal{A}$$
	$\omega(\lambda)$ l'ordre (multiplicatif) de $\lambda$ dans $\U$.

	Si $\omega(\lambda)=m$, on a $gr(\lambda)=\U_{m}$ donc il existe $k\in\Z$ tel que $\lambda^{k}=e^{\frac{2\mathrm{i}\pi}{m}}\in\mathcal{A}$ (car $\lambda^{k}\in\Sp(M^{k})$). Supposons que $\{\omega(\lambda)\bigm| \lambda\in\mathcal{A}\}$ non borné. Alors il existe $(m_{k})_{k\in\N}$ tel que $m_{k}\xrightarrow[k\to+\infty]{}+\infty$ et $e^{\frac{2\mathrm{i}\pi}{m_{k}}}\in\mathcal{A}$. Alors 
	$$\underbrace{e^{2\mathrm{i}\lfloor\frac{m_{k}}{2}\rfloor \frac{\pi}{m_{k}}}}_{\xrightarrow[k\to+\infty]{} e^{i\pi}=-1}\in\mathcal{A}$$
	ce qui est impossible car $\vert\lambda+1\vert\geqslant2-\mu>0$. On peut donc noter
	$$m=\underset{\lambda\in\mathcal{A}}{\vee}\omega(\lambda)$$
	et pour tout $M\in G$, pour tout $\lambda\in\Sp(M)$, $\lambda^{m}=1$. Or $M$ est diagonalisable, donc $M^{m}=I_{n}$.
\end{proof}

\begin{proof}
	Si $M\in\mathcal{G}_{q}$, $P(X)=X^{q}-1$ annule $M$ donc $M$ est diagonalisable à valeurs propres dans $\U_{q}$. Réciproquement, si $M$ est diagonalisable et $\Sp_{\C}(M)\subset\U_{q}$ alors il existe $P\in GL_{n}(\C)$ avec 
	$$M=P\diag(\lambda_{1},\dots,\lambda_{n})P^{-1}$$
	et donc 
	$$M^{q}=P\diag(\lambda_{1}^{q},\dots,\lambda_{n}^{q})P^{-1}=I_{n}$$

	Si $M\in\mathcal{G}_{q}$ n'est pas une homothétie, il existe $\lambda\neq\mu\in\Sp_{\C}(M)^{2}$ et $P\in GL_{n}(\C)$ tel que 
	$$M=P\begin{pmatrix}
		\lambda & &\\
		& \mu & &\\
		& & \ddots
	\end{pmatrix}P^{-1}$$
	Soit $k\in\N^{*}$ tel que 
	$$M=P\begin{pmatrix}
		\lambda & \frac{1}{k}&\\
		& \mu & &\\
		& & \ddots
	\end{pmatrix}P^{-1}\xrightarrow[k\to+\infty]{}M$$
	Or 
	$$\begin{pmatrix}
		\lambda & \frac{1}{k}\\
		0 & \lambda
	\end{pmatrix}\text{  est semblable }\begin{pmatrix}
		\lambda & 0\\
		0 & \mu
	\end{pmatrix}$$
	car $\chi_{A}=(X-\lambda)(X-\mu)$ donc est diagonalisable. Donc $M_{k}\sim M$ et $M_{k}\in\mathcal{G}_{q}$ et $M$ n'est pas isolé.

	Montrons le petit lemme suivante: soit $\Vert\cdot\Vert$ une norme sur $\C^{n}$ et $\vertiii{\cdot}$ la norme subordonnée, soit $\lambda\in\C$ et $M\in\M_{n}(\C)$ et $\varepsilon>0$. Si $\vertiii{M-\lambda I_{n}}\leqslant\varepsilon$ alors $\Sp_{\C}(M)\subset\overline{B(\lambda,\varepsilon)}$. En effet, soit $X$ un vecteur propre de $M$ associé à $\mu\in\Sp_{\C}(M)$. On a 
	$$\Vert (M-\lambda I_{n})X\Vert=\vert\mu-\lambda \vert \Vert X\Vert\leqslant\vertiii{M-\lambda I_{n}}\Vert X\Vert\leqslant\varepsilon\Vert X\Vert$$
	donc $\vert\mu-\lambda\vert\leqslant\varepsilon$.

	Pour $\varepsilon=\sin(\frac{\pi}{q})>0$ et $\lambda\in\U_{q}$; si $M\in B_{\vertiii{\cdot}}(\lambda I_{n},\varepsilon)\cap \mathcal{G}_{q}$ alors pour tout $\mu\in\Sp_{\C}(M)$, on a $\vert\lambda-\mu\vert\leqslant\sin(\frac{\pi}{q})$ donc $\lambda=\mu$. Donc si $M=\lambda I_{n}$
	 alors $M$ est isolé (avec $\lambda\in\U_{q}$). Donc les matrices scalaires sont isolées.
\end{proof}
\section{Fonction d'une variable réelle}

\begin{proof}
	Tout d'abord, $\deg(L_{n})=n$ et  son coefficient dominant et $\frac{(2n)!}{2^{n}(n!)^{2}}$.
	\begin{enumerate}
		\item Soit $f\in\mathcal{C}^{0}([0,1],\R)$. $-1$ et $1$ sont racines d'ordre $n$ de $P_{n}$ donc pour tout $k\in\{0,\dots,n-1\}$ $P_{n}^{(k)}(-1)=P_{n}^{(k)}(-1)=0$. Ainsi, on a par intégrations par parties successives:
		\begin{equation}(f|L_{n})=(-1)^{n}\int_{-1}^{1}f^{(n)}(t)P_{n}(t)dt\end{equation}
		Notamment, si $P\in\R_{n-1}[X]$, $P^{(n)}=0$ et $(P|L_{n})=0$. En particulier, pour tout $m<n$, $\deg(L_{m})\leqslant n-1$ et $(L_{m}|L_{n})=0$ donc $(L_{n})_{n\in\N}$ est orthogonale. Notons dès maintenant que l'on peut calculer la norme de $L_{n}$ grâce aux intégrales de Wallis:
		
		\begin{align}
			\Vert L_{n}\Vert_{2}^{2}
			&=(L_{n}|L_{n})\\
			&=(-1)^{n}\int_{-1}^{1}L_{n}^{(n)}(t^{2}-1)^{n}dt\\
			&=\frac{(2n)!}{2^{2n}(n!)^{2}}\int_{-1}^{1}(1-t^{2})^{n}dt
		\end{align}
		On pose $t=\cos(\theta)$ d'où $dt=-\sin(\theta)d\theta$, d'où
		\begin{align}
			\int_{-1}^{1}(1-t^{2})^{n}dt
			&=\int_{0}^{\pi}\sin(\theta)^{2n+1}d\theta\\
			&=2I_{2n+1}\text{ [Wallis] }
		\end{align}
		On a classiquement $I_{n+2}=\frac{n+1}{n+2}I_{n}$.
		D'où
		\begin{align}
			I_{2n+1}
			&=\frac{2n}{2n+1}\times\frac{2n-2}{2n-1}\times\dots\times\frac{2}{3}\times \underbrace{I_{1}}{=1}\\
			&=\frac{2^{2n}(n!)^{2}}{(2n+1)!}
		\end{align}
		d'où
		\begin{equation}\Vert L_{n}\Vert_{2}^{2}=\frac{(2n)!}{2^{2n}(n!)^{2}}\times 2\times \frac{2^{2n}(n!)^{2}}{(2n+1)!}=\frac{2}{2n+1}\end{equation}

		\item On utilise la formule de Leibniz en écrivant $X^{2}-1=(X+1)(X-1)$.
		\item On montre le résultat par récurrence sur $k\in\{0,\dots,n\}$ en invoquant le théorème de Rolle. On trouve donc que $L_{n}=P_{n}^{(n)}$ s'annule au moins $n$ fois sur $]-1,1[$. Or $\deg(L_{n})=n$, donc ces zéros sont simples et ce sont les seuls.
		\item $(L_{0},\dots,L_{n})$ est une base de $\R_{n}[X]$ (étagée en degré). Donc il existe $(\alpha_{n,0},\dots,\alpha_{n,k})\in\R^{k+1}$ tel que $XL_{n-1}=\sum_{k=0}^{n}\alpha_{n,k}L_{k}$. Si $k\leqslant n-3$, on a
		\begin{equation}(XL_{n-1} L_{k})=\alpha_{n,k}\Vert L_{k}\Vert_{2}^{2}=(L_{n-1}XL_{k})=0\end{equation}
		car $\deg(XL_{k})=k+1\leqslant n-2$. Donc 
		\begin{equation}XL_{n-1}=\alpha_{n,n-2}L_{n-2}+\alpha_{n,n-1}L_{n-1}+\alpha_{n,n}L_{n}\end{equation}
		Pour calculer les coefficients, on fait tout simplement les produits scalaires:
		\begin{equation}(Xl_{n-1}|L_{n-1})=\int_{-1}^{1}tL_{n-1}(t)^{2}dt\end{equation}
		Or $P_{n}$ est paire, donc $L_{n}$ est de la parité de $n$ et donc $L_{n}^{2}$ est paire puis $XL_{n}^{2}$ est impaire. Donc $\alpha_{n,n-1}=0$.

		\begin{align}
			(XL_{n-1}|L_{n-2})
			&=\alpha_{n,n-2}\underbrace{\Vert L_{n-2}\Vert_{2}^{2}}_{=\frac{2}{2n-3}}\\
			&=(-1)^{n}\int_{-1}^{1}P_{n-1}(t)\underbrace{(XL_{n-2})^{(n-1)}(t)}_{\frac{(2n-4)!(n-1)}{2^{n-2}(n-2)!}}
		\end{align}
		Par ailleurs,
		\begin{align}
			(-1)^{n-1}\int_{-1}^{1}P_{n-1}(t)dt
			&=\frac{1}{2^{n-1}(n-1)!}\underbrace{\int_{-1}^{1}(1-t^{2})^{n-1}dt}_{2I_{2n-1}}\\
			&=\frac{1}{2^{n-1}(n-1)!}\times 2\times\frac{2^{2n-2}(n-1)!)^{2}}{(2n-1)!}\\
			&=\frac{2^{n}(n-1)!}{(2n-1)!}
		\end{align}
		donc $\frac{\alpha_{n,n-2}}{\alpha_{n,n}}=\frac{n-1}{n}$. D'où le résultat.
	\end{enumerate}
\end{proof}

\begin{proof}
	On forme \function{g}{[a,b]}{\R}{x}{\underbrace{\Delta f(x_{0},\dots,x_{n-1},x)}_{\varphi(x)}-\underbrace{\prod_{i=0}^{n-1}(x-x_{i})A}_{P(x)}}
	On a $g(x_{n})=0$. On suppose les $(x_{i})_{1\leqslant i\leqslant n}$ distincts, et on pose 
	\begin{equation}A=\frac{V(x_{0},\dots,x_{n})}{\prod_{i=0}^{n-1}(x_{n}-x_{i})}\end{equation}
	$g$ est de classe $\mathcal{C}^{n}$ et pour tout $i\in\{0,\dots,n\}$, on a $g(x_{i})=0$.
	Donc il existe $\xi\in]a,b[$ tel que $g^{(n)}(\xi)=0$ (théorème de Rolle appliqué $n$ fois. $\deg(P)=n$ et son coefficient dominant est $A$ donc $P^{(n)}(\xi)=An!=\varphi^{(n)}(\xi)$.

	On développe maintenant $\varphi(x)$ par rapport à la dernière colonne:
	\begin{equation}\varphi(x)=f(x)\times V_{n}(x_{0},\dots,x_{n-1})+Q(X)\end{equation}
	avec $\deg(Q)\leqslant n-1$ et $V_{n}(x_{0},\dots,x_{n-1})=\prod_{0\leqslant j<i\leqslant n-1}(x_{i}-x_{j})$ (déterminant de Vandermonde). On a donc 
	\begin{equation}\varphi^{(n)}(x)=f^{(n)}(x)\prod_{0\leqslant j<i\leqslant n-1}(x_{j}-x_{i})\end{equation}
	et en reportant, on a 
	\begin{equation}\frac{f^{(n)}(\xi)}{n!}=\frac{A}{\prod_{0\leqslant i<j\leqslant n-1}(x_{j}-x_{i})}=\Delta f(x_{0},\dots,x_{n})\end{equation}
\end{proof}

\begin{proof}
	On utilise le développement de Taylor avec reste intégral.
	\begin{equation}f(0)=f\Bigl(\frac{1}{2}\Bigr)-\frac{1}{2}f'\Bigl(\frac{1}{2}\Bigr)+\int_{\frac{1}{2}}^{0}-tf''(t)dt\end{equation}
	et de même
	\begin{equation}f(1)=f\Bigl(\frac{1}{2}\Bigr)-\frac{1}{2}f'\Bigl(\frac{1}{2}\Bigr)+\int_{\frac{1}{2}}^{1}(1-t)f''(t)dt\end{equation}
	D'où
	\begin{align}
		A(f)
		&=f(0)-f\Bigl(\frac{1}{2}\Bigr)+f(1)-f\Bigl(\frac{1}{2}\Bigr)\\
		&=\int_{0}^{\frac{1}{2}}tf''(t)dt+\int_{\frac{1}{2}}^{1}(1-t)f''(t)dt\\
		&\leqslant\int_{0}^{\frac{1}{2}}tdt+\int_{\frac{1}{2}}^{1}(1-t)dt\\
		&=\frac{1}{4}
	\end{align}
	Et c'est atteint pour $f(t)=\frac{t^{2}}{4}$.
\end{proof}

\begin{proof}
	Pour tout $(x,h)\in\R^{2}$, $f(x+h)-f(x-h)=2hf'(x)$ donc 
	\begin{equation}
		\label{eq:7.1}
		f'(x)=\frac{1}{2}(f(x+1)-f(x-1))
	\end{equation}
	donc $f'$ est $\mathcal{C}^{1}$ et donc $f$ est $\mathcal{C}^{2}$. On fixe alors $x$ et on dérive deux fois~\eqref{eq:7.1} en fonction de $h$. On a alors
	\begin{equation}f''(x+h)=f''(x-h)\end{equation}
	pour tout $(x,h)\in\R^{2}$ donc $f''$ est constante et $f$ est polynômiale de degré 2.

	Réciproquement, si $f(x)=ax^{2}+bx+c$, on a bien la relation de l'énoncé.
\end{proof}

\begin{proof}
	\phantom{}
	\begin{enumerate}
		\item Soit $a>0$, \function{\tau_{a}}{\R}{]a,+\infty[}{x}{\frac{f(x)-f(a)}{x-a}}
		est croissante. Donc il existe $l=\lim\limits_{x\to+\infty}\tau_{a}(x)\in\overline{\R}$. On écrit alors 
		\begin{equation}\frac{f(x)}{x}=\frac{f(x)-f(a)}{x-a}\times \frac{x-a}{x}+\frac{f(a)}{x}\xrightarrow[x\to+\infty]{}l\end{equation}

		\item S'il existe $a<b\in(\R_{+}^{*})^{2}$ tel que $f(a)<f(b)$, alors $\tau_{a}(b)>0$. Comme $\tau_{a}$ est croissante, $l\geqslant\tau_{a}(b)>0$. Par contraposée, si $l\geqslant0$, $f$ est décroissante.
		\item Posons pour tout $x\in\R_{+}^{*}$, $\varphi(x)=f(x)-lx$. Pour $x<y$, on a 
		\begin{equation}\frac{\varphi(y)-\varphi(x)}{y-x}=\frac{f(y)-f(x)}{y-x}-l\leqslant0\end{equation}
		Donc $\varphi$ est décroissante et $\lim\limits_{x\to+\infty}\varphi(x)\in\overline{\R}$ existe.
	\end{enumerate}
\end{proof}

\begin{proof}
	\phantom{}
	\begin{enumerate}
		\item On forme \function{g}{[0,1]}{\R}{x}{\frac{1}{\frac{1}{p}+x}}
		Alors 
		\begin{equation}\sum_{k=0}^{np}\frac{1}{n+k}=\frac{1}{np}\sum_{k=0}^{np}\frac{1}{\frac{1}{p}+\frac{k}{np}}\xrightarrow[n\to+\infty]{}\int_{0}^{1}\frac{dx}{\frac{1}{p}+x}=\ln(p+1)=l_{p}\end{equation}

		\item On note $f(x)=f(0)+xf'(0)+x\varepsilon(x)$ avec $\varepsilon(x)\xrightarrow[\varepsilon\to0]{}0$. 
		
		Soit $\varepsilon_{0}>0$. Il existe $\alpha_{0}>0$ tel que si $0<x<\alpha_{0}$, alors $\vert\varepsilon(x_{0})\vert\leqslant\varepsilon_{0}$, et il existe $N_{0}\in\N$ tel que pour tout $n\geqslant N_{0}$, $\frac{1}{n}\leqslant\alpha_{0}$. Alors pour tout $n\geqslant N_{0}$, pour tout $k\in\{0,\dots,np\}$, 
		\begin{equation}\frac{1}{k+n}\Rightarrow \Biggl\vert\varepsilon\Bigl(\frac{1}{k+n}\Bigr)\Biggr\vert\leqslant\frac{\varepsilon_{0}}{p}\end{equation}
		et
		\begin{equation}\Biggl\vert\sum_{k=0}^{np}\frac{\varepsilon(\frac{1}{k+n})}{k+n}\Biggr\vert\leqslant\sum_{k=0}^{np}\frac{\frac{\varepsilon_{0}}{p}}{k+n}\leqslant\frac{\varepsilon_{0}}{p}\frac{np+1}{n+1}\leqslant\varepsilon_{0}\end{equation}

		On a donc
		\begin{equation}v_{n}=\sum_{k=0}^{np}\frac{1}{n+k}f'(0)+\sum_{k=0}^{np}\frac{\varepsilon(\frac{1}{n+k})}{n+k}\xrightarrow[n\to+\infty]{}\ln(p+1)f'(0)\end{equation}

		\item On peut penser à $f\colon x\mapsto\sqrt{x}$ continue et $f(0)=0$. De plus,
		\begin{equation}\sum_{k=0}^{np}\frac{1}{\sqrt{n+k}}\geqslant\frac{np+1}{\sqrt{n(p+1)}}\xrightarrow[n\to+\infty]{}+\infty\end{equation}
		donc $v_{n}$ diverge.

		\item On écrit $f(x)=f(0)+xf'(0)+\frac{x^{2}}{2!}f''(0)+x^{2}\varepsilon(x)$ avec $\varepsilon(x)\xrightarrow[\varepsilon\to+\infty]{}0$. Ainsi, 
		\begin{equation}v_{n}=\sum_{k=0}^{np}\frac{f''(0)}{2(n+k)^{2}}+\sum_{k=0}^{bp}\frac{\varepsilon(\frac{1}{k+n})}{(k+n)^{2}}\end{equation}
		Soit $\varepsilon>0$, il existe $N\in\N$ tel que pour tout $n\geqslant N$, pour tout $k\in\{0,\dots,np\}$, $\vert\varepsilon(\frac{1}{n+k})\vert\leqslant\varepsilon$ et donc 
		\begin{equation}\Biggl\vert\sum_{k=0}^{np}\frac{\varepsilon(\frac{1}{n+k})}{(n+k)^{2}}\Biggr\vert\leqslant\sum_{k=0}^{np}\frac{\varepsilon}{(n+k)^{2}}\end{equation}
		donc 
		\begin{equation}\sum_{k=0}^{np}\frac{\varepsilon(\frac{1}{n+k})}{(n+k)^{2}}=O\Biggl(\sum_{k=0}^{np}\frac{f''(0)}{2}\times\frac{1}{(n+k)^{2}}\Biggr)\end{equation}
		puis
		\begin{equation}v_{n}\underset{n\to+\infty}{\sim}\sum_{k=0}^{np}\frac{f''(0)}{2(n+k)^{2}}\end{equation}
		Or 
		\begin{align}
			\sum_{k=0}^{np}\frac{1}{(n+k)^{2}}
			&=\frac{1}{(np)^{2}}\sum_{k=0}^{np}\frac{1}{(\frac{1}{p}+\frac{k}{np})^{2}}\\
			&=\frac{1}{np}\times \underbrace{\frac{1}{np}\sum_{k=0}^{np}\frac{1}{(\frac{1}{p}+\frac{k}{np})^{2}}}_{\xrightarrow[n\to+\infty]{}\int_{0}^{1}\frac{dx}{(\frac{1}{p}+x)^{2}}}
		\end{align}
		donc 
		\begin{equation}v_{n}\underset{n\to+\infty}{\sim}\frac{f''(0)p}{n(p+1)}\end{equation}
	\end{enumerate}
\end{proof}

\begin{proof}
	Supposons que $f'$ ne tend pa vers 0 en $+\infty$: il existe $\varepsilon_{0}>0,\forall A>0,\exists x_{A}\geqslant A,\vert f'(x_{A})\vert\geqslant\varepsilon_{0}>0$. Par continuité uniforme, il existe $\alpha_{0}\geqslant0$, $\forall(x,y)\in(\R_{+})^{2}$, si $\vert x-y\vert\leqslant\alpha_{0}$ alors $\vert f'(x)-f'(y)\vert\leqslant\frac{\varepsilon_{0}}{2}$. Alors pour tout $t\in[x_{A}-\alpha,x_{A}+\alpha]$, on a 
	\begin{equation}\vert f'(t)\vert\geqslant \vert f'(x_{A})\vert-\vert f'(x_{A})-f'(t)\vert\geqslant\varepsilon_{0}-\frac{\varepsilon_{0}}{2}\geqslant\frac{\varepsilon_{0}}{2}\end{equation}
	et pour $A=n$, pour tout $n\in\N,\exists x_{n}\geqslant n,\forall t\in[x_{n}-\alpha,x_{n}+\alpha],\vert f'(t)\vert\geqslant\frac{\varepsilon_{0}}{n}$. D'après le théorème des valeurs intermédiaires, $f'$ est de signe constant sur $[x_{n}-\alpha,x_{n}+\alpha]$. Quitte à changer $f$ en $-f$, on peut supposer qu'il existe une infinité de $n\in\N$ tels que $f'>0$ sur les $[x_{n}-\alpha,x_{n}+\alpha]$. Alors
	\begin{equation}f(x_{n}+\alpha_{0})-f(x_{n}-\alpha_{0})=\int_{x_{n}-\alpha_{0}}^{x_{n}+\alpha_{0}}f'(t)dt\geqslant\varepsilon_{0}\alpha_{0}>0\end{equation}
	mais comme $\lim\limits_{x\to+\infty}f(x)\in\R$, on a 
	\begin{equation}\lim\limits_{n\to+\infty}f(x_{n}+\alpha_{0})-f(x_{n}-\alpha_{0})=0\end{equation}
	d'où la contradiction.

	Si $f\in\mathcal{C}^{1}(\R_{+},\C)$, on applique ce qui précède à $\Im(f)$ et $\Re(f)$. 

	Si $f'$ n'est pas uniformément continue, ce n'est plus valable, par exemple 
	\begin{equation}f(x)=\frac{\sin(x^{2})}{x}\xrightarrow[x\to+\infty]{}0\end{equation}
	car $\vert f(x)\vert\leqslant\frac{1}{x}$ et 
	\begin{equation}f'(x)=\underbrace{-\frac{1}{x^{2}}\sin(x^{2})}_{\xrightarrow[x\to+\infty]{}0}+\underbrace{\frac{2x\cos(x^{2})}{x}}_{\text{n'a pas de limite en }+\infty}\end{equation}
\end{proof}

\begin{proof}
	Soit $x\in\R$ et $h\neq0$, on a 
	\begin{equation}\frac{f(x+h)-f(x)}{h}=g(x+\frac{h}{2})\xrightarrow[h\to0]{}g(x)\end{equation}
	par continuité de $g$. Donc $f$ est dérivable et $f'=g$. Par ailleurs, pour $y=\frac{1}{2}$, on a 
	\begin{equation}f'(x)=f(x+\frac{1}{2})-f(x-\frac{1}{2})\end{equation}
	par récurrence $f$ est $\mathcal{C}^{\infty}$.

	En outre, en fixant $x$ et en dérivant la relation de départ deux fois par rapport à $y$, on a 
	\begin{equation}f''(x+y)-f''(x-y)=0\end{equation}
	Donc $f''$ est constante donc $f$ est un polynôme de degré plus petit que 2.

	Réciproquement, on vérifie que ces fonctions marchent (avec $f'=g$).
\end{proof}

\begin{proof}
	On a 
	\begin{equation}S_{n}=\sum_{k=1}^{n-1}\frac{1}{2}(f(k)+f(k+1))-\int_{k}^{k+1}f(t)dt\end{equation}
	On note $F(x)=\int_{1}^{x}f(t)dt$ de classe $\mathcal{C}^{2}$.

	On a
	\begin{equation}F(b)=F(a)+F'(a)(b-a)+\int_{a}^{b}F''(t)(b-t)dt\end{equation}
	Pour $a=k$ et $b=k+\frac{1}{2}$, on a 
	\begin{equation}F(k+\frac{1}{2})=F(k)+\frac{1}{2}F'(k)+\int_{k}^{k+\frac{1}{2}}(k+\frac{1}{2}-t)f'(t)dt=F(k)+\frac{1}{2}F'(k)+\int_{0}^{\frac{1}{2}}uf'(k+\frac{1}{2}-u)du\end{equation}
	et pour $a=k+1,b=k+\frac{1}{2}$,
	\begin{equation}F(k+\frac{1}{2})=F(k+1)-\frac{1}{2}F'(k+1)+\int_{k+1}^{k+\frac{1}{2}}(k+\frac{1}{2}-t)f'(t)dt=F(k+1)-\frac{1}{2}F'(k+1)+\int_{0}^{\frac{1}{2}}uf'(k+\frac{1}{2}+u)du\end{equation}

	On a donc
	\begin{equation}\frac{1}{2}(f(k)-f(k+1))-\int_{k}^{k+1}f(t)dt=\int_{0}^{\frac{1}{2}}u(f'(k+\frac{1}{2}+u)-f'(k+\frac{1}{2}-u))du\end{equation}
	d'où
	\begin{equation}S_{n}=\int_{0}^{\frac{1}{2}}u\sum_{k=1}^{n-1}\underbrace{f'(k+\frac{1}{2}+u)-f'(k+\frac{1}{2}-u)}_{\geqslant0\text{ car }u\geqslant0\text{ et }f'\text{ croissante}}du\end{equation}
	et 
	$f'(k+\frac{1}{2}+u)-f'(k+\frac{1}{2}-u)\leqslant f'(k+1)-f'(k)$ d'où 
	\begin{equation}S_{n}\leqslant\underbrace{\int_{0}^{\frac{1}{2}}udu}_{=\frac{1}{8}}(f'(n)-f'(1))\end{equation}
\end{proof}

\begin{proof}
	\phantom{}
	\begin{enumerate}
		\item D'après l'inégalité de Taylor-Lagrange, on a 
		\begin{equation}
		\left\{
			\begin{array}[]{l}
				\Vert A\Vert\leqslant\frac{h^{2}}{2}M_{2}\\
				\Vert B\Vert\leqslant\frac{h^{2}}{2}M_{2}
			\end{array}
		\right.
		\end{equation}
		On a $B-A-f(x-h)+f(x+h)=2hf'(x)$ d'où 
		\begin{equation}\Vert f'(x)\Vert\leqslant\frac{hM_{2}}{2}+\frac{M_{0}}{h}\end{equation}
		Donc $f'$ est bornée sur $\R$. On a ensuite un majorant qui dépend de $h$ que l'on peut optimiser, et on trouve la borne demandée.

		\item L'inégalité de Taylor-Lagrange donne à nouveau
		\begin{equation}\forall k\in\{1,\dots,n-1\},\Vert A_{k}\Vert\leqslant\frac{k^{n}}{n!}M_{n}\end{equation}
		On forme alors
		\begin{equation}
		\begin{pmatrix}
			A_{1}-f(x+1)\\
			\vdots\\
			A_{k}-f(x+k)\\
			\vdots\\
			A_{n}-f(x+n)
		\end{pmatrix}
		=
		\underbrace{
		\begin{pmatrix}
			-1 & -1 & \dots & \frac{-1}{(n-1)!}\\
			\vdots & \vdots & & \vdots\\
			-1 & -k & \dots & \frac{-k^{n-1}}{(n-1)!}\\
			\vdots & \vdots & & \vdots\\
			-1 & -n & \dots & \frac{-n^{n-1}}{(n-1)!}\\
		\end{pmatrix}}_{=M}
		\begin{pmatrix}
			f(x)\\
			\vdots\\
			f^{(k)}(x)\\
			\vdots\\
			f^{(n-1)}(x)
		\end{pmatrix}
		\end{equation}

		On a 
		\begin{equation}\det(M)=\frac{(-1)^{n}}{1!\times 2!\times\dots\times (n-1)!}V(1,\dots,n)\end{equation}
		où $V$ est le déterminant de Vandermonde. Donc $\det(M)\neq0$. On peut former les $f^{(j)}(x)$ en fonction des $(A_{i}-f(x+i))_{1\leqslant i\leqslant n}$: il existe $(\alpha_{1},\dots,\alpha_{n})\in\R^{n}$ tel que pour tout $x\in\R$, $f^{(j)}(x)=\sum_{i=1}^{n}\alpha_{i}(A_{i}-f(x+i))$. Donc 
		\begin{equation}\Vert f^{(j)}(x)\Vert\leqslant\sum_{i=1}^{n}\vert\alpha_{i}\vert\Bigl(\frac{n}{n!}M_{n}+M_{0}\Bigr)\end{equation}
		Donc $f^{(j)}$ est bornée pour tout $j\in\{1,\dots,n-1\}$.
	\end{enumerate}
\end{proof}

\begin{proof}
	\phantom{}
	\begin{enumerate}
		\item 
		\begin{equation}l_{\sigma,\gamma}=\sum_{i=0}^{n-1}\Bigl\Vert\int_{a_{i}}^{a_{i+1}}\gamma'(t)dt\Bigr\Vert\leqslant\sum_{i=0}^{n-1}\int_{a_{i}}^{a_{i+1}}\Vert\gamma'(t)\Vert dt=\int_{a}^{b}\Vert\gamma'(t)\Vert dt\end{equation}

		\item On a 
		\begin{align}
			\Bigl\vert l_{\sigma,\gamma}-\sum_{i=0}^{n-1}\Vert\gamma'(a_{i})\Vert(a_{i+1}-a_{i})\Bigr\vert
			&=\Bigl\vert\sum_{i=0}^{n-1}\Vert\gamma(a_{i+1})-\gamma(a_{i})\Vert-\Vert\underbrace{(a_{i+1}-a_{i})}_{>0}\gamma'(a_{i})\Vert\Bigr\vert\\
			&\leqslant\sum_{i=0}^{n-1}\Vert\gamma(a_{i+1})-\gamma(a_{i})-(a_{i+1}-a_{i})\gamma'(a_{i})\Vert\\
			&\leqslant\sum_{i=0}^{n-1}\int_{a_{i}}^{a_{i+1}}\Vert\gamma'(t)-\gamma'(a_{i})\Vert dt
		\end{align}

		\item $\Vert\gamma'\Vert$ est continue donc 
		\begin{equation}\int_{a}^{b}\Vert\gamma'(t)\Vert dt=\lim\limits_{\delta(\sigma)\to0}\sum_{i=0}^{n-1}\Vert\gamma'(a_{i})\Vert(a_{i+1}-a_{i})\end{equation}
		Donc $\alpha_{0}$ existe.

		$\gamma'$ est continue sur $[a,b]$ donc uniformément continue sur $[a,b]$, et il existe $\alpha_{1}>0$ tel que pour tout $(x,y)\in[a,b]^{2}$, on a 
		\begin{equation}\vert x-y\vert\leqslant\alpha_{}\Rightarrow\Vert\gamma'(x)-\gamma'(y)\Vert\leqslant\frac{\varepsilon}{2(b-a)}\end{equation}
		Alors si $\delta(\sigma)\leqslant\alpha_{1}$, pour tout $i\in\{0,\dots,n-1\}$, pour tout $t\in[a_{i},a_{i+1}]$, on a
		\begin{equation}\vert t-a_{i}\vert\leqslant(a_{i+1}-a_{i})\leqslant\alpha_{1}\end{equation}
		d'où 
		\begin{equation}\Vert \gamma'(a_{i})-\gamma'(t)\Vert\leqslant\frac{\varepsilon}{2(b-a)}\end{equation}
		et d'après la question 2, on a donc 
		\begin{equation}\Bigl\vert l_{\sigma,\gamma}-\sum_{i=0}^{n-1}\Vert\gamma'(a_{i})\Vert(a_{i+1}-a_{i})\Bigr\vert\leqslant\frac{\varepsilon}{2}\end{equation}

		Finalement, si $@d(\sigma)\leqslant\min(\alpha_{0},\alpha_{1})$, on a 
		\begin{equation}\Bigl\vert l_{\sigma,\gamma}-\int_{a}^{b}\Vert\gamma'(t)\Vert dt\Bigr\vert\leqslant\varepsilon\end{equation}
		Donc 
		\begin{equation}l(\gamma)=\int_{a}^{b}\Vert\gamma'(t)\Vert dt\end{equation}

		\item On a 
		\begin{equation}\gamma'(t)=\begin{pmatrix}
			-R\sin(t)\\
			R\cos(t)
		\end{pmatrix}\end{equation}
		donc $\Vert\gamma'(t)\Vert=R$ et $l(\gamma)=2\pi R$.
	\end{enumerate}
\end{proof}

\begin{proof}
	\phantom{}
	\begin{enumerate}
		\item Pour tout $t\in I$, on a 
		\begin{equation}\gamma(t)=\vert\gamma(t)\vert e^{\mathrm{i}\theta_{1}(t)}=\vert\gamma(t)\vert e^{\mathrm{i}\theta_{2}(t)}\end{equation}
		donc 
		\begin{equation}e^{\mathrm{i}(\theta_{1}(t)-\theta_{2}(t))}=1\end{equation}

		Ainsi, pour tout $t\in I$, il existe $k(t)\in\Z$ telle que $\theta_{2}(t)-\theta_{1}(t)=2k(t)\pi$. On a 
		\begin{equation}k(t)=\frac{\theta_{2}(t)-\theta_{1}(t)}{2\pi}\end{equation}
		qui est continue et à valeurs entières, donc constante égale à $k_{0}$ d'après le théorème des valeurs intermédiaires.

		\item Si $\gamma(t)=x(t)+\mathrm{i}y(t)$, 
		\begin{equation}\vert\gamma(t)\vert=\sqrt{x(t)^{2}+y(t)^{2}}\end{equation}
		Comme $\sqrt{\cdot}$ est $\mathcal{C}^{\infty}$ sur $\R_{+}^{*}$, par composition, $f$ est $\mathcal{C}^{k}$. On a alors
		\begin{equation}f(t)=e^{\mathrm{i}\theta(t)}\Rightarrow f'(t)=\mathrm{i}\theta'(t)e^{\mathrm{i\theta(t)}}=\mathrm{i}\theta'(t)f(t)\end{equation}
		Donc 
		\begin{equation}\theta(t)=-\mathrm{i}\frac{f'(t)}{f(t)}\end{equation}

		De plus, on a 
		\begin{equation}\theta(t)=\theta(t_{0})-\mathrm{i}\int_{t_{0}}^{t}\frac{f'(u)}{f(u)}du\end{equation}
		pour $t_{0}\in I$.

		\item On fixe $t_{0}\in I$. Soit $\theta_{0}$ un argument de $\gamma(t_{0})$, on pose 
		\begin{equation}\theta(t)=\theta_{0}-\mathrm{i}\int_{t_{0}}^{t}\frac{f'(u)}{f(u)}du\end{equation}
		
		Comme $\frac{f'}{f}$ est $\mathcal{C}^{k-1}$, $\theta$ est bien $\mathcal{C}^{k}$. On forme $g(t)=e^{\mathrm{i}\theta(t)}$ qui est de classe $\mathcal{C}^{k}$. On a 
		\begin{equation}g'(t)=\mathrm{i}\theta'(t)g(t)=\frac{f'(t)}{f(t)}g(t)\end{equation}
		donc $\Bigl(\frac{g}{f}\Bigr)'=0$, donc $\frac{g}{f}$ est constante sur $I$ et $g(t_{0})=e^{\mathrm{i}\theta_{0}}=f(t_{0})$ donc $g=f$ sur $I$. Ainsi, pour tout $t\in I$, on a $\vert f(t)\vert=\vert e^{\mathrm{i}\theta(t)}\vert=1$ et si $\theta(t)=a(t)+\mathrm{i}(t)$, on a donc 
		\begin{equation}e^{\mathrm{i}\theta(t)}=e^{-b(t)}e^{\mathrm{i}a(t)}\end{equation}
		donc $b(t)=0$ et $\theta(t)\in\R$.
	\end{enumerate}
\end{proof}

\section{Suites et séries de fonctions}
\section{Séries entières}
\section{Intégration}
\section{Espaces préhilbertiens}
\section{Espaces euclidiens}
\section{Calcul différentiel}
\section{\'Equation différentielles linéaires}


\end{document}