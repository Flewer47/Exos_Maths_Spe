\documentclass[12pt]{article}
\usepackage[french]{babel}
\usepackage[utf8]{inputenc}
\usepackage[T1]{fontenc}
\usepackage{amsmath}
\usepackage{amssymb}
\usepackage{color}
\definecolor{persianplum}{rgb}{0.44, 0.11, 0.11}
\usepackage{url}
\usepackage[breaklinks]{hyperref}
\hypersetup{
	colorlinks=true,
	linkcolor=persianplum,
	filecolor=blue,
	citecolor=black,      
	urlcolor=cyan,
}

\textwidth=18cm \textheight=23cm \oddsidemargin=-1.00cm
\evensidemargin=-1.00cm
\parindent=1cm
\topmargin=-2cm

\usepackage{amsthm}
\newtheorem{solution}{Solution}[section]
\theoremstyle{remark}
\newtheorem{remark}{Remarque}[section]

\newcommand{\K}{\mathbb{K}} \newcommand{\R}{\mathbb{R}}
\newcommand{\C}{\mathbb{C}} \newcommand{\Q}{\mathbb{Q}}
\newcommand{\N}{\mathbb{N}} \newcommand{\Z}{\mathbb{Z}}
\newcommand{\U}{\mathbb{U}} \newcommand{\E}{\mathbb{E}}
\newcommand{\M}{\mathcal{M}} \renewcommand{\L}{\mathcal{L}}
\renewcommand{\P}{\mathbb{P}} \newcommand{\im}{\emph{Im}}
\DeclareMathOperator{\sgn}{sgn} \DeclareMathOperator{\diag}{diag}
\DeclareMathOperator{\rg}{rg} \DeclareMathOperator{\Tr}{Tr}
\DeclareMathOperator{\Sp}{Sp} \DeclareMathOperator{\mat}{mat}
\DeclareMathOperator{\com}{com} \DeclareMathOperator{\conv}{conv}
\newcommand{\vertiii}[1]{{\left\vert\kern-0.25ex\left\vert\kern-0.25ex\left\vert{}#1
\right\vert\kern-0.25ex\right\vert\kern-0.25ex\right\vert}}
\newcommand{\function}[5]{
	$$
	\begin{array}{rccl}
		#1: & #2 & \to & #3 \\
		& #4 & \mapsto & #5
	\end{array}
	$$
}

\begin{document}

\begin{titlepage}
	\centering
	\vspace*{\fill}
	\Huge \textit{\textbf{Solutions Exercices MP/MP$^*$}}
	\vspace*{\fill}
\end{titlepage}

\cleardoublepage

\tableofcontents

\cleardoublepage

\section{Algèbre Générale}
\cleardoublepage
\section{Séries numériques et familles sommables}
\cleardoublepage
\section{Probabilités sur un univers dénombrable}
\cleardoublepage
\section{Calcul matriciel}
\cleardoublepage
\section{Réduction des endomorphismes}


\begin{solution}
	Pour le sens indirect, soit $\lambda\in\Sp_{\C}(M)$. Pour tout $p\in\N$, $\lambda\in\Sp_{\C}(M_{p})$ donc $\det(M_{p}-\lambda I_{n})=0$. Par continuité du déterminant, on a $0=\det(M_{p}-\lambda I_{n})\xrightarrow[p\to+\infty]{}\det(-\lambda I_{n})$. Donc $\lambda=0$ et $\Sp_{\C}(M)=\{0\}$ donc $M$ est nilpotente.

	Pour le sens direct, soit $u\in\L(\C^{n})$ canoniquement associée à $M$. On trigonalise $u$ sur une base $\mathcal{B}=(\varepsilon_{1},\dots,\varepsilon_{n})$ avec $u(\varepsilon_{1})=0,u(\varepsilon_{2})=a_{1,2}\varepsilon_{1},\dots,u(\varepsilon_{n})=a_{1,n}\varepsilon_{1}+\dots+a_{n-1,n}\varepsilon_{n-1}$. Posons pour $i\in\{1,\dots,n\}$, $\varepsilon_{i,p}=\frac{\varepsilon_{i}}{p^{i-1}}$. On pose $\mathcal{B}_{p}=(\varepsilon_{1,p},\dots,\varepsilon_{n,p})$ et $M_{p}=\mat\limits_{B_{p}}(u)$, semblable à $M$ et $M_{p}\xrightarrow[p\to+\infty]{}0$ car $\Vert M_{p}\Vert\leqslant\frac{1}{p}\Vert M_{1}\Vert$.
\end{solution}

\begin{solution}
	On pose $u\in\L(\C^{n})$ canoniquement associée à $M$. 

	Pour le sens indirect, si $M$ n'est pas diagonalisable, il existe une base $B=(\varepsilon_{1},\dots,\varepsilon_{n})$ de $\C^{n}$ telle que 
	$$\mat\limits_{\mathcal{B}}(u)=D+N$$
	où $D$ est diagonale et $N$ est nilpotente (décomposition de Dunford). En reprenant les bases $\mathcal{B}_{p}$ définies à l'exercice précédent, on a
	$$\mat\limits_{\mathcal{B}_{p}}(u)=D+N_{p}\xrightarrow[p\to+\infty]{}D$$
	Si $D\in S_{M}$, alors $M$ est diagonalisable ce qui est exclu par hypothèse. Donc $S_{M}$ n'est pas fermé.

	Pour le sens direct, si $M$ est diagonalisable, soit $(M_{p})_{p\in\N}\in(S_{M})^{\N}$ avec $M_{p}\xrightarrow[p\to+\infty]{}M'$. Soit $\lambda\in\C$. On a $\chi_{M_{p}}(\lambda)=\det(\lambda I_{n}-M_{p})=\chi_{M}(\lambda)$ car $M$ et $M_{p}$ sont semblables. Par continuité du déterminant, on a $\chi_{M'}(\lambda)=\chi_{M}(\lambda)$, donc $\chi_{M'}=\chi_{M}$. De plus, $A\mapsto\Pi_{M}(A)$ (polynôme minimal) est continue sur $\M_{n}(\C)$ et pour tout $p\in\N$, on a $\Pi_{M}(M_{p})=0$ donc $\Pi_{M}(M')=0$. $M'$ est donc annulée par $\Pi_{M}$, donc $M'$ est diagonalisable et comme $\chi_{M}=\chi_{M'}$, $M$ et $M'$ ont les mêmes valeurs propres avec les mêmes multiplicités. Donc $M'\in S_{M}$.
\end{solution}

\begin{remark}
	Le polynôme caractéristique est une fonction continue de la matrice, mais c'est faux pour le polynôme minimal, par exemple pour 
	$$M_{p}=\begin{pmatrix}
		\frac{1}{p} &0\\
		0 & \frac{2}{p}
	\end{pmatrix}$$
	On a $M_{p}\xrightarrow[p\to+\infty]{}0$ et $\Pi_{M_{p}}=(X-\frac{1}{p})(X-\frac{2}{p})\xrightarrow[p\to+\infty]{} X^{2}\neq X=\Pi_{M_{\infty}}$ donc $\lim\limits_{p\to+\infty}\Pi_{M_p}\neq\Pi_{\lim\limits_{p\to+\infty}M_{p}}$.
\end{remark}

\cleardoublepage
\section{Espaces vectoriels normés}

\begin{solution}
	\phantom{}
	\begin{enumerate}
		\item A $(x,y)\in\R^{2}$ fixé, la fonction \function{\varphi}{\R}{\R}{t}{x\cos(t)+y\sin(2t)}
		est bornée, donc le $\sup$ sur $\R$ existe. Pour la séparation, prendre $t=0$ et $t=\frac{\pi}{4}$. Pour l'inégalité triangulaire, montrer l'inégalité à $t$ fixé puis passer au $\sup$ sur $\R$.
		
		\item Si $\vert x\vert+\vert y\vert\leqslant1$, alors $N(x,y)\leqslant 1$ donc on a la première inclusion. 
		
		Si $N(x,y)\leqslant 1$, utiliser $t=0$ pour avoir $\vert x\vert\leqslant1$ et $t=\frac{\pi}{4}$ puis $t=-\frac{\pi}{4}$ pour pouvoir justifier
		$$\vert 2y\vert\leqslant \Biggl\vert x\frac{\sqrt{2}}{2}+y\Biggr\vert+\Biggl\vert y-x\frac{\sqrt{2}}{2}\Biggr\vert\leqslant 2$$
		et donc $\vert y\vert\leqslant1$. D'où la deuxième inclusion. 

		\item On fixe $(x,y)\in S_{N}(0,1)\cap(\R_{+})^{2}$. $\varphi$ est $2\pi$-périodique, $\varphi(\pi-t)=\varphi(t)$ et $\sup\limits_{t\in\R}\vert\varphi(t)\vert=1$. On peut donc se limite à un intervalle de longueur $2\pi$ pour l'étude de $\varphi$. 
		
		On note que si $t\in[-\pi,0]$, $\cos(t)$ et $\sin(2t)$ sont de signes opposés. Donc
		$$\vert\varphi(t)\vert\leqslant x\vert\cos(t)\vert+y\vert\sin(2t)\vert=\vert\varphi(-t)\vert$$
		et $-t\in[0,\pi]$. Donc le $\sup$ est atteint sur $[0,\pi]$.

		On note maintenant, comme $\vert\varphi(\pi-t)\vert=\vert\varphi(t)\vert$ sur $[0,\frac{\pi}{2}]$, que si $t\in[\frac{\pi}{4},\frac{\pi}{2}]$,
		$$0\leqslant\varphi(t)=x\underbrace{\cos(t)}_{\in[0,\frac{\sqrt{2}}{2}]}+y\sin(2t)\leqslant x\underbrace{\cos(\frac{\pi}{2}-t)}_{\in[\frac{\sqrt{2}}{2},1]}+y\sin(2\times (\frac{\pi}{2}-t))=\varphi(\frac{\pi}{2}-t)$$

		Donc le $\sup$ est atteint sur $[0,\frac{\pi}{4}]$. Soit maintenant $t_{0}\in[0,\frac{\pi}{4}]$ tel que $\varphi(t_{0})$ réalise le $\sup$ (existe car $\varphi$ est continue sur un compact). Comme c'est aussi le $\sup$ sur $\R$ qui est ouvert, on a la condition d'Euler du premier ordre: $\varphi'(t_{0})=0$.

		On a donc $x\cos(t_{0})+y\sin(2t_{0})=1$ et $-x\sin(t_{0})+2y\cos(2t_{0})=0$. On en déduit les valeurs de $x$ et $y$ en fonction de $t_{0}$, en faisant attention que $\cos(t_{0})\neq0$ sinon $\sin(t_{0})=0$ aussi ce qui n'est pas le cas, et au cas où $t_{0}=0$.

		Réciproquement, s'il existe $t_{0}\in[0,\frac{\pi}{4}]$ tel que $x$ et $y$ s'écrivent de la façon demandée, alors $t_{0}$ est l'unique point satisfaisant $\varphi(t_{0})=1$ et $\varphi'(t_{0})=0$. Mais alors le $\sup$ de $\varphi$ sur $[0,\frac{\pi}{4}]$ est atteint en un point $t_{1}$ qui vérifie les mêmes choses, donc $t_{1}=t_{0}$ d'où $N(x,y)=1$.
	\end{enumerate}
\end{solution}

\begin{solution}
	\phantom{}
	\begin{enumerate}
		\item Pour l'inégalité triangulaire, introduire la forme bilinéaire symétrique positive sur $E$ \function{\varphi}{E\times E}{\R}{(f,g)}{f(0)g(0)+\int_{0}^{1}f'(t)g'(t)dt}
		Alors $N(f)=\sqrt{\varphi(f,f)}$ et on utilise l'inégalité de Minkowski.
		\item Pour $x\in[0,1]$, écrire $\vert f(x)\vert=\vert f(0)+f(x)-f(0)\vert$, $f(x)-f(0)=\int_{0}^{x}f'(t)dt$, utiliser Cauchy-Schwarz avec $f'$ et $1$ puis que $\sqrt{a}+\sqrt{b}\leqslant\sqrt{2}\sqrt{a+b}$, pour enfin passer au $\sup$ sur $x$.
		\item Utiliser, pour $n\in\N^{*}$, la fonction \function{f_n}{[0,1]}{\R}{t}{t^n}
	\end{enumerate}
\end{solution}

\begin{solution}
	Si $f$ est ouverte, $f(\R^{n})$ est un sous-espace vectoriel ouvert de $R^{p}$. Donc $f$ est surjective.

	Si $f$ est surjective, on prend $F$ un supplémentaire de $\ker(f)$ dans $\R^{n}$ avec $\dim(\ker(f))=n-p$ et $\dim(F)=p$. Soit $(e_{1},\dots,e_{p})$ une base de $F$ et $(e_{p+1},\dots,e_{n})$ une base de $\ker(f)$. On vérifie que $(f(e_{1},\dots,f(e_{p}))$ est une base de $\R^{p}$. On définit \function{N_1}{\R^n}{\R}{\sum_{i}^{n}x_{i}e_{i}}{\max\limits_{1\leqslant i\leqslant n}\vert x_{i}\vert}
	norme sur $\R^{n}$ et \function{N_2}{\R^p}{\R}{\sum_{i}^{p}y_{i}f(e_{i})}{\max\limits_{1\leqslant i\leqslant p}\vert y_{i}\vert}
	norme sur $\R^{p}$.

	Soit $\Theta$ un ouvert de $\R^{n}$, soit $y_{0}\in f(\Theta)$, il existe $x_{0}\in\Theta\colon y_{0}=f(x_{0})$. Si $x_{0}=\sum_{i=1}^{n}\alpha_{i}e_{i}$, alors $y_{0}=\sum_{i=1}^{p}\alpha_{i}f(e_{i})$. Comme $\Theta$ est un ouvert, il existe $r_{0}>0$ tel que 
	$$B_{N_{1}}(x_{0},r_{0})\subset\Theta$$
	Soit $y=\sum_{i=}^{p}\beta_{i}f(e_{i})\in\R^{p}$, si $N_{2}(y-y_{0})<r_{0}$, pour tout $i\in\{1,\dots,p\}$, $\vert\beta_{i}-\alpha_{i}\vert<r_{0}$ et 
	$$y=f\Biggl(\sum_{i=1}^{p}\beta_{i}e_{i}+\sum_{i=p+1}^{n}\alpha_{i}e_{i}\Biggr)\overset{\text{def}}{=}f(x)$$
	avec $N_{1}(x-x_{0})=\max\limits_{1\leqslant i\leqslant p}\vert\beta_{i}-\alpha_{i}\vert<r_{0}$. Ainsi $x\in\Theta$ et $y\in f(\Theta)$, donc $B_{N_{2}}(y_{0},r_{0})\subset f(\Theta)$ et $f(\Theta)$ est un ouvert.
\end{solution}

\begin{solution}
	\phantom{}
	\begin{enumerate}
		\item Classique.
		\item $$\vert f(x)\vert\leqslant\vert f(0)\vert+\vert f(x)-f(0)\vert\leqslant\vert f(0)\vert+\kappa(f)x\leqslant N(f)$$
		car $x\leqslant 1$, donc $N_{\infty}\leqslant N$. Pour la non-équivalence, prendre \function{f_n}{[0,1]}{\R}{t}{t^n}
		\item On a $\vert f(0)\vert\leqslant N_{\infty}(f)$ donc $N(f)\leqslant N'(f)$. Ensuite, $N_{\infty}\leqslant N$ donne $N'\leqslant N+\kappa\leqslant 2N$. Donc $N$ est $N'$ sont équivalentes.
	\end{enumerate}
\end{solution}

\begin{remark}
	Exemple de normes qui, en dimension infinie, ne se dominent pas mutuellement. On prend $(e_{i})_{i\in I}$ une base (de Hamel), $J=(i_{n})_{n\in\N}\subset I$ dénombrable. Si $x=\sum_{i\in I}x_{i}e_{i}$, on peut vérifier que 
	$$N_{1}(x)=\sum_{n\in\N}\vert x_{i_{n}}\vert+\sum_{i\in I\setminus J}\vert x_{i}\vert$$
	et
	$$N_{2}(x)=\sum_{n\in\N}n\vert x_{i_{2n}}\vert+\sum_{n\in\N}\frac{1}{n+1}\bigl\lvert x_{i_{2n+1}}\bigr\rvert+\sum_{i\in I\setminus J}\vert x_{i}\vert$$
	ne se dominent pas.
\end{remark}

\begin{solution}
	Il existe $\alpha>0$ tel que $B_{\Vert\cdot\Vert_{\infty}}(I_{n},\alpha)\subset G$. Soient $i\neq j$ et $\lambda\in\C$. Il existe $p\in\N^{*}$ tel que $\frac{\vert\lambda\vert}{p}<\alpha$. Alors 
	$$\Biggl\lVert T_{i,j}\Biggl(\frac{\lambda}{p}\Biggr)-I_{n}\Biggr\rVert_{\infty}=\Biggl\lvert\frac{\lambda}{p}\Biggr\rvert<\alpha$$
	donc $T_{i,j}(\lambda)\in G$ ($T_{i,j}$ est la matrice de transvection: $T_{i,j}(\lambda)=I_{n}+\lambda E_{i,j}$).

	Ainsi,
	$$T_{i,j}(\lambda)=\Biggl(T_{i,j}\Biggl(\frac{\lambda}{p}\Biggr)\Biggr)^{p}\in G$$

	Soit $\delta=\rho e^{\mathrm{i}\theta}\in\C^{*}$. On a $\lim\limits_{n\to+\infty}\rho^{\frac{1}{p}}e^{\mathrm{i}\frac{\theta}{p}}=1$ donc il existe $p\in\N^{*}$ tel que $\vert\rho^{\frac{1}{p}}e^{\mathrm{i}\frac{\theta}{p}}-1\vert<\alpha$.
	
	On a alors
	$$\Biggl\lVert D_{n}\Bigl(\rho^{\frac{1}{p}}e^{\mathrm{i}\frac{\theta}{p}}\Bigr)-I_{n}\Biggr\rVert_{\infty}<\alpha$$
	donc $D_{n}(\delta)=D_{n}(\rho^{\frac{1}{p}}e^{\mathrm{i}\frac{\theta}{p}})^{p}\in G$ (matrice de dilatation).

	Comme les matrices de transvection et de dilatation engendrent $GL_{n}(\C)$, on a bien $G=GL_{n}(\C)$.
\end{solution}

\begin{remark}
	C'est faux sur $\R$. Contre-exemple: matrices de déterminant positif.
\end{remark}

\begin{solution}
	Si $f$ n'est pas continue en 0, il existe $\varepsilon_{0}>0$ tel que pour tout $\alpha>0$, il existe $h\in E$ avec $\Vert h\Vert\leqslant\alpha$ et $\Vert f(h)\Vert>\varepsilon_{0}$. On prends $\alpha_{n}=\frac{1}{n+1}$, d'où $\Vert nh_{n}\Vert\leqslant1$ mais $\underbrace{\Vert f(nh_{n})\Vert}_{\leqslant M}>n\varepsilon_{0}\xrightarrow[n\to+\infty]{}+\infty$. Donc $f$ est continue en $0$. Comme $f$ est linéaire, pour tout $x\in E$,
	$$\lim\limits_{\Vert h\Vert\to0}f(x+h)=\lim\limits_{\Vert h\Vert\to0}f(x)+f(h)=f(x)$$
	donc $f$ est continue.

	On a $f(px)=p(fx)$ pour tout $p\in\Z$ puis $qf(\frac{p}{q}x)=f(px)=pf(x)$ pour tout $(p,q)\in\Z\times\N^{*}$ donc pour tout $r\in\Q$, $f(rx)=rf(x)$.
	Soit $\lambda\in\E$, il existe une suite de rationnels telle que $\lim\limits_{n\to+\infty} r_{n}=\lambda$. Comme $f$ est continue, on a 
	\begin{align*}
		f(\lambda x)
		&=\lim\limits_{n\to+\infty}f(r_{n}x)\\
		&=\lim\limits_{n\to+\infty}r_{n}f(x)\\
		&=\lambda f(x)
	\end{align*}
	Donc $f$ est linéaire.
\end{solution}

\begin{remark}
	Soit $e_{0}=1$ et $e_{1}=\sqrt{2}$ et $(e_{i})_{i\in I}$ une $\Q$-base de $\R$ ($0\in I$). On définie 
	$$f\Bigl(\sum_{i\in I}\lambda_{i} e_{i}\Bigr)=\lambda_{0}e_{0}+\sqrt{2}\sum_{i\in I\setminus\{0\}}\lambda_{i}e_{i}$$
	$f$ vérifie $f(x+y)=f(x)+f(y)$, mais si $(r_{n})_{n\in\N}$ est une suite de rationnels tendant vers $\sqrt{2}$, $f(r_{n})=r_{n}\to\sqrt{2}\neq f(\sqrt{2})=2$.
\end{remark}

\begin{solution}
	\phantom{}
	\begin{enumerate}
		\item On a $\alpha(A)\subset \overline{A}$ donc $\overline{\mathring{\overline{A}}}\subset\overline{A}$ donc $\alpha(\alpha(A))\subset\alpha(A)$. Comme $\alpha(A)$ est un ouvert inclus dans $\overline{\mathring{\overline{A}}}\subset\overline{A}$ donc $\alpha(A)\subset\alpha(\alpha(A))$.

		\item Si $\beta(A)=\overline{\mathring{A}}$, on montre aussi que $\beta(\beta(A))=\beta(A)$. On a donc $A,\overline{A},\mathring{A},\overline{\mathring{A}},\mathring{\overline{A}},\overline{\mathring{\overline{A}}}$ et $\mathring{\overline{\mathring{A}}}$ et c'est tout.
	\end{enumerate}
\end{solution}

\begin{solution}
	\phantom{}
	\begin{enumerate}
		\item Si $d_{A}=d_{B}$, 
		$$\overline{A}=\{x\in E\bigm| d_{A}(x)=0\}=\{x\in E\bigm| d_{B}(x)=0\}=\overline{B}$$
		Réciproquement, soit $x\in E$ et $\varepsilon>0$, il existe $a_{1}\in\overline{A}$, $\Vert x-a_{i}\Vert\leqslant d_{\overline{A}}(x)+\frac{\varepsilon}{2}$ (par définition de l'inf). Il existe $a_{2}\in A$, $\Vert a_{1}-a_{2}\Vert\leqslant\frac{\varepsilon}{2}$ (par définition de la fermeture). Ainsi,
		$$d_{A}(x)\leqslant\Vert x-a_{2}\Vert\leqslant\Vert x-a_{1}\Vert+\Vert a_{1}-a_{2}\Vert\leqslant d_{\overline{A}}(x)+\varepsilon$$
		Ceci valant pour tout $\varepsilon>0$, $d_{A}(x)\leqslant d_{\overline{A}}(x)$. Comme $A\subset\overline{A}$, $d_{\overline{A}}\leqslant d_{A}$, on a $d_{A}=d_{\overline{A}}=d_{\overline{B}}=d_{B}$.

		\item Soit $x\in A$, on a $d_{B}(x)=\vert d_{B}(x)-d_{A}(x)\vert\leqslant\rho(A,B)$ donc $\sup\limits_{x\in A}d_{B}(x)\leqslant\rho(A,B)$, de même pour $\sup\limits_{y\in B}d_{A}(y)$ donc on on a un première inégalité.
		
		Réciproquement, soit $x\in E$ et $\varepsilon>0$, il existe $a\in A$ et $b\in B$ tel que $\Vert x-a\Vert\leqslant d_{A}(x)+\varepsilon$ et $\Vert x-b\Vert\leqslant d_{B}(x)+\varepsilon$.
		On a alors
		$$d_{A}(x)\leqslant\Vert x-a\Vert\leqslant\Vert a-b\Vert+\Vert x-b\Vert\leqslant d_{B}(x)+\varepsilon+\alpha(A,B)$$
		Ceci vaut pour tout $\varepsilon>0$, donc $d_{A}(x)\leqslant d_{B}(x)+\alpha(A,B)$. De même, $d_{B}(x)\leqslant d_{A}(x)+\alpha(A,B)$ donc $\rho(A,B)\leqslant\alpha(A,B)$.
	\end{enumerate}
\end{solution}

\begin{solution}
	\phantom{}
	\begin{enumerate}
		\item Soit $(y_{n})_{n\in\N}\in P(F)^{\N}$ qui converge vers $y\in\C$ donc il existe $(x_{n})\in F^{\N}$ telle que l'on ait pour tout $n\in\N$, $P(x_{n})=y_{n}$. $(x_{n})_{n\in\N}$ est bornée car $\lim\limits_{z\to+\infty}\vert P(z)\vert=+\infty$ (car $P$ est non constant), donc on peut extraire (Bolzano-Weierstrass) $x_{\sigma(n)}\to x$ et $x\in F$ car $F$ est fermé. Par continuité de $z\mapsto P(z)$ sur $\C$, on a $y=P(x)\in P(F)$.
		
		\item Soit $\Theta$ un ouvert de $\C$, soit $y\in P(\Theta),\exists x\in\Theta$ tel que $P(x)=y$ et il existe $r>0$, $B(x,r)\subset\Theta$. Soit $y'\in\C$, supposons que pour tout $x'\in\C$ tel que $P(x')=y'$, on a $\vert x-x'\vert>r$. Soit $Q(X)=P(X)-y'=a\prod_{i=1}^{n}(X-x_{i})$ non constant où $a$ est le coefficient dominatrice de $P$. Par hypothèse, pour tout $i\in\{1,\dots,n\}\colon\vert x_{i}-x\vert>r$ (car $P(x_{i})=y'$), ainsi 
		$$\vert Q(x)\vert=\vert y-y'\vert\geqslant\vert a\vert r^{n}$$
		Par contraposée, si $\vert y-y'\vert\leqslant\frac{\vert a\vert r^{n}}{2}$, alors il existe $x'\in\C$ tel que $P(x')=y'$ et $\vert x'-x\vert<r$.Ainsi, $x'\in B(x,r)\subset\Theta$ et $y'\in P(\Theta)$. Donc $B(y,\vert a\vert r^{n})\subset P(\Theta)$ et $P(\Theta)$ est un ouvert.
	\end{enumerate}
\end{solution}

\begin{solution}
	\phantom{}
	\begin{enumerate}
		\item Si $P\notin\mathcal{S}$, il existe $z_{0}\in\C\setminus\R$ tel que $P(z_{0})=0$ et $\vert\Im(z_{0})\vert^{n}>0=P(z_{0})$. Par contraposée, si pour tout $z\in\C$, $\vert P(z)\vert\geqslant\vert\Im(z_{})\vert^{n}$,alors $P\in\mathcal{S}$.

		Réciproquement, si $P=\prod_{i=1}^{n}(X-\lambda_{i})\in\mathcal{S}$ avec $(\lambda_{i})_{1\leqslant i\leqslant n}$ réels, soit $z=a+ib\in\C$. On a
		$$\vert P(z)\vert=\prod_{i=1}^{n}\vert a-\lambda_{i}+ib\vert\geqslant\vert b\vert^{n}$$
		
		\item Soit $(P_{p})_{p\in\N}\in\mathcal{S}^{\N}$ telle que $P_{p}\xrightarrow[p\to+\infty]{}P\in F$. Soit $z\in\C$, on a pour tout $p\in\N$, $\vert P_{p}(z)\vert\geqslant\vert\Im(z)\vert^{n}$ donc quand $p\to+\infty$, $\vert P(z)\vert\geqslant\vert\Im(z)\vert^{n}$ donc $P\in\mathcal{S}$ et $S$ est fermé.
		
		\item Soit $(M_{p})_{p\in\N}$ une suite de matrice trigonalisable sur $\R$ qui converge vers $M\in\M_{n}(\R)$. Ib bite $\chi_{p}$ le polynôme caractéristique de $M_{p}$. Pour tout $p\in\N$, $\chi_{p}\in\mathcal{S}$ et $\chi_{p}\xrightarrow[p\to+\infty]{}\chi_{M}$. Comme $\mathcal{S}$ est fermé, $\chi_{M}\in \mathcal{S}$ et $M$ est trigonalisable sur $\R$.
	\end{enumerate}
\end{solution}

\begin{solution}
	\phantom{}
	\begin{enumerate}
		\item $\varphi$ est linéaire et $\dim(\K_{m-1}[X]\times\K_{n-1}[X])=m+n+=\dim(\K_{n+m-1}[X])$.
		
		Si $\varphi$ est bijective, elle est surjective et il existe $(U,V)\in\K[X]^{2}$ tel que $UA+BV=1$ et d'après le théorème de Bézout, on a $A\wedge B=1$.

		Réciproquement, si $\varphi$ n'est pas surjective, il existe $(U,V)\in(\K_{m-1}[X]\times\K_{n-1}[X])\setminus\{(0,0)\}$ tel que $\varphi(U,V)=0$ d'où $AU=-BV$. Soit $\delta=A\wedge B$, on écrit $A=\delta A_{1}$ et $B=\delta B_{1}$ avec $A_{1}\wedge B_{1}=1$ et on a $A_{1}U=-B_{1}V$. D'après le théorème de Gauss, on a $A_{1}\mid V$ et $B_{1}\mid U$. Si $U=0$, on a $V=0$ et de même si $V=0$, on a $U=0$. On peut donc supposer $U\neq0$ et $V\neq 0$, et on a alors $\deg(A_{1})\leqslant\deg(V)\leqslant n-1<n=\deg(A)$ mais $A=\delta A_{1}$ donc $\deg(\delta)\geqslant1$ et $A\wedge B\neq 1$.

		\item $\Phi$ est continue car $R_{A,B}$ est un polynôme en les coefficients de $A$ et $B$.
		
		\item Comme on est dans $\C$, $\Delta=\{P\in\C_{p}[X]\bigm| P\wedge P'=1\}=\{P\in\C_p[X]\bigm| R_{P,P'}\neq0\}$. $\Phi_{P,P'}$ est continue d'après la question précédente, $\delta=\Phi_{P,P'}^{-1}(\C^{*})$ donc $\Delta$ est ouvert.
		
		Sur $\R$, on n'a pas la caractérisation de scindé à racines simples si et seulement si $P\wedge P'=1$ (contre-exemple: $P=X^{2}+1$). Dans $\R_{3}[X]$, $X$ est scindé à racines simples et $X(1+\varepsilon X)^{2}\xrightarrow[\varepsilon\to0]{}X$ et $-\frac{1}{\varepsilon}$ est racine double, donc $\Delta$ n'est pas ouvert.
	\end{enumerate}
\end{solution}

\begin{remark}
	On peut cependant considérer 
	$$\Delta_{n}=\{P\in\C_{p}[X]\bigm| P\text{ scindé à racines simples sur }\R\text{ et }\deg(P)=n\}$$
	Si $\lambda_{1}<\lambda_{2}<\dots<\lambda_{n}$ sont les racines (distinctes) de $R$ sur $\R$, on choisit $\alpha_{0}\in]-\infty,\lambda_{1}$, $\alpha_{n}\in]\lambda_{n},+\infty[$ et $\alpha_{i}\in]\lambda_{i},\lambda_{i+1}[$ si $i=1,\dots,n-1$. 

	Pour tout $k\in\{0,\dots,n-1\}$, on a $P(\alpha_{k})P(\alpha_{k+1})<0$ (car les racines de $P$ provoquent des changements de signe). Soit \function{\Psi}{\R_n[X]}{\R^n}{Q}{(Q(\alpha_{k})Q(\alpha_{k+1}))_{0\leqslant k\leqslant n-1}}
	$\Psi$ est continue sur $\R_{n}[X]$ et $\Psi(P)\in(\R_{-}^{*})^{n}$ qui est ouvert, donc il existe $r>0$ tel que si $\Vert P-Q\Vert<r$, alors $\Psi(Q)\in(\R_{-}^{*})^{n}$. Donc $Q$ change $n$ fois de signe, et admet au moins $n$ racines. Mais $\deg(Q)=n$, donc $Q$ est scindé à racines simples sur $\R$, donc $\Delta_{n}$ est ouvert dans $\{P\in\R[X]\bigm|\deg(P)=n\}$.
\end{remark}

\begin{remark}
	$$\{M\in\M_{n}(\C)\bigm|M\text{ diagonalisable à racines simples}\}=\{M\in\M_{n}(\C)\bigm|\chi_M\text{ sciné à racines simples}\}$$
	est un ouvert de $\M_{n}(\C)$ car $M\mapsto\chi_{M}$ est continue sur $\M_{n}(\C)$, et c'est aussi vrai sur $\R$.
\end{remark}

\begin{solution}
	\phantom{}
	\begin{enumerate}
		\item Soit \function{f}{\M_n(\R)}{\M_n(\R)}{A}{A^{n}}
		$f$ est continue et $F=f^{-1}(\{0\})$ donc $F=\overline{F}$.

		Soit $M_{0}\in F$, $X^{n}$ annule $M_{0}$ donc $M_{0}$ est trigonalisable: on écrit $M_{0}$ dans une base où les coefficients diagonaux sont tous nuls. Soit alors $M_{\varepsilon}$ la même matrice dans la même base en rajoutant simplement $\varepsilon$ en première position de la diagonale. Alors $M_{\varepsilon}\xrightarrow[\varepsilon\to0]{}M_{0}$ et $M_{\varepsilon}\notin F$ donc $\mathring{F}=\emptyset$. Notons que cela signifie que $F$ est dense.

		\item La norme dérive du produit scalaire $(A|B)\mapsto\Tr(A^{\mathsf{T}}B)$. Soit $M\in F$, on a $\Vert M-I_{n}\Vert^{2}=\Vert M\Vert^{2}+\Vert I_{n}\Vert^{2}-2(M|I_{n})$. On a $(M|I_{n})=\Tr(M)=0$ car $M$ est nilpotente. Donc $\Vert M-I_{n}\Vert^{2}$ est minimale pour $\Vert M\Vert^{2}$ minimale, donc pour $M=0\in F$. Donc $d(I_{n},F)=\Vert I_{n}\Vert=\sqrt{n}$ (et la distance est atteinte pour $0_{\M_n(\R)}$).
	\end{enumerate}
\end{solution}

\begin{solution}
	\phantom{}
	\begin{enumerate}
		\item $A\mapsto\det(A)$ est continue et $GL_{n}(\K)=\det^{-1}(\K^{*})$ est donc ouvert. Si $A\in\M_{n}(\K)$, pour $p\in\N$, on pose $A_{p}=A-\frac{1}{p+1}I_{n}$. Comme $\Sp(A)$ est fini, il existe $N\in\N$, tel que pour tout $p\geqslant N$, $\frac{1}{p+1}\notin\Sp(A)$. Donc pour tout $p\geqslant N$, $A_{p}\in GL_{n}(\K)$, et $A_{p}\xrightarrow[p\to+\infty]{}A$ donc $GL_{n}(\K)$ est dense dans $\M_{n}(\K)$.
		\item On fixe $B\in\M_{n}(\K)$. Soit $A\in GL_{n}(\K)$. On écrit $BA=A^{-1}(AB)A$ donc $AB$ et $BA$ sont semblables donc $\chi_{AB}=\chi_{BA}$. Comme, à $B$ fixé, $A\mapsto\chi_{AB}$ et $A\mapsto\chi_{BA}$ sont continues sur $\M_{n}(\K)$, on a le résultat par densité.
	\end{enumerate}
\end{solution}

\begin{solution}
	\phantom{}
	\begin{enumerate}
		\item On a $v_{p}\circ(id_{E}-u)=(id_{E}-u)\circ v_{p}=\frac{1}{p}(id_{E}-u^{p})$, donc $\Vert v_{p}\circ(id_{E}-u)\Vert\leqslant\frac{1}{p}(\Vert id_{E}\Vert+\Vert u^{p}\Vert)\xrightarrow[p\to+\infty]{}0$.
		
		Soit $x\in\ker(u-id_{E})\cap\im(u-id_{E})$, on a $u(x)=x$ et il existe $y\in E$, $x=(u-id_{E})(y)$. On a $v_{p}(x)=\frac{1}{p}(px)=x$ et $v_{p}(x)=v_{p}\circ(u-id_{E})(y)\xrightarrow[p\to+\infty]{}0$ d'où $x=0$. Le théorème du rang permet de conclure.

		\item Soit $x\in E$, on écrit $x=x_{1}+x_{2}$ avec $\Pi(x)=x_{1}$ et $x_{2}=(u-id_{E})(y_{2})$. Alors $v_{p}(x)=x_{1}+v_{p}\circ(u-id_{E})(y_{2})\xrightarrow[p\to+\infty]{}x_{1}=\Pi(x)$.
	\end{enumerate}
\end{solution}

\begin{solution}
	\phantom{}
	\begin{enumerate}
		\item Pour tout $x\in A$, $f_{n}(x)\in A$ car $A$ est convexe. Soit $(x,y)\in A^{2}$, on a
		$$\Vert f_{n}(x)-f_{n}(y)\Vert=\Bigl(1-\frac{1}{n}\Bigr)\Vert f(x)-f(y)\Vert\leqslant\Bigl(1-\frac{1}{n}\Bigr)\Vert x-y\Vert$$
		Donc $f_{n}$ est $(1-\frac{1}{n})$-lipschitzienne. On forme \function{g_n}{A}{\R}{x}{\Vert f_n(x)-x\Vert}
		qui est continue. Soit $x_{n}\in A$ telle que $g_{n}(x_{n})=\min\limits_{x\in A}g_{n}(x)$ (existe car $A$ est compact et $g_{n}$ continue). On a $x_{n}\in A$, d'où $f_{n}(x_{n})\in A$ et 
		$$g_{n}(f_{n}(x_{n}))=\Vert f_{n}(f_{n}(x_{n}))-f_{n}(x_{n})\Vert\leqslant\Bigl(1-\frac{1}{n}\Bigr)\Vert f_{n}(x_{n})-x_{n}\Vert=\Bigl(1-\frac{1}{n}\Bigr)g_{n}(x_{n})$$
		Si $g_{n}(x_{n})\neq0$, alors on aurait $g_{n}(f(x_{n}))<g_{n}(x_{n})$ ce qui n'est pas possible. Donc $g_{n}(x_{n})=0$ et $f_{n}(x_{n})=x_{n}$.

		Soit $y_{n}$ un autre point fixe, on a 
		$$\Vert f_{n}(x_{n})-f_{n}(y_{n})\Vert=\Vert x_{n}-y_{n}\Vert\leqslant\Bigl(1-\frac{1}{n}\Bigr)\Vert x_{n}-y_{n}\Vert$$
		donc $x_{n}=y_{n}$.

		\item On a $(x_{n})_{n\in\N}\in A^{\N}$ et on extrait (car $A$ est compact) et on a 
		$$x_{\sigma(n)}\xrightarrow[n\to+\infty]{}x\in A$$
		On a 
		$$f_{\sigma(n)}(x_{\sigma(n)})=x_{\sigma(n)}=\underbrace{\frac{1}{\sigma(n)}f(x_{0})}_{\xrightarrow[n\to+\infty]{}0}+\underbrace{\Bigl(1-\frac{1}{\sigma(n)}\Bigr)f(x_{\sigma(n)})}_{\xrightarrow[n\to+\infty]{}f(x)}$$
		par continuité de $f$. Donc $f(x)=x$.

		\item Soit $(x,y)\in A^{2}$, points fixes de $f$, et $t\in[0,1]$, on pose $z=tx+(1-t)y$. On a 
		\begin{align*}
			\Vert x-y\Vert
			&=\Vert f(x)-f(y)\Vert\\
			&\leqslant \Vert f(x)-f(z)\Vert+\Vert f(z)-f(y)\Vert\\
			&\leqslant\Vert x-z\Vert+\Vert z-y\Vert\\
			&=(1-t)\Vert x-y\Vert+t\Vert x-y\Vert\\
			&=\Vert x-y\Vert
		\end{align*}
		On a donc égalité partout: $\Vert f(x)-f(y)\Vert=\Vert f(x)-f(z)\Vert+\Vert f(z)-f(y)\Vert$ et $\Vert f(x)-f(z)\Vert=\Vert x-z\Vert$, $\Vert f(z)-f(y)\Vert=\Vert z-y\Vert$ car $f$ est $1$-lipschitzienne.

		Comme la norme est euclidienne, il existe $\lambda\in\R_{+}$ tel que $f(x)-f(z)=\lambda(f(z)-f(y))$ d'où $f(x)+\lambda f(y)=(\lambda+1)f(z)$ d'où $f(z)=\frac{x+\lambda y}{\lambda+1}=t'x+(1-t')y$ avec $t'=\frac{1}{\lambda+1}\in[0,1]$. En reportant, on a 
		$$\Vert f(x)-f(z))\Vert=\Vert x-t'x-(1-t')y\Vert=(1-t')\Vert x-y\Vert=\Vert x-z\Vert=(1-t)\Vert x-y\Vert$$
		Si $x\neq y$, alors $t=t'$ et $f(z)=tx+(1-t)y=z$.

		\item Soit dans $\R^{2}$, $\overline{B_{\Vert\cdot\Vert}(0,1)}=[-1,1]^{2}=A$. Soit \function{f}{A}{A}{(x,y)}{(x,\vert x\vert)}
		On a 
		\begin{align*}
			\Vert f(x_{1},y_{1})-f(x_{2},y_{2})\Vert_{\infty}
			&= \Vert (x_{1},\vert x_{1}\vert)(x_{2},\vert x_{2}\vert)\Vert_{\infty}\\
			&=\max\{\vert x_{1}-x_{2}\vert, \bigl\vert\vert x_{1}\vert-\vert x_{2}\vert\bigr\vert\}\\
			&=\vert x_{1}-x_{2}\vert\\
			&\leqslant\Vert (x_{1},y_{1})-(x_{2},y_{2})\Vert_{\infty}
		\end{align*}
		Donc $f$ est 1-lipschitzienne, on a $f(x,y)=(y,x)$ si et seulement si $y=\vert x\vert$. Donc ici, $F$ n'est pas convexe.
	\end{enumerate}
\end{solution}

\begin{solution}
	\phantom{}
	\begin{enumerate}
		\item On a pour tout $(x,y)\in E^{2}$, $f(x+y)=f(x)+f(y)$ et par récurrence, pour tout $n\in\Z$, $f(nx)=nf(x)$. Pour $r=\frac{p}{q}\in\Q$, on a $f(qrx)=qf(rx)=f(px)=pf(x)$ donc $f(rx)=rf(x)$. Par densité de $\Q$ dans $\R$ et continuité de $f$, on a pour tout $\lambda\in\R$, $f(\lambda x)=\lambda f(x)$. Donc $f$ est linéaire.
		
		Pour $\K=\C$, cela ne marche pas. Contre-exemple: la conjugaison dans $\C$.

		\item On étudie la série, pour $x$ fixé de terme général 
		$$\Vert v_{n+1}(x)-v_{n}(x)\Vert=\frac{1}{2^{n}}\Vert f(2^{n+1}x)-2f(2^{n}x)\Vert\leqslant\frac{M}{2^{n+1}}$$
		qui est donc convergente. Donc $(v_{n})_{n\in\N}$ converge.

		\item On a $v_{0}(x)=f(x)$, donc $\sum_{n=0}^{+\infty}v_{n+1}(x)-v_{n}(x)=g(x)-f(x)$. $f$ étant continue, $v_{n}$ l'est aussi, et pour tout $n\in\N$, comme pour tout $x\in E$, $\Vert (v_{n+1}-v_{n})(x)\Vert\leqslant\frac{M}{2^{n+1}}$, donc $g$ est continue.
		
		\item On a, pour tout $(x,y)\in E^{2}$,
		$$\Vert v_{n}(x+y)-v_{n}(x)-v_{n}(y)\Vert=\Vert \frac{1}{2^{n}}f(2^{n}(x+y))-\frac{1}{2^{n}}(f(2^{n}x)+f(2^{n}y))\Vert\leqslant\frac{M}{2^{n}}$$
		Donc quand $n\to+\infty$, $g(x+y)=g(x)+g(y)$.

		On a pour tout $x\in E$, 
		$$\Vert g(x)-f(x)\Vert=\Bigl\lVert\sum_{n=0}^{+\infty}v_{n+1}(x)-v_{n}(x)\Bigr\Vert\rVert\leqslant\sum_{n=0}^{+\infty}\Vert v_{n+1}(x)-v_{n}(x)\Vert\leqslant\sum_{n=0}^{\infty}\frac{M}{2^{n}}=M$$

		Soit maintenant $h$ linéaire continue telle que $h-f$ soit bornée, soit $M'=\sup\limits_{x\in E}\Vert h(x)-f(x)\Vert$. On a donc 
		$$\Vert v_{n}(x)-h(x)\Vert=\Bigl\Vert\frac{1}{2^{n}}f(2^{n}x)-\frac{1}{2^{n}}h(2^{n}x)\Bigr\Vert\leqslant\frac{M'}{2^{n}}$$
		car $h$ est linéaire. Donc quand $n\to+\infty$, $g(x)=h(x)$ car $\lim\limits_{n\to+\infty}v_{n}(x)=g(x)$.
	\end{enumerate}
\end{solution}

\begin{solution}
	En particulier, pour $t=f(0)$, $f^{-1}(\{f(0)\})=\{x\in E\bigm| f(x)=f(0)\}$ est borné (car compact). Donc il existe $A$ tel que $f^{-1}(\{f(0)\})\subset\overline{B(0,A)}$. Par contraposée, pour tout $x\in E$, si $\Vert x\Vert>A$, alors $f(x)\neq f(0)$.

	On montre alors que $E\setminus\overline{B(0,A)}$ est connexe par arcs (faire le tour de la boule par l'extérieur).

	$f$ étant continue, d'après le théorème des valeurs intermédiaires, on a soit pour tout $x\in E\setminus\overline{B(0,A)}$, $f(x)>f(0)$ soit $f(x)<f(0)$. Quitte à remplacer $f$ par $-f$, on se place dans le cas $f(x)>f(0)$. Comme on est en dimension finie sur $\overline{B(0,A)}$ compact, $f$ atteint son minimum et ce minimum est plus petit que $f(0)$, c'est donc un minimum global.
\end{solution}

\begin{remark}
	C'est faux pour $n=1$. Contre-exemple: $f=id_{\R}$.
\end{remark}

\begin{solution}
	Si c'était le cas, on prend un cercle $\mathcal{C}$ compact (et connexe par arcs). $f(\mathcal{C})$ est compact connexe par arc dans $\R$. On note $f(\mathcal{C})=[a,b]$ (avec $a<b$ car $f$ injective). Si $x\in\mathcal{C}$ est tel que $f(x)=\frac{a+b}{2}$, on $\underbrace{f(\mathcal{C}\setminus\{x\})}_{\text{connexe par arc}}=\underbrace{[a,b]\setminus\Bigl\{\frac{a+b}{2}\Bigr\}}_{\text{pas connexe par arc}}$ donc une telle fonction n'existe pas.
\end{solution}

\begin{solution}
	\phantom{}
	\begin{enumerate}
		\item Pour tout $n\in\N$, $\Vert e_{n}\Vert_{l^{1}}=1$ et $\vert K_{n}\vert=\vert\varphi(e_{n})\vert\leqslant\vertiii{\varphi}$ donc $(K_{n})_{n\in\N}$ est bornée. On note $M=\sup\vert K_{n}\vert\leqslant\vertiii{\varphi}$.
		
		Soit maintenant $u=(u_{n})_{n\in\N}\in l^{1}$. On a, pour $N\in\N$, 
		$$\Biggl\lVert u-\sum_{n=0}^{N}u_{n}e_{n}\Biggr\rVert_{1}=\sum_{n=N+1}^{\infty}\vert u_{n}\vert\xrightarrow[N\to+\infty]{}0$$
		(reste d'une série convergente). Par continuité de $\varphi$, on a donc 
		$$\vert \varphi(u)\vert\leqslant\sum_{n=0}^{\infty}\vert u_{n}\vert \vert K_{n}\vert\leqslant M\Vert u\Vert_{1}$$

		Ainsi, $\vertiii{\varphi}\leqslant M$ et donc $\vertiii{\varphi}=M$.

		\item $F$ est linéaire et une isométrie d'après la question précédente, donc injective. Soit $(K_{n})_{n\in\N}\in l^{\infty}$. On définit \function{\varphi}{l^1}{\R}{u=(u_n)_{n\in\N}}{\sum_{n=0}^{\infty}u_{n}K_{n}}
		Elle est bien définie car $\sum_{n=0}^{+\infty}\vert u_{n}\vert<+\infty$ et $(K_{n})_{n\in\N}$ est bornée. Elle est linéaire, et continue car $\vert\varphi(u)\vert\leqslant\Vert(K_{n})_{n\in\N}\Vert_{\infty}\Vert u\Vert_{1}$.

		Enfin, pour tout $n\in\N,\varphi(e_{n})=K_{n}$. Donc $F(\varphi)=(K_{n})_{n\in\N}$ et $F$ est surjective. Donc $F$ est une isométrie bijective et le dual topologique de $l^{1}$ est équivalent à $l^{\infty}$.
	\end{enumerate}
\end{solution}

\begin{solution}
	\phantom{}
	\begin{enumerate}
		\item Soit $\varphi$ une forme linéaire non nulle telle que $K=\ker(\varphi)$/ Si $F$ est dense, $\varphi$ est discontinue. Soit $(a,b)\in(E\setminus H)^{2}$ et $(x_{n})_{n\in\N}\in H^{\N}$ qui converge vers $b-a$ (existe car $H$ est dense). La suite $(a+x_{n})_{n\in\N}$ converge vers $b$. Pour $n\in\N$, on a $\varphi(a+x_{n})=\varphi(a)\neq0$, et pour $t\in[0,1]$, $\varphi(t(a+x_{n})+(1-t)(a+x_{n+1}))=\varphi(a)\neq0$. Donc $[a+x_{n},a+x_{n+1}]\subset E\setminus H$.
		
		Soit $\gamma:[0,1]\to E\setminus H$ telle que 
		$$
		\left\{
			\begin{array}[]{rcll}
				\gamma(t) & = &\alpha_{n}t+\beta_{n}\in[a+x_{n},a+x_{n+1}]\subset E\setminus H &\text{si }t\in[1-\frac{1}{n},1-\frac{1}{n+1}]\\
				\gamma(1) & = &b&\\
				\gamma(t) &= & a+tx_{0}&\text{si }t\in[0,\frac{1}{2}]
			\end{array}
		\right.
		$$

		On cherche à définir $\alpha_{n}$ et $\beta_{n}$: on veut $\gamma(1-\frac{1}{n})=a+x_{n}$ et $\gamma(1-\frac{1}{n+1})=a+x_{n+1}$ (pour la continuité en se raccordant au $x_{n}$). En résolvant le système, on trouve $\alpha_{n}=n(n+1)(x_{n}-x_{n+1})$ et $\beta_{n}=a+x_{n}-(n-1)(n+1)(x_{n}-x_{n+1})$.

		Soit alors $\varepsilon>0$, il existe $N\in\N$ tel que pour tout $n\geqslant N\colon\Vert x_{n}+a-b\Vert<\varepsilon$ et pour tout $n\geqslant N$, pour tout $t\in[1-\frac{1}{n},1-\frac{1}{n+1}[$, $\gamma(t)\in[a+x_{n},a+x_{n+1}]\subset B(b,\varepsilon)$ par convexité de la boule. Donc $\lim\limits_{t\to 1}\gamma(t)=b$ et $\gamma$ est continue. Donc $E\setminus H$ est connexe par arcs.

		\item Soit $\varphi$ une forme linéaire telle que $\ker(f)=H$ est fermé. Alors $\varphi$ est continue (à redémontrer). Soit $x\in E\setminus H$, on a $\varphi(x)\varphi(-x)<0$ et d'après le théorème des valeurs intermédiaires, si $E\setminus H$ était connexe par arcs, $\varphi$ s'annulerait sur $E\setminus H$ ce qui n'est pas vrai. Donc $E\setminus H$ n'est pas connexe par arcs.
		
		\item Si $\K=\C$, si $H$ est dense alors $E\setminus H$ est connexe par arc d'après la première question. Si $H$ est fermé, soit $\varphi$ une forme linéaire continue telle que $\ker(f)=H$. Soit $(x_{1},x_{2})\in(E\setminus H)^{2}$. 
		
		\begin{itemize}
			\item Si $\frac{\varphi(x_{1})}{\varphi(x_{2})}\notin\R_{-}^{*}$, alors pour tout $t\in[0,1]$, $\varphi(\underbrace{tx_{1}+(1-t)x_{2}}_{\in E\setminus H})\neq0$ et on peut relier directement $x_{1}$ et $x_{2}$.
			\item Sinon, il existe $\theta\in\R,(\rho,\rho')\in(\R_{+}^{*})^{2}$ avec $\varphi(x_{1})=\rho e^{\mathrm{i}\theta}$ et $\varphi(x_{2})=\rho'e^{\mathrm{i}(\theta+\pi)}$. Alors $x_{3}=ix_{1}$ est tel que $[x_{1},x_{3}]\subset E\setminus H$ et $[x_{2},x_{3}]\subset E\setminus H$ (on contourne l'origine par une rotation de l'angle $\frac{\pi}{2}$). Par conséquent, on peut utiliser $x_{3}$ pour relier $x_{1}$ et $x_{2}$ donc $E\setminus H$ est connexe par arcs.
		\end{itemize}
	\end{enumerate}
\end{solution}

\begin{solution}
	Soit \function{\varphi}{\R_{+}^{*}}{\R}{x}{((x,\sin(\frac{1}{x})))}
	$\varphi$ est continue et $\Gamma)\varphi(\R_{+}^{*})$ est connexe par arcs.

	On a $\overline{\Gamma}=\Gamma\cup\Gamma'$ avec $\Gamma'=\{(0,y)\bigm| y\in[-1,1]\}$. En effet, pour tout $y\in[-1,1]$, on pose $x_{k}=\frac{1}{\arcsin(y)+2k\pi}$. On a $\sin(\frac{1}{x_{k}})=y\xrightarrow[k\to+\infty]{}y$ donc $(0,y)=\lim\limits_{k\to=+\infty}(x_{k},\sin(\frac{1}{x_{k}}))\in\overline{\Gamma}$.

	Réciproquement, si $(x,y)\in\overline{\Gamma}$, il existe $(x_{k})\in(\R_{+}^{*})^{\N}$ avec $x=\lim\limits_{k\to+\infty}x_{k}$ et $y=\lim\limits_{k\to+\infty}\sin(\frac{1}{x_{k}})$. Si $x>0$, par continuité, $y=\sin(\frac{1}{x})$ et $(x,y)\in\Gamma$. Si $x=0$, $y\in[-1,1]$ donc $(x,y)\in\Gamma'$.

	Si $\overline{\Gamma}$ est connexe par arcs, il existe \function{\gamma}{[0,1]}{\overline{\Gamma}}{t}{(x(t),y(t))}
	continue telle que $\gamma(0)=(0,0)$ et $\gamma(1)=(\frac{1}{\pi},0)$. La première projection $t\mapsto x(t)$ est continue avec $x(0)=0$ et $x(1)=\frac{1}{\pi}$. On définit maintenant $t_{1}=\sup\{t\in[0,1]\bigm| x(t)=0\}$. Par continuité, $x(t_{1})=0$ et donc $t_{1}<1$. Donc pour tout $t>t_{1}$, $x(t)>0$ et $\gamma(t)=(x(t),\sin(\frac{1}{x(t)}))$ pour $t>t_{1}$ et $\gamma(t_{1})=(0,y_{1})$ avec $y_{1}\in[-1,1]$.

	Or, -1 et 1 n'appartiennent pas simultanément à $]y_{1}-\frac{1}{2},y_{1}+\frac{1}{2}[$. On peut supposer que $1\notin]y_{1}-\frac{1}{2},y_{1}+\frac{1}{2}[$. Comme $\gamma$ est continue, il existe $t_{2}>t_{1}$ tel que pour tout $t\in]t_{1},t_{2}]$, $\sin(\frac{1}{x(t)})\in]y_{1}-\frac{1}{2},y_{1}+\frac{1}{2}[$. Or $x(t_{2})>0$ et $x(t_{1})=0$ donc il existe $k\in\N^{*}$, $t_{0}\in]t_{1},t_{2}]$ tel que $x(t_{0})=\frac{1}{2k\pi+\frac{\pi}{2}}$ (théorème des valeurs intermédiaires). Mais alors $\sin(\frac{1}{x(t_{0})})=1\notin]y_{1}-\frac{1}{2},y_{1}+\frac{1}{2}[$ ce qui contredit ce qui précède.

	Donc $\overline{\Gamma}$ n'est pas connexe par arcs.
\end{solution}

\begin{solution}
	\phantom{}
	\begin{enumerate}
		\item Pour tout $n\in\N$, $u_{n}\in K$ car $u_{n}$ est le barycentre de $(a,T(a),\dots,T^{n}(a))$ et $K$ est convexe. Comme $K$ est compact, on peut extraire $u_{\sigma(n)}\xrightarrow[n\to+\infty]{}u\in K$. Alors
		$$(id_{E}-T)(u_{\sigma(n)})=\frac{1}{\sigma(n)+1}(id_{E}-T^{\sigma(n)+1})(a)$$
		d'où 
		$$\rVert(id_{E}-T)(u_{\sigma(n)})\lVert\leqslant\frac{1}{\sigma(n)+1}\times 2M\xrightarrow[n\to+\infty]{}0$$
		avec $M=\sup\limits_{x\in K}\Vert x\Vert$ (existe car $K$ est compact donc borné). Par continuité de $T$, on a $T(u)=u$.

		\item Posons $F'=\{u\in K\bigm| T(u)=u\}$ fermé car $K'=K\cap\Bigl((\underbrace{id_{E}-T}_{\text{continu}})^{-1}\{0\}\Bigr)$.
		Donc $K'$ est compact et non vide d'après la première question. De plus, pour tout $(u_{1},u_{2})\in K'^{2}$, pour tout $t\in[0,1]$, par linéarité de $T$, on a 
		$$T(tu_{1}+(1-t)u_{2})=tu_{1}+(1-t)u_{2}$$
		donc $K'$ convexe. De plus, comme $U\circ T=T\circ U$, pour tout $u\in K'$, on a $T(U(u))=U(T(u))=U(u)$ donc $U(u)\in K'$. On applique alors la question 1 à $K'$ est il existe $y\in K'\colon U(y)=y$ et $T(y)=y$.
	\end{enumerate}
\end{solution}

\begin{solution}
	\phantom{}
	\begin{enumerate}
		\item C'est le théorème du rang car $\rg(u)\leqslant n\leqslant p-2$, et $H=\{(\alpha_{1},\dots,\alpha_{p})\bigm|\sum_{i=1}^{p}\alpha_{i}=\}$ est de dimension $p-1$ donc $H\cap\ker(u)\neq\{0\}$ (formule de Grassmann).
		
		\item On a 
		$$\sum_{i=1}^{p}(\lambda_{i}+t\alpha_{i})x_{i}=\sum_{i=1}^{p}\lambda_{i}x_{i}+t\sum_{i=1}^{p}\alpha_{i}x_{i}=x$$
		et
		$$\sum_{i=1}^{p}(\lambda_{i}+t\alpha_{i})=\sum_{i=1}^{p}\lambda_{i}+t\sum_{i=1}^{p}\alpha_{i}=1$$

		Soit $I_{+}=\{i\in\{1,\dots,p\}\bigm|\alpha_{i}>0\}$ et $I_{-}=\{i\in\{1,\dots,p\}\bigm|\alpha_{i}<0\}$. On a $I_{+}\neq\emptyset$ et $I_{-}\neq\emptyset$ car $\sum_{i=1}^{p}\alpha_{i}=0$ et $(\alpha_{1},\dots,\alpha_{p})\neq(0,\dots,0)$. Soit $t\geqslant0$. Pour tout $i\in I_{+}$, $\lambda_{i}+t\alpha_{i}\geqslant0$. Pour $i\in I_{-}$, $\lambda_{i}+t\underbrace{\alpha_{i}}_{<0}\geqslant 0$ si et seulement si $t\leqslant-\frac{\lambda_{i}}{\alpha_{i}}$. Prenons alors 
		$$t=\min\limits_{i\in I_{-}}\Bigl(-\frac{\lambda_{i}}{\alpha_{i}}\Bigr)$$
		On au aussi pour tout $i\in I_{-}$, $\lambda_{i}+t\alpha_{i}\geqslant 0$ et il existe $i_{0}\in I_{-}$ tel que $\lambda_{i_{0}}+t\alpha_{i_{0}}=0$.

		\item Par récurrence descendante, on se ramène à n+1 points car si $x$ est barycentre de $p$ points avec $p\geqslant n+2$, alors il est barycentre de $p-1$ points.
		
		\item Soit $A=\{(\lambda_{1},\dots,\lambda_{n+1})\in\R_{+}^{n+1}\bigm|\sum_{i=1}^{n+1}\lambda_{i}=1\}$ fermé et borné en dimension finie donc compact. Soit \function{f}{A\times K^{n+1}}{\conv(K)}{((\lambda_{1},\dots,\lambda_{n}),(x_{1},\dots,x_{n+1}))}{\sum_{i=1}^{n+1}\lambda_{i}x_{i}}
		$f$ est surjective et continue, donc $\conv(K)$ est l'image continue d'un compact donc $\conv(K)$ est compact.
	\end{enumerate}
\end{solution}

\cleardoublepage
\section{Fonction d'une variable réelle}

\begin{solution}
	On note $A_{h}=\{\vert\varphi(x)-\varphi(y)\vert\bigm|(x,y)\in I^{2}\text{ et }\vert x-y\vert\leqslant h\}$.
	\begin{enumerate}
		\item $\omega_{\varphi}$ est bien défini car $\vert\varphi(x)-\varphi(y)\vert\leqslant 2\Vert\varphi\Vert_{\infty}$). Si $0<h\leqslant h'$, alors $A_{h}\subset A_{h'}$ donc $\sup(A_{h})\leqslant\sup(A_{h'})$ donc $\omega_{\varphi}(h)\leqslant\omega_{\varphi}(h')$.
		\item Soit $(h,h')\in(\R_{+}^{*})^{2}$, soit $(x,y)\in I^{2}$ tel que $\vert x-y\vert\leqslant h+h'$ (où on peut supposer que $x\leqslant y$).
		\begin{itemize}
			\item Si $y\in[x,x+h]$, alors $\vert x-y\vert\leqslant h$ donc $\vert\varphi(x)-\varphi(y)\vert\leqslant\omega_{\varphi}(h)\leqslant\omega_\varphi(h)+\omega_{\varphi}(h')$
			\item Si $y\in[x+h,x+h+h']$, $\vert\varphi(x)-\varphi(y)\vert\leqslant\vert\varphi(x)-\varphi(x+h)\vert+\vert\varphi(x+h)-\varphi(y)\vert\leqslant\omega_\varphi(h)+\omega_{\varphi}(h')$ car $\vert x-(x+h)\vert\leqslant h$ et $\vert x+h-y\vert\leqslant h'$.
		\end{itemize}
		Donc $\omega_{\varphi}(h+h')\leqslant\omega_\varphi(h)+\omega_\varphi(h')$.
		\item Par récurrence sur $n\in\N$, on a $\omega_\varphi(nh)=n\omega_\varphi(h)$. Si $\lambda\in\R_{+}^{*}$, on a $\lambda h\leqslant(\lfloor \lambda\rfloor+1)h$ et par croissance et ce qui précède, on a 
		$$\omega_\varphi(\lambda h)\leqslant(\lfloor\lambda\rfloor+1)\omega_\varphi(h)\leqslant(\lambda+1)\omega_\varphi(h)$$
		\item Soit $\varepsilon>0$. $\varphi$ étant uniformément continue, il existe $\alpha>0$ tel que pour tout $(x,y)\in I^{2}$, si $\vert x-y\vert\alpha$ on a $\vert\varphi(x)-\varphi(y)\vert\leqslant\varepsilon$ et on a pour $h\leqslant\alpha$, $\omega_\varphi(h)\leqslant\varepsilon$ d'où $\lim\limits_{h\to0}\omega_\varphi(h)=0$.
		
		Soit alors $h_{0}>0$ fixé et $h>0$,
		\begin{itemize}
			\item si $h_{0}\leqslant h$, on a $0\leqslant\omega_\varphi(h)-\omega_\varphi(h_0)\leqslant\omega_\varphi(h-h_0)$.
			\item si $h\leqslant h_{0}$, on a $0\leqslant\omega_\varphi(h_0)-\omega_\varphi(h)\leqslant\omega_\varphi(h_0-h)$.
		\end{itemize}
		Dans tous les cas, on a $\vert\omega_\varphi(h)-\omega_\varphi(h_{0})\vert\leqslant\omega_\varphi(\vert h_{0}-h\vert)$. Donc on a bien $\lim\limits_{h\to h_{0}}\omega_\varphi(h)=\omega_\varphi(h_{0})$. Donc $\omega_{\varphi}$ est continue (et même uniformément).
	\end{enumerate}
\end{solution}

\cleardoublepage
\section{Suites et séries de fonctions}
\cleardoublepage
\section{Séries entières}
\cleardoublepage
\section{Intégration}
\cleardoublepage
\section{Espaces préhilbertiens}
\cleardoublepage
\section{Espaces euclidiens}
\cleardoublepage
\section{Calcul différentiel}
\cleardoublepage
\section{\'Equation différentielles linéaires}


\end{document}